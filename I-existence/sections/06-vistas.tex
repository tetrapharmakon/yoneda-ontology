\section{Vistas on ontologies}\label{vistas}
\epigraph{¿No basta un solo término repetido para desbaratar y confundir la serie del tiempo? ¿Los fervorosos que se entregan a una línea de Shakespeare no son, literalmente, Shakespeare?}{\cite{confutacion}}
Looking at our \autoref{sec_coins} it's surprising how mathematically consistent Borges' universe becomes when it is described through category theory. As we already said, this has to be regarded as an experiment, a literary \emph{divertissement}, and a test-bench for our main claim, that as abstract as it may seem ontology has a quantitative content.

There is clearly no point to just analyse Borges' literary work: we can go further. The present section has this purpose, while sketching somelong term goals our work can aspire to become. We begin with a more academic discussion of classical idealism \emph{à la} Berkeley; of course, in light of our \autoref{incendiata}, this can still be linked to Tl\"on's universe, as Borges has always been fascinated by, and mocked, classical idealism (see \cite{confutacion}).

To this, it follows a more wide-ranging discourse; surely a less substantiated one, but we aim at laying a foundation for future work, ours or others', in a research track that we feel is fertile and promising.
\subsection{Answers to the idealist}\label{berkelei}
Let us reconsider \autoref{bla}; although in passing, we mentioned famous Berkeley's view of perception as a bundle of stimulation incapable to cohere. Let's say clearly that endeavouring on such a wide ground as classical idealism isn't the purpose of our work; yet, in his novel Borges regards Tl\"on's language and philosophy as a concrete realisation of Berkeley's theory of knowledge.


	Within topos theory, a proposition $p$'s truth values can be more nuanced, as it can be false with a certain strength $t : I$. Berkeleyan instantaneism aims to obtain (once he accepted that coins exist only a "bit" in some points) that $\sum_u p(u,c,d) = (\bot,1)$.
	
	
	He cannot do it by construction, but if he wanted to he must provide an alternative model of the paradox that is equally valid (however, it seems to us difficult not to speak in terms of `` forces '' or their equivalents, in single `` instants '').
	
	To be even more picky, when asked if the coins exist or not the correct answer is another question: `` In which $I$ are you? ''. And, given $I$ totally ordered: `` At what point do you want calculate force? ''.
	
	The point can be a temporal, spatial or any other kind of reference (depending on the structure of $U$), the important thing is that the set has the properties indicated by us.
	
	To be able to "talk" inside the topos, and answer the question about existence, it's necessary to first provide the "data" (observer, subset of $C$, day of the week) and then we can assign a truth value endowed with strength.
	
	Replacing classical notions (identity, persistence-in-time) with precise notions suggested by the language we propose, therefore also means changing natural language. When the idealist says that in the istants in which he closes his eyes the table in front of him no longer exists, he is talking about `` strength '' in the sense defined here: he only has to make it explicit. He puts himself in a `` non-classic '' universe (a topos) with multiple truth values and simply has to calculate \emph{what is} the value of $p$.


The expansion of truth-values \footnote{Ad esser precisi: $\{0,1\}$ è ``fisso''; la varietà di risposte è data dalla scelta di $I$ e dalla struttura di $U$} translates into an expansion of the possible answers to be given to the coins question. We don't show the `` correct '' one but the topos in which to find it, the context of discourse in which it's necessary to place, intuitively a fuzzy context, which avoids the barren dichotomies in which the debate had stalled for a few centuries.

Note (as anticipated in \ref{existence}) that we were able to model the paradox without using a temporal logic framework; it's legitimate interpreting the elements of $I$ as instants, but not necessary. Temporal description is a sub-case of the more general model we have provided. Even a presentist or a Borges' coherent idealist (who deduces the nonexistence of time from the Berkeleyan theses) can establish the truth of $p$ without renouncing his ontological stance, but must then accept the result of the calculation.

Another advantage that (not ontly) the idealist could appreciate is given by the density of $I$. Property that allows, when there is disagreement between two interlocutors $X$ and $Y$ on the degree of existence of the coins, (mainly due to the functional dependence of the propositions on the configurations), to find an intermediate force between the force assigned by $X$ and the one assigned by $Y$. This is a way of formalizing intersubjectivity within Tl\"on.

The metaphysical stance that emerging from the work seems halfway between empiricism and idealism. %La critica storica dell'idealismo al metodo scientifico è l'uso di prove indirette di esistenza (come le dimostrazioni non costruttive che non piacciono agli intuizionisti).%

In our fuzzy vision of existence we can assume that the invisible boat that no one sees, except because the water moves, does not exist, or assume that the boat is there but with strenght less than the maximum. \footnote{Come si è detto in \ref{existence} è un tentativo di risoluzione epistemica del paradosso. Cfr. Quando si parla di ``maggiore è il numero di osservatori più intenso è il grado di esistenza delle monete''}

If you use this language you cannot simply say that the boat does not exist because it's not observable, a position that does not explain the movement of the water \footnote{Even the metaphysician who does not assume the existence of causation must explain the relationship (whatever it is, if there is one) or in any case the correlation between the phenomenon `` the boat moves '' (which presupposes the existence of the boat) and the phenomenon `` water moves '', as it is not connected to other phenomena, for example, the existence of clouds}; in short, we cannot forget the detail for which, although it is legitimate to think that the world disappears when I close my eyes (so I can not assume that it has strenght 1), it nevertheless reappears when I reopen them (for which it cannot globally have strength 0).

	If the arrow "disappears" and no other observer experiences it at any given moment, evidently it exists with strenght less than 1. Being \emph{inside} Tl\"on this is a concession to the epistemic vision of its inhabitants, who they cannot know what happens to the coins of the other observers; and in this sense, being unable to calculate the global strenght, the model is unusable by the tl\"onians. We can properly formalize their ontology, to be able to say something more than those who live in it.

	However, we can answer the idealist if he speaks the language of categories. \footnote{Here we take an idealistic interlocutor as an example of "extreme" metaphysics, and because it was in the target of Borges' novel to provide a critical and alternative reading, but this is not the claim of the paper}



\subsection{By their fruits you will know them}\label{frutti}
% Possiamo dire, figurativamente, che l'ontologia è il modo che usiamo per ``apparecchiare'' il mondo dentro un determinato linguaggio. Questo è ciò che gli ontologi possono dire, il luogo nel quale lavorare.
% Dove sia il linguaggio e ``qual è il principio che ispira la scelta di un determinato linguaggio (e quindi di un'ontologia)' ci sembra una domanda di interesse antropologico, o forse neurofisiologico, ma è fuori dalla portata di ciò che l'ontologia operativa può dire \footnote{\de{Dovrebbe essere una domanda metaontologica ma qui la posizione che assumo è che la nostra metaontologia dice che la risposta non rientra nel campo delle sue possibilità}\endde} (se non ricascando nelle pastoie dell'impostazione classica, priva degli ``ambienti' adatti dentro i quali parlare delle cose).

% Rapporto tra ``esistenza'' e ``comportamento nel modello'': rinunciamo alla caratterizzazione classica di esistenza? qualunque essa sia (presumibilmente quella orribile quineana). Sembra di sì. Quindi
% \begin{center}
% 	``$x$ esiste (nel modello $\mathcal{M}$)'' := ``insieme dei comportamenti di $x$ in $\mathcal{M}$''
% \end{center}
% \de{va precisata meglio ma è una def relativizzata ad un contesto (come promettevamo in 1.4) e ci permette di parlare di esistenza senza ricorrere a identità e persistenza nel tempo. Molto più agevole}\endde].

% Questo è quello che intendiamo con ``esistenza'' a livello metaontologico. La nozione poi si specifica a seconda del mondo (topos) che analizziamo in quel momento. Ad esempio in Tl\"on esistere significa avere una forza strettamente maggiore di 0.

The main point of our paper can be summarised very concisely: an ``ontology'' is a category $\clO$, inside which ``Being unravels''. Every existence theory shall be reported, and is relative, to a fixed ontology $\clO$, the ``world we live in''; such existence theory coincides with the internal language $\clL(\clO)$ of the ontology/category (from now on we employ the two terms as synonyms), in the syntax\hyp{}semantics adjunction of \cite{syntax-semantics_duality}.

So determined, the internal language of an ontology $\clO$ is the collection of ``things that can be said'' about the elements of the ontology.

If, now, ontology is the study of Being, and if we are structuralist in the metatheory (cf. \autoref{sec_intro}), we cannot know beings but through their attributes. Secretly, this is \emph{Yoneda lemma} (cf. \cite[1.3.3]{Bor1}), the statement that the totality of modes of understanding a ``thing'' $X$ coincides with the totality of modes your language allows you to probe $X$. Things do not exist out of an ontology; objects do not exist out of a category; types do not exist out of a type theory. In relational structures, objects are known via their modes of interaction with other objects, and these are modeled as morphisms $U \to X$; Yoneda lemma posits that we shall ``know objects by their morphisms'': the object $X:\clC$ coincides with the totality of all morphisms $U\to X$, organised in a coherent bundle (a functor $\clC(\firstblank,X) : \clC \to \Set$).

All in all, an ontology is a mode of understanding the attributes of Being: a category, be it in an Aristotelic or in a structuralist sense. As a consequence \emph{meta}ontology, i.e. the totality of such ways of understanding, must coincide with the general theory of such individual modes: with \emph{category theory}.

We should say no more on the matter: everything else pertains to \emph{meta}ontology.

Indeed, questions as ``where is language'' and what general principles inspire it might have an anthropological or even neurophysiological answer; not an ontological one. Or at least, not without paying a price: operative ontology as sketched here has limits. It can't speak of Being out of the one ignited by itself. (read as: category theory \emph{has limits}: it cannot speak efficiently of objects out of a fixed universe of discourse. Implicit: category theory also has merits.)
\subsection{Metaontology} \label{metaon}
The gist of our \autoref{bla} is that $X,Y,Z$ can't assess the existence of the coins classically; they just have access to partial information allowing neither a global statement of existence on the set $C$ of coins lost by $X$, nor an unbiased claim about the meaning thereof. Coins are untouched as a global lost conglomerate, yet their strength of existence is very likely to change \emph{locally}.

% Gli osservatori di Tl\"on non riescono a dire nulla sulle monete degli altri, neanche con strength. Il nostro modello è inutilizzabile \emph{dentro} Tl\"on; ed è dovuto alla diversità di linguaggio dell'abitante di Tlon. Sospendiamo il giudizio su cosa venga ``prima', se l'ontologia o il linguaggio (gli autori hanno posizioni divergenti ma la ricerca assume pragmaticamente il secondo) ma senza dubbio essi sono strettamente correlati.

This begs the important questions of to which extent language determines ontology. And to which extent it constrains its expressive power? To which extent the inhabitants of Tl\"on fail to see what is exactly ``a topos further'' (persistence of existence through time). To what extent \emph{we} fail to see that\dots

% Possiamo imprecisamente definire l'ontologia come ciò che gli oggetti \emph{sono} \footnote{cioè come si comportano nel modello} all'interno di un linguaggio.

% Prendiamo lo stato dell'arte dell'ontologia in filosofia analitica: l'apostolo quineano di turno ti dice che ``ontologia'' è il dominio di oggetti su cui variano i quantificatori \cite{?}. Questa def sufficientemente operativa dice ``di meno'' di quello che vorrebbe l'impostazione classica (così come la nostra; ma, come sempre, è l'unico modo per essere precisi); ma dice ``di meno'' anche della nostra. Se ontologia è il comportamento degli oggetti nel modello, è ``irrilevante'' cosa io ritenga di poter inserire nel dominio (struttura di $U$) (il realista concettuale dice che sono oggetti anche $\top$ e $\bot$ \footnote{O meglio sono oggetti ciò che $\top$ e $\bot$ denotano} mentre l'empirista puro no) (e così via di chiacchiere su ``ce lo metto o no il tempo nel mio modello?''): il dibattito può rimanere sul modello. In questa maniera abbiamo un agone intersoggettivo di attribuzione dei significati (vabbè, si fa per ridere: cfr. Umberto Eco, ovunque) dentro il quale \emph{fare} ontologia (anzi, ontologie).

Far from claiming we resolved such metaontological issues, here we make an hopefully clarifying statement: according to Quine and his school, ontology can be defined as the ``domain over which [logical, or natural language's] quantifiers run''. This is not wrong; in fact, it is perfectly compatible with our views. But we work at a raised complexity level, in the following sense. In our framework, ontology \emph{à la} Quine still is a category, because (cf. \autoref{quantifezzi}) a quantifier can be described as a certain specific kind of functor
\[\forall,\exists : PX \to PY\]
between powersets regarded as \emph{internal} categories of our ambient ontology $\clO$; in light of this, it's easy to imagine this pattern to continue: if Quine calls metaontology what we call ontology, i.e. the metacategory grounding his propositional calculus is a single ambient category \emph{among many}, we live ``one universe higher'' in the cumulative hierarchy of foundations and meta-foundations: our ontology possesses a higher dimensionality, and harbours Quinean theory as an internal structure (see \autoref{internista} for the definition of an internal category). So, it shall exist, \emph{somewhere} -at least in some secret way, hidden from the comprehension of men- a language that calls ontology what we call metaontology (and thus a language that declassifies Quine's to a sub-ontology -whatever this means).

Finding the bottom of this tower of turtles is the aim of the track of research, of very ancient tradition, within which we want to insert the present work.

Of course our work does not make a single comment on how, if, and when, this ambitious foundational goal can be achieved. We just find remarkable that framing Quine's definition in this bigger picture is not far from the so-called practice of ``negative thinking'' in category theory: negative thinking is the belief that a high-dimensional/complex entity can be understood by means of the analogy with its low-dimensional/complexity counterparts; see \cite{nlab:category-order,nlab:neg-think} for a minimal introduction to the principle, \cite{baez2010lectures} for a practical introduction, and see \cite{gowers2007} for what Gowers calls ``backwards generalisation''.
% \begin{italian}
% 	Riassumendo: per ''ontologia'' in generale intendiamo una theory of existence, namely una teoria dei modi in cui gli oggetti si relazionano gli uni agli altri per essere compresi da noi (Yoneda Lemma risultato centrale).
% 	Più propriamente è allora il caso di parlare di (operative) ontologies, al plurale, che sono le categorie col loro linguaggio interno. Vale a dire (1.3) gli ''ambienti'' dentro i quali studiare le relazioni tra gli oggetti. Come sappiamo da 1.2 questa concezione relazionale dell'ontologia è ispirata dall'essere strutturalisti, ma solo nella metateoria. %per cui la nostra ontologia, come si è detto sopra, è già una metaontologia%

% 	Qualunque discorso su come gerarchizzare le varie ontologie, sulle relazioni tra esse, è quindi un discorso sulle categorie, ed è la nostra (informale) metaontologia qui accennata.

% 	Le grandi domande esistenziali non pertengono, perciò, a nessuno di questi domini. The reasons why we implemented this extreme reduction to this historical matters are fully explained in 1.4 and are essentially practical.

% 	Ogni altro discorso crediamo debba partire da qui.
% \end{italian}
\subsection{Conclusion}
The circle apparently closes on, and motivates better, our initial foundational choice: categories and their theory correspond 1:1 to ontology and its theory. However, countless important issues remain open: in what sense this is satisfying? In what sense the scope of our analysis is not limited by this choice? What's his foundation? Is this a faithful way to describe such an elusive concept as ``Being''?

None of these questions is naive; in fact, each legitimately pertains to metaontology, and has no definitive answer. More or less our stance is as follows: approaching problems in ontology with a reasonable amount of mathematical knowledge is fruitful. Yet, the problem of what is a foundation for that mathematics remains (fortunately!) wide open; it pertains to metaontology, whose ambitious effort is to clarify ``what there is''. We believe the philosophers' job to work in synergy with quantitative knowledge, approaching the issue with complementary tools.