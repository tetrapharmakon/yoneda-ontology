\documentclass[a4paper, 11pt]{article}
\usepackage[T1]{fontenc}
\usepackage[utf8]{inputenc}
\usepackage[italian]{babel}

\usepackage{xcolor,amsmath,amsfonts}

\newenvironment{fo}[1]{\color{red} \textbf{Fouche said}: #1}{\color{black}}
\newenvironment{de}[1]{\color{blue} \textbf{Denta said}: #1}{\color{black}}

\begin{document}

\title{Appunti per ``Yoneda Lemma as a Structural \textit{Eadem Sunt}''}
\author{D. Denta}
\maketitle

\tableofcontents
\section{CT come metalinguaggio} 

\textit{approfondire e dire meglio} 


 [Il linguaggio categoriale ci permette di rilevare gli pseudo-problemi delle formulazioni tradizionali dell'ontologia contemporanea più avanzata, di dissolverli e/o risolverli, concentrandosi su ciò che rimane, lo ``scheletro'' del problema, espresso con poche nozioni basilari della \textit{Homotopy Type Theory}].

\fo{SACRO DIO CANE, gli a capo con il doppio backslash no però, eh.}\endfo
Se ci è concesso un gioco di parole, nostra ambizione è suggerire non, come paventa la tradizione analitica, un \textit{uso corretto del linguaggio}, quanto piuttosto un \textit{uso del linguaggio corretto}, il quale è, ovviamente, quello della \textbf{CT}.
[Una delle caratteristiche messe in evidenza dal linguaggio di \textbf{CT} è l'inutilità e impossibilità di parlare sensatamente di ontologia in senso ``assoluto''; si può farlo solo relativamente a una data teoria $T$, come già alcuni pensatori (ad esempio, Peter Geach) avevano suggerito.
\fo{Ok. Io porrei così la questione (ed è il motivo per cui avevo chiesto di stare attenti a linguaggio VS metalinguaggio): sin dal lavoro di Lawvere riguardante la semantica funtoriale, le categorie sono le teorie che vogliamo interpretare, ma anche i luoghi in cui vogliamo interpretare quelle teorie: una teoria è infatti una categoria --piccola-- che realizza un dato linguaggio logico $L$, e allo stesso tempo una categoria --grande-- è ``l'Universo'' all'interno del quale vogliamo fare Matematica; vi sono diversi punti su cui insistere, alcuni non molto chiari persino ai matematici; per esempio il fatto che non esiste una matematica sola; ne esiste una --con la sua algebra, la sua analisi e la sua teoria dei gruppi-- per ogni categoria entro la quale vogliamo

	In sintesi: in teoria delle categorie sono oggetti puramente sintattici: è il contesto a incarnare questa sintassi in una semantica, a dare significato ai simboli. Il linguaggio è più povero, perché ora non si parla più di un numero 6, ma di "qualsiasi cosa il numero 6 significhi, in relazione al contesto dove esso è inserito"; la matematica è quella cosa che si fa nelle categorie, e per ogni teoria matematica serve una classe di categorie diversa (abeliane per fare geometria algebrica; topos o CCC per fare logica; categorie modello per fare topologia algebrica; categorie algebriche per fare algebra universale\dots

	La teoria delle categorie dice una cosa che i matematici tendono a non capire bene, e che genera una certa confusione: tutta la matematica che un professore associato a Berkeley che si occupa di, diciamo, combinatoria, o sistemi dinamici ha visto nella sua carriera, tutta quella che ha prodotto, che i suoi colleghi gli hanno insegnato, che lui ha insegnato ai suoi studenti, su cui ha formulato congetture entusiasmanti... tutta quella matematica succede dentro \emph{una sola} categoria. Io invece ne conosco almeno due, e conosco dei modi per prendere la matematica nel suo complesso, ossia l'intera teoria dei gruppi, l'intera teoria degli anelli, l'intera teoria di Galois, l'intera topologia algebrica, e portarla da una categoria all'altra.

	In questo senso la teoria delle categorie "è più grande della matematica nel suo complesso" e ne fa da fondazione; in questo senso è un metalinguaggio dove è possibile rifare la matematica nella sua interezza. L'idea, che è anche ciò che i matematici tendono a non capire bene, e che senza la teoria delle categorie non si riesce nemmeno a enunciare, è che \emph{le teorie matematiche sono a loro volta oggetti matematici}, e in quanto tali sono passibili dello stesso studio di cui sono passibili gli oggetti di cui quelle teorie parlano. Ovviamente questo è impreciso, perché scopre il fianco a paradossi terribili: altrettanto ovviamente, il motivo per cui sto essendo impreciso è che l'idea è molto piu importante della sua formalizzazione. Una formalizzazione in questo senso, però, è possibile.

	Prevedibilmente, gli oggetti matematici in questione sono proprio le categorie, che di CT sono l'oggetto di studio: "la teoria dei gruppi" è quell'oggetto matematico che chiamo la categoria dei gruppi. Ma posso andare più in là e definire una "teoria" come una particolare categoria T, e considerare i suoi "modelli", ossia i modi di realizzare la teoria T "dentro" un'altra categoria, per esempio quella degli insiemi, o quella dei gruppi abeliani, etc. Allora, se chiamo questi modelli "funtori", e scrivo $T \to Set$ per uno di tali funtori, la totalità di questi modelli forma una categoria che "dice le stesse cose che la matematica classica dice sulla teoria che corrisponde a $T$". Incidentalmente "l'unificazione della matematica attraverso la teoria dei topos" è una generalizzazione di questo problema: se scrivo $(T,Set)$ per la categoria dei modelli di $T$, cosa implica aver trovato che $(T,Set)\cong (S,Set)$ (molto: $T,S$ non possono essere troppo lontane come teorie, se hanno gli stessi modelli, no?)

	Niente più e niente meno che un teorema del tipo: se due gruppi agiscono nello stesso modo sugli stessi insiemi, quanto distanti sono quei gruppi? Solo che qui non si parla di gruppi, ma di teorie matematiche nella loro interezza; e non di fumosi concetti da filosofo, no, sono definizioni precise, che parlano con rigore di cose che senza questo linguaggio non è nemmeno pensabile dire.

	Allora tanto vale studiare tutte quelle cose con cui si fa matematica, nella loro interezza. E la parte di matematica che si occupa di studiare le categorie, cioè le varie parti della matematica come singoli oggetti matematici nella loro mutua interazione come si chiamerà\dots?}\endfo
Nell'ottica relazionale ispirata dall'uso dei concetti categoriali, è naturale ammettere la ``relatività ontologica''. Questione sulla quale in filosofia si dibatte da decenni e che, in \textbf{CT}, è naturalmente implicata. Alcuni esempi classici dell'ontologia (vedremo qui il paradosso di Borges, e in un lavoro successivo, le sfere distinte di Black) sono più agevolmente decodificabili, e risolvibili, analizzati in riferimento al contesto, cioè $dom(T)$ di una data $T$].
\fo{Sì. Vale la pena notare che il nostro claim sarà ``sia nel paradosso delle nove monete di rame, sia nell'esperimento mentale di Black, esistono una categoria $\mathcal C_{\text{Bor}}$ e una categoria $\mathcal C_\text{Bla}$ con oportune proprietà, per cui il ragionamento paradossale (\emph{le nove monete non sono esistite nella notte tra mercoledì e giovedì}, e \emph{le due sfere non sono due perché non vi è modo di distinguerle}) svaniscono.''
La natura figmentale dell'esempio delle due sfere si riassume così: un interlocutore pensa le sfere in uno spazio euclideo; l'altro interlocutore in uno spazio affine. A generare il paradosso è semplicemente l'incapacità dei due (e di Black) di notare che la conversazione avviene \emph{riguardo agli stessi oggetti} ma in categorie diverse: per $A$, nella categoria dove c'è una mappa tra le due sfere quando hanno lo stesso centro e lo stesso raggio --sono quindi ``euclideanamente uguali''; per $B$, nella categoria dove c'è un isomorfismo tra le due sfere quando tra loro c'è una trasformazione \emph{affine}. Allora $A$ vede diverse cose che $B$ vede uguali, perché $A$ opera un quoziente rispetto a una relazione di equivalenza più fine di quella che opera $B$. Questo esempio da solo permette di introdurre praticamente tutti gli strumenti di cui renderemo conto in ``no identity w/out homotopy'':
\begin{itemize}
	\item Questo esempio rende lampante il fatto che una categoria è determinata dai suoi morfismi molto più che dai suoi oggetti; $A$ e $B$ sono convinti di parlare delle stesse cose, perché $|\mathcal C_A|=|\mathcal C_B|$, ma la seconda ha molti più morfismi, e quindi può far collassare molti più oggetti rispetto alla sua nozione di ``uguaglianza categoriale'' (dobbiamo stabilire una volta per tutte dei nomi per le cose cui vogliamo riferirci).
	\item mette in chiaro in un esempio specifico il fatto che la relazione di identità non è primitiva; è contestuale; è indotta dalla nozione di uguaglianza che ogni categoria si porta dietro. Ed è binaria, perché è un caso particolare di una nozione di omotopia (meglio: è la nozione di isomorfismo in una categoria dell'omotopia, che resta indotta dalla nozione di omotopia di cui $\mathcal C$ era dotata. No identity without homotopy, appunto).
	\item Questo è né più né meno che il contenuto strutturale del programma di Klein: le figure dello spazio restano sempre le stesse, raccolte nella classe $\mathbb S$; ciò che cambia è il gruppo(ide) $\mathcal G$ che facciamo agire sulle figure e di cui prendiamo la categoria delle azioni $\mathbb S /\!\!/\mathcal G$; ``il mondo'' con la sua nozione di identità è l'insieme di Bishop
	      \[\Big\{ \pi_0(\mathbb S/\!\!/\mathcal G) \mid [x] \equiv [y] \iff \exists \varphi : x \to y\Big\}\]
	\item Permette di prevedere cosa penserebbe del problema un ipotetico interlocutore $C$ che vivesse in un mondo proiettivo (una sfera e un'iperbole sono lo stesso oggetto) o in uno topologico (una sfera e un cubo sono lo stesso oggetto), etc. Ovviamente si può ragionare anche viceversa: basta trovare un contesto dove due oggetti spazialmente coincidenti non sono ``lo stesso'' oggetto. La sfida è che bisogna uscire da fondazione insiemistica (l'assioma di estensionalità è precisamente l'asserto per cui due cose che hanno gli stessi punti sono una cosa sola); invece di impelagarsi nella costruzione precisa di un controesempio (ce ne sono, in fondazioni type-teoretiche), penso sia meglio dire semplicemente: ``ecco, vedete? Il principio di identità è  a tutti gli effetti l'assioma di estensionalità: voi trovate assurdo che a dichiarare uguali due enti non sia sufficiente dire che hanno gli stessi atomi. Eppure questo è possibile: voi classici non avete la nozione di identità più forte in assoluto, state nel mezzo; e la metateoria risultante dal prendere una nozione di identità più stringente è semmai ancor piu interessante di quella cui siete abituati. Quindi la nozione di uguaglianza non è \emph{immutabile, scolpita nel tempo}\footnote{Non sei l'unico che apprezza il cinema impegnato}. E' un assioma: se lo vuoi lo prendi, altrimenti non lo prendi. E se non lo prendi si apre un mondo, perché ammetti che due cose sono uguali anche quando non hanno gli stessi punti, oppure che \emph{non basta} avere gli stessi punti per essere uguali, nello stesso senso in cui non è sufficiente che due insiemi siano in biiezione affinché siano omeomorfi, o isomorfi come gruppi, etc.'' Chiaramente, questo lastrica la strada al lemma di Yoneda, che ``è l'assioma di estensionalità in CT'' (lo introdurrei esattamente con queste parole, e spenderei una parte congrua del lavoro a darne una introduzione \emph{ad usum delphini} che invece che di trasformazioni naturali e gruppianellicampi parli di ontologia e di estensionalità).
\end{itemize}
Qual è il punto di tutto questo? Che una sfera è molte cose: è uno spazio topologico, è una varietà algebrica, è una varietà differenziale, è un gruppo di Lie (non in dimensione 2, ma per esempio in dimensione 1 o 3), è una superficie di Riemann, è questo ed è quell'altro. $A$ e $B$ nell'esempio di Black parlano uno di una sfera $S^2_A$ che appartiene a una categoria $A$, e l'altro di una sfera $S^2_B$ che appartiene a una categoria $B$. Il paradosso nasce quindi da un uso scorretto e impreciso del formalismo.

Per quanto riguarda Borges invece, in un topos di insiemi variabili ${\sf Set}/I$, indicizzati da un opportuna base $I$ sconnessa (ad esempio, ma è solo un esempio, $[0,1/3)\cup [1/3,1]$), le nove monete di rame non hanno ragione di ``perseverare nell'esistenza'' nella notte tra mercoledì e giovedì, nel senso che la funzione che ne mappa l'esistenza può assumere valore ${\sf true}$ in certi momenti e $\sf 0$ in altri. Quello indicizzato da $[0,1/3)\cup (2/3,1]$ è un mondo che ``smette di esistere'' per un po', nella notte tra mercoledì e giovedì, e ``torna ad esserci'' subito dopo, senza che i suoi abitanti siano stati spostati o alterati in alcun modo. Questo perché una funzione che assume valori diversi (mettiamo costantemente 1 e costantemente 0) su diverse componenti connesse ha una discontinuità in $1/3$ \emph{per noi} ma non per il linguaggio interno della categoria ${\sf Set}/I$.

Ritengo comunque doveroso precisare che la vera ragione per cui il paradosso delle nove monete non sta in piedi è molto più primitiva: il rame non può arrugginire.}\endfo
\section{Introduzione di entità astratte (matematiche) in ontologia}
Sviluppare questo appunto di Carnap sull'introduzione di qualunque tipo di entità astratte in semantica (ma vale anche in ontologia). Per un filosofo non è una ovvietà, e l'articolo di Carnap è un esempio di tentativo di spiegare come si procede nello sviluppo di una disciplina:
\begin{quotation}
	we take the position that
	the introduction of the new ways of speaking does not need any theoretical justification
	because it does not imply any assertion of reality [...].  it is a
	practical, not a theoretical question; it is the question of whether or not to accept the new
	linguistic forms. The acceptance cannot be judged as being either true or false because it is
	not an assertion. It can only be judged as being more or less expedient, fruitful, conducive to
	the aim for which the language is intended. Judgments of this kind supply the motivation for
	the decision of accepting or rejecting the kind of entities. [Carnap, 1956]
\end{quotation}
\section{The structuralist way}
Summary: \textit{nozione (in un certo senso ``a priori'') di struttura, suo development in mathematical history, rapporto con \textbf{CT}, sottigliezze del bourbakismo}
\begin{quotation}
	\begin{flushright}
		\textit{Le due posizioni} [strutturalismo e deduttivismo] \textit{dicono poco,ma non dicono sciocchezze, non corrono il rischio di essere falsificate.} [Lolli, 2002]
	\end{flushright}
\end{quotation}
\begin{quotation}
	\begin{flushright}
		\textit{CT emphasized structure instead substance}. [Kr\"omer, 2007]
	\end{flushright}
\end{quotation}
Nel corso del '900 la direzione dell'evoluzione della matematica ha portato la disciplina inizialmente a frazionarsi in differenti sotto-discipline, con loro oggetti e linguaggi specifici, e poi a trovare un'inattesa unificazione sotto la nozione portante di \textbf{struttura}, e lo strumento formale che meglio ne caratterizzato il concetto, le \textbf{categorie}.
Questo processo ha portato spontaneamente ad una revisione epistemologica della matematica e ha ispirato, nell'evolversi degli strumenti operativi, una revisione sia dei suoi fondamenti che della sua ontologia. Per molti studiosi è innegabile che
\begin{quotation}
	[the] mathematical uses of the tool CT and epistemological
	considerations having CT as their object cannot be separated, neither historically
	nor philosophically. [Kr\"omer, 2007]
\end{quotation}
Ciò è però avvenuto prescindendo sia dallo specifico dibattito fondazionale (attivissimo all'epoca in cui apparse il gruppo Bourbaki, \fo{Ci sono molte cose da aggiungere qui; penso sia impossibile rendere conto dell'intero dibattito sull'eventualità di una fondazione categoriale della matematica in maniera concisa; certamente le 1-categorie non sono sufficienti. Le $\infty$-categorie, con HoTT, sono già molto meglio (HoTT e la teoria degli $\infty$-topos sono semantiche per MLDTT con univalenza (V), e il credo contemporaneo è che MLDTT+V \emph{sia} una fondazione della matematica; io non ne so abbastanza per esporre un parere --sono fermamente convinto che MLDTT+V sia estremamente espressiva. O ignoriamo la questione, o cerchiamo un HoTTista, o diciamo qualcosa di poco impegnato. Io voto per rimandare l'intersezione ad HoTT a un altro lavoro --di questo trittico o meno)}\endfo che ha avviato il processo strutturale) sia da più generali considerazioni filosofiche, anche di mera filosofia della matematica.

La nozione di \textit{struttura} è, in un certo senso, un concetto ``a priori''; vale a dire uno strumento utile nella pratica matematica e per caratterizzare un particolare ontological viewpoint. Non solo non è definito formalmente ma caso per caso i matematici stabiliscono quando esso possiede o meno significato (nel contesto di una $T$ data).
\fo{Cosa intendi con questo? Una struttura \emph{si può} definire formalmente --in effetti, in molti modi.}\endfo
La pratica matematica, nella via strutturalista, produce una ontologia ``naturale'', che alcuni, in seguito, sentirono il dovere di caratterizzare con più precisione. D'altronde, analogamente a quanto suggeriva Carnap rispetto alla semantica,
\begin{quotation}
	mathematicians creating
	their discipline were apparently not seeking to justify the constitution of the
	objects studied by making assumptions as to their ontology. [Kr\"omer, 2007]
\end{quotation}
Ma, al netto dei tentativi (anche dello stesso gruppo Bourbaki), \fo{Per incis, il rapporto Bourbaki-categorie è sempre stato un po' conflittuale  sebbene tra le fila di B. ci fossero tanti ``categoristi''. Invece del dibattito storico io spingerei su un punto tecnico: da una iniziale diatriba, e anche una certa confusione iniziale, sono emersi con naturalezza dei concetti che ad oggi sono la pietra angolare della matematica strutturale e della logica categoriale; questi strumenti non solo sono d'aiuto al dibattito sullo stato ontologico delle entità matematiche, e sull'impegno che volta per volta si prende (``ci sono'' le cose?) chi ne parla; non solo questo, ma risolvono vari problemi.}\endfo ciò che conta è che l'abitudine a ragionare in termini di strutture abbia prodotto implicite posizioni epistemologiche e ontologiche.
Questione che meriterebbe una lunga riflessione autonoma. Per i nostri scopi basti enunciare una distinzione che Kr\"omer riprende in parte da [Corry, 1996]: quella tra \textbf{structuralism} e \textbf{structural mathematics}:
\begin{itemize}
	\item[\textbf{(1)}] Structuralism: \textit{the philosophical
		      position regarding structures as the subject matter of mathematics}
	\item[\textbf{(2)}] Structural Mathematics: \textit{the methodological approach to look in a given problem
		      “for the structure”}
\end{itemize}
\textbf{(1)} implica \textbf{(2)} ma non è necessaria l'implicazione inversa. Tuttavia è essa spesso venuta spontanea nella riflessione teorica durante la storia della matematica recente (l'ontologia ``naturale'' appunto). L'uso della \textbf{CT} come metalinguaggio, nonostante la compromissione storica con lo strutturalismo, non rende tuttavia automatico il passaggio da \textbf{(2)} a \textbf{(1)}. ma suggerisce che l'ontologia non solo dipende dalla ``ideologia'' (in senso quineano) di $T$, cioè dalla potenza espressiva della teoria, ma è influenzata dal modello epistemologico che l'uso dello stesso linguaggio formale ispira.

L'utilità della distinzione di Kr\"omer è però un'altra: invece di incespicarsi in una definizione possibilmente non ambigua di \textit{struttura} (con le conseguenze indesiderate che potrebbe avere nella pratica operativa) si può ridurre (o ridefinire) la filosofia \textbf{(1)} alla metodologia \textbf{(2)}, dicendo che:
\begin{quotation}
	\textbf{structuralism is the claim that mathematics
		is essentially structural mathematics} [Kr\"omer, 2007]
\end{quotation}
(la pratica operativa che ``entra'' nella definizione di strutturalismo evita il decennale dibattito delle humanities sui medesimi concetti).

Ciò è equivalente a dire: la pratica strutturale è essa stessa la sua filosofia.

Gli storici tentativi di spiegazione del termine ``struttura'' attuati da Bourbaki negli anni a seguire dalla pubblicazione degli \textit{Elements}, sono la prima sistematica elaborazione di una filosofia che si accordasse con la fecondità operativa della \textit{structural mathematics}. Il suo obiettivo è quello di ``\textit{assembling of all possible ways in which given set can be endowed with certain structure}'' [Kr\``omer, 2007], e per farlo elabora, nel programmatico \textit{The Architecture of Mathematics} (redatto dal solo Dieudonné), pubblicato nel 1950, una strategia formale. Pur specificando che ``\textit{this definition is not sufficiently general for the needs of mathematics}'' [Bourbaki, 1950], codifica una serie di operational steps tramite i quali una struttura su un insieme è ``assembled set-theoretically''. Adotta, insomma, una prospettiva ``riduzionista'' nella quale
\begin{quotation}
	the structureless sets are
	the raw material of structure building which in Bourbaki’s analysis is “unearthed”
	in a quasi-archaeological, reverse manner; they are the most general objects which
	can, in a rewriting from scratch of mathematics, successively be endowed with
	ever more special and richer structures. [K\"omer, 2007]
\end{quotation}
A conti fatti, dunque, nello strutturalismo bourbakista la nozione di \textit{set} non sparisce definitivamente davanti a quella di struttura.Il lavoro filosofico da fare per ottenere uno struttualismo integrale era ancora lunga. Va detto poi che
\begin{quotation}
	like any other scientist's
	system of images of knowledge, Bourbaki's own system has been subject
	to criticism, it has evolved through the years, and, occasionally, it has
	included ideas that are in opposition to the actual work whose setting the
	images provide. Since Bourbaki gathered together various leading mathematicians,
	it has also been the case that members of the group professed changing
	beliefs, often conflicting with one another at the level of the images of knowledge. [Corry, 1996]
\end{quotation}
Corry - in riferimento a Bourbaki - preferisce parlare di \textit{images of mathematics} piuttosto che di \textit{philosophy}. Il nostro percorso sembrerebbe quindi procedere da una ``immagine'' ad una peculiare metodologia, che a sua volta informa una ``filosofia spontanea'' e giunge ad avere gli strumenti per intervenire su una qualunque ontologia; grazie alle categorie gli enti di cui ci si occupa sono divenuti così astratti da poter essere applicati al di fuori del campo matematico. [qui quello che sembra voglia dire non è proprio quello che voglio dire. Da eliminare quasi certamente].

Quello che Dieudonné auspicava come ``\textit{the unifying role of the mathematical structure}'' è svolto dalla \textbf{CT} che ha mostrato di catturare perfettamente la nozione di struttura, [in nota: ``\textbf{CT} can bee seen a theoretical treatment of what mathematicians used to call structure'' [Kr\"omer, 2007]] senza le impurità bourbakiste o i difetti di altre nozioni meno aderenti alla pratica effettiva, pur avendo uno storico \textit{fil rouge} con quella particolare ``immagine'' della matematica alla base dello strutturalismo originario.
\fo{Quello che è contenuto tra la precedente nota e questa mi sfugge: probabilmente dovremmo parlarne a voce e mi devi far capire di cosa stai parlando \emph{davvero} qui...}\endfo
\section{Per una ontologia operativa}
cenni a proposte già esistenti di ontologia relazionale (ciò che c'è di più vicino ad una prospettiva di tipo strutturalista) [Whitehead, Brentano, Simons (\textit{Parts. A Study in Ontology}), Kuhlmann, \textit{Structural Realism} di Landry e Rickles] \textit{questa è roba da trattare meglio nel paper sulla mereologia}; usare la storia novecentesca della matematica come esempio di ciò che dovrebbe fare l'ontologia col ``nuovo'' linguaggio di cui la forniamo. [esempio delineato nel paragrafo precedente]


Il nostro bersaglio principale è la nozione di \textit{identità}, la quale \marginpar{$(ID_1)$} una volta sostituita con opportune procedure della \textbf{CT}, a seconda dei contesti (di qui la relatività ontologica), in particolare se sostituita con la nozione di \textit{omotopia tra mappe continue} - inficerebbe anche la nozione, indefinita e vaga, di \textit{esistenza}. Difatti nella prospettiva classica, quineana, l'esistenza - tradizionalmente cardine della ricerca ontologica - si definisce tramite la nozione, più (illusoriamente) semplice e primitiva, di identità: ``\textit{A esiste}'' sse ``\textit{qualcosa è identico ad A}'' (questo ``qualcosa'' è una variabile vincolata ad un quantificatore esistenziale. Ricordiamo che, nella prospettiva classica, l'impegno ontologico varia sul dominio della teoria, il quantificatore è un semplice operatore).

Il motivo per cui riteniamo di dover agire in questa direzione è dovuto agli innumerevoli problemi che la nozione di identità classica (criterio di Leibniz, sue varianti, ma anche definizioni successive in sua vece) si porta dietro, rilevati da molti filosofi nel corso del secolo passato, e non superabili rifiutando solo l'identità leibniziana o abbracciando la prospettiva mereologica (ma ci riserviamo di parlarne in lavori successivi). Il motivo per cui molti filosofi, pur sottolineando l'inadeguatezza della nozione, non hanno mai seriamente proposto di sostituirla, crediamo sia per mancanza sia di un linguaggio adatto sia di alternative teoriche rigorose in esso espimibili.
L'\textit{Homotopy Type Theory}, e più in generale la \textbf{CT}, rispondono a queste esigenze, e attuano quella sostituzione finora mai realizzata.
Come si è accennato nel precedente paragrafo l'uso di questi strumenti concettuali ha aiutato la pratica matematica e ha involontariamente ispirato una visione epistemologica, e poi ontologica, della disciplina, vale a dire dei suoi oggetti di studio. A coloro che obiettano che bisogna prima sapere \textit{cos'è} una struttura prima di lavorare con essa noi rispondiamo, con Kr\"omer, che
\begin{quotation}
	this reproach is empty and one tries to explain the clearer by the more obscure when giving priority to ontology in such situations [...]. Structure occurs in the dealing with something and does
	not exist independently of this dealing. [Kr\"omer, 2007]
\end{quotation}
Aumentiamo le capacità teoretiche della disciplina e ne restringiamo le ambizioni, facendo dipendere qualunque riflessione su domini di oggetti dalla teoria che si propone di descriverli (e, quindi, dal suo specifico linguaggio, traducibile efficacemente in termini categoriali).

La categoria $C$ produce la sua \textbf{duale} $C^{op}$ che ha stessi oggetti e stesse frecce di $C$ ma le frecce sono nel verso opposto, cioè con interscambio tra dominio e codominio. Si ha anche che $(C^{op})^{op} = C$, cioè la \textbf{biduale} di $C$ è sè stessa.

Alcune conseguenze per la cosiddetta ``verità categoriale'': se $\alpha$ è una proposizione del linguaggio delle categorie, chiamiamo \textbf{duale} di $\alpha$ la proposizione $\alpha^{op}$ che si ottiene sostituendo in essa i vari concetti con i corrispettivi duali (dominio con codominio e viceversa, etc). Quindi:
\begin{quote}
	Se $\alpha$ è \textbf{vera} in $C$ allora $\alpha^{op}$ è vera in $C^{op}$
\end{quote}
In particolare, se $\alpha$ è una \textbf{verità categoriale}, vale a dire una quote vera in \textit{ogni} categoria allora $\alpha$ è vera in ogni categoria che sia duale a qualche categoria. In ogni categoria duale di ogni categoria in cui valga $\alpha$ deve valere anche $\alpha^{op}$, quindi in ogni categoria biduale di una qualche categoria vale $\alpha^{op}$. Ma, come sappiamo,ogni categoria è la propria biduale. Perciò:
\begin{quote}
	Se $\alpha$ è una \textbf{verità categoriale} lo è anche $\alpha^{op}$
\end{quote}
La ``creazione'' di una categoria duale non è necessariamente qualcosa di matematicamente significativo; non è cioè detto che l'oggetto prodotto sia utilizzabile o ben definibile: [riportare esempi dal Casari]
Il linguaggio categoriale, quindi, permette di vedere la differenza tra la semplice manipolazione di concetti operativamente utili e la produzione di enti (categorie) effettivamente cogenti. Un vero e proprio esercizio di concettualizzazione, che unisce rigore a immaginazione creativa. Ad esempio nella categoria $\textbf{Pfn}$ non c'è una moltiplicazione di enti ma una moltiplicazione dei concetti utili a formare un ente (nello specifico una tripla di oggetti) che si distingua sensatamente da enti simili (cosa che la \textit{Set Theory} non permette di fare, nel caso in esame e in molti altri).

Sappiamo che nella storia della \textbf{CT} l'accettazione dell'evidenza dei risultati è avvenuta a prescindere da un dibattito intorno alla scarsa precisione e coerenza logica degli stessi. Le categorie sono nate come strumento concettuale e, senza preoccuparsi delle sottigliezze della ricerca fondazionale, [mettere in nota: La pratica matematica ha fatto il lavoro che il solo dibattito fondazionale non poteva fare, e talvolta ostacolava (Poincarè ironizzava sulle dimostrazioni dei Prinicpia di 1+1=2 non a caso). Ha fatto filosofia calcolando!]
hanno catturato efficacemente tutte le nozioni della matematica moderna, rivelandosi utili e feconde. Nostro claim è che si riveleranno tali anche con le usuali nozioni metafisiche. Si tratta di adottare, in fondo, una prospettiva pragmatica:
\begin{quotation}
	that structural mathematics is characterized as an activity by a treatment of things as if one were dealing with structures. From the pragmatist viewpoint, we do not know much more about structures than how to deal with them, after all.
\end{quotation}
La ``traduzione'' dei problemi dell'ontologia nel linguaggio di \textbf{CT} permette di manipolare meglio nozioni (non solo, come si sa, matematiche) ma metamatematiche e metafisiche, e ci dota di un approccio più compatto e di una visione più ``leggera'' e occamista delle questioni vertenti su oggetti e esistenza. Non giustifichiamo questo approccio a priori ma ne assumiamo la fecondità già provata in letteratura, soprattutto paragonata a quella degli approcci set-theoretic (di cui già è informata la totalità delle ontologie formali).\\
Memori delle storiche osservazioni di Carnap, non riteniamo che questo approccio ``operativo'' all'ontologia (che non è puro \textit{problem solving} ma anche chiarificazione concettuale) implichi necessariamente l'adesione incondizionata ad uno strutturalismo filosofico integrale - o a sue varianti specifiche come la teoria \textbf{ROS} -, esattamente come non avviene nel passaggio dalla \textit{structural mathematics} allo strutturalismo vero e proprio (o al bourbakismo). La sua importanza è principalmente metodologica. (\textit{Au contraire} risulta necessario per chi appoggia posizioni strutturaliste al di fuori della matematica cominciare a fare ontologia in termini categoriali, nelle modalità qui indicate). Tuttavia il suo valore euristico è tale che invitiamo a porsi comunque in questa prospettiva; argomentazioni indipendenti, poi, come dimostra la vicenda Bourbaki, potranno essere valutate e giustificate separatamente senza inficiare i risultati raggiunti.

Ed infine fughiamo il dubbio che l'intento sia una completa matematizzazione della filosofia: non c'è nessuna aspirazione programmatica, solo il tentativo di presentare questioni note in linguaggio differente, e risolverle in tal modo.
\section{Categorizzare Borges}
Applicheremo la nostra ``ontologia categoriale'' (più semplicemente: l'uso del linguaggio di \textbf{CT} per trattare questioni ontologiche) all'elenco delle mini-metafisiche borgesiane di \textit{Tl\"on} esposte - a guisa di sintesi di tutte le simpatiche ``follie'' della storia della filosofia - nel racconto che apre \textit{Ficciones}. Tutto il racconto è una specie di laboratorio per coloro che vogliono maneggiare diverse teorie metafisiche, anche di poco variabili le une dalle altre, e permette a noi di mostrare come pochi concetti categoriali siano sufficienti a trattare una gran numero di teorie differenti (sempre in un'ottica relazionale).

Ci concentremo in particolare sul paradosso dei \textit{Nine Copper Coins}, diffusamente affrontato da Borges, unica stentata teoria ''realista`` esprimibile nella mentalità idealistica degli abitanti di Tl\"on.
[Sviluppare meglio questa parte: il racconto di Borges è un piccolo laboratorio di metafisica divertente, e vale la pena sfruttarlo un po'].


\fo{Da dove iniziamo allora; per esempio dal ricostruire il nucleo del programma di Erlangen. Sicché
	\begin{itemize}
		\item Definizione di gruppo(ide);
		\item Definizione di azione di un gruppo(ide) su un insieme/categoria;
		\item Definzione dell'action groupoid e delle sue componenti connesse: il $\pi_0$ dell'action groupoid di una certa azione è l'oggetto di studio della geometria (breve digressione su come questa geometria sia una geometria discreta, i.e. algebra lineare senza topologia, o su campi finiti, così come continua, i.e. geodiff su varietà, e azioni di gruppi di Lie che sono gruppi strutturali di fibrati);
		\item Esempi vari dalla geometria classica: qual è il gruppo, qual è l'insieme su cui agisce, nei vari tipi di geometria:
		      \[proiettiva \supset affine \supset lineare \supset euclidea\]
		\item Come questo, a partire dalla geometria, diventa una teoria ontologica: l'identità è sempre una identità nel $\pi_0$ di un action groupoid.
		\item Allora, l'esempio delle sfere di Black è quello di due bambini che litigano per un giocattolo che chiamano in modi diversi.
	\end{itemize}}\endfo

nozione di struttura caso particolare di quella di funtore; superamento di Bourbaki; visione di Lawvere (i 2 paper salvati); [provare a fare una sintesi di questo percorso]; algebra = studio di categorie che hanno proprietà utili per studiare operazioni finitarie definite su un insieme. 

\textbf{Cose da fare:} chiarire parte sul metalinguaggio; sintesi collegamento Bourbaki-Lawvere; ricostruire dibattito sull'identità.  


\newpage

Raccolgo qui un po' di cose sparse cui ho pensato io. Mi sembra del materiale che fa un buon incipit.

\section{Sulla natura programmatica di questa serie}
La matematica, all'interno dello scibile umano, dibatte attorno a tre indefiniti fondamentali frutto dell'esperienza sensibile: la forma, la misura e l'inferenza. Dall'appercezione che esistono entità estese nello spazio e durevoli nel tempo, alla misura di \emph{quanto} esse sono estese, alla costruzione di una rete di relazioni concettuali tra le entità estese, che renda conto del processo che genera conoscenza e la rende riproducibile. Contaminazioni tra le teorie costruite a questo modo sono ovviamente possibili e comuni; in effetti la matematica avviene proprio nelle aree di mutua competenza tra la logica, l'algebra e la geometria. 

Si può persino dire di più, attribuendo alla matematica una natura duplice: essa è sì un linguaggio, ma è anche il presupposto al linguaggio, ciò che lo rende possibile, il prerequisito affinché esso avvenga: è un ur-linguaggio che dà lo schema generativo per ``tutti'' i linguaggi. E' un linguaggio i cui elementi sono le regole per darsi un linguaggio che veicoli informazione e consenta la deduzione (questo fa, anche il più prosodico dei testi): è un \emph{meme}, in un senso speciale, perché il meme non è questa o quella jpeg, quanto piuttosto la classe di equivalenza di tutte le variazioni possibili su uno stesso scheletro sintattico; è un metaoggetto, è un modo di darsi degli oggetti.

In questa prospettiva è evidente che la matematica (non la sua storia, non la sua filosofia, la sua \emph{pratica}) sia utile ad approcciare le domande essenziali dell'ontologia: cosa sono ``le cose'', cosa rende le cose ciò che sono e non, piuttosto, diverse.

In tale prospettiva è auto-evidente \footnote{Vietato usare le parole "evidente" o "auto-evidente" D.D.} che l'abitudine al pensiero corretto (la pratica della matematica ha questa come definizione possibile) sia l'unico modo onesto di spiegare la cogenza degli enti; ma è deprimente e innegabile che un certo dibattito filosofico si sia reso impermeabile al linguaggio matematico. A volte ciò è fatto con una certa sufficienza nei confronti degli ingegni minuti, e altre volte semplicemente con l'ingenuità dei profani. Riparare a uno strappo avvenuto molto tempo fa, dovuto a diverse finalità e diverso lessico specifico, non è cosa cui possa ambire il lavoro di due sole persone. Se, però, il lettore di queste note chiede una motivazione estesa per il nostro lavoro, un progetto di ampio respiro in cui esso si inserisca, in breve un \emph{programma}, la troverà ora: esiste una matematica il cui scopo è risolvere alcuni problemi della filosofia, nello stesso senso in cui certa matematica ``risolve'' il problema del moto dei corpi celesti. Non annichilisce completamente la domanda: propone dei modelli entro i quali formularla; mette in evidenza ciò che è banale conseguenza degli assiomi di quel modello e, per contro, ciò che non lo è, ciò che per essere risolto chiede che il linguaggio sia espanso, modificato, affilato. Avvicinare da matematici questa disciplina mai nominata, che è poi solamente ciò che la filosofia dovrebbe essere fin dall'inizio: pensiero igienico e bene informato, è lo scopo di questi testi.

Non potendo risolvere il problema dell'ontologia (un tale fine sarebbe megalomane e mal posto), ci proponiamo qui perlomeno di iniziare a scardinare alcune credenze assodate di un certo cattivo ``filosofo quadratico medio''; puntare il dito su alcune problematiche che la prosa è incapace di notare, perché le manca il lessico specifico; suggerire che \emph{nel linguaggio giusto}, quando le parole significano la cosa giusta e sono strumenti di episteme invece che mere formule magiche, alcune questioni essenziali dell'ontologia recente si dissolvono in un filo di fumo, e altre diventano semplicemente ``la domanda sbagliata'': non quesiti sciocchi o falsi, bensì domande che non si dovevano fare, che non significano né ciò che i loro propalatori speravano, né altro.

Questo linguaggio giusto, igienico, non è la matematica; ma in quanto presupposto al linguaggio, la matematica ha un'enorme capacità igienica a determinare da cosa la \emph{characteristica universalis} debba essere composta. Lo scopo di questo lavoro è svelare alcuni frammenti del suo lessico, \emph{mostrando} al di là della vana speculazione che un certo modo di operare ha rendimento superiore ad altri, che parlano del, e attorno al, nulla. Siamo poi tanto certi di stare parlando nel modo giusto, che lo facciamo sfacciatamente, scomodando e destrutturando il più difficile dei problemi, per poi restituirlo intatto, ma completamente cambiato: il principio di identità. 

Cosa significa che \emph{due cose sono, invece, una} è un problema che ci arrovella \marginpar{$(ID_2)$} fin da quando otteniamo la ragione e la parola; ciò perché il problema è tanto elementare quanto sfuggente: l'unica maniera in cui possiamo esibire ragionamento certo è il calcolo; del resto, se la sintassi non vede che l'uguaglianza in senso più stretto possibile, la prassi deve diventare in fretta capace di una maggiore elasticità: per un istante ho postulato che ci fossero ``due'' cose, non una. E non è forse questo a renderle due? E questa terza cosa che le distingue, è davvero diversa da entrambe?


\section{Introduction}
\subsection{CT as metalanguage}
\begin{quotation}
	\begin{flushright}
		"\textit{La filosofia dovrebbe imitare i metodi delle scienze che ottengono dei successi}" (C.S.Peirce, 1868)
	\end{flushright}
\end{quotation}

Schema: 
\begin{itemize}
	\item \textit{Captatio benevolentiae} (finta): Ci rendiamo conto che fare filosofia utilizzando modelli matematici è storicamente sospetto di riduzionismo o formalismo
	\item matematica come Ur-linguaggio (eviterei il primo capoverso di Section 6 perchè può sembrare una frettolosa definizione di matematica, data per buona (anche se ne uscirebbe fuori un bel piccolo paperino))
	\item il linguaggio giusto per fare ontologia: la \emph{Category Theory}; Fosco cerca di "tecnicalizzare" prima parte rossa $(F_R)$ 
	\item piccola nota storica: cenni al concetto di struttura in matematica e al bourbakismo
	\item sottolineare utilità (ontologica) dei concetti emersi in quel periodo storico (non tanto nel dibattito fondazionale, quanto nella pratica matematica "strutturale"); cenno distinzione Kr\"omer-Corry tra structuralism e structural mathematics (soprattutto per dire quella roba per cui la pratica strutturale è essa da sola la filosofia necessaria alla matematica)
	\item Come Lawvere allarga gli orizzonti del dibattito: dalle strutture ai funtori (e rende ancor più evidenti le conclusioni del punto precedente); "on our choice of metatheory and foundation" 
	\item detto questo: (più breve e secco di ciò che sembra): cosa è una \emph{teoria} in \textbf{CT} (è una categoria, ovviamente; prima parte rossa di Fo); le categorie sono sia la teorie che gli oggetti delle teorie che i "contesti" delle teorie; metalinguaggio adatto a fare matematica e - noi diciamo - anche ontologia (le teorie sono teorie in fondo, che siano matematiche o metafisiche; si tratta nel secondo caso magari di dover fare un lavoro preliminare di chiarificazione/formalizzazione e poi lavorarci \emph{dentro} una categoria)
\end{itemize}

[\emph{qui ci vuole un qualche incipit da concordare}]

Per quanto possa sembrare sospetto usare la matematica per risolvere questioni tradizionalmente di pertinenza della filosofia, ciò che noi riteniamo di poter fare è fornire un linguaggio adeguato entro il quale parlare di ontologia, prendendolo dalla matematica; se ci è concesso un gioco di parole non si tratta, come la tradizione in filosofia analitica ha da sempre paventato, di fare un \emph{uso corretto del linguaggio} quanto piuttosto un \emph{uso del linguaggio corretto}. Tale linguaggio è preso appunto dalla matematica, \marginpar{\textit{qui forse }$(F_R)$} ed è la \emph{Category Theory} (d'ora in poi \textbf{CT}). Mostreremo come le categorie siano state feconde nella pratica matematica e come possano esserlo analogamente in ontologia. Il motivo è che con esse è possibile (idea prima di esse addirittura inesprimibile) trattare le teorie matematiche come oggetti matematici \footnote{Ogni categoria ha un suo linguaggio interno, come si vedrà, e ogni classe di categorie modella una logica; a loro volta le categorie devono stare in una categoria più grande (in \emph{Type Theory} c'è un postulato che permette di evitare gli ovvi paradossi del caso, stabilendo una gerarchia di tipi; una volta scritto un tipo "esiste", è un oggetto esplicitamente e manifestamente linguistico, senza che si pongano i problemi che dichiareremo ed eviteremo più sotto)}. 

Una \emph{teoria}, infatti, non è altro che una \emph{small category} che realizza \marginpar{\emph{ampliare e/o dire "meglio}} un dato linguaggio logico $L$, ma è anche vero che l'intera matematica si trova \emph{dentro} una categoria (grande). Parlando figurato, le teorie sono categorie e i "luoghi" in cui interpretiamo le teorie sono a loro volta categorie. In \textbf{CT} le categorie sono, quindi, oggetti puramente sintattici, ed è il contesto in cui si opera a determinare una semantica. Già in semantica logica una \emph{interpretazione} $\mathcal{I}$ di un enunciato $\alpha$ è una funzione che associa elementi di un insieme (di solito l'insieme dei valori di verità) alle variabili libere in $\alpha$. Nella storia della \textbf{CT} (caratterizzata da un rifiuto dell'impostazione \emph{set-theoretic} a livello fondazionale) una serie di generalizzazioni e raffinamenti dello stesso procedimento hanno condotto alla nozione di funtore e quindi alla semantica funtoriale di Lawvere \footnote{Lawvere non fa altro, dopo aver costruito il sistema formale "elementare", che presentare una teoria della categoria di tutte le categorie che fornisca modelli per la th elementare (quindi ci dà un linguaggio sintattico, in cui le categorie non sono altro che \emph{termini}, e poi una metateoria nella quale poter considerare categorie di categorie etc, che è poi alla fine una teoria funtoriale, dove ogni ffbf di ETAC diventa una formula di BT in cui si specifica su quale modello operano i termini):
\begin{quotation}
	If $\Phi$ is any theorem of elementary theory of abstract categories, then $ \forall \mathcal{A} (\mathcal{A} \models \Phi)$ is a theorem of basic theory of category of all categories
\end{quotation}
e aggiunge, un po'ambiguamente, "\textit{every object in a world described by basic theory is, at least, a category}". Altri esempi: $\Delta_i$ in ETAC (che indica dom/cod) diventa $\mathcal{A} \models \Delta_i$ in BT; $\forall x [\dots]$ in ETAC diventa $\forall x [x \in \mathcal{A} \rightarrow \dots]$ in BT etc etc}. 

Un funtore $F$ da una categoria $\mathcal{A}$ a una categoria $\mathcal{B}$ è una mappa denotata da $F: \mathcal{A} \to \mathcal{B}$ che associa ad ogni oggetto $X$ di $\mathcal{A}$ un oggetto $F(X)$ di $\mathcal{B}$ e ad ogni morfismo $f: X \to Y$ di $\mathcal{A}$ il morfismo $F(f): F(X) \to F(Y)$ di $\mathcal{B}$, tale da soddisfare le seguenti proprietà:
\begin{itemize}
	\item Per ogni $f: X \to Y$ e $g: Y \to Z$ vale $F(g \circ f)= F(f) \circ F(g)$
	\item $F(id_X) = id_{F(X)}$ per ogni $X \in \mathcal{A}$
\end{itemize}

Se, come si è detto, una teoria non è altro che una categoria $T$, i funtori sono i "modelli" di $T$ entro cui interpretarla. 
 In questo senso è corretto affermare che \textbf{CT} è il metalinguaggio con il quale si può fare tutta la matematica. Nulla vieta, a questo punto, di considerare qualuque teoria ontologica come una categoria, previo lavoro di chiarificazione/formalizzazione che lo stesso linguaggio categoriale ci permette di eseguire. In esso possiamo "tradurre" i problemi classici dell'ontologia, fornire modelli entro i quali formularne meglio presupposti e domande, evidenziare ciò che è banale conseguenza degli assiomi di quel modello e ciò che non lo è, risolverli e, in alcuni casi, dissolverli, rivelandone la natura figmentale. Si tratta di fornire un \emph{ambiente} ben definito nel quale questioni ritenute oggetto di dibattito filosofico possano illuminarsi in modi nuovi o scomparire. E questo non per qualche perverso istinto riduzionistico, ma per poterne parlare in termini efficaci e nel linguaggio adatto a inquadrarli: tentare, con gli strumenti più avanzati e raffinati dell'astrazione matematica, di rispondere a delle domande, produrre conoscenza, e non solo dibattito; inscrivere antiche o recenti questioni in un nuovo paradigma, volto a superare e al contempo far avanzare la ricerca.
 
 Come ogni paradigma lo dotiamo di una sintassi con la quale "nominare" concetti e dare definizioni, e di una semantica che produca modelli, e quindi contesti, entro i quali "guardare" le teorie; questa sintassi e questa semantica non ce le inventiamo: sono già nella matematica e da lì le preleviamo. [Forse qui infilare inciso su identità come nozione relativa alla teoria di riferimento, anticipando cose da dire in 7.2].  
 
 \textbf{Nota}: questo discorso (o almeno l'ultimo capoverso), che va ampliato e scritto meglio, è forse più proficuo metterlo, come già in schema, \emph{dopo} che si è fatta breve carrellata Bourbaki-Lawvere e spiegato la rilevanza pratica e ontologica delle nozioni emerse.  




\subsection{Towards an Operative Ontology: the notion of identity}
\begin{itemize}
	\item dalle brevi riflessioni storico-filosofiche sopra si sottolinea la fecondità di questo approccio nella \emph{pratica} matematica (che ha ricadute anche sulla sua ontologia) (cit. di Kr\"omer quando ridicolizza l'atteggiamento filosofico di voler sapere \emph{prima} cos'è una struttura, senza preoccuparsi di verificare che sia utile e funzioni)
	\item (forse cit. Carnap, tra l'altro quella roba indicata nelle bozze la dice anche per l'ontologia non solo per la semantica) quindi bisogna fare ontologia in modo operativo (cioè bisogna appunto \emph{farla}) utilizzando il linguaggio categoriale; cosa fa questo linguaggio? (mix tra inizio delle mie note e soprattutto Section 6 (o forse meglio quando si parla del linguaggio in 7.1) e parte finale Section 4 (cioè i filosofi devono capire che quando operano in questo modo non stanno sostenendo automaticamente tesi metafisiche forti, tipo uno strutturalismo integrale))
	\item cenni a relatività ontologica praticamente implicita e scontata in questo approccio (dipendenza dell'ontologia dalla teoria in cui stiamo operando): intuizione che a qualche filosofo era venuta, inascoltato. Aiuta a vedere con occhio diverso concetti base come l'\emph{ontological committment} 
	\item noi speroniamo l'identità: $(ID_1) + (ID_2)$
	\item seconda parte rossa Fo (solo sulle monete di rame)
	
\end{itemize}
Si può pensare di mettere una nota programmatica nelle conclusioni. (in generale non mi ci vuole niente a mettere insieme questo schema in un discorso compiuto (forse solo la parte su Lawvere è più ostica anche se appunti che linkano tutte le cose ce li ho (perlopiù tratti dal paper del 65(?)). La \textit{Homotopy Type Theory} è una immensa riflessione sul principio di identità. \textbf{Rilevante}: nella metateoria che utilizziamo le categorie sono i "luoghi" dentro i quali si fa matematica. Idea base da far emergere nell'introduzione. 

Motivi per i quali un filosofo analitico che si occupa di ontologia dovrebbe accettare l'approccio del nostro lavoro (magari esplicitare nelle concluding remarks):
\begin{itemize}
	\item (principale) chiarisce definitivamente la questione della relatività ontologica (o meglio, della \emph{context-dependence}) di alcune nozioni (principalmente l'identità); il che vuol dire tirarsi fuori da un dibattito che spesso blocca la ricerca su questi temi. Non ha senso parlare di ontologia in senso \emph{assoluto} (è una intuizione che hanno già in molti, se non tutti, e vedersela emergere con chiarezza da un approccio lo rende convincente)
	\item chiarisce roba come le sfere di Black nel concreto e lo può fare potenzialmente con ogni esperimento mentale possibile. Quand'anche non riuscissero a digerire l'idea di una ontologia totalmente operativa, molti di loro sono da decenni calati nella mentalità \emph{problem solving} per cui è efficace che un approccio risolva dei problemi.
	\item (sarò ultra mega azzardato e impreciso ma:) le categorie è \emph{come se} formalizzassero una nozione di "contesto"; avere un ambiente comodo e rigoroso nel quale riflettere sugli esperimenti mentali, o i concetti e le definizioni, che interessano agli ontologi è senza dubbio un modo per fare filosofia in un laboratorio "controllato", dal quale non può fuoriuscire nulla più di ciò che si vuole indagare in quel momento; altro sogno a cui aspirano molti senza dirlo. In breve: avere una nozione di contesto (o "luogo in cui stanno le teorie") è anelito decennale, specie di recente, e mette l'acquolina in bocca. 
	\item gli esempi tratti dalla storia della matematica ci spingono poi a chiarire che questo approccio non implica adesioni a teorie metafisiche "forti" (e neanche a un "matematismo" radicale che spaventa molti di loro). Avere un linguaggio che sia una specie di macchina-risolvi-problemi, tra l'altro dotato di concetti molto più interessanti e profondi di quelli insiemistici a cui l'ontologia formale è abituata, senza dover aderire a metafisiche prestabilite, è persuasivo oltre ogni arzigogolo da retore.    
\end{itemize} 
Queste sono le "strategie retoriche" che devono emergere insieme alle idee base, sia nella parte introduttiva che conclusiva; a quest'ultima deve fare da controcanto la parte in cui applichiamo la \textbf{CT}: lì vanno sempre sottolineate, dopo i calcoli, quali semplici conseguenze ontologiche se ne traggono. Anche se sono certo che non resisteremo a fare un po' quelli che tirano fuori il coniglio dal cappello. 

nella microcategoria ogni categoria modella una logica, ma esse stesse stanno nella categoria di tutte le categorie (che per evitare paradossi è una cat in un universo più grande; in type th c'è un postulato, catena di tipi, gerarchia etc); una volta scritto un tipo esiste, è più palesemente una entità linguistica; 
il linguaggio interno di una cat mi permette di dire cos'è una struttura e di fare tutta la matematica che mi serve, (basta ad esempio la categoria $Set$); la \textbf{CT} fornisce i modelli per fare matematica (e non solo, distinzione Agazzi teorie concrete/astratte della mat) [che non ha ragion d'essere almeno dagli anni '70, vale a dire che la distinzione si appiattisce sulle teorie astratte, lo sono tutte nel senso di Agazzi, e per questa possiamo usare strutture matematiche per trattare di oggetti non classicamente matematici come le teorie ontologiche];  [frase su tg: tu sai cos'è una categoria ma per scrivere la ct devi metterti in un'altra categoria]; cose scritte sul quaderno blu 


prendi la retta reale e prendi gli intervalli nellì'oridine di inclusione inverso. questo insieme parzialemnte ordinato ha una topologia canonica (di Scott), laquale ha segmenti iniziali, finali. Al variare degli intervalli tu prendi tutti gli intervalli contenenti quello dato. su questo spazio puoi prendere la cat dei fasci; gli intrvalli sono r, e un intervallo è tanto grande quanto l'ordine è piccolo. 

A un dato tempo t p è vera, a un dato s è falsa; tra t e s ci sono altre cose, per cui si usano come modelli ordini densi (ecco perché la retta reale). 

In $Set/I$ nel ling interno

$p : U \to \Omega_I$

$\Omega_I$ classif sottoggetti, $\Omega_I = \{0,1\}\times I \to I$.

$\{ m : A \hookrightarrow B\}$ $\cong$ $\{ \chi : B \to \Omega_I\}$

$p : U \to \{0,1\}\times I$

$(p : U \to \{0,1\}\times I \to I) = u : U \to I$

$$
\begin{array}{ccc}
U &\overset{p}\to& \{0,1\}\times I \\
u\downarrow && \downarrow\pi \\
I &=& I
\end{array}
$$
$\pi p(x : U) = u(x : U)$
$p(x) = (\epsilon, u(x))$ $\epsilon =0,1$

$p(x) = (\epsilon, t_x)$

$A = \{x : U \mid p(x) = (1,t_x), t_x > 0\} = p^\leftarrow(\{1\}\times (0,1])$ \\
$B = \{x : U \mid p(x) = (1,1)\} = p^\leftarrow((1,1))$ \\
$E = \{t : I \mid \exists x : U, p(x)=(1,t)\}$ 

$p(x) = (0,1/2)$

$p(x) = (1,1/2)$


$p(x) = (0,1/3)$

$p(x) = (0,1/\pi)$

$p(x) = \text{la freccia esiste}$
$p$ è super vera
$p(x)=(1,1)$

$p$ è vera un po', non tanto
$p(x) =(1,t<1)$

$p$ è falsa un po'
$p(x) =(0,t<1)$

$\pi p$ non deve essere continua; quando è continua la freccia perdura nell'esistenza, e questi oggetti divini si chiamano hrön \\
$\pi_I \circ p : U \to \{0,1\}\times I \to I$


\subsection{risposte all'idealista}
All'idealista, il berkeleyano puro, che volesse dedurre, da questa "esistenza ad intermittenza" della freccia, ch'essa in realtà non sussiste, potremmo obiettare che sta parlando male. In questo modello è scorretto dire che $p$ è falsa in tutti i punti di $I$, ma che semmai è "falsa con forza $t$" dove $t \in I$. 

Ad essere ancora più pignoli, alla domanda se la freccia esista o meno la risposta corretta è a sua volta una domanda: "In quale $I$ sei? Qual è il tuo classificatore di sottoggetti?". Indicato il proprio punto (che sia un riferimento temporale, spaziale o di qualunque altro genere) allora si può rispondere mettendo un numero reale al posto di $t$, indicando cioè la "forza" di esistenza dell'oggetto. 

Sostituire nozioni classiche (identità, persistenza-nel-tempo) con nozioni precise suggerite dal linguaggio che proponiamo, vuol dire perciò modificare anche il linguaggio naturale. Quando l'idealista afferma che negli istanti in cui chiude gli occhi la sua casa non esiste più, sta anch'egli parlando di "forza" nel senso qui definito: lo deve solo esplicitare. Si mette in un universo (un topos) "non-classico" con più valori di verità e deve semplicemente calcolare \emph{qual è} il valore di $p$. 

Il modello fa più di quello che prometteva: è una descrizione adeguata del mondo di Tl\"on ma anche della persistenza nel tempo in $@$ (in quale relazione il nostro mondo stia con Tl\"on è da vedere). 

Da notare che abbiamo potuto modellizzare il paradosso senza impiegare un framework di logica temporale; è lecito interpretare gli elementi di $I$ come istanti, ma non necessario. La descrizione in termini temporali è un sottocaso del modello più generale che abbiamo fornito. Persino un presentista o l'idealista coerente di Borges (che dalle tesi berkeleyane deduce l'inesistenza del tempo) può stabilire la verità di $p$ senza abiurare alla sua posizione ontologica, ma deve poi accettare il risultato del calcolo. 

L'ampliamento dei valori di verità si traduce in un ampliamento delle risposte possibili da dare alla domanda sulla freccia. Non indichiamo quella "corretta" ma il topos in cui trovarla, il contesto di discorso dal quale non si può prescindere, intuitivamente un contesto fuzzy, che evita le sterili dicotomie in cui era impantanato il dibattito da qualche secolo.  Se ci mettiamo in un contesto paraconsistentista è possibile inoltre modellizzare tutto "dialogo ontologico" tra soggetti, plausibilmente con un approccio non aggiuntivo, e una semantica di Rescher-Brandom o una Lewis' strategy. La densità di $I$ è la proprietà che permette l'accordo intersoggettivo in Tl\"on.     

\end{document} 
