\section{Introduction}\label{sec_intro}
\epigraph{El mundo, desgraciadamente, es real.}{\cite{confutacion}}
This is the first piece of a series of works that will hopefully span through a certain amount of time, and touch a pretty wide range of topics. Its purpose is to adopt a wide-ranging approach to a fragment of elementary problems in a certain branch of contemporary philosophy. In particular, our work aims to lay a foundation, or a transcription, of a few problems in ontology by means of pure mathematics; in particular, by means of the branch of mathematics known as category theory.

As authors, we are aware that such an ambitious statement of purpose must be adequately motivated. This is the scope of the initial section of the present first manuscript.
\subsection{What is this series}

Since forever, mathematics studies three fundamental indefinites: form, measure, and inference. Apperception makes us recognise that there are extended entities in space, persisting in time. From this, the necessity to measure how much these entities are extended, and to build a web of conceptual relations between them, explaining how they arrange ``logically''.
Contaminations between these three archetypal processes are cetainly possible and common; mathematics happens exactly at the crossroad where algebra, geometry and logic intersect.

We can even say more: mathematics is a language; metamathematics done through mathematics (if such a thing even exists) exhibits the features of a \emph{ur-language}, a generative scheme for ``all'' possible languages. It is a language whose elements are the rules to give oneself a language, conveying information, and allowing deduction. It is a metaobject: a scheme to generate objects/languages.

Taken this tentative definition, mathematics (not its history, not its philosophy, but its \emph{practice}) is serves as a powerful tool to tackle the essential questions of ontology: what ``things'' are, what makes them what they are and not different.

Quantitative thinking is a consistent (some would say ``honest'') way of approaching this deep problem, cogency of entities; however, it is undeniable that a certain philosophical debate has become hostile to mathematical language. A tear in the veil that occurred a long time ago, due to different purposes and different specific vocabulary, can not be repaired by two people only. If, however, the reader of these notes asks for an extended motivation for our work, a wide-ranging project in which it fits, a long-term goal, in short a \emph{program}, they will find it now: there is a piece of mathematics whose purpose is to solve philosophical problems, in the same way certain mathematics ``solves'' the motion of celestial bodies. It does not completely annihilate the question: it proposes models within which one can reformulate it; it highlights what is trivial consequence of the axioms of that model and what instead is not, and requires the language to expand, to be modified, sharpened. Our aim is to approach this never-mentioned discipline as mathematicians. %in realtà questo ultimo pezzetto lo avevo interpolato nella subsubsection ultima di 1.3; vedere dove sta meglio%

Sure, solving once and for all the problems posited by ontology would be megalomaniac; we do not claim such a thing. Instead, more modestly, we propose a starting point unhinging some well-established beliefs (above all else, the idea that ontology is too general to be approached quantitatively); we humbly point the finger at some problems that prose is unable to notice, because it lacks a specific and technical language; we suggest that \emph{in such a language}, when words mean precise things and are tools of epistemic research instead of mere magic spells, a few essential questions of recent ontology dissolve in a thread of smoke, and others simply become ``the wrong question'': not false, just meaningless as the endeavour to attribute a temperature to consciousness or justice.

We shall say at the outset that such a language is not mathematics; but mathematics has an enormous hygienic capacity to hint at what the \emph{characteristica universalis} should be made of.

It may seem suspicious to employ mathematics to tackle questions that traditionally pertain to philosophy, we believe we can provide a more adequate language, taken from mathematics, within which to frame some deep ontological questions. The reader will allow a tongue-in-cheek here: when coming to ontology, it is not a matter of making a \emph{correct use of language}, but rather an \emph{use of the correct language}. 

This language is inherently mathematical, as only mathematics has the power to substantiate an analysis in quantitative terms. This language is \emph{category theory}, as only category theory has the power to speak about totality of ``all'' structures of a given kind in a compelling way, treating mathematical \emph{theories} as mathematical \emph{objects}.%\footnote{We will see that each category has its own \emph{internal language}, a formal language in which all first-order logic can be concocted; in turn categories, taken all together sit in a larger category, a \emph{meta}category; cf. the notion of \emph{cumulative hierarchy}, or the hierarchy of types in whatever model of type theory.}

As it is currently organised, our work will attempt to cover the following topics:
\begin{itemize}
	\item the present manuscript, \emph{Existence}, aims at providing a sufficiently expressive theory of existence. Having a foundational r\^ole, the scope of most of our remarks is of course more wide-ranging and aimed at building the fundamentals of our toolset (mainly, category theory and categorical logic, as developed in \cite{mac1992sheaves,JohnstonePT,lambek1988introduction}). As both a test-bench for our theory, and a literary \emph{divertissement}, we propose a category-theoretic solution of Borges' paradoxes present in \cite{fictions}.
	\item A second chapter \cite{black}, currently in preparation, addresses the problem of \emph{identity}, and in particular its context-dependent nature. Our proof of concept here consists of a solution to Black's ancient ``two spheres'' paradox; this time the solution is provided by Klein's famous \emph{Erlangen programme} group-theoretic devices. The two interlocutors of Black's imaginary dialogue inhabit respectively an Euclidean and an affine world: this affects their perception of the ``two'' spheres, and irredemably prevents them from mutual understanding.
	\item A third chapter \cite{homot}, currently in preparation, addresses again the problem of identity, but this time through the lens of algebraic topology; in recent years homotopy theory defied a well\hyp{}established ontological assumptions such as the identity principle; the many commonalities between category theory and homotopy theory suggest that ``identity'' is not a primitive concept, but instead depends on our concrete representation of mathematical entities. When $X,Y$ are objects in a category $\clC$, there is often a class of ``equivalences'' $W \subseteq \hom(\clC)$ prescribing that $X,Y$ shouldn't be distinguished; equality (better, some sort of homotopy \emph{equivalence}) is then defined \emph{ex post} in terms of $W$, changing as the ambient category $\clC$ does; this yields a $W$-parametric notion of identity $\equiv_W$, allowing categories to be categorified versions of \emph{Bishop sets}, i.e. pairs $(S,\rho)$ where $\rho$ is an equivalence relation on $S$ prescribing a $\rho$-equality.
\end{itemize}
\subsection{On our choice of metatheory and foundation}
During '900 the direction of the evolution of mathematics brought the discipline to devide in different sub-disciplines at first, with their own specific objects and languages, to then find unexpected unification under a one an only notion, the notion of \emph{structure}, using the formal tool that has best characterized the concept, the \emph{categories}. This process has spontaneously led to epistemological revision of mathematics and has inspired, in the development of operational tools, a revision of both his foundations and his ontology. For many scholars is undeniable that

\begin{quote}
	[the] mathematical uses of the tool \CT and epistemological
	considerations having \CT as their object cannot be separated, neither historically
	nor philosophically. \cite{kromer2007tool}
\end{quote}
This happened regardless of both the specific foundational debate \cite{,,,}.

The mathematical practice in the structuralist way produces a "natural" ontology, that same, later, felt the need to characterized more precisely. After all, similarly to what Carnap \footnote{Some words that philosophers should keep in mind, on the lawfulness of the use of abstract entities (specifically mathematical) in semantic reflection, also valid in ontology:
	\begin{quote}
		we take the position that the introduction of the new ways of speaking does not need any theoretical justification because it does not imply any assertion of reality [...].  it is a practical, not a theoretical question; it is the question of whether or not to accept the new linguistic forms. The acceptance cannot be judged as being either true or false because it is not an assertion. It can only be judged as being more or less expedient, fruitful, conducive to the aim for which the language is intended. Judgments of this kind supply the motivation for the decision of accepting or rejecting the kind of entities. \hfill \cite{carnap1956meaning}
	\end{quote}} suggested regarding semantics,

\begin{quote}
	mathematicians creating their discipline were apparently not seeking to justify the constitution of the	objects studied by making assumptions as to their ontology.\hfill  \cite{kromer2007tool}
\end{quote}
Beyond the attempts (ever those th Bourbaki group), what matter is that the habit of reasoning in terms of structure has suggested implicit epistemological and ontological attitudes. This matter would deserve an exhausting independent inquiry.

 For our objectives it's enough to declare a differentiation that Kr\"omer elaborate, inspired by [Corry, 1996]: the different between \textbf{structuralism} and \textbf{structural mathematics}:

\begin{itemize}
	\item[\textbf{(1)}] Structuralism: \textit{the philosophical
		      position regarding structures as the subject matter of mathematics}
	\item[\textbf{(2)}] Structural Mathematics: \textit{the methodological approach to look in a given problem
		      “for the structure”}
\end{itemize}
\begin{remark}
	\textbf{(1)} implies \textbf{(2)} but the opposite is not always true.
	
	That is, one can do structural mathematics without being structuralists; that is, taking different, or even opposite, positions with respect to structuralism itself. That it has often come spontaneously to embrace this position in the theoretical reflection in the recent history of mathematics is quite different (the "natural" ontology). 
	
	The use of \theory{CT} as meta-language, despite the historical link with structuralism, it doesn't nevertheless make authomatic the transition from \textbf{(2)} to \textbf{(1)}, but suggests that the ontology is not only dependent to "ideology" (in quinean sense) of the theory, namely it expressive power, but is influenced by the epistemological model inspired by use of formal language itself. 
	 
\end{remark}

The usefulness of Kr\"omer's distinction, however, is another: instead of stumbling in a possibly not ambigous definition of \textit{structure} (with the unwanted consequences that could arise in the operational practice), the philosophy \textbf{(1)} can be reduce (or redefine) to the methodology \textbf{(2)}, saying that: 

\begin{quote}
	\emph{structuralism is the claim that mathematics
		is essentially structural mathematics} \cite{kromer2007tool}
\end{quote}

(the operative practice that "intervenes" in definition of structuralism avoid the decade-long debate of the humanities on this concepts).


This is the same thing as saying: the structural practice is its philosophy itself. 

The historical attempts of explaining the term "structure" by Bourbaki in the years following publication of the \textit{Elements} \cite{,,,}, had the first systematic elaboration of a philosophy that could accorded with rate of success of \textit{structural mathematics}. His target it is to "\textit{assembling of all possible ways in which given set can be endowed with certain structure}" \cite{kromer2007tool}, and to do so elaborate, in the programmatic paper \textit{The Architecture of Mathematics} (written by Dieudonné alone), published in 1950, a formal strategy. While specifying that "\textit{this definition is not sufficiently general for the needs of mathematics}" [Bourbaki, 1950], he codifies a series of operational steps through which a structure on set is "assembled set-theoretically". Adopting therefore a redcutionist perspective in which 

\begin{quote}
	the structureless sets are the raw material of structure building which in Bourbaki’s analysis is ``unearthed'' in a quasi-archaeological, reverse manner; they are the most general objects which can, in a rewriting from scratch of mathematics, successively be endowed with ever more special and richer structures.\hfill  \cite{kromer2007tool}
\end{quote}
On balance in the bourbakist structuralism the notion of \emph{set} doesn't disappear definitively in front of the notion of structure. The path towards an ``integral'' structuralism was still long.



In his influential paper \cite{lajolla} William Lawvere proposes a foundation of mathematics based on category theory. To appreciate the depth and breadth of such an impressive piece of work, however, the word ``foundation'' must be taken in the particular sense intended by mathematicians:
\begin{quote}
	[\dots\unkern] a single system of first-order axioms in which all usual mathematical objects can be defined and all their usual properties proved.
\end{quote}
Such a position sounds at the same time a bit cryptic to unravel, and unsatisfactory; Lawvere's (and others') stance on the matter is that a foundation of mathematics is \emph{de facto} just a set $\clL$ of first order axioms organised in a Gentzen-like deduction system. The deductive system so generated reproduces mathematics as we know and practice it, i.e. provides a formalisation for something that already exists and needs no further explanation, and that we call ``mathematics''.

It is not a vacuous truth that $\clL$ exists somewhere: the fact that the theory so determined has a nontrivial model, i.e the fact that it can be interpreted inside a given familiar structure, is both the key assumption we make, and the less relevant aspect of the construction itself; showing that $\clL$ ``has a model'' is --although slightly improperly-- meant to ensure that, \emph{assuming the existence of a naive set theory} (i.e., assuming the prior existence of structures called ``sets''), axioms of $\clL$ can be satisfied by a naive set. Alternatively, and more crudely: assuming the existence of a model of ZFC, $\clL$ has a model \emph{inside that model of ZFC}.\footnote{However, ensuring that a given theory has a model isn't driven by psychological purposes only: on the one hand, purely syntactic mathematics would be very difficult to parse, as opposed to the unformalised, more colloquial practice of mathematical development; on the other hand (and this is more important), the only thing syntax can see is equality, and truth. To prove that a given statement is false, one either checks all possible syntactic derivations leading to $\varphi$, finding none --this is unpractical, to say the least-- or they \emph{find a model} where $\lnot\varphi$ holds.}

A series of works attempting to unhinge some aspects of ontology through category theory should at least try to tackle such a simple and yet diabolic question as ``where'' are the symbols forming the first-order theory of ETCC. And yet, everyone just believes in sets, and solves the issue of ``where'' they are with a leap of faith from which all else follows.

This might appear somewhat circular: aren't sets in themselves already a mathematical object? How can they be a piece of the theory they aim to be a foundation of? In his \cite{lolli1977categorie} the author addresses the problem as follows:
\begin{quote}
	Quando un matematico parla di modelli non ha infatti l'impressione di uscire dall'ambito insiemistico. Questa impressione, che è corretta, è giustificata dalla possibilità di rappresentare i linguaggi formali con gli oggetti della teoria degli insiemire di studiare in essa le relazioni tra le strutture e le rappresentazioni dei simboli. Quando l'attenzione è rivolta ai modelli di una teoria degli insiemi, certe questioni sofistiche non possono però essere più evitate. La domanda spontanea sulla relazione che intercorre tra gli insiemi che so­no modelli di una teoria e gli insiemi di cui parla la teoria non è altro che una domanda sulla relazione tra la metateoria semantica insiemi­stica e la teoria in esame, o teoria oggetto. Le due teorie possono coincidere, anzi la metateoria può essere anche una sottoteoria pro­pria della teoria oggetto. [\dots\unkern]
	
	Si sceglie un insieme i cui elementi, di solito insiemi finiti, rappresentano i simboli del linguaggio che si vuole studiare, quindi con una operazione di concatenazione che può essere la coppia ordinata si definisce l'insieme delle parole $L$-delle for­mule ben formate, dei termini, l'operazione che a una formula asso­cia le sue variabili libere e così via per tutte le nozioni sintattiche.
\footnote{Authors' translation: \emph{}}
\end{quote}
The idea that a subtheory $L'\subset L$ of the object theory can play the r\^ole of metatheory might appear baffling; in practice, the submodel doesn't exactly play such r\^ole: instead, there are two possible solutions. Pure platonism assumes the existence of a hierarchy of universes harbouring the object theory; pure syntacticism exploits G\"odel's completeness theorem: every proof is a finite object, and every theorem proved in the metatheory is just a finite string of symbols. No need for a model.

Platonism has limits: in a fixed a class theory $\sfC$ ($\sf MK$, Morse-Mostowski; or $\sf NBG$, Von Neumann-Bernays-G\"odel), there's an object $V$ that plays the r\^ole of the universe of sets; in $V$, all mathematics can be enacted. Of course, consistency of $\sfC$ is only granted by an act of faith.

Syntacticism has limits: following it, one abjures any universality mathematics might claim. But syntacticism also has merits: undeniably (disgracefully, luckily) the World is real. And reality is complex enough to contain languages as purely syntactical objects; the percussion of a log with oxen bones, rather than prophecies over the entrails of a lamb, or intuitionistic type theory, all have the same purpose: intersubjective convection of meaning, deduced by a bundle of perceptions, so to gain advantage, i.e. predictive power, over them. Of course: intuitionistic type theory is just \emph{slightly} more effective than hepatomancy.

Let's say at the outset this point of view makes absolutely no claim of originality; research on this matter is fervid in communities other than the philosophical one (cf. \cite{,,,}). 

Accepting that knowledge is obtained by collision and retro-propagation between Reality and the perceptual bundle it generates, the urge to define seemingly abstract concepts like learning, conscience, and knowledge, together with precious continuous feedback coming from real objects, evidently determine an undeniable primacy of quantitative thinking, this time intended as machine learning and artificial intelligente, that sets (it should, if only more philosopers knew linear algebra) a new bar for research in philosophy of mind.

However, we refrain from entering such a deep rabbit-hole, as it would have catastrophic consequences on the quality, length, and depth of our exposition. 

The usual choice for mathematicians is to assume that, wherever and whatever they are, these symbols ``are'', and our r\^ole in unveiling mathematics is \emph{descriptive} rather than generative \footnote{Dal punto di vista costruttivista non si può legittimamente affermare che gli assiomi ``creino'' gli oggetti matematici, dal che la risposta alla domanda ``dove sono?'' sarebbe ``da nessuna parte''. L'unica cosa che si può dire certamente è che essi ``precisano, in modo rigoroso ed esatto, ancorché implicito, il \emph{significato} degli oggetti matematici'' [Agazzi,\cite{?}] (e sembra che anche in matematica, come è uso in filosofia del linguaggio, sia assennato tenere ferma la distinzione tra significato di un'espressione e sua denotazione). Noi ci fermiamo su questa soglia.}

This state of affairs has, to the best of our moderate knowledge on the subject, various possible explanations:
\begin{itemize}
	\item On one hand, it constitutes the heritage of Bourbaki's authoritarian stance on formalism in pure mathematics;
	\item on the other hand, a different position would result in barely no difference for the ``working class''; mathematicians are irreducible pragmatists, somewhat blind to the consequences of their philosophical stances.
\end{itemize}
So, symbols and letters do not exist outside of the Gentzen-like deductive system we specified together with $\clL$.

As arid as it may seem, this perspective proved itself to be quite useful in working mathematics; consider for example the type declaration rules of a typed functional programming language: such a concise declaration as
\begin{lstlisting}[ language=Haskell
                , basicstyle=\ttfamily\small
                , keywordstyle=\color{blue!75}
                , morekeywords={S,Z}
                  ]
  data Nat = Z | S Nat
\end{lstlisting}
makes no assumption on ``what'' \verb|Z| and \verb|S:: Nat -> Nat| are; instead, it treats these constructors as meaningful formally (in terms of the admissible derivations a well-formed expression is subject to) and intuitively (in terms of the fact that they model natural numbers: every data structure that has those two constructors must be the type $\bbN$ of natural numbers).

In altre parole possiamo tradurre "formally" con "synctactically" (di modo che, ad esempio, gli assiomi di \theory{PA} sono significanti per via delle derivazioni formali che ci permettono di fare) e "intuitively" con "semantically" (gli assiomi di \theory{PA} si presume abbiano almeno un modello, quello standard).

Taken as an operative rule, this reveals exactly what is our stance towards foundations: we are ``structuralist in the metatheory'', meaning that we treat the symbols of a first-order theory or the constructors of a type system irregardless of their origin, provided the same relation occur between criptomorphic collections of labeled atoms.
\begin{italian}
\begin{remark}
	Ergo per noi nel metalinguaggio gli oggetti del linguaggio sono strutture (o li trattiamo come tali), sospendendo il giudizio su cosa effettivamente \emph{siano} fuori dal metalinguaggio. Possiamo intuitivamente schematizzare così:
	\begin{center}
		\begin{tabular}{lccr}\toprule
			$L$                   & Objects    & Denotation & Ontology   \\
			\midrule
			\textbf{Language}     & Categories & Theories   & ??         \\
			\midrule
			\textbf{Metalanguage} & Categories & ``Places'' & Structures
		\end{tabular}
	\end{center}
	Da qui già si può notare, e lo si dirà ancora in 1.4, come questa impostazione non comporti affatto sostenere una forma di strutturalismo metafisico ``forte'', solo una versione ``weak'' a livello metalinguistico.

	E' d'aiuto essersi posti in un'ottica relazionale, implicita nella teoria che usiamo (sul rapporto stretto tra strutturalismo e uso di ontologie relazionali [cfr. \cite{??}]) e quindi possiamo parlare di relazioni tra oggetti senza dire cosa siano gli oggetti, se non, formalmente, i termini posti ai lati estremi dei funtori. Sappiamo dire cosa sono gli oggetti della teoria dal punto di vista della metateoria, in base a come li trattiamo nella pratica operativa (cioè strutture e relazioni tra esse), ma non cosa sono \textit{nella} teoria, se non simboli, sulla cui fondazione non ci pronunciamo.
\end{remark}
\end{italian}
In this precise sense we are thus structuralists in the metatheory, and yet we do so with a grain of salt, maintaining a transparent approach to the consequences and limits of this partialisation. On the one hand, pragmatism works \footnote{Ed invero noi stiamo chiedendo agli ontologi di diventare ``pragmatisti'', o operativisti}; it generates rules of evaluation for the truth of sentences. On the other hand, this sounds like a Munchhausen-like explanation of its the value, in terms of itself. Yet there seems to be no way to do better: answering the initial question would give no less than a foundation of language.

And this for no other reason that ``our'' metatheory is something near to a structuralist theory of language; thus, a foundation for such a metatheory shall inhabit a meta-metatheory\dots{} and so on.

Thus, rather than trying to revert this state of affairs we silently comply to it as everyone else does; but we feel contempt after a brief and honest declaration of intents towards where our metatheory lives. Such a metatheory hinges again on work of Lawvere, and especially on the series of works on functorial semantics.
\subsection{Categories as places}\label{as_places}
The present section has double, complementary purposes: we would like to narrow the discussion down to the particular flavour in which we interpret the word ``category'', but also to expand its meaning so as to encompass its r\^ole as a foundation for mathematics. More or less, the idea is that a category is both an algebraic structure (a microcosm) and a metastructure in which \emph{all} other algebraic structures can be interpreted (a macrocosm).\footnote{As an aside, we shall at least mention the dangers of too much a naive approach towards the micro/macrocosm dichotomy: if all algebraic structures can be interpreted in a category, and categories are algebraic structures, there surely is such a thing as the theory of categories \emph{internal} to a given one. And large categories shall be thought as categories internal to the ``meta''category (unfortunate but unavoidable name) of categories. There surely is a well-developed and expressive theory of internal categories (see \cite[8]{Bor1}); but our reader surely has understood that the two ``categories'', albeit bearing the same name, shall be considered on totally different grounds: one merely is a structure; the other is a foundation for that, and others, structure.}

More in detail, a category provides with a sound graphical representation of the defining operation of a certain type of structure $\fkT$ (see \autoref{unialg} below; we take the word \emph{universal} in the sense of \cite[XV.1]{grillet2007abstract}).
Such a perspective allows to concretely build an object representing a given (fragment of ) a language $L$, and a topos (see \autoref{eletop}) obtained as sort of a universal semantic interpretation of $L$ as internal language. This construction is a classical piece of categorical logic, and will not be recalled here: the reader is invited to consult \cite[II.12, 13, 14]{lambek1988introduction}, and in particular
\begin{quote}
	J. Lambek proposed to use the \emph{free topos} [on a type theory/language] as ambient world to do mathematics in; [\dots\unkern] Being syntactically constructed, but universally determined, with higher-order intuitionistic type theory as internal language, [Lambek] saw [this structure] as a reconciliation of the three classical schools of philosophy of mathematics, namely formalism, platonism, and intuitionism.
\end{quote}
In light of this, and in light of our \autoref{funsemanzi}, we stress once again that categories live on different, almost opposite, grounds: as syntactic objects, that can be used to model \emph{language}, and as semantic objects, that can be used to model \emph{meaning}.

Interpretation, as defined in logical semantics \cite{gamut1991logic}, can be seen as a function $t: L^\star \to \Omega$ that associates elements of a set $\Omega$ to the free variables of a formula $\alpha$ in a language $L$; along the history of category theory, subsequent refinements of this fundamental idea led to revolutionary notions as that of functorial semantics or (elementary) topos.
As an aside, it shall be noted that the impulse towards this research was somewhat motivated by the refusal of set-theoretic foundations, opposed to type-theoretic ones.

In the following subsection, we give a more fine-grained presentation of the philosophical consequences that a ``metatheoretic structural'' perspective has on mathematical ontology.
\subsubsection{Theories and their models}
In \cite{lajolla} the author W. Lawvere builds a formal language \theory{ETAC} encompassing ``elementary'' category theory, and a theory \theory{ETCC} for the category of all categories, yielding a model for \theory{ETAC}. In this perspective category theory has a syntax \CT; in \CT, categories are terms. In addition, we are provided with a \emph{meta}theory, in which we can consider categories of categories, etc.:%e poi una metateoria nella quale poter considerare categorie di categorie etc, che è poi alla fine una teoria funtoriale, dove ogni ffbf di ETAC diventa una formula della \emph{basic theory} di ETCC in cui si specifica su quale modello operano i termini:
\begin{quote}
	If $\Phi$ is any theorem of elementary theory of abstract categories, then $\forall \clA (\clA \models \Phi)$ is a theorem of basic theory of category of all categories. \hfill \cite{lajolla}
\end{quote}
After this, the author makes the rather ambiguous statement that ``\textit{every object in a world described by basic theory is, at least, a category}''. This is a key observation: what is the world described by \theory{ETAC}, what are its elements?

We posit that the statement shall be interpreted as follows: categories in mathematics carry a double nature. They surely are the structures in which the entities we are interested to describe organise themselves; but on the other hand, they inhabit a single, big (meta)category of all categories. Such a big structure is fixed once and for all, at the outset of our discussion, and it is the \emph{place} in which we can provide concrete models for ``small'' categories. To fix ideas with a particular example: we posit that there surely is such a thing as ``the category of groups''. But on the other hand, groups are just very specific kinds of sets, so groups are but a substructure of the only category that exists.% In altre parole, da un lato esiste la categoria dei gruppi; dall'altro essa è semplicemente una sottocategoria dell'Universo, o ``dell'unica categoria che esiste'': quella degli insiemi.

Sure, such an approach is quite unsatisfactory in a structural perspective. It bestows the category of sets with a privileged role that it does not have: sets are just \emph{one} of the possible choices for a foundation of mathematics. Instead, we would like to disengage the (purely syntactic) notion of structure from the (semantical) notion of interpretation.

The ``categories as places'' philosophy now provides such a disengagement, in order to approach the foundation of mathematics agnostically.%: it is not important what the foundational model contains, its universal property is important.

In this perspective one can easily fit various research tracks in categorical algebra, \cite{Janelidze2004}, functorial semantics \cite{lawvere1963functorial,hyland2007category}, categorical logic \cite{lambek1988introduction}, and topos theory \cite{JohnstonePT} that characterised the last sixty years of research in category theory.

Lawvere's \emph{functorial semantics} was introduced in the author's PhD thesis \cite{lawvere1963functorial} in order to provide a categorical axiomatisation of universal algebra, the part of mathematical logic whose subject is the abstract notion of mathematical structure: a semi-classical reference for universal algebra, mingled with a structuralist perspective, is \cite{manes2012algebraic}; see also \cite{sankappanavar}. We shall say at the outset that a more detailed and technical presentation of the basic ideas of functorial semantics is given in our \autoref{funsemanzi} below; here we aim neither at completeness nor at self-containment.

Everything starts with the following definition:
\begin{definition}\label{unialg}
	A \emph{type $\fkT$ of universal algebra} is a pair $(T,\underline{\alpha})$ where $T$ is a set called the (\emph{algebraic}) \emph{signature} of the theory, and $\underline\alpha$ a function $T \to \bbN$ that assigns to every element $t\in T$ a natural number $n_t: \bbN$ called the \emph{arity} of the function symbol $t$.
\end{definition}
\begin{definition}
	A (\emph{universal}) \emph{algebra} of type $\fkT$ is a pair $(A,f^A)$ where $A$ is a set and $f^A: T \to \prod_{t\in T} \Set(A^{n_t},A )$ is a function that sends every function symbol $t: T$ to a function $f^A_t: A^{n_t} \to A$; $f^A_t$ is called the $n_t$-ary operation on $A$ associated to the function symbol $t: T$.
\end{definition}
We could evidently have replaces $\Set$ with another category $\clC$ of our choice, provided the object $A^n: \clC$ still has a meaning for every $n: \bbN$ (to this end, it suffices that $\clC$ has finite products; we call such a $\clC$ a \emph{Cartesian} category). A universal algebra of type $\fkT$ in $\clC$ is now a pair $(A,f^A)$ where $A: \clC$ and $f^A: T \to \prod_{t\in T} \clC(A^{n_t},A )$; it is however possible to go even further, enlarging the notion of ``type of algebra'' even more.

The abstract structure we are tryin to classify is a \emph{sketch} (the terminology is neither new nor unexplicative: see \cite{ehresmann1968esquisses,coppey1984leccons, Bor2}) representing the most general arrangement of operations $f^A: A^n \to A$ and properties thereof\footnote{Examples of such properties are (left) alternativity: for all $x,y,z$, one has $f^A(x,f^A(x,y)) = f^A(f^A(x,x),y)$; associativity: $f^A(x,f^A(y,z)) = f^A(f^A(x,y),z)$; commutativity: $f^A(x,y)=f^A(y,x)$; and so on.} that coexist in an object $A$; such a sketch is pictorially represented as a (rooted directed) graph, modeling arities of the various function symbols determining a given type of algebra $\fkT$ (see also \cite[XV.3]{grillet2007abstract} for the definition of \emph{variety of algebras}).%: un'algebra universale soggetta a equazioni, ossia una coppia $(A,R)$ ove $A$ è un'algebra di tipo $\fkT$ e $R\subseteq A^\star \times A^\star$ è un sottoinsieme di coppie di parole in un opportuno monoide di Kleene modificato).

Given the theory $\fkT$ and the graph $G_{\fkT}$ that it represents, the category $\mathcal{L}_{\fkT}$ generated by $G_\fkT$ ``is'' the theory we aimed to study, and every functor $A: \clL_{\fkT} \to \Set$ with the property that $A([n+m]) \cong A[n] \times A[m]$ concretely realises via its image a representation of $\fkT$ in $\Set$.

More concretely, there is a ``theory of groups''. Such a theory determines a graph $G_{\cate{Grp}}$ built in such a way to generate a category $\clL=\clL_{\cate{Grp}}$ with finite products. \emph{Models} of the theory of groups are functors $\clL \to \Set$ uniquely determined by the image of the ``generating object'' $[1]$ (the set $G=G[1]$ is the underlying set, or the \emph{carrier} of the algebraic structure in study; in our \autoref{unialg} the carries is just the first member of the pair $(A,f^A)$); the request that $G$ is a product preserving functor entails that if $\clL$ is a theory and $G: \clL \to \Set$ one of its models, we must have $G[n]=G^n = G \times G \times\dots\times G$, and thus each function symbol $f: [n]\to [1]$ describing an abstract operation on $G$ receives an interpretation as a concrete function $f: G^n \to G$.

Until now, we interpreted our theory $\clL$ in sets; but we could have chosen a different category $\clC$ at no additiona cost, provided $\clC$ was endowed with finite products, in order to speak of the object $A^n = A\times \cdots\times A$ for all $n: \bbN$. In this fashion, we obtain the $\clC$-models of $\clL$, instead of its $\Set$-models: formally, and conceptually, the difference is all there.

Yet, the freedom to disengage language and meaning visibly has deep consequences: suddenly, and quite miracolously, we are allowed to speak of groups internal to the category of sets, i.e. functors $\clL_{\cate{Grp}} \to \Set$, topological groups, i.e. functors $\clL_{\cate{Grp}} \to \cate{Top}$ (so multiplication and inversion are continuous maps \emph{by this very choice}, without additional requests); we can treat monoids in the category of $R$-modules, i.e. $R$-algebras \cite[]{}, and monoids in the category of posets \cite{} (i.e. quantales \cite{}) all on the same conceptual ground.

\subsubsection{The r\^ole of toposes}
Among many different Cartesian categories in which we can interpret a given theory $\clL$, toposes play a special r\^ole; this is mostly due to the fact that the \emph{internal language} every topos carries (in the sense of \autoref{da_lang}) is quite expressive.

To every theory $\clL$ one can associate a category, called the \emph{free topos} $\clE(\clL)$ on the theory (see \cite{lambek1988introduction}), such that there is a natural bijection between the $\clF$-models of $\clL$ and (a suitable choice of) morphisms of toposes\footnote{We refrain to enter the details of the definition of a morphism of toposes, but we glimpse at the definition: given two toposes $\clE,\clF$ a morphism $(f^*,f_*): \clE \to clF$ consists of a pair of \emph{adjoint} functors (see \cite[3]{Bor1}) $f^*: \clE \leftrightarrows \clF: f_*$ with the property that $f^*$ commutes with finite limits (see \cite[2.8.2]{Bor1})} $\clE(\clL) \to \clF$:
\[\cate{Mod}(\clL, \clF) \cong \hom(\clE(\clL), \clF).\]
In the present subsection we analyse how the construction of models of $\clL$ behaves when the semantics takes value in a category of presheaves.

Let's start stating a plain tautology, that still works as blatant motivation for our interest in toposes opposed as more general categories for our semantics. Sets can be canonically identified with the category $[1,\Set]$, so models of $\clL$ are tautologically identified to its $[1,\Set]$-models. It is then quite natural to wonder what $\clL$-models become when the semantics is taken in more general functor categories like $[C,\Set]$. This generalisation is compelling to our discussion: in case $C$ is a discrete category, we get back the well-known category of variable sets $\Set/C$ of \autoref{variabbo_set}.

Now, it turns out that $[C,\Set]$-model for an algebraic theory $\clL$, defined as functors $\clL \to [C,Set]$ preserving finite products, correspond precisely to functors $C\to \Set$ such that each $Fc$ is a $\clL$-model: this gives rise to the following ``commutative property'' for semantic interpretation:
\begin{quote}
	$\clL$-models in $[C,\Set]$ are precisely those models $C \to \Set$ that take value in the subcategory $\cate{Mod}_{\clL}(\Set)$ of models for $\clL$. In other words we can ``shift'' the $\cate{Mod}(-)$ construction in and out $[C,\Set]$ at our will:
	\[
		\cate{Mod}_{\clL(\Omega)}([C,\Set]) \cong [C, \cate{Mod}_{\clL(\Omega)}(\Set)]
	\]
\end{quote}
As the reader can see, the procedure of interpreting a given ``theory'' inside an abstract finitely complete category $\clK$ is something that is only possible when the wrd theory is interpreted as a category, and when a model of the theory as a functor. This discipline goes under many names: the one we will employ, i.e. \emph{categorical}, or \emph{functorial}, semantics \cite{lawvere1963functorial}, \emph{internalisation} of structures, \emph{categorical algebra}.

The internalisation paradigm sketched above suggests how ``small'' mathematicians often happily develop their mathematics without ever exiting a single (large) finitely complete category $\clK$, without even suspecting the presence of models for their theories outside $\clK$. To a category theorist, ``groups'' as abstract structures behave similarly to the disciples of the sect of the Phoenix \cite{fenix}: ``the name by which they are known to the world
is not the same as the one they themselves pronounce.'' They are a different, deeper structure than the one intended by their users.

By leaving the somewhat unsatisfying picture that ``all categories are small'' and by fixing a semantic universe like $Set$), \emph{each} category works as a world in which one can speak mathematical language (i.e. ``study models for the theory of $\Omega$-structures'' as long as $\Omega$ runs over all possible theories).

As already hinted in \autoref{} categories exhibit a double nature: they are the theories we want to study, but they also are the \emph{places} where we want to realise those theories;looking from high enough, there is plenty of other places where one can move, other than the category $\Set$ of sets and functions (whose existence is, at the best of our knowledge, consequence of a postulate; we similarly posit that there exists a model for \theory{ETAC}). Small categories model theories, it has a \emph{syntax}, in that they describe a relational structure using compositionality; but large categories offer a way to interpret the syntax, so being a \emph{semantics}. A large relational structure is fixed once and for all, lying on the background, in which all other relational structures are interpreted.

It is nearly impossible to underestimate the profundity of this disengagement: synta and semantics, once separated and given a limited ground of action, acquire their meaning.\footnote{Of course, this is dialectical opposition at its pinnacle, and by no means a sterile approach to category theory; see \cite{lawvere1996unity} for a visionary account of how ``dialectical philosophy can be modeled mathematically''.}

More technically, in our \autoref{da_lang} we recall how this perspective allows to intepret different kinds of logics in different kinds of categories: such an approac leads very far, to the purported equivalence between different flavours of logic\emph{s} and different classes of categories; the particular shape of semantics that you can interpret in $K$ is no more, no less than a reflection of the nice categorical properties of $K$ (e.g., having finite co/limits, nice choices of factorisation systems \cite[5.5]{Bor1}, \cite{FK}, a subobject classifier; or the property of the posets $\cate{Sub}(A)$ of subobjects of an object $A$ of being a complete, modular, distributive lattice\dots --in light of \autoref{da_lang}, this last property has to do with the internal logic of the category: propositions are the set, or rather the \emph{type}, of ``elements'' for which they are ``true''; and in nice cases (like e.g. in toposes), they are also arrows with codomain a suitable \emph{type of truth values} $\Omega$).
\subsubsection{Categories are universes of discourse}
Somehow, the previous section posits that category theory as a whole is ``bigger than the mathematics itself'', and it works as one of its foundations; a category is a totality where all mathematics can be re-enacted; in this perspective, \theory{ETAC} works as a metalanguage in which we develop our approach. This is not very far from current mathematical practice, and in particular from Mac Lane's point of view:
\begin{quote}
	We can [\dots\unkern] ``ordinary'' mathematics as carried out exclusively within [a universe] $U$ (i.e. on elements of $U$) while $U$ itself and sets formed from $U$ are to be used for the construction of the desired large categories.\hfill \cite[I.6]{McL}
\end{quote}
since the unique large category we posit is essentially ``the universe''. But this approach goes further, as it posits that \emph{mathematical theories are in themselves mathematical objects}, and as such, subject to the same analysis we perform on the object of which those theories speak about.%  L'idea è che \emph{le teorie matematiche sono a loro volta oggetti matematici}, e in quanto tali sono passibili dello stesso studio di cui sono passibili gli oggetti di cui quelle teorie parlano.

All in all, our main claim is that categories are of some use in clarifying a few philosophical problems, especially for what concerns the objects of discourse of ontology. The advantages of this approach are already visible in mathematics; we are trying to export them in the current philosophical debate.

Among many, we record
\begin{itemize}
	\item the possibility of reading theories in terms of relations: this allows to suspend our ontological committment on the nature of objects: they are given, but embedded in a relational structure. This relational structure, i.e. a category modeling the ontology in study is the object of our discourse
	\item a sharper, more precise, and less time-consuming conceptualisation process, which becomes a purely \emph{context-dependent} process, i.e. a function of the aforementioned relational structure.
\end{itemize}
Of course, a relational structure as mentioned above is nothing but a certain kind of category: we then posit that \emph{an ontology is a category}.

Of course, as extremist as it may seem, our position fits into an already developed open debate; for example, these are the opening lines of a paper by J.P. Marquis:
\begin{quote}
	[...] \emph{to be} is \emph{to be related}, and the ``essence'' of an ``entity'' is given by its relations to its ``environment''
	\hfill \cite{Marquis1997}
\end{quote}
Such a point of view is only acceptable when its fruitfulness has been determined by clarifying and formalising the relationship between an ontology regarded as ``all that there is'', and a (large) category as ``the place we inhabit''.

\begin{italian}
Sappiamo che nella storia della \CT l'accettazione dell'evidenza dei risultati è avvenuta a prescindere da un dibattito intorno alla scarsa precisione e coerenza logica degli stessi. Le categorie sono nate come strumento concettuale e, senza preoccuparsi delle sottigliezze della ricerca fondazionale, hanno catturato efficacemente tutte le nozioni della matematica moderna, rivelandosi utili e feconde. Nostro claim è che si riveleranno tali anche con le usuali nozioni metafisiche. Si tratta di adottare, in fondo, una prospettiva pragmatica:
\begin{quote}
	that structural mathematics is characterized as an activity by a treatment of things as if one were dealing with structures. From the pragmatist viewpoint, we do not know much more about structures than how to deal with them, after all. \hfill \cite{kromer2007tool}
\end{quote}
La ``traduzione'' dei problemi dell'ontologia nel linguaggio di \CT permette di manipolare meglio nozioni (non solo, come si sa, matematiche) ma metamatematiche e metafisiche, e ci dota di un approccio più compatto e di una visione più ``leggera'' e occamista delle questioni vertenti su oggetti e esistenza. Non giustifichiamo questo approccio a priori ma ne testimoniamo la fecondità già provata in letteratura\footnote{Cf. Mt7,16, giusto per ingraziarsi i severiniani.}, soprattutto paragonata a quella degli approcci set-theoretic (di cui già è informata la totalità delle ontologie formali).

In \CT possiamo ``tradurre'' i problemi classici dell'ontologia, fornire modelli entro i quali formularne meglio presupposti e domande, evidenziare ciò che è banale conseguenza degli assiomi di quel modello e ciò che non lo è, risolverli e, in alcuni casi, dissolverli, rivelandone la natura figmentale. Si tratta di fornire un \emph{ambiente} ben definito nel quale questioni ritenute oggetto di dibattito filosofico possano illuminarsi in modi nuovi o scomparire. E questo non per qualche perverso istinto riduzionistico, ma per poterne parlare in termini efficaci e nel linguaggio adatto a inquadrarli: tentare, con gli strumenti più avanzati e raffinati dell'astrazione matematica, di rispondere a delle domande, produrre conoscenza, e non solo dibattito; inscrivere antiche o recenti questioni in un nuovo paradigma, volto a superare e al contempo far avanzare la ricerca.

Come ogni paradigma lo dotiamo di una sintassi con la quale ``nominare'' concetti e dare definizioni, e di una semantica che produca modelli, e quindi contesti, entro i quali ``guardare'' le teorie; questa sintassi e questa semantica non ce le inventiamo: sono già nella matematica e da lì le preleviamo.
\subsection{Towards an Operative Ontology}


Riassumiamo i vantaggi del fare ontologia usando la teoria delle categorie:
\begin{itemize}
	\item Prima di tutto in questo modo ontologia, ci si permetta la battuta, la si \emph{fa} effettivamente. Vale a dire, come mostreremo nel resto del lavoro e in altri successivi, si affrontano di petto le questioni e le si risolvono. Stiamo perciò suggerendo un approccio \emph{problem solving}
	\item Lo strutturalismo ``debole'' implicito in questa visione è da noi mantenuto solo a livello metateorico, e consente di guardare alle relazioni tra oggetti all'interno delle teorie. Anche qui, l'approccio relazionale è assunto in quanto proficuo in senso pratico, operativo, e - come vedremo al termine del paragrafo - non implica l'adesione incondizionata a uno strutturalismo ``forte''.
	\item Come conseguenza fondamentale del punto precedente nessun concetto viene studiato in senso ``assoluto'' (qualunque cosa ciò significhi) ma relativamente al contesto in cui opera, e alla teoria che stiamo adottando per definirlo.
\end{itemize}

Come si è detto in 1.2 l'uso di questi strumenti concettuali ha aiutato la pratica matematica e ha involontariamente ispirato una visione epistemologica, e poi ontologica, della disciplina, vale a dire dei suoi oggetti di studio. A coloro che obiettano che bisogna prima sapere \textit{cos'è} una struttura prima di lavorare con essa noi rispondiamo, con Kr\"omer, che
\begin{quote}
	this reproach is empty and one tries to explain the clearer by the more obscure when giving priority to ontology in such situations [...]. Structure occurs in the dealing with something and does
	not exist independently of this dealing. \cite{kromer2007tool}
\end{quote}
Memori delle osservazioni di Carnap [nota \dots], non riteniamo che questo approccio ``operativo'' all'ontologia (che non è puro \textit{problem solving} ma anche chiarificazione concettuale) implichi necessariamente l'adesione incondizionata ad uno strutturalismo filosofico integrale - o a sue varianti specifiche come la teoria \textbf{ROS} -, esattamente come abbiamo visto non avvenire nel passaggio dalla \textit{structural mathematics} allo strutturalismo vero e proprio (o al bourbakismo). La sua importanza è principalmente metodologica. (\textit{Au contraire} risulta necessario per chi appoggia posizioni strutturaliste al di fuori della matematica cominciare a fare ontologia in termini categoriali, nelle modalità qui indicate).
\end{italian}
 \subsubsection{Existence: Persistence of Identity?} 
 Ontology rests upon the principle of identity: it is this very principle that our category-theoretic approach aims to unhinge. 
 \begin{italian}
 E tuttavia formalizzare il concetto intuitivo di identità si rivela una questione estremamente spinosa: cosa significa che \emph{due cose sono, invece, una} è un problema che ci arrovella fin da quando otteniamo la ragione e la parola; ciò perché il problema è tanto elementare quanto sfuggente: l'unica maniera in cui possiamo esibire ragionamento certo è il calcolo; del resto, se la sintassi non vede che l'uguaglianza in senso più stretto possibile, la prassi deve diventare in fretta capace di una maggiore elasticità: per un istante ho postulato che ci fossero ``due'' cose, non una. E non è forse questo a renderle due? E questa terza cosa che le distingue, è davvero diversa da entrambe?

 Usciti dalle nebbie delle speculazioni tradizionali, i filosofi a cavallo tra '800 e '900 si sono posti questo complicato problema: due vie sono possibili: la risposta fregeana \cite{} per cui ''$x$ esiste'' se e solo se ''$x$ è identico a qualcosa [banalmente, a sè stesso]'' alla soluzione logica quineana per cui ''essere è essere il valore di una variabile [vincolata]''. 
 
 Il primo approccio si rivela comodo solo se si è realisti concettuali, ma è scarsamente informativo (cosa è l'uguaglianza tra due oggetti apparentemente diversi? Quel qualcosa che ci ha fatto sospettare lo fossero non è abbastanza a renderli tali?); coinvolge la nozione di identità, ed anzi scarica su questa l'onere di definire esistenza; questa non è la strada giusta: l'identità di fatto non esiste, perché ogni identità è un'identificazione, e ogni uguaglianza una relazione di equivalenza; la questione è tuttavia complessa abbastanza da dedicarvi un lavoro a parte (in effetti, due \cite{,}) di questo polittico.
 
 Il secondo tipo di approccio ha ispirato l'interesse per la nozione di \emph{ontological committment} (l'insieme di assunti che ``si danno per scontati'' quando si parla di ontologia, o se ne partecipa) e per la conseguente definizione di ontologia (di una teoria) come ''dominio di oggetti su cui variano i quantificatori'' (cf. \cite{}): namely una teoria qualsiasi è impegnata sulle entità su cui variano i quantificatori dei suoi enunciati. 
 	
 	Vedremo in \autoref{metaon} che la concezione Quineana fitta nella nostra visione di ontologia categoriale come conseguenza di una internalizzazione. 
 	
 	Esiste una terza via, meno diffusa in letteratura ma decisamente meno opinabile: definire l'esistenza tramite la persistenza nel tempo. Diciamo che ''$x$ esiste'' se e solo se ''$x$ è identico a sè stesso in ogni frame temporale $\la T,<\ra$'', dove $T$ è un insieme non vuoto di istanti e $<$ una relazione binaria in $T$ (e la relazione di esistenza in $T$ è allora una relazione $(x,t)\mapsto x\mathrel{\tilde\in} t$; ``$x$ esiste in $T$'' se per ogni $t : T$ si ha $x\mathrel{\tilde\in} t$).
 	
 	Come si vede questa definizione cattura una nozione intuitiva di esistenza, impiegando sia l'identità (con tutti i problemi che essa comporta) che il tempo, o meglio una opportuna logica temporale nella quale far "persistere" le entità. (E' facile scrivere cosa significa la relazione ``$x$ esiste in $T$'' in termini di (L)TL)
 	
	 Uno dei risultati di questo paper è che possiamo definire l'esistenza in maniera altrettanto intuitiva, senza riferirsi all'identità, né a un frame temporale; in effetti, fornendo un concetto più generale, dentro il quale si troveranno anche gli altri.
	 
	 Porremo la questione nel seguente modo: ciò che è variabile relativo ad $x$ è il grado, o \emph{forza} della sua esistenza; l'esistenza ``classica'' è esistenza in massimo grado nel linguaggio interno del topos che fa da Universo (per noi, un universo borgesiano); lì saremo in grado di indicare il ''grado'' di esistenza degli oggetti che lo abitano, senza presupporre di muoverci attraverso istanti di tempo (come potrebbe suggerire l'esempio delle monete) o punti dello spazio (come la freccia).%, oppure in altri inimmaginabili modi, magari attraverso diverse dimensioni. 
	 A seconda della struttura del dominio possiamo scegliere la logica da utilizzare e così il ''contesto'' più adatto all'intuizione che abbiamo dell'universo nel linguaggio naturale. 
 	
 	La persistenza nel tempo non è perciò rimossa dalla descrizione, o negata; piuttosto, inglobata. E' un sottocaso del modello generale, precisamente quello in cui la proposizione sull'esistenza dell'oggetto è vera con forza 1 i tutti gli istanti (cfr. \autoref{}, nota 16).
 	
 	Va da sè che, molto informalmente, l'esistenza in this conception non è altro che la modalità di "presenza" degli oggetti all'interno di un modello. E' quindi letteralmente ciò che possiamo \emph{farci} con gli oggetti, come possiamo porci rispetto a essi. 
 	
 	Non è solo una nozione operativa di esistenza, vicina peraltro al nostro senso comune: per noi le cose esistono se possiamo toccarle, vederle, postularle (quando invisibili) in base a ipotesi su rapporti di causazione che hanno con entità osservabili, descriverle, contarle, utilizzarle; e ciò è indipendente dal \emph{come} esse esistano. E di conseguenza è anche una visione epistemica. 
 	
 	Bypassata la domanda ''se le cose esistono'', in base alle nostre scelte metateoriche e fondazionali, l'esistenza riguarda i modi tramite i quali le cose entrano in relazione l'una con l'altra. Questo ci permette, ecco i vantaggi dell'ontologia categoriale, di sfruttare la visione strutturalista e poter descrivere e render cogenti non solo il nostro mondo ma realtà distanti, come Tl\"on, fornite di un'ontologia diversa. 
 	
 	In linguaggio matematico questo "modo di comportarsi delle cose" non è altro che lo studio delle relazioni tra gli oggetti di una categoria.
 	
 	Infine sarà, l'esistenza, anche context-dependent; varierà a seconda del linguaggio interno della teoria (cioè della categoria) nella quale operiamo. E questo permette di formalizzare la banale intuizione, spesso sfuggente agli occhi degli ontologi, per cui l'esistenza in un mondo come Tl\"on sarà presumibilmente diversa dalla nostra. L'ovvia constatazione che cambiando ontologia cambia il concetto di ''esistere'' diventa qui una cosneguenza automatica dell'uso di un linguaggio matematico. 
 \end{italian}


% Ontology rests upon the principle of identity. It is this very principle that here we aim to unhinge.

% Cosa significa che \emph{due cose sono, invece, una} è un problema che ci arrovella fin da quando otteniamo la ragione e la parola; ciò perché il problema è tanto elementare quanto sfuggente: l'unica maniera in cui possiamo esibire ragionamento certo è il calcolo; del resto, se la sintassi non vede che l'uguaglianza in senso più stretto possibile, la prassi deve diventare in fretta capace di una maggiore elasticità: per un istante ho postulato che ci fossero ``due'' cose, non una. E non è forse questo a renderle due? E questa terza cosa che le distingue, è davvero diversa da entrambe?

% Ciò che risulterà evidente è che la nozione di identità è, appunto, \emph{context-dependent}, e questo risolve il dibattito che, almeno da [Geach,\dots], impegna i filosofi, in merito alla sua eventuale relatività ontologica.

% Sostituire la nozione classica vuol dire rivedere i fondamenti dell'ontologia: la stessa nozione centrale di \emph{esistenza}, nella tradizione quineana, si definisce tramite la nozione, più (illusoriamente) semplice e primitiva, di identità: ``\textit{A esiste}'' sse ``\textit{qualcosa è identico ad A}'' (questo ``qualcosa'' è una variabile vincolata ad un quantificatore esistenziale).

% Il motivo per cui riteniamo di dover agire in questa direzione è dovuto agli innumerevoli problemi che la nozione di identità classica (criterio di Leibniz, sue varianti, ma anche definizioni successive in sua vece) si porta dietro, rilevati da molti filosofi nel corso del secolo passato, e non superabili rifiutando solo l'identità leibniziana o abbracciando la prospettiva mereologica (ma ci riserviamo di parlarne in lavori successivi). Il motivo per cui molti filosofi, pur sottolineando l'inadeguatezza della nozione, non hanno mai seriamente proposto di sostituirla, crediamo sia per mancanza sia di un linguaggio adatto sia di alternative teoriche rigorose in esso espimibili.
% L'\textit{Homotopy Type Theory}, e più in generale la \CT, rispondono a queste esigenze, e attuano quella sostituzione finora mai realizzata. \fo{Esempio di come HoTT sia strutturalista nella metateoria è che in HoTT si può definire cos'è una categoria, e dopo averlo fatto si scopre che la sintassi interpreta ``essere uguali'' per due oggetti/termini di tipo categoria $A,B: \mathcal C$ come ``essere isomorfi'', o come ``essere omotopi'' quando $\mathcal C$ viene interpretato come un tipo \emph{di omotopia}, $A,B: \mathcal C$ come punti di questo spazio, e $A=_{\mathcal C}B: \sf Prop$ come un'omotopia tra $A$ e $B$. Vale anche la pena notare che in teoria dei tipi la relatività ontologica della nozione di identità \emph{è un assunto}: ogni tipo $X$ è equipaggiato con una ``sua'' nozione di identità $=_X$ che è locale, è ``la sua'' e nulla a a che vedere, a priori, con $=_Y$ per un altro tipo $Y$. Ogni uguaglianza istanziata per termini di tipo diverso è quindi inammissibile \emph{nel linguaggio} ancor prima che nella semantica.}\endfo



