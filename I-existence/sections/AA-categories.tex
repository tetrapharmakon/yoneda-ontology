\section{Category theory}
\epigraph{El atanor está apagado -repitió- y están llenos de polvo los alambiques. En este tramo
	de mi larga jornada uso de otros instrumentos.}{\cite{arena}}
\subsection{Fundamentals of CT}
Throughout the paper we employ standard basic category-theoretic terminology, and thus we refrain from giving a self contained exposition of elementary definitions. Instead, we rely on famous and wide-spread sources like \cite{Bor1,Bor2,McL,riehlcontext,leinster2014basic,simmons2011introduction}.

Precise references for the basic definitions can be found
\begin{itemize}
	\item for the definition of category, functor, and natural transformation, in \cite[1.2.1]{Bor1}, \cite[I.2]{McL}, \cite[1.2.2]{Bor1}, \cite[I.3]{McL}, \cite[1.3.1]{Bor1}.
	\item The Yoneda lemma is stated as \cite[1.3.3]{Bor1}, \cite[III.2]{McL}.
	\item For the definition of co/limit and adjunction, in \cite[2.6.2]{Bor1}, \cite[III.3]{McL}, \cite[2.6.6]{Bor1}, \cite[III.4]{McL} (consider in particular the definitions of \emph{pullback}, \emph{product}, \emph{terminal object}).
	\item For the definition of accessible and locally presentable category in \cite[5.3.1]{Bor2}, \cite[5.2.1]{Bor2}, \cite{Adamek1994}.%, \cite[]{}.
	\item Basic facts about ordinal and cardinal numbers can be found in \cite{kunen}; another comprehensive reference on basic and non-basic set theory is \cite{jech2013set}.
	\item The standard source for Lawvere functorial semantics is Lawvere's PhD thesis \cite{lawvere1963functorial}; more modern accounts are \cite{hyland2007category}.
	\item Standard references for topos theory are \cite{mac1992sheaves,JohnstonePT}. See in particular \cite[VI.5]{mac1992sheaves} and \cite[5.4]{JohnstonePT} for what concerns the Mitchell-Bénabou language of a topos.
\end{itemize}
% \subsection{Toposes}\leavevmode
% For us, an \emph{ordinal number} will be an isomorphism class of well\hyp{}ordered sets, and a \emph{cardinal number} is any ordinal which is not in bijection with a smaller ordinal. Every set $X$ admits a unique \emph{cardinality}, i.e. a least ordinal $\kappa$ with a bijection $\kappa \cong X$ such that there are no bijections from a smaller ordinal. We freely employ results that depend on the axiom of choice when needed. A cardinal $\kappa$ is \emph{regular} if no set of cardinality $\kappa$ is the union of fewer than $\kappa$ sets of cardinality less than $\kappa$; all cardinals in the following subsection are assumed regular without further mention.

% Let $\kappa$ be a cardinal; we say that a category $\clA$ is $\kappa$\hyp{}filtered if for every category $\clJ\in\Qat_{<\kappa}$ with less than $\kappa$ objects, $\clA$ is injective with respect to the cone completion $\clJ\to \clJ^\rhd$; this means that every diagram
% \[
% 	\vcenter{\xymatrix{
% 			\clJ\ar[d]\ar[r]^D & \clA \\
% 			\clJ^\rhd\ar@{.>}[ur]_{\bar D}
% 		}}
% \]
% has a dotted filler $\bar D : \clJ^\rhd \to \clA$.

% We say that a category $\clC$ admits filtered colimits if for every filtered category $\clA$ and every diagram $D : \clA \to \clC$, the colimit $\colim D$ exists as an object of $\clC$. Of course, whenever an ordinal $\alpha$ is regarded as a category, it is a filtered category, so a category that admits all $\kappa$\hyp{}filtered colimits admits all colimits of chains
% \[
% 	C_0 \to C_1 \to \cdots \to C_\alpha \to\cdots
% \]
% with less than $\kappa$ terms. A useful, completely elementary result is that the existence of colimits over all ordinals less than $\kappa$ implies the existence of $\kappa$\hyp{}filtered colimits; this relies on the fact that every filtered category $\clA$ admits a cofinal functor (see \cite[§2.11]{Bor1}) from an ordinal $\alpha_\clA$.

% We say that a functor $F : \clA \to \clB$ \emph{commutes with} or \emph{preserves} filtered colimits if whenever $\clJ$ is a filtered category, $D : \clJ \to \clA$ is a diagram with colimit $L=\colim_\clJ D_j$, then $F(L)$ is the colimit of the composition $F\circ D$. Anoter common name for such an $F$ is a \emph{finitary} functor, or a functor \emph{with rank $\omega$}.
% \begin{definition}\label{accepre}
% 	Let $\clC$ be a category;
% 	\begin{itemize}
% 		\item We say that $\clC$ is \emph{$\kappa$\hyp{}accessible} if it admits $\kappa$\hyp{}filtered colimits, and if it has a \emph{small} subcategory $\clS\subset \clA$ of $\kappa$\hyp{}presentable objects such that every $A\in\clA$ is a $\kappa$\hyp{}filtered colimit of objects in $\clS$.
% 		\item We say that $\clC$ is \emph{(locally) $\kappa$\hyp{}presentable} if it is accessible and cocomplete.
% 	\end{itemize}
% 	The theory of presentable and accessible categories is a cornerstone of \emph{categorical logic}, i.e. of the translation of model theory into the language of category theory.

% 	Accessible and presentable categories admit \emph{representation theorems}:
% 	\begin{itemize}
% 		\item A category $\clC$ is accessible if and only if it is equivalent to the ind\hyp{}completion $\text{Ind}_\kappa(\clS)$ of a small category, i.e. to the completion of a small category $\clS$ under  $\kappa$\hyp{}filtered colimits;
% 		\item A category $\clC$ is presentable if and only if it is a full reflective subcategory of a category of presheaves $i : \clC \to \Qat(\clS^\op,\Set)$, such that the embedding functor $i$ commutes with $\kappa$\hyp{}filtered colimts.
% 	\end{itemize}
% \end{definition}
% All categories of usual algebraic structures are accessible, and they are locally presentable as soon as they are cocomplete; an example of a category which is $\aleph_1$\hyp{}presentable but not $\aleph_0$\hyp{}presentable: the category of metric spaces and short maps.

% We now glance at \emph{topos theory}:
% \begin{definition}\label{eletop}
% 	An \emph{elementary topos} is a category $\clE$
% 	\begin{itemize}
% 		\item which is \emph{Cartesian closed}, i.e. each functor $\firstblank\times A$ has a right adjoint $[A, \firstblank]$;
% 		\item having a \emph{subobject classifier}, i.e. an object $\Omega\in\clE$ such that the functor $\text{Sub} : \clE^\op\to \Set$ sending $A$ into the set of isomorphism classes of monomorphisms $\var{U}{A}$ is representable by the object $\Omega$.
% 	\end{itemize}
% 	The natural bijection $\clE(A,\Omega)\cong\text{Sub}(A)$ is obtained pulling back the monomorphism $U\subseteq A$ along a \emph{universal arrow} $t : 1\to \Omega$, as in the diagram
% 	\[
% 		\vcenter{\xymatrix{
% 				U \pb\ar[r]\ar[d]& 1\ar[d]^t \\
% 				A \ar[r]_{\chi_U}& \Omega
% 			}}
% 	\]
% 	so, the bijection is induced by the map $\var{U}{A}\mapsto \chi_U$.
% \end{definition}
% \begin{definition}\label{grotop}
% 	A \emph{Grothendieck topos} is an elementary topos that, in addition, is locally presentable.
% \end{definition}
% The well-known \emph{Giraud theorem} gives a proof for the difficult implication of the following \emph{recognition principle} for Grothendieck toposes:
% \begin{theorem}
% 	Let $\clE$ be a category; then $\clE$ is a Grothendieck topos if and only if it is a left exact reflection of a category $\Qat(\clA^\op,\Set)$ of presheaves on a small category $\clA$.
% \end{theorem}
% (recall that a \emph{left exact reflection} of $\clC$ is a reflective subcategory $\clR\hookrightarrow \clC$ such that the reflector $r : \clC \to \clR$ preserves finite limits. It is a reasonably easy exercise to prove that a left exact reflection of a Grothendieck topos is again a Grothendieck topos; Giraud proved that all Grothendieck toposes arise this way.)
% \subsection{A little primer on algebraic theories}\label{funsemanzi}
% The scope of this short subsection is to collect a reasonably self-contained account of functorial semantics. It is unrealistic to aim at such a big target as providing a complete account of it in a single appendix; the reader is warmly invited to parallel their study with more classical references as \cite{lawvere1963functorial}.
% \begin{definition}[Lawvere theory]\label{lo_tiori}
% 	A \emph{Lawvere theory} is a category having objects the natural numbers, and where the sum on natural numbers has the universal property of a categorical product, as defined e.g. in \cite[2.1.4]{Bor1}.
% \end{definition}
% Let us denote $[n]$ the typical object of $\clL$. Unwinding the definition, we deduce that in a Lawvere theory $\clL$ the sum of natural numbers $[n+m]$ is equipped with two morphisms $[n] \leftarrow [n+m] \to [m]$ exhibiting the universal property of the product.

% Every Lawvere theory comes equipped with a functor $p : \cate{Fin}^\op \to \clC$ that is the identity on objects and preserves finite products. A convenient shorthand to refer to the Lawvere theory $\clL$ is thus as the functor $p$, or as the pair $(p,\clL)$.
% \begin{definition}
% 	The category $\cate{Law}$ of Lawvere theories has objects the Lawvere theories, understood as functors $p : \cate{Fin}^\op \to \clL$, and morphisms the functors $h :  p\to q$ such that the triangle
% 	\[\xymatrix{
% 			& \cate{Fin}^\op \ar[dr]^q \ar[dl]_p & \\
% 			\clL \ar[rr]_h && \clM
% 		}\]
% 	is commutative. It is evident that $\cate{Law}$ is the subcategory of the undercategory $\cate{Fin^\op}/\Qat$ (see e.g. \cite[I.6]{McL}for a precise definition) made by those functors that preserve finite products.
% \end{definition}
% \begin{remark}
% 	The category $\textsf{Law}$ has no nonidentity 2-cells; this is a consequence of the fact that a natural transformation $\alpha : h \To k$ that makes the triangle ``commute'', i.e. $\alpha * p = \id_q$ must be the identity on all objects.
% \end{remark}
% \begin{example}[The trivial theories]
% 	The category $\cate{Fin}^\op$, opposite to the category of finite sets and functions, is the initial object in the category  $\cate{Law}$; the terminal object is constructed as follows: the category $\clT$ has objects the natural numbers, and $\clT([n],[m])=\{*\}$ for every $n,m: \bbN$. It is evident that given this definition, there is a unique identity-on-objects functor $\clL \to \clT$ for every other Lawvere theory $(p,\clL)$.
% \end{example}
% \begin{definition}[Model of a Lawvere theory]
% 	A \emph{model} for a Lawvere theory $(p ,\clL)$ consists of a product-preserving functor $L : \clL \to \Set$. The subcategory $[\clL,\Set]_\times \subset [\clL, \Set]$ of models of the theory $\clL$ is \emph{full}, i.e. a morphism of models $L \to L'$ consists of a natural transformation $\alpha : L \Rightarrow L'$ between the two functors.
% \end{definition}
% Observe that the mere request that $\alpha : L \to L'$ is a natural transformation between product preserving functors means that $\alpha_{[n]} : L[n] \to L'[n]$ coincides with the product $(\alpha_{[1]})^n : L[1]^n \to L'[1]^n$.
% \begin{proposition}
% 	Let $p : \cate{Fin}^\op\to \clL$ be a Lawvere theory. Then, the following conditions are equivalent for a functor $L : \clL \to \Set$:
% 	\begin{itemize}
% 		\item $L$ is a model for the Lawvere theory $(p,\clL)$;
% 		\item the composition $L\circ p : \cate{Fin}^\op \to \Set$ preserves finite products;
% 		\item there exists a set $A$ such that $L\circ p = \Set(j[n],A)$.
% 	\end{itemize}
% \end{proposition}
% \begin{corollary}\label{da_pull}
% 	The square
% 	\[
% 		\xymatrix{
% 			\cate{Mod}(p,\clL) \ar[d]_u \ar[r]^r & [\clL,\Set]\ar[d]^{p^*} \\
% 			\Set \ar[r]_{N_j} & [\cate{Fin}^\op,\Set]
% 		}
% 	\]
% 	is a pullback of categories. The functor $u$ is completely determined by the fact that $u(L) = L[1]$, $r$ is an inclusion, and $N_j(A) = \lambda F.\Set(F,A)$ is the functor induced by the inclusion $j : \cate{Fin} \subset \Set$.
% \end{corollary}
% \begin{corollary}
% 	The category of models $\cate{Mod}(p,\clL)$ of a Lawvere theory is a locally presentable, accessibly embedded, complete and cocomplete subcategory of $[\clL,\Set]$. Moreover, the forgetful functor $u : \cate{Mod}(p,\clL) \to \Set$ of \autoref{da_pull} is \emph{monadic} in the sense of \cite[4.4.1]{Bor2}. A complete proof of all these facts is in \cite[3.4.5]{Bor2}, \cite[3.9.1]{Bor2}, \cite[5.2.2.a]{Bor2}. A terse argument goes as follows: the functors $p^*, N_j$ are accessible right adjoints between locally presentable categories; therefore, so is the pullback diagram: $r$ is a fully faithful, accessible right adjoint, and $u$ is an accessible right adjoint, that moreover reflects isomorphisms. It can be directly proved that it preserves the colimits of split coequalizers, and thus the adjunction $f \dashv u$ is monadic by \cite[4.4.4]{Bor2}.
% \end{corollary}
% The last technical remark that we collect sheds a light on the discorso prolisso in \autoref{as_places}: the models of a thelory $\clL$ interpreted in the category of models of a theory $M$ correspond to the models of a theory $\clL \otimes \clM$, defined by a suitable universal property:
% \[
% 	\cate{Mod}(\clL\otimes \clM,\Set)  \cong
% 	\cate{Mod}(\clL, \cate{Mod}(\clM,\Set))  \cong
% 	\cate{Mod}(\clM, \cate{Mod}(\clL,\Set)).
% \]
% \begin{definition}
% 	Given two theories $\clL$ and $\clM$ it is possible to construct a new theory called the \emph{tensor product} $\clL \otimes \clM$; this new theory can be characterized by the following universal property: the models of $\clL \otimes \clM$ consist of the category of $\clL$-models interpreted in the category of $\clM$-models or, equivalently (and this is remarkable) of $\clM$-models interpreted in the category of $\clL$-models.
% \end{definition}
% \begin{theorem}
% 	(\cite[4.6.2]{Bor2}) There is an equivalence between the following two categories:
% 	\begin{itemize}
% 		\item $\cate{Law}$, regarded as a non-full subcategory of the category $\textsf{Fin}^\op/\Qat$, i.e. where a morphism of Lawvere theories consists of a functor $h : \clL \to \clM$ that preserves finite products;
% 		\item \emph{finitary} monads, i.e. those monads that preserve filtered colimits, and morphisms of monads in the sense of \cite[4.5.8]{Bor2}.
% 	\end{itemize}
% \end{theorem}
% \begin{proof}
% 	The proof goes as follows: given a Lawvere theory $p : \textsf{Fin}^\op \to \clL$, we have shown that the functor $u : \cate{Mod}(p) \to \Set$ in the pullback square \autoref{da_pull} has a left adjoint $f : \Set \to \cate{Mod}(p)$; the composition $uf$ is thus a monad on $\Set$. This is functorial, when a morphism of monads is defined
% \end{proof}
% \begin{definition}
% 	\label{internista}
% 	Let  $\clC$ be a category with finite limits. An \emph{internal category} $\underline{A}$ consists of a pair $(A_0, A_1)$ of objects of $\clC$, endowed with morphisms $s,t,c,i$ as in the following diagram
% 	\[
% 		\xymatrix{
% 		A_1 \times_{A_0} A_1 \ar[r]|-c \ar@<6pt>[r]^-{p_0}  \ar@<-6pt>[r]_-{p_1} & A_1 \ar@<6pt>[r]^s  \ar@<-6pt>[r]_t & A_0 \ar[l]|i
% 		}
% 	\]
% 	where $A_1 \times_{A_0} A_1$ is obtained pulling back $s,t$. These data must satisfy the following axioms, resembling the category axioms:
% 	\begin{enumtag}{ic}
% 		\item $i$ equalises the pair $(s,t)$, i.e. $s\circ i= t\circ i$, and this composition makes the identity $1_{A_0}$;
% 		\item $t\circ p_1 = t\circ c$ and $s\circ p_0 = s \circ c$;
% 		\item $c\circ\la 1_{A_1},i\circ s\ra = c\circ\la i\circ t, 1_{A_1}\ra$, and this composition makes the identity $1_{A_1}$;
% 		\item $c$ is \emph{associative}, i.e. $c \circ (1_{A_1}\times_{A_0} c)= c \circ (c \times_{A_0} 1_{A_1})$.
% 	\end{enumtag}
% \end{definition}
% These axioms are meant to re-enact the definition of category, as an abstract ``object of objects'' and ``object of morphisms'', endowed with maps $s,t: A_1 \to A_0$ sending every morphism to its \emph{source} and \emph{target}, with an \emph{identity selector} map $i : A_0 \to A_1$ and with a composition partial binary operation $c : A_1 \times A_1 \to A_1$ whose domain is the object of \emph{composable} arrows. Cf. \cite[Ch. 8]{Bor1} for a torough discussion.
% It has to be noted however that the theory of categories is \emph{not} an algebraic theory in the sense of our \autoref{lo_tiori}, and this just because the composition operation is only \emph{partially} defined over the domain of composable arrows, i.e. on the pullback $A_1\times_{A_0} A_1$. Theories whose syntax allows for a number of partially defined relation and function symbol are termed \emph{essentially algebraic}. See \cite{Adamek1994} for a torough discussion of the topic.