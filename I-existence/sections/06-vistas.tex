\section{Vistas on ontologies}\label{vistas}
\epigraph{¿No basta un solo término repetido para desbaratar y confundir la serie del tiempo? ¿Los fervorosos que se entregan a una línea de Shakespeare no son, literalmente, Shakespeare?}{\cite{confutacion}}
Looking at our \autoref{sec_coins} it's surprising how mathematically consistent Borges' universe becomes when it is described through category theory. As we already said, this has to be regarded as an experiment, a literary \emph{divertissement}, and a test-bench for our main claim, that as abstract as it may seem ontology has a quantitative content.

There is clearly no point to just analyse Borges' literary work: we can go further. The present section has this purpose, while sketching somelong term goals our work can aspire to become. We begin with a more academic discussion of classical idealism \emph{à la} Berkeley; of course, in light of our \autoref{incendiata}, this can still be linked to Tl\"on's universe, as Borges has always been fascinated by, and mocked, classical idealism (see \autoref{confutacion}).

To this, it follows a more wide-ranging discourse; surely a less substantiated one, but we aim at laying a foundation for future work, ours or others', in a research track that we feel is fertile and promising.
\subsection{Answers to the idealist}\label{berkelei}
Let us reconsider \autoref{bla}; although in passing, we mentioned famous Berkeley's view of perception as a bundle of stimulation incapable to cohere. Let's say clearly that endeavouring on such a wide ground as classical idealism isn't the purpose of our work; yet, in his novel Borges regards Tl\"on's language and philosophy as a concrete realisation of Berkeley's theory of knowledge.

\begin{italian}
Within topos theory, è scorretto dire che $p$ è falsa in ``assoluto'', ma che semmai è ``falsa con forza $t$'' dove $t \in I$. Il berkeleyano vorrebbe ottenere (una volta accettato che le monete esistano solo un po' in alcuni punti) che $\sum_u p(u,c,d) = (\bot,1)$. 

Non può per costruzione, ma se volesse deve fornire un modello alternativo del paradosso che sia altrettando valido (ci sembra sia difficile comunque non parlare in termini di ``forze'' o loro equivalenti, in ``istanti'' singoli).

Ad essere ancora più pignoli, alla domanda se le monete esistano o meno la risposta corretta è a sua volta una domanda: ``In quale $I$ sei?'' ed essendo $I$ totalmente ordinato ``In quale suo punto vuoi calcolare la forza?''. 

Il punto può essere un riferimento temporale, spaziale o di qualunque altro genere (a seconda della struttura di $U$), l'importante è che l'insieme abbia le proprietà da noi indicate.

Per poter ''parlare'' all'interno del topos, e rispondere alla domanda sull'esistenza, è necessario prima fornire i "dati" (osservatore, subset di $C$, giorno della settimana) e poi possiamo assegnare un valore di verità dotato di forza. 

Sostituire nozioni classiche (identità, persistenza-nel-tempo) con nozioni precise suggerite dal linguaggio che proponiamo, vuol dire perciò modificare anche il linguaggio naturale. Quando l'idealista afferma che negli istanti in cui chiude gli occhi il tavolo davanti a lui non esiste più, sta anch'egli parlando di ``forza'' nel senso qui definito: lo deve solo esplicitare. Si mette in un universo (un topos) ``non-classico'' con più valori di verità e deve semplicemente calcolare \emph{qual è} il valore di $p$.

L'ampliamento dei valori di verità \footnote{Ad esser precisi: $\{0,1\}$ è ``fisso''; la varietà di risposte è data dalla scelta di $I$ e dalla struttura di $U$} si traduce in un ampliamento delle risposte possibili da dare alla domanda sulle monete. Non indichiamo quella ``corretta'' ma il topos in cui trovarla, il contesto di discorso dal quale non si può prescindere, intuitivamente un contesto fuzzy, che evita le sterili dicotomie in cui era impantanato il dibattito da qualche secolo.

Da notare (come anticipato in 1.4) che abbiamo potuto modellizzare il paradosso senza impiegare un framework di logica temporale; è lecito interpretare gli elementi di $I$ come istanti, ma non necessario. La descrizione in termini temporali è un sottocaso del modello più generale che abbiamo fornito. Persino un presentista o l'idealista coerente di Borges (che dalle tesi berkeleyane deduce l'inesistenza del tempo) può stabilire la verità di $p$ senza abiurare alla sua posizione ontologica, ma deve poi accettare il risultato del calcolo.

Altro vantaggio che l'idealista, e non solo, potrebbe apprezzare è dato dalla densità di $I$. Proprietà che permette, nel caso di disaccordo tra due interlocutori $X$ e $Y$ sul grado di esistenza della moneta, (per via principalmente della dipendenza funzionale delle proposizioni dalle configurazioni), di trovare una forza intermedia compresa tra la forza assegnata da $X$ e quella assegnata da $Y$. In un certo senso questo è un modo di formalizzare l'intersoggettività all'interno di Tl\"on. 



La posizione metafisica che emerge dal lavoro sembra una via di mezzo tra empirismo e idealismo. 

La critica storica dell'idealismo al metodo scientifico è l'uso di prove indirette di esistenza (come le dimostrazioni non costruttive che non piacciono agli intuizionisti). 

Nella nostra visione fuzzy dell'esistenza si può assumere che la barca invisibile che nessuno vede, se non perché l'acqua si sposta, non esista, o assumere che la barca ci sia ma con una forza minore del massimo \footnote{Come si è detto in 1.4 è un tentativo di risoluzione epistemica del paradosso. cfr. Quando si parla di ``maggiore è il numero di osservatori più intenso è il grado di esistenza delle monete''}. 

Se si usa questo linguaggio non si può semplicemente affermare che la barca non esiste perché non è osservabile, posizione che non spiega lo spostamento dell'acqua \footnote{Anche il metafisico che non assume avvenga il processo di causazione deve spiegare la relazione (qualunque essa sia, se c'è) o comunque la correlazione tra il fenomeno ``la barca si muove'' (che presuppone l'esistenza della barca) e il fenomeno ``l'acqua si sposta'', il quale non è collegato, ad esempio, al fenomeno dell'esistenza delle nuvole}; in breve non ci si può scordare del dettaglio per cui, anche se è lecito pensare che il mondo sparisca quando chiudo gli occhi (per cui posso non assumere che abbia forza di esistenza massima), nondimeno riappare quando li riapro (per cui globalmente non può avere forza 0).

Se la freccia "sparisce" e nessun altro osservatore in un dato istante ne fa esperienza, evidentemente essa esiste con forza minore di 1. Essendo \emph{dentro} Tl\"on questa è una concessione alla visione epistemica dei suoi abitanti, i quali non possono sapere cosa ne è delle monete degli altri osservatori; e in questo senso, non potendo calcolare la forza globale, il modello è inutilizzabile dai tl\"oniani. Possiamo formalizzare adeguatamente la loro ontologia, per poter dire qualcosa in più rispetto a chi vive in essa. 

All'idealista tuttavia possiamo rispondere, se egli parla il linguaggio delle categorie \footnote{Preme comunque notare che prendiamo qui come interlocutore l'idealismo in quanto esempio di metafisica "estrema", e perchè era negli obiettivi del racconto di Borges fornirne una lettura critica e alternativa, ma non è questo il claim del paper. }.    

\end{italian}

\subsection{By their fruits you will know them}\label{frutti}
% Possiamo dire, figurativamente, che l'ontologia è il modo che usiamo per ``apparecchiare'' il mondo dentro un determinato linguaggio. Questo è ciò che gli ontologi possono dire, il luogo nel quale lavorare.
% Dove sia il linguaggio e ``qual è il principio che ispira la scelta di un determinato linguaggio (e quindi di un'ontologia)' ci sembra una domanda di interesse antropologico, o forse neurofisiologico, ma è fuori dalla portata di ciò che l'ontologia operativa può dire \footnote{\de{Dovrebbe essere una domanda metaontologica ma qui la posizione che assumo è che la nostra metaontologia dice che la risposta non rientra nel campo delle sue possibilità}\endde} (se non ricascando nelle pastoie dell'impostazione classica, priva degli ``ambienti' adatti dentro i quali parlare delle cose).

% Rapporto tra ``esistenza'' e ``comportamento nel modello'': rinunciamo alla caratterizzazione classica di esistenza? qualunque essa sia (presumibilmente quella orribile quineana). Sembra di sì. Quindi
% \begin{center}
% 	``$x$ esiste (nel modello $\mathcal{M}$)'' := ``insieme dei comportamenti di $x$ in $\mathcal{M}$''
% \end{center}
% \de{va precisata meglio ma è una def relativizzata ad un contesto (come promettevamo in 1.4) e ci permette di parlare di esistenza senza ricorrere a identità e persistenza nel tempo. Molto più agevole}\endde].

% Questo è quello che intendiamo con ``esistenza'' a livello metaontologico. La nozione poi si specifica a seconda del mondo (topos) che analizziamo in quel momento. Ad esempio in Tl\"on esistere significa avere una forza strettamente maggiore di 0.

The main point of our paper can be summarised very concisely: an ``ontology'' is a category $\clO$, inside which ``Being unravels''. Every existence theory shall be reported, and is relative, to a fixed ontology $\clO$, the ``world we live in''; such existence theory coincides with the internal language $\clL(\clO)$ of the ontology/category (from now on we employ the two terms as synonyms), in the syntax\hyp{}semantics adjunction of \cite{}.

So determined, the internal language of an ontology $\clO$ is the collection of ``things that can be said'' about the elements of the ontology.

If, now, ontology is the study of Being, and if we are structuralist in the metatheory (cf. \autoref{sec_intro}), we cannot know beings but through their attributes. Secretly, this is \emph{Yoneda lemma} (cf. \cite[1.3.3]{Bor1}), the statement that the totality of modes of understanding a ``thing'' $X$ coincides with the totality of modes your language allows you to probe $X$. Things do not exist out of an ontology; objects do not exist out of a category; types do not exist out of a type theory. In relational structures, objects are known via their modes of interaction with other objects, and these are modeled as morphisms $U \to X$; Yoneda lemma posits that we shall ``know objects by their morphisms'': the object $X:\clC$ coincides with the totality of all morphisms $U\to X$, organised in a coherent bundle (a functor $\clC(\firstblank,X) : \clC \to \Set$).

All in all, an ontology is a mode of understanding the attributes of Being: a category, be it in an Aristotelic or in a structuralist sense. As a consequence \emph{meta}ontology, i.e. the totality of such ways of understanding, must coincide with the general theory of such individual modes: with \emph{category theory}.

We should say no more on the matter: everything else pertains to \emph{meta}ontology. 

Indeed, questions as ``where is language'' and what general principles inspire it might have an anthropological or even neurophysiological answer; not an ontological one. Or at least, not without paying a price: operative ontology as sketched here has limits. It can't speak of Being out of the one ignited by itself. (read as: category theory \emph{has limits}: it cannot speak efficiently of objects out of a fixed universe of discourse. Implicit: category theory also has merits.)
\subsection{Metaontology} \label{metaon}
The gist of our \autoref{bla} is that $X,Y,Z$ can't assess the existence of the coins classically; they just have access to partial information allowing neither a global statement of existence on the set $C$ of coins lost by $X$, nor an unbiased claim about the meaning thereof. Coins are untouched as a global lost conglomerate, yet their strength of existence is very likely to change \emph{locally}.

% Gli osservatori di Tl\"on non riescono a dire nulla sulle monete degli altri, neanche con strength. Il nostro modello è inutilizzabile \emph{dentro} Tl\"on; ed è dovuto alla diversità di linguaggio dell'abitante di Tlon. Sospendiamo il giudizio su cosa venga ``prima', se l'ontologia o il linguaggio (gli autori hanno posizioni divergenti ma la ricerca assume pragmaticamente il secondo) ma senza dubbio essi sono strettamente correlati.

This begs the important questions of to which extent language determines ontology. And to which extent it constrains its expressive power? To which extent the inhabitants of Tl\"on fail to see what is exactly ``a topos further'' (persistence of existence through time). To what extent \emph{we} fail to see that\dots

% Possiamo imprecisamente definire l'ontologia come ciò che gli oggetti \emph{sono} \footnote{cioè come si comportano nel modello} all'interno di un linguaggio.

% Prendiamo lo stato dell'arte dell'ontologia in filosofia analitica: l'apostolo quineano di turno ti dice che ``ontologia'' è il dominio di oggetti su cui variano i quantificatori \cite{?}. Questa def sufficientemente operativa dice ``di meno'' di quello che vorrebbe l'impostazione classica (così come la nostra; ma, come sempre, è l'unico modo per essere precisi); ma dice ``di meno'' anche della nostra. Se ontologia è il comportamento degli oggetti nel modello, è ``irrilevante'' cosa io ritenga di poter inserire nel dominio (struttura di $U$) (il realista concettuale dice che sono oggetti anche $\top$ e $\bot$ \footnote{O meglio sono oggetti ciò che $\top$ e $\bot$ denotano} mentre l'empirista puro no) (e così via di chiacchiere su ``ce lo metto o no il tempo nel mio modello?''): il dibattito può rimanere sul modello. In questa maniera abbiamo un agone intersoggettivo di attribuzione dei significati (vabbè, si fa per ridere: cfr. Umberto Eco, ovunque) dentro il quale \emph{fare} ontologia (anzi, ontologie).

Far from claiming we resolved such metaontological issues, here we make an hopefully clarifying statement: according to Quine and his school \cite{}, ontology can be defined as the ``domain over which [logical, or natural language's] quantifiers run''. This is not wrong; in fact, it is perfectly compatible with our views. But we work at a raised complexity level, in the following sense. In our framework, ontology \emph{à la} Quine still is a category, because (cf. \autoref{quantifezzi}) a quantifier can be described as a certain specific kind of functor
\[\forall,\exists : PX \to PY\]
between powersets regarded as \emph{internal} categories of our ambient ontology $\clO$; in light of this, it's easy to imagine this pattern to continue: if Quine calls metaontology what we call ontology, i.e. the metacategory grounding his propositional calculus is a single ambient category \emph{among many}, we live ``one universe higher'' in the cumulative hierarchy of foundations and meta-foundations: our ontology possesses a higher dimensionality, and harbours Quinean theory as an internal structure (see \autoref{internista} for the definition of an internal category). So, it shall exist, \emph{somewhere} -at least in some secret way, hidden from the comprehension of men- a language that calls ontology what we call metaontology (and thus a language that declassifies Quine's to a sub-ontology -whatever this means).

Finding the bottom of this tower of turtles is the aim of the track of research, of very ancient tradition, within which we want to insert the present work.

Of course our work does not make a single comment on how, if, and when, this ambitious foundational goal can be achieved. We just find remarkable that framing Quine's definition in this bigger picture is not far from the so-called practice of ``negative thinking'' in category theory: negative thinking is the belief that a high-dimensional/complex entity can be understood by means of the analogy with its low-dimensional/complexity counterparts; see \cite{nlab:category-order,nlab:neg-think} for a minimal introduction to the principle, \cite{baez2010lectures} for a practical introduction, and see \cite{gowers2007} for what Gowers calls ``backwards generalisation''.


\begin{italian}
	Riassumendo: per ''ontologia'' in generale intendiamo una theory of existence, namely una teoria dei modi in cui gli oggetti si relazionano gli uni agli altri per essere compresi da noi (Yoneda Lemma risultato centrale). 
	Più propriamente è allora il caso di parlare di (operative) ontologies, al plurale, che sono le categorie col loro linguaggio interno. Vale a dire (1.3) gli ''ambienti'' dentro i quali studiare le relazioni tra gli oggetti. Come sappiamo da 1.2 questa concezione relazionale dell'ontologia è ispirata dall'essere strutturalisti, ma solo nella metateoria. %per cui la nostra ontologia, come si è detto sopra, è già una metaontologia%
	
	Qualunque discorso su come gerarchizzare le varie ontologie, sulle relazioni tra esse, è quindi un discorso sulle categorie, ed è la nostra (informale) metaontologia qui accennata. 
	
	Le grandi domande esistenziali non pertengono, perciò, a nessuno di questi domini. The reasons why we implemented this extreme reduction to this historical matters are fully explained in 1.4 and are essentially practical. 
	
	Ogni altro discorso crediamo debba partire da qui.   
\end{italian}
\subsection{Conclusion}
The circle apparently closes on, and motivates better, our initial foundational choice: categories and their theory correspond 1:1 to ontology and its theory. However, countless important issues remain open: in what sense this is satisfying? In what sense the scope of our analysis is not limited by this choice? What's his foundation? Is this a faithful way to describe such an elusive concept as ``Being''?

None of these questions is naive; in fact, each legitimately pertains to metaontology, and has no definitive answer. More or less our stance is as follows: approaching problems in ontology with a reasonable amount of mathematical knowledge is fruitful. Yet, the problem of what is a foundation for that mathematics remains (fortunately!) wide open; it pertains to metaontology, whose ambitious effort is to clarify ``what there is''. We believe the philosophers' job to work in synergy with quantitative knowledge, approaching the issue with complementary tools.