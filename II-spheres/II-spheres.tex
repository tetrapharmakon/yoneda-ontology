\documentclass{amsart}

\makeatletter
\def\@settitle{\begin{center}%
  \baselineskip14\p@\relax
  \bfseries
  \uppercasenonmath\@title
  \@title
  \ifx\@subtitle\@empty\else
     \\[1ex]\uppercasenonmath\@subtitle
     \footnotesize\mdseries\@subtitle
  \fi
  \end{center}%
}
\def\subtitle#1{\gdef\@subtitle{#1}}
\def\@subtitle{}
\makeatother

\author{Fouche and Denta}
\title{Categorical ontology II}
\subtitle{A geometric look on identity}
\usepackage{fouche}
\begin{document}
\maketitle
\begin{abstract}
  [\dots]

  As both a test-bench for the language and proof of concept we offer a possible homotopy-theoretic approach to Black's ancient two spheres' problem. The interlocutors of Black's imaginary dialogue inhabit respectively an Euclidean world and an affine world, and this affects their perception of the ``two'' spheres.
\end{abstract}
\section{Introduction}
\section{Two exactly similar spheres}
\epigraph{This is my rifle. There are many like it, but this one is mine.}{\emph{The Rifleman's Creed}}
La \emph{aporia delle due sfere identiche} è stata enunciata per la prima volta da M. Black in \cite{} per incrinare la validità acritica con cui il principio di identità degli indiscernibili viene adottato in ontologia.

Tale ``paradosso'' si riassume circa come segue: si supponga che l'universo sia costituito solo da due sfere perfettamente identiche, con medesime proprietà formali e materiali. Esse risulterebbero assolutamente indistinguibili. Per determinarne l'identità numerica dovrebbe bastare l'unica proprietà che presumibilmente non hanno in comune, cioé la locazione spaziale. Tuttavia:

"[A:] Ognuna delle due sfere sarà certamente diversa dall'altra per essere a una certa distanza da quell'altra, ma a distanza nulla da sé stessa; vale a dire essa avrà almeno una relazione con sé stessa - l'essere a distanza nulla da o l'essere nello stesso luogo di - che non ha con l'altra. (...)

B: (...) Ognuna avrà la caratteristica relazionale di essere a una distanza di 2 miglia, diciamo, dal centro di una sfera di un diametro, ecc. E ognuna avrà la caratteristica relazionale (...) di essere nello stesso luogo di se stessa. le due sono simili per questo riguardo come per chiunque altro.

A: Ma ogni sfera occupa un luogo diverso; e questo varrà a distinguerle

B: Ciò suona come se voi pensaste che i luoghi abbiano una qualche esistenza indipendente (...). [qui il ragionamento sui luoghi é di natura relazionale, e credo sia inevitabile nella nostra prospettiva ragionare cos\'e] Dire che due sfere sono in luoghi diversi equivale appunto a dire che c'é una certa distanza tra 2 sfere e abbiamo visto che questo non varrà a distinguerle. Ognuna é a una certa distanza - invero la stessa distanza - dall'altra

Dopo una discussione sofferta, e non povera di argomenti, i due interlocutori non arrivano a un accordo; il lettore è invitato a notare che si possono produrre molti argomenti in sfavore della buona posizione di questa aporia, dai più ingenui (se marchio una delle due sfere con un segno rosso, sull'altra ne appare uno identico? Se sì, ``le sfere'' sono una sola; se no, erano due.), ai più intricati.

Lo scopo della sezione finale di questo articolo è utilizzare il linguaggio introdotto nelle precedenti per ascrivere, definitivamente e in modo incontrovertibile, questo paradosso a un uso scorretto del linguaggio.


In breve: sia nel paradosso delle nove monete di rame, sia nell'esperimento mentale di Black, esistono una categoria $\mathcal C_{\text{Bor}}$ e una categoria $\mathcal C_\text{Bla}$ con oportune proprietà, per cui il ragionamento paradossale (\emph{le nove monete non sono esistite nella notte tra mercoledì e giovedì}, e \emph{le due sfere non sono due perché non vi è modo di distinguerle}) svaniscono.

In particolare, nel nostro linguaggio, in particolare in quello introdotto in \ref{}, la natura figmentale dell'aporia delle due sfere si riassume così: un interlocutore, $A$, pensa le sfere in uno spazio euclideo; l'altro interlocutore, $B$, le pensa in uno spazio affine.

A generare il paradosso è l'incapacità dei due (e di Black) di notare che la conversazione avviene sì riguardo agli stessi oggetti, ma \emph{in categorie diverse}, o meglio, in diversi action groupoid associati ad uno stesso spazio geometrico $\mathbb S$: per $A$, nella categoria dove c'è una mappa tra le due sfere quando hanno lo stesso centro e lo stesso raggio --sono quindi ``euclideanamente uguali''; per $B$, nella categoria dove c'è un isomorfismo tra le due sfere quando tra loro c'è una trasformazione \emph{affine}.

Allora, semplicemente, $A$ vede diverse cose che $B$ vede uguali, perché $A$ opera un quoziente rispetto a una relazione di equivalenza più fine di quella che opera $B$. Questo esempio da solo introduce praticamente tutti gli strumenti di cui rendiamo conto in questo lavoro:
\begin{itemize}
	\item rende lampante il fatto che una categoria è determinata dai suoi morfismi molto più che dai suoi oggetti; $A$ e $B$ sono convinti di parlare delle stesse cose, perché $|\mathcal C_A|=|\mathcal C_B|$, ma la seconda ha molti più morfismi, e quindi può far collassare molti più oggetti rispetto alla sua nozione di ``uguaglianza categoriale'' (dobbiamo stabilire una volta per tutte dei nomi per le cose cui vogliamo riferirci).
	\item mette in chiaro in un esempio specifico il fatto che la relazione di identità non è primitiva; è contestuale; è indotta dalla nozione di uguaglianza che ogni categoria si porta dietro. Ed è binaria, perché è un caso particolare di una nozione di omotopia (meglio: è la nozione di isomorfismo in una categoria dell'omotopia, che resta indotta dalla nozione di omotopia di cui $\mathcal C$ era dotata. No identity without homotopy, appunto).
	\item Questo è né più né meno che il contenuto strutturale del programma di Klein: le figure dello spazio restano sempre le stesse, raccolte nella classe $\mathbb S$; ciò che cambia è il gruppo(ide) $\mathcal G$ che facciamo agire sulle figure e di cui prendiamo la categoria delle azioni $\mathbb S /\!\!/\mathcal G$; ``il mondo'' con la sua nozione di identità è l'insieme di Bishop
	      \[\Big\{ \pi_0(\mathbb S/\!\!/\mathcal G) \mid [x] \equiv [y] \iff \exists \varphi : x \to y\Big\}\]
	\item Permette di prevedere cosa penserebbe del problema un ipotetico interlocutore $C$ che vivesse in un mondo proiettivo (una sfera e un'iperbole sono lo stesso oggetto) o in uno topologico (una sfera e un cubo sono lo stesso oggetto), etc. Ovviamente si può ragionare anche viceversa: basta trovare un contesto dove due oggetti spazialmente coincidenti non sono ``lo stesso'' oggetto. La sfida è che bisogna uscire da fondazione insiemistica (l'assioma di estensionalità è precisamente l'asserto per cui due cose che hanno gli stessi punti sono una cosa sola); invece di impelagarsi nella costruzione precisa di un controesempio (ce ne sono, in fondazioni type-teoretiche: per esempio i tipi $\mathbb N : \sf Ord$ e $\mathbb N : \sf Mon$ sono diversi -hanno diverse proprietà universali), penso sia meglio dire semplicemente: ``ecco, vedete? Il principio di identità è  a tutti gli effetti l'assioma di estensionalità: voi trovate assurdo che a dichiarare uguali due enti non sia sufficiente dire che hanno gli stessi atomi. Eppure questo è possibile: voi classici non avete la nozione di identità più forte in assoluto, state nel mezzo; e la metateoria risultante dal prendere una nozione di identità più stringente è semmai ancor piu interessante di quella cui siete abituati. Quindi la nozione di uguaglianza non è \emph{immutabile, scolpita nel tempo}\footnote{Non sei l'unico che apprezza il cinema impegnato}. E' un assioma: se lo vuoi lo prendi, altrimenti non lo prendi. E se non lo prendi si apre un mondo, perché ammetti che due cose sono uguali anche quando non hanno gli stessi punti, oppure che \emph{non basta} avere gli stessi punti per essere uguali, nello stesso senso in cui non è sufficiente che due insiemi siano in biiezione affinché siano omeomorfi, o isomorfi come gruppi, etc.'' Chiaramente, questo lastrica la strada al lemma di Yoneda, che ``è l'assioma di estensionalità in CT'' (lo introdurrei esattamente con queste parole, e spenderei una parte congrua del lavoro a darne una introduzione \emph{ad usum delphini} che invece che di trasformazioni naturali e gruppianellicampi parli di ontologia e di estensionalità).
\end{itemize}
Qual è il punto di tutto questo? Che una sfera è molte cose: è uno spazio topologico, è una varietà algebrica, è una varietà differenziale, è un gruppo di Lie (non in dimensione 2, ma per esempio in dimensione 1 o 3), è una superficie di Riemann, è questo ed è quell'altro. $A$ e $B$ nell'esempio di Black parlano uno di una sfera $S^2_A$ che appartiene a una categoria $A$, e l'altro di una sfera $S^2_B$ che appartiene a una categoria $B$. Il paradosso nasce quindi da un uso scorretto e impreciso del formalismo.

\section{The actual work}
% Da dove iniziamo allora; per esempio dal ricostruire il nucleo del programma di Erlangen. Sicché
% 	\begin{itemize}
% 		\item Definizione di gruppo(ide);
% 		\item Definizione di azione di un gruppo(ide) su un insieme/categoria;
% 		\item Definzione dell'action groupoid e delle sue componenti connesse: il $\pi_0$ dell'action groupoid di una certa azione è l'oggetto di studio della geometria (breve digressione su come questa geometria sia una geometria discreta, i.e. algebra lineare senza topologia, o su campi finiti, così come continua, i.e. geodiff su varietà, e azioni di gruppi di Lie che sono gruppi strutturali di fibrati);
% 		\item Esempi vari dalla geometria classica: qual è il gruppo, qual è l'insieme su cui agisce, nei vari tipi di geometria:
% 		      \[proiettiva \supset affine \supset lineare \supset euclidea\]
% 		\item Come questo, a partire dalla geometria, diventa una teoria ontologica: l'identità è sempre una identità nel $\pi_0$ di un action groupoid.
% 		\item Allora, l'esempio delle sfere di Black è quello di due bambini che litigano per un giocattolo che chiamano in modi diversi.
% 	\end{itemize}
\begin{definition}[Groupoid]
A \emph{groupoid} is a category where all morphisms are invertible; a \emph{group} is a groupoid with a single object $\star$. Endomorphisms of $G$ are the \emph{elements} of the group; the set of elements of $G$ becomes a monoid under composition.
\end{definition}
\begin{definition}[Actegory]
Let $\clC$ be a category, and $\clG$ be a groupoid; a (left) actegory structure on $\clC$ consists of a functor
\[
\lambda : \clG \times \clC \to \clC
\]
such that
\begin{itemize}
  \item
\end{itemize}
This amounts to the choice of an \emph{action} of the groupoid $\clG$ over the categry $\clC$, i.e. of a strong monoidal functor $\lambda : \clG \to [\clC,\clC]$.
\end{definition}
\begin{remark}
  A \emph{right} actegory structure consists not of a functor $\rho : \clC \times \clG \to \clC$ such that axioms ... are satisfied; instead it consists of a left actegory structure with respect to the groupoid $\clG^\op$. This complies with the fact that in classical algebra a right group action is a left action of the opposite group $G^\op$.
\end{remark}
\begin{notation}
  We denote a left $\clG$-actegory structure on $\clC$ as $\lhd_\clG$ or simply $\lhd$; so, the action of $G \in\clG$ on $C \in \clC$ is denoted $G\lhd C$, and similarly for $\varphi\lhd f$. In particular, the action of a group $G$ on $\clC$ is completely determined by a homomorphism from $G$ to the group of self-equivalences of $\clC$.
\end{notation}
\begin{definition}[Action groupoid of an actegory]
Let $(\clC,\lhd_\clG)$ be an actegory; we define the \emph{action groupoid} of $(\clC,\lhd_\clG)$ as the category having the same objects of $\clC$, and where there is a morphism $X\to G\lhd X$ for every $X\in\clC, G\in\clG$.
\end{definition}
\begin{example}
un po' di esempi
\end{example}
\begin{example}
di action groupoid
\end{example}
\begin{example}
nella natura
\end{example}
\begin{definition}[Connected components functor]
Every directed graph $\underline G$, with set of vertices $V$ and set of edges $E$, defines a quotient set of $V$ by the equivalence relation generated by $A\approx B$ if there is an arrow $A\to B$; the symmetric and transitive closure of this relation yields a quotient $V/R$ that is usually denoted as the set of \emph{connected components} $\pi_0(\underline G)$. This is a functor ${\sf Gph} \to \Set$, and if we now regard a category $\clC$ as a mere directed graph we obtain a well-defined set $\pi_0(\clC)$ of connected components of a category.
\end{definition}
\subsection{Klein program, with groupoids}
\begin{remark}[In the discrete as well as in the continuous]
Breve digressione su come questa geometria sia una geometria discreta, i.e. algebra lineare senza topologia, o su campi finiti, così come continua, i.e. geodiff su varietà, e azioni di gruppi di Lie che sono gruppi strutturali di fibrati
\end{remark}
\begin{example}
  A roundup of examples from all geometry
\[proiettiva \supset affine \supset lineare \supset euclidea\]
\end{example}
Effects on ontology: identity always is an identity in the $\pi_0$ of a certain action groupoid.

\section{Two exactly similar spheres, again}
\hrulefill

\subsubsection{Existence: Persistence of Identity?} 
 Ontology rests upon the principle of identity: it is this very principle that our category-theoretic approach aims to unhinge. 
 \begin{italian}
 E tuttavia formalizzare il concetto intuitivo di identità si rivela una questione estremamente spinosa: cosa significa che \emph{due cose sono, invece, una} è un problema che ci arrovella fin da quando otteniamo la ragione e la parola; ciò perché il problema è tanto elementare quanto sfuggente: l'unica maniera in cui possiamo esibire ragionamento certo è il calcolo; del resto, se la sintassi non vede che l'uguaglianza in senso più stretto possibile, la prassi deve diventare in fretta capace di una maggiore elasticità: per un istante ho postulato che ci fossero ``due'' cose, non una. E non è forse questo a renderle due? E questa terza cosa che le distingue, è davvero diversa da entrambe?

 Usciti dalle nebbie delle speculazioni tradizionali, i filosofi a cavallo tra '800 e '900 si sono posti questo complicato problema: due vie sono possibili: la risposta fregeana \cite{} per cui ''$x$ esiste'' se e solo se ''$x$ è identico a qualcosa [banalmente, a sè stesso]'' alla soluzione logica quineana per cui ''essere è essere il valore di una variabile [vincolata]''. 
 
 Il primo approccio si rivela comodo solo se si è realisti concettuali, ma è scarsamente informativo (cosa è l'uguaglianza tra due oggetti apparentemente diversi? Quel qualcosa che ci ha fatto sospettare lo fossero non è abbastanza a renderli tali?); coinvolge la nozione di identità, ed anzi scarica su questa l'onere di definire esistenza; questa non è la strada giusta: l'identità di fatto non esiste, perché ogni identità è un'identificazione, e ogni uguaglianza una relazione di equivalenza; la questione è tuttavia complessa abbastanza da dedicarvi un lavoro a parte (in effetti, due \cite{,}) di questo polittico.
 
 Il secondo tipo di approccio ha ispirato l'interesse per la nozione di \emph{ontological committment} (l'insieme di assunti che ``si danno per scontati'' quando si parla di ontologia, o se ne partecipa) e per la conseguente definizione di ontologia (di una teoria) come ''dominio di oggetti su cui variano i quantificatori'' (cf. \cite{}): namely una teoria qualsiasi è impegnata sulle entità su cui variano i quantificatori dei suoi enunciati. 
 	
 	Vedremo in \autoref{metaon} che la concezione Quineana fitta nella nostra visione di ontologia categoriale come conseguenza di una internalizzazione. 
 	
 	Esiste una terza via, meno diffusa in letteratura ma decisamente meno opinabile: definire l'esistenza tramite la persistenza nel tempo. Diciamo che ''$x$ esiste'' se e solo se ''$x$ è identico a sè stesso in ogni frame temporale $\la T,<\ra$'', dove $T$ è un insieme non vuoto di istanti e $<$ una relazione binaria in $T$ (e la relazione di esistenza in $T$ è allora una relazione $(x,t)\mapsto x\mathrel{\tilde\in} t$; ``$x$ esiste in $T$'' se per ogni $t : T$ si ha $x\mathrel{\tilde\in} t$).
 	
 	Come si vede questa definizione cattura una nozione intuitiva di esistenza, impiegando sia l'identità (con tutti i problemi che essa comporta) che il tempo, o meglio una opportuna logica temporale nella quale far "persistere" le entità. (E' facile scrivere cosa significa la relazione ``$x$ esiste in $T$'' in termini di (L)TL)
 	
	 Uno dei risultati di questo paper è che possiamo definire l'esistenza in maniera altrettanto intuitiva, senza riferirsi all'identità, né a un frame temporale; in effetti, fornendo un concetto più generale, dentro il quale si troveranno anche gli altri.
	 
	 Porremo la questione nel seguente modo: ciò che è variabile relativo ad $x$ è il grado, o \emph{forza} della sua esistenza; l'esistenza ``classica'' è esistenza in massimo grado nel linguaggio interno del topos che fa da Universo (per noi, un universo borgesiano); lì saremo in grado di indicare il ''grado'' di esistenza degli oggetti che lo abitano, senza presupporre di muoverci attraverso istanti di tempo (come potrebbe suggerire l'esempio delle monete) o punti dello spazio (come la freccia).
	  
	 A seconda della struttura del dominio possiamo scegliere la logica da utilizzare e così il ''contesto'' più adatto all'intuizione che abbiamo dell'universo nel linguaggio naturale. 
 	
 	La persistenza nel tempo non è perciò rimossa dalla descrizione, o negata; piuttosto, inglobata. E' un sottocaso del modello generale, precisamente quello in cui la proposizione sull'esistenza dell'oggetto è vera con forza 1 i tutti gli istanti (cfr. \autoref{}, nota 16).
 	
 	Va da sè che, molto informalmente, l'esistenza in this conception non è altro che la modalità di "presenza" degli oggetti all'interno di un modello. E' quindi letteralmente ciò che possiamo \emph{farci} con gli oggetti, come possiamo porci rispetto a essi. 
 	
 	Non è solo una nozione operativa di esistenza, vicina peraltro al nostro senso comune: per noi le cose esistono se possiamo toccarle, vederle, postularle (quando invisibili) in base a ipotesi su rapporti di causazione che hanno con entità osservabili, descriverle, contarle, utilizzarle; e ciò è indipendente dal \emph{come} esse esistano. E di conseguenza è anche una visione epistemica. 
 	
 	Bypassata la domanda ''se le cose esistono'', in base alle nostre scelte metateoriche e fondazionali, l'esistenza riguarda i modi tramite i quali le cose entrano in relazione l'una con l'altra. Questo ci permette, ecco i vantaggi dell'ontologia categoriale, di sfruttare la visione strutturalista e poter descrivere e render cogenti non solo il nostro mondo ma realtà distanti, come Tl\"on, fornite di un'ontologia diversa. 
 	
 	In linguaggio matematico questo "modo di comportarsi delle cose" non è altro che lo studio delle relazioni tra gli oggetti di una categoria.
 	
 	Infine sarà, l'esistenza, anche context-dependent; varierà a seconda del linguaggio interno della teoria (cioè della categoria) nella quale operiamo. E questo permette di formalizzare la banale intuizione, spesso sfuggente agli occhi degli ontologi, per cui l'esistenza in un mondo come Tl\"on sarà presumibilmente diversa dalla nostra. L'ovvia constatazione che cambiando ontologia cambia il concetto di ''esistere'' diventa qui una cosneguenza automatica dell'uso di un linguaggio matematico. 
 \end{italian}
\end{document}
