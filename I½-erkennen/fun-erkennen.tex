\documentclass[a4paper]{../birkjour}

\usepackage{../fouche-copy}
\RequirePackage{xparse}

\ExplSyntaxOn
\NewDocumentCommand{\makeabbrev}{mmm}
 {
  \yoruk_makeabbrev:nnn { #1 } { #2 } { #3 }
 }

\cs_new_protected:Npn \yoruk_makeabbrev:nnn #1 #2 #3
 {
  \clist_map_inline:nn { #3 }
   {
    \cs_new_protected:cpn { #2 } { #1 { ##1 } }
   }
 }
\ExplSyntaxOff

\makeabbrev{\textbf}{bf#1}{
  a,b,c,d,e,g,h,i,j,k,l,m,n,o,p,q,r,t,u,v,w,x,y,z,%
  A,B,C,D,E,G,H,I,J,K,L,M,N,O,P,Q,R,T,U,V,W,X,Y,Z }
\makeabbrev{\boldsymbol}{bs#1}{%
    a,b,c,d,e,f,g,h,i,j,k,l,m,n,o,p,q,r,s,t,u,v,w,x,y,z,%
    A,B,C,D,E,F,G,H,I,J,K,L,M,N,O,P,Q,R,S,T,U,V,W,X,Y,Z }
\makeabbrev{\mathsf}{sf#1}{
  a,b,c,d,e,f,g,h,i,j,k,l,m,n,o,p,q,r,s,t,u,v,w,x,y,z,%
  A,B,C,D,E,F,G,H,I,J,K,L,M,N,O,P,Q,R,S,T,U,V,W,X,Y,Z }
\makeabbrev{\mathfrak}{fk#1}{
  a,b,c,d,e,f,g,h,j,k,i,l,m,n,o,p,q,r,s,t,u,v,w,x,y,z,%
  A,B,C,D,E,F,G,H,I,J,K,L,M,N,O,P,Q,R,S,T,U,V,W,X,Y,Z }
\makeabbrev{\mathcal}{cl#1}{
  A,B,C,D,E,F,G,H,I,J,K,L,M,N,O,P,Q,R,S,T,U,V,W,X,Y,Z }
\makeabbrev{\mathbb}{bb#1}{
  A,B,C,D,E,F,G,H,I,J,K,L,M,N,O,P,Q,R,S,T,U,V,W,X,Y,Z }
\makeabbrev{\underline}{u#1}{
  a,b,c,d,e,f,g,h,j,k,i,l,m,n,o,p,q,r,s,t,u,v,w,x,y,z,%
  A,B,C,D,E,F,G,H,I,J,K,L,M,N,O,P,Q,R,S,T,U,V,W,X,Y,Z }


\makeatletter
\def\@settitle{\begin{center}%
  \baselineskip14\p@\relax
  \bfseries
  \uppercasenonmath\@title
  \@title
  \ifx\@subtitle\@empty\else
     \\[1ex]\uppercasenonmath\@subtitle
     \footnotesize\mdseries\@subtitle
  \fi
  \end{center}%
}
\def\subtitle#1{\gdef\@subtitle{#1}}
\def\@subtitle{}
\makeatother

\newcommand{\var}[3][]{%
\left[%
\begin{smallmatrix}
  #2            \\%
  #1 \downarrow \\%
  #3%
\end{smallmatrix}%
\right]%
}
%
\newcommand{\cvar}[3]{%
\begin{xsmallmatrix}{0em}%
  & #1         \\ 
  #2 & \downarrow \\ 
  & #3%
\end{xsmallmatrix}%
}

\newcommand{\po}[1][dr]{\save*!/#1+1.5pc/#1:(1,-1)@^{|-}\restore}
\newcommand{\pb}[1][dr]{\save*!/#1-1.5pc/#1:(-1,1)@^{|-}\restore}
\def\theory#1{\textsf{#1}}
\def\CT{\theory{CT}\@\xspace}
\setlength{\epigraphwidth}{.6\textwidth}
\setcounter{tocdepth}{1}
%
\def\ra{\rangle}
\def\la{\langle}
\def\lr#1#2{\la #1,#2\ra}
\def\tr{\textsf{t}}
\newcommand{\true}{\texttt{t}}
\def\id{\text{id}}

%== authors' info
\author{Dario Dentamaro}
\address{ Dario \textsc{Dentamaro}: 
        }
\email{dio@cane.it}
%
\author{Fosco Loregian}
\address{ Fosco \textsc{Loregian}:%
          Tallinn University of Technology,%
          Institute of Cybernetics, Akadeemia tee 15/2,%
          12618 Tallinn, Estonia%
        }
\email{fosco.loregian@taltech.ee}
\email{fosco.loregian@gmail.com}

%== metadata
\title{Categorical ontology I}
\subtitle{Existence}

\usepackage{ xspace
           , proof
           , fontawesome
           , listings
           }

\renewcommand*{\ttdefault}{cmvtt}

\def\ang#1{\langle #1 \rangle}
\def\lsem{\textcolor{gray!60}{[}}
\def\rsem{\textcolor{gray!60}{]}}
\def\trn#1{\lsem #1\rsem}

\def\f#1{\textcolor{red}{#1}}

\newcounter{quad_i}
\newcommand{\qd}[1][]{\ifthenelse{\equal{#1}{}}{\quad}{\setcounter{quad_i}{0}\whiledo{\value{quad_i}<#1}{\quad\stepcounter{quad_i}}}}
\newcommand{\funsp}[2]{(#1 \Rightarrow #2)}

\title{Categorical Ontology: Functorial erkennen}
%== authors' info
% \author{Blinded authors}
%
%
%
\author{Dario \textsc{Dentamaro}}
\address{Università degli Studi di Firenze,\newline Dipartimento di Matematica e Informatica
        }
\email{dario.dentamaro@stud.unifi.it}
% \email{dariodentamaro26@gmail.com}
%
\author{Fosco \textsc{Loregian}}
\address{ Tallinn University of Technology,\newline %
          Institute of Cybernetics, Akadeemia tee 15/2, \newline %
          12618 Tallinn, Estonia }
\email{fosco.loregian@taltech.ee}
% \email{fosco.loregian@gmail.com}


\usepackage{booktabs}
% \usepackage{braket}
%== `braket' confligge con alcune macro già esistenti 
\newcommand{\bk}[1]{\langle #1\rangle}
\begin{document}
\scriptsize
\begin{abstract}
  The present paper approaches ontology and metaontology through mathematics, and more precisely through category theory. We exploit the theory of \emph{elementary toposes} to claim that a satisfying ``theory of existence'', and more at large ontology itself, can both be obtained through category theory. In this perspective, an \emph{ontology} is a mathematical object: it is a category, the universe of discourse in which our mathematics (intended at large, as a theory of knowledge) can be deployed. The \emph{internal language} that all categories possess prescribes the modes of existence for the objects of a fixed ontology/category.

  This approach resembles, but is more general than, fuzzy logics, as most choices of $\clE$ and thus of $\Omega_\clE$ yield nonclassical, many-valued logics.
  
  Framed this way, ontology suddenly becomes more mathematical: a solid corpus of techniques can be used to backup philosophical intuition with a useful, modular language, suitable for a practical foundation. As both a test-bench for our theory, and a literary \emph{divertissement}, we propose a possible category-theoretic solution of Borges' famous paradoxes of Tl\"on's ``nine copper coins'', and of other seemingly paradoxical construction in his literary work. We then delve into the topic with some vistas on our future works.
\end{abstract}
\maketitle

\tableofcontents

\epigraph{Life is the life of the world to come, which a man earns by means of the letters.}{Ozar Eden Ganuz}
\section{Semantical conception of theories}
\subsection{The Two Dictionaries}
Along the XXth century there have been many attempts towards a formal definition of a scientific theory. The \emph{Wiener Kreis}' verificationist paradigm, and Neurath's theory of `protocollar statements', has given the initial input to elaborate a completely semantic framework for scientific theories, and spurred the search for a pan-linguistic vision of philosophy of science \cite{Weinb}.

The formal account in which --among others-- Carnap \cite{carnapfound} provided his notion of `theory' is known in the literature as \emph{syntactical conception of theories} or 'received view' \cite{krause-foundation,krause2011axiomatization,giunti2016}. Albeit the term `semantic' is due to later developments, the field of epistemology that logical neopositivism started can legitimately be labeled a `semantics of theories', because some of its features, if not the underlying ideology, are the same throughout the work of Carnap \cite{carnap56,carnapfound},  Beth \cite{?}, and Suppe \cite{suppe89}.

In this approach, a `theory' becomes a structure $(F_\clL, \clK)$ where $F_\clL$ is a formal system, and $\clK$ the totality of all its interpretations, or \emph{models}. We take this as the starting point of our analysis, originally inspired by Carnapian views, and we separate further $F_\clL$ into two `vocabularies' $(\clT,\clO)$ (intended, in modern terms, as two syntactic categories carved from two first order theories), one of them representing the \emph{pure} or \emph{theoretical} terms, and the other representing the \emph{applied} or \emph{observational} terms. 

In short, for us a ``scientific theory'' is a triple $\bk{(\clT,\clO), \clK}$ whose first two elements form the `underlying logic' $F_\clL=(\clT,\clO)$ and where $\clK$ is a (possibly large) category of models or ``interpretations''. This is not a new idea, as the habit of identifying a sort of mathematical structure, in a way that is independent from the cohort of its syntactic presentations, permeates classical universal algebra since the early work of Lawvere (see also \cite{,} for applications to categorical logic and other disciplines). 

Tangentially, a similar approach has been used to `axiomatise' a notion of evolutionary theory in \cite{biologia}; we also mention the pioneering work of Rosen \cite{} towards the axiomatisation of structures of living systems using category theory.
\begin{definition}[Wiener Circle theory] \cite{krause-foundation}
	A theory $T$ consists of:
	\begin{itemize}
	\item A formal language $\clF_\clL$ formed by a non-logical vocabulary $\overline{\clV}$, which is further divided into two sub-vocabularies: $\clT$ for theoretical terms and $\clO$ for observational terms.\footnote{There are no restrictions on the choice of language here:
	\begin{quotation}
		one already commits the theory with being characterized not only by its postulates and correspondence rules but also with a specific vocabulary and language. [...] Given that a theory is identified with its linguistic formulation (in axiomatic terms), it would result in impossible formulations of the same theory in alternative vocabularies. \hspace{\fill}\cite{krause-foundation}
	\end{quotation}}
	\item A logical vocabulary, namely a set of logical axioms endowed with derivation rules and a notion of consequence relation, called $\clV$. So $\clF_\clL= (\clT,\clO) \cup \overline{\clV}$.
	\item A set of sentences in $\clT$ called \emph{theoretical postulates}.
	\item An informal semantics for observational terms whose relating terms of $\clO$ with observable objects and events \footnote{Questo è fortemente dipendente dalla meaning theory neopositivistica \cite{} ed è un passaggio problematico per motivi chiariti più avanti}. 
	\item A set of sentences called \emph{correspondance rules} relating theoretical terms with observational terms. 
	\end{itemize}    
\end{definition}
In the Carnapian --and in general te neopositivistic-- account a theory can be expressed as a sentence formed by terms $\tau_1, \dots, \tau_k$ taken from both the dictionaries of $\clF_\clL$. 

In the Wiener Kreis paradigm, the formal specification of $\clO$ is left unclear; Carnap \cite{carnapfound} posits the existence of \emph{correspondance rules} between $\clO$ and $\clT$, associating to each term $o$ of $\clO$, or O-term, its companion in $\clT$, or the T-term $\tau$ derived from $o$.\footnote{In general, Carnap holds that $\clO \subset \clT$, but at the same time he blurs the features of this identification of observational terms as `types of T-terms'.}

We can maintain a similar idea, just phrased in a slightly more precise way: we posit the existence of a function $\varphi$ that translates O-terms into T-terms. So, a Wiener definition of a theory is a suitable set of pairs $\{\bk{\tau,\varphi(\tau)} \mid \tau\in\clT\}\subseteq \clT\times\clO$, where $\varphi: \clT \to \clO$ is called a \emph{translation function}. In this way it is trivially true that all the terms of a theory are in the first dictionary. This is akin to the \emph{graph} of the translation function $\varphi$, and not by chance: see \autoref{da_collage}.\footnote{In passing, it is worth to notice that Carnap's intuition fits nicely in our functorial framework: assume $\clT, \clO$ exhibit some kind of structure, and that $i : \clO\subseteq\clT$ \emph{as substructures} (e.g., assume that they are some sort of ordered sets, and that the order on $\clO$ is induced by the inclusion); then, a \emph{left} (resp., \emph{right}) \emph{translation function} $\varphi_L$ (resp., $\varphi_R$), is a left (resp., right) adjoint for the inclusion $i : \clO\hookrightarrow \clT$. We will not expand further on this idea, but see \autoref{carnap_translation_functors}.}

The neopositivistic approach is to build an observational version of a theory $T$ following a procedure first outlined by Ramsey \cite{?}. La Ramsey-sentence of a theory, non potendo nessun termine $\tau_k \in \clF_\clL$ uscire da $\clT$, sia un ri-tradurre nel linguaggio di $\clO$ il contenuto della teoria, sostituendo i termini teorici "puri" con delle variabili (dimenticando di specificare un dominio di oggetti da cui prenderle).
\begin{definition} [Ramsey sentence]
	Given a translation function $\varphi: \clT \to \clO$, the \emph{Ramsey sentence} $T^\text{Ra}$ (at $\xi\in\clT$) of a theory $T$ is obtained as  
	\[ 
		T^\text{Ra} = \{\bk{[\tau/\xi], \varphi([\tau/\xi])} \mid \tau\in\clT\} 
	\]
\end{definition}
In questa semantica si dice allora che $T^\text{Ra}$ rappresenta il \emph{contenuto osservativo} della teoria e in generale vale che: $T \cong T^\text{Ra}$. 
{\color{red} Cosa vuol dire questo?}

\medskip
We find no epistemological motivation to follow such an approach (were it only because it is cumbersome, and it raises more questions than it is able to solve);  Non ci pare, al netto degli obiettivi riduzionistici del Wiener Kreis \cite{}, che ci siano motivi epistemologicamente validi per eseguire un tale procedimento, che del resto non ha avuto molto seguito nella letteratura successiva, ma la distinzione carnapiana è ripresa nel nostro framework, per trarre alcune conclusioni on relations between theoretical and observational core \autoref{sec:orge11c3c4} e su come allargare la nozione stessa di teoria \autoref{}.

I termini di $\clO$ si riferiscono a fenomeni ma sono appunto 'termini': anche nei futuri sviluppi semantici il trattamento formale impedisce di uscire davvero dalla sintassi. Parafrasando Duhem si potrebbe dire che già agli albori della semantics of theory "tutta l'osservazione [in fisica] è carica di teoria" dove per teoria si intende la categoria sintattica $\clT$.

Le ambiguità carnapiane sul dominio di oggetti sui quali verterebbe la definizione viennese noi, coerenti col testo, le risolviamo indicando come dominio $\clT$. Cosa sia il mondo puro delle osservazioni al quale farebbe riferimento la Ramsey-version della teoria non è chiaro, se non un altro sotto-dizionario teorico, per l'appunto. Non a caso il neopositivismo da un iniziale fisicalismo approda ad un'ottica convenzionalista \cite{?}, in seguito ai falliti tentativi di formalizzazione di un contesto osservazionale extra-teorico. 

Seguendo \cite{psillos} si può dare una lettura strutturalista a questi attempts. Si può dire che una teoria $T$ is logically equivalent to the conjunction
\[T^\text{Ra} \land (T^\text{Ra} \rightarrow T)
\] where the second member is the \emph{meaning postulate}:
\begin{quotation}
	Carnap notes that this conditional has no factual content and takes it to be a meaning postulates \cite{psillos}
\end{quotation}

This is a kind of 'if-then' realism: the subject of study of scientific knowledge is not the world, but instead the conditions that, given certain assumptions, happen to be true in the structural world that the theories describe.

The meaning postulate is nothing but the Wiener Kreis' version of the famous \emph{demarcation problem} between science and metaphysics: the attempt to elaborate a distinguishability criterion between a proposition belonging to empirical sciences, and a metaphysical (or, more broadly, a non-scientifical) one. Such criteria have always been either too strict (taking sensorial experience as ultimate judge of a scientific statement), or too large (from \cite{schwarz2009twisted}: \emph{Physics is a part of Mathematics devoted to the calculation of integrals of the form $\int g(x) e^{f(x)}dx$}): our approach addresses this problem.

Tolti i due estremi, gli enunciati protocollari da un lato \cite{?}, i discorsi di Heidegger dall'altro \cite{?}, esistono tutti i casi intermedi per i quali a meaning criterion (o una operazione come la drastica traduzione dei costrutti teorici in "referti osservativi puri" tramite ramseyfication) impedisce concretamente di individuare la demarcazione. 
%Classical objections range from Popper \cite{?} to more recent sociology of science.

Negli anni, oltre a tramontare le aspirazioni fisicaliste della cosiddetta received view, si sono dissolti anche gli approcci generali al problema del trattamento formale delle teorie scientifiche, e si sono moltiplicati gli studi su linguaggi specifici dell'impresa scientifica \cite{}. 

Proveremo che a profunctorial approach è un modo per riconsiderare nozioni più larghe e 'universali', costeggiando anche il demarcation problem. In più chiarendo come si induce una interpretazione sulla categoria 'mondo' e quale ruolo svolgono le versioni funtoriali degli strumenti classici di questo campo, fin qui criticati \autoref{inducing_herme}. 

\subsection{Our contribution}
From a neo-positivistic stance, the distinction between theoretical and empirical is purely formal: it is not due to the hypothetical nature of the former (even an empirical law can be hypothetical), but to the fact that the two kinds of law contain different types of terms \cite{carnap56}. This purports a purely linguistic approach to epistemological issues. 

In fact, the present work pushes in this direction even more: the profunctorial formulation of scientific theories deletes even more forcefully any intrinsic distinction that might be between the observational and the theoretical\fshyp{}linguistic structure of a theory.

In profunctorial terms, thanks to \autoref{da_collage} and standard category\hyp{}theoretic arguments, the distinction between observational and theoretical vanishes \emph{in the mathematical model}: first of all, the bicategory defined in \autoref{def_profu}, with a mutuation from \cite{benabou2000distributors}, is auto-dual; this means that a profunctor $\fkR : \clT \pto \clO$ admits a `mirror image' $\fkR^\op : \clO \pto \clT$;\footnote{This is reminiscent of the fact that, as observed in \autoref{sec:org7dd09e1}, a relation has not a privileged domain of definition; clearly, the category $\mathsf{Cat}$ has a nontrivial involution given by $\op$ing a category, and this renders the auto-duality slightly more visible in the case of categorified relations (i.e., profunctors).} second, and certainly more decisive a comment towards our thesis, as outlined in \autoref{resoudre_la_tension} a generic profunctor $\fkR : \clT \pto\clO$ yields the ``collage'' of the observational and theoretical categories $\clT,\clO$ `glued along' $\fkR$; in simple terms, a new category $\clT\uplus_p\clO$, fitting in a span 
\[ \vcenter{\xymatrix{
	& \clT\uplus_p \clO \ar[dr]\ar[dl]& \\ 
	\clT  && \clO 
}} \] (cf. \autoref{def:dfib} and in particular \autoref{collage_explaned}) allowing to recover the theoretical and observational terms.

In our discussion, however, we require $(\clT,\clO)$ to satisfy an additional admissibility condition, that is the existence of a meaningful relation between the theoretical world $\clT$ and the observational world $\clO$; this notion of `meaningful relation' between structured high-level systems is again captured a well-known mathematical object, a \emph{profunctor} \cite{benabou2000distributors} between the two syntactic categories $\clT,\clO$.

Proposing the fundamental features of a ``general theory of scientific theories'' in terms of profunctors is the main contribution of the present work.

\medskip
We conclude this introductory section with a paragraph discussing about the ``nature''' of the categories $\clT,\clO$.

As we already observed in a previous work \cite{catont1}, the problem of locating the syntactic objects embodying a linguistic theory can be easily solved from an esperientialist stance: the world undeniably exists, and it is a sufficiently complex structure to contain the ``concrete'' constituents of a formal system. Therefore, we derive the primitive symbols of language from a portion of the world.

This problem, and its proposed solution, reflect unavoidably on the way in which the categories $\clT,\clO$ are built. In our model the world is a (possibly large) category $\clW$, unfathomable and given since the beginning of time, to which we can only access through \emph{probing} functors $\phi : \clL \to \clW$ (cf. \autoref{canvas_scienza}) representing small `accessible' categories construed from parts of $\clW$ that we can experience. Te request that $\clW$ is sufficiently expressive now translates into the request that as a category $\clW$ contains enough ``traces'' of functors like $\phi$; this (cf. \autoref{mondo_yalda}) translates formally in the request that any such $\phi$ admits a \emph{colimit} (cf. \cite[Ch. 2]{Bor1}) in $\clW$.

When things are put in this perspective, a few remarks are in order: 
\begin{itemize}
	\item This closes the circle over the problem of representation of a world $\clW$ in terms of a portion $\clT$ to which we have hermeneutical access, and from which we have carved a language. In fact, such a representation happens through `canvas functors' $\phi : \clL \to \clW$ that, thanks to the cocompleteness property of $\clW$, extend uniquely to representation functors $[\clL^\op,\Set] \leftrightarrows \clW$. 
	\item On the other hand, `the world' as a whole is unknowable, strictly speaking: instead of $\clW$, we can access to an `observational fragment' $\clO$, from which we recover, now exploiting the cocompleteness of $[\clO^\op,\Set]$, a further representation $[\clL^\op,\Set] \leftrightarrows [\clO^\op,\Set]$. In general, this is all that can be said; such a picture is already capable of determining, by elementary means, an equivalence of categories (i.e., an equivalence of models) between the observational and the theoretical \emph{nuclei} of $[\clT^\op,\Set] \leftrightarrows [\clO^\op,\Set]$: we discuss the matter in \autoref{nuclei}, and \autoref{resoudre_la_tension}.
	\item Additional assumptions on the canvas $\phi : \clL\to \clW$, however, can refine our analysis: we can infer that the totality of models $[\clL^\op,\Set]$ \emph{contains a copy} of the world $\clW$. In this precise sense, assuming what is outlined in the definition of \science in \autoref{canvas_scienza}, language prevails: the unfathomable world is a full subcategory of the class of all modes in which the language of $\clT$ can be interpreted. 
\end{itemize}
As bold a statement as it might seem, this has fruitful consequences: see for example \autoref{remark_yuggoth_1}, \autoref{remark_yuggoth_2}.
\subsubsection*{Structure of the paper}
Section 2 and 3 outline the mathematical background we need throughout the work; section 4 introduces our mai notions: a canvas, a world, a theory, a science. Sections 5 and 6 will provide evidence that the two dictionaries, theoretical and observable, live in a very tight relation; a broadly intended meaning for the notion of \emph{theory} asks for a precise understanding of the category of adjunctions between the theoretical and observational side.
\section{Semantical conception of theories}
During the XXth century it was considered necessary to develop a formal treatment of scientific theories. The Wiener Kreis verificationist paradigm/account, and the Neurath theory of `protocollar statements', was the input to elaborate a completely semantical framework for working with scientific theories, and the clue of a pan-linguistic vision of philosophy of science.

For the sake of strictness the formal account in which Carnap and associates provide their notion of `theory' is known in literature as \emph{syntactical conception of theories} \cite{.} while the introduction of term `semantic' is due to later developments. But the field of epistemology that the logical neopositivism started one can call `semantics of theories', because some characteristics, and above all the underlying ideology, are the same from Carnap to Beth to Suppes, up to the recent canonical uses of physical handbooks.

	[scrivere quali sono queste caratteristiche]

	[sintesi delle varie concezioni; le teorie come classi di modelli $\clK$;
		le teorie come oggetti formali]

semantica non standard per teorie empiriche in cui le teorie sono sistemi formali e tutte le nozioni diventano oggetti matematici; più propriamente una \emph{teoria} diventa una struttura $(F_\clL, \clK)$ dove $F_\clL$ è il vero sistema formale e $\clK$ è la classe di tutti i suoi modelli. La nostra strategia è separare ulteriormente $F_\clL$ in due `vocabolari' (per noi, le categorie sintattiche di teorie al primo ordine), uno $P_{F_\clL}$ che rappresenta i termini puri (nel senso di Plantinga) e uno $A_{F_\clL}$ che rappresenta i termini \emph{applicati}. Ergo una teoria $\mathbf{T}$ sarà una particolare tripla $\bk{(P_{F_\clL},A_{F_\clL}), \clK}$ in cui la prima coppia configura una logica (uno `spazio degli stati' che configura una logica).

La coppia $(P_{F_\clL},A_{F_\clL})$ è poi soggetta a una ulteriore condizione di ammissibilità, cf. \ref{}, chiedendo che esista un profuntore tra le due categorie sintattiche $P_{F_\clL},A_{F_\clL}$.

La specificazione del dominio di $A_{F_\clL}$ determina il tipo di teoria che stiamo considerando (scientifica, strettamente empirica, logico-matematica, metafisica).

Dire che $A_{F_\clL}$ determina le \emph{tipizzazioni} della teoria significa
dire che svolge lo stesso ruolo della legge $\beta$ nella semantica dello spazio degli
stati, mentre la classe $\clK$ è isomorfa all'insieme $\mathcal{M}$ dello spazio
degli stati. Il tipo di $\beta$ determina il tipo di $\mathcal{M}$ che determina il
tipo di $\mathbf{T} = (\mathcal{M}, \beta)$. Idem nel nostro approccio: $A_{F_\clL}
	= \{\alpha_1,\dots,\alpha_n\}$ determina il tipo, che implementa una logica che determina
la classe $\clK$.
%si può generalizzare tutto ciò dicendo appunto che a seconda di come è strutturata la classe $A_{\mathcal{F_L}}$ noi possiamo distinguere le teorie anche al di fuori dell'ambito strettamente scientifico. Questa considerazione va a scomparire proprio con ciò che emerge in sezione 6: eliminando questa distinzione ottieni un account più generale di trattamento delle teorie%

\subsection{The Two Dictionaries}
Nella concezione neopositivistica la distinzione tra legge teorica e legge empirica non è dovuta alla natura ipotetica della prima (anche una legge empirica può esserlo) quanto dal fatto che i due tipi di legge contengono tipi differenti di termini \cite{}. La distinzione è quindi formale, e indica una approccio prettamente linguistico a questioni epistemologiche.
\begin{remark}\label{hint_at_collage}
	Anche in questa visione `sintattica' \cite{ } una teoria è sempre una struttura che contiene un sistema formale $\mathcal{F_L}$ e la classe $\clK$ dei suoi modelli. La strategia carnapiana per rendere conto della presenza di entità `osservazionali' e quindi, a rigore, non formalizzabili, all'interno di teorie scientifiche è quella di considerare due diversi dizionari: $\mathcal{V_T}$ che contiene \emph{termini teorici} e $\mathcal{V_O}$ che contiene \emph{termini osservativi}. Intuitivamente $\mathcal{F_L} = \mathcal{V_T} \cup \mathcal{V_O}$, ma piu precisamente $\mathcal{F_L} = \mathcal{V_T} \uplus_\varphi \mathcal{V_O}$.
\end{remark}

Per derivare una legge empirica da una teorica Carnap introduce delle \emph{correspondance rules} ma senza definirle adeguatamente. Possiamo analogamente fornire il framework `viennese' di una \emph{funzione di traduzione} $\varphi: \mathcal{V_O} \to \mathcal{V_T}$ tale che $\omega_j \mapsto \varphi (\omega_j)$ \footnote{In generale Carnap sembra assumere che $\mathcal{V_O} \subset \mathcal{V_T}$ ma specifica comunque che è errato dire che gli O-terms siano esempi di T-terms.}.
%qui va aggiunta la funzione di "applicazione" $\psi: \mathcal{V_T} \to \mathcal{V_O}$ che fa il percorso inverso di quella di traduzione. è un modo migliore per rappresentare il dibattito di inizio 900: quale delle due funziona meglio? Non esiste la risposta corretta (per i neopositivisti presumibilmente la Ramsey version della def 2.1 coinvolge $\psi$ e ha come oggetti elementi di $\mathcal{V_O}$). La risposta storicamente più accurata è "entrambe alternativamente". Bene, altra cosa risolta dalla sezione 6% 

\begin{definition} [Wiener Definition]
	Una teoria $\mathbf{T}$ è una coppia $\bk{\tau_i, \varphi (\omega_j)}$ dove $\tau_i$, $\varphi(\omega_j) \in \mathcal{V_T}$.
\end{definition}



[\cite{} Carnap da pag. 299; importante la 314]

\subsection{\emph{Was Sind und was sollen die Erkennen?}}

La strategia carnapiana è figlia della distinzione di Moritz Schlick \cite{.} tra \emph{kennen} e \emph{erkennen} ... [spiegare la manfrina e la nostra `traduzione']

In questo paragrafo parlerei della questione `sì ma cosa sono gli `osservativi' nella nostra semantica funtoriale?', dell'arbitrarietà della divisione in due categorie sintattiche, per comodità nel trattamento di determinate teorie, e introdurrei alla tensione tra teorico e osservazionale che si sviluppa formalmente in seguito (cenno storico in nota al perchè i neopositivisti fanno la ramseyfication e perchè a noi non interessa (citare lo Weinberg)).
%forse tutta sta parte confluisce in section 6%


\section{Nerve and realisations}\label{sec_3_nervi}
\label{sec:org1a423df}
We start by recalling the universal property of the category of presheaves over $\clC$:
let $\clC$ be a small category, $\clW$ a cocomplete category; then, precomposition with the Yoneda embedding $\yon_{\clC} : \clC \to [\clC^\op, \Set]$ determines a functor
\[\Qat([\clC^\op, \Set], \clW)\xto{\firstblank\circ \yon_{\clC}} \Qat(\clC,\clW),\]
that restricts a functor $G : [\clC^\op, \Set]\to \clW$ to act only on representable functors, confused with objects of $\clC$, thanks to the fact that $\yon_\clC$ is fully faithful. We then have that
\begin{theorem}\label{yext_are_good}\leavevmode
	\begin{enumtag}{ye}
		\item The universal property of the category $[\clC^\op, \Set]$ amounts to the existence of a left adjoint $\Lan_{\yon_{\clC}}$ to precomposition, that has invertible unit (so, the left adjoint is fully faithful).
	\end{enumtag}
	This means that $\Qat(\clC,\clW)$ is a full subcategory of $\Qat([\clC^\op, \Set], \clW)$. Moreover
	\begin{enumtag}{yi}
		\item The essential image of $\Lan_{\yon_{\clC}}$ consists of those $F : [\clC^\op, \Set] \to \clW$ that preserve all colimits.
		\item If $\clW = [\clE^\op, \Set]$, this essential image is equivalent to the subcategory of left adjoints $F : [\clC^\op, \Set] \to [\clE^\op, \Set]$.
	\end{enumtag}
\end{theorem}
As a consequence of this,
\begin{definition}[Nerve and realisation contexts]\label{nr_para}\index{Nerve!--- context}
	Any functor $F : \clC\to \clW$ from a small category $\clC$ to a (locally small) \emph{cocomplete} category $\clW$ is called a \emph{nerve\hyp{}realisation context} (a NR \emph{context} for short).
\end{definition}
Given a NR context $F$, we can prove the following result:
\begin{proposition}[Nerve-realisation paradigm]\label{nervereal}
	The left Kan extension of $F$ along the Yoneda embedding $\yon_\clC : \clC\to [\clC^\op, \Set]$, i.e. the functor
	\[L_F=\Lan_{\yon_\clC} F : [\clC^\op, \Set]\to \clW\]
	is a left adjoint, $L_F\dashv N_F$. $L_F$ is called the $\clW$-\emph{realisation functor} or the \emph{Yoneda extension} of $F$, and its right adjoint the $\clW$-\emph{coherent nerve}.
\end{proposition}
\begin{proof}
	From a straightforward computation, it follows that if we define $N_F(D)$ to be $C\mapsto \clW(F C,D)$, this last set becomes canonically isomorphic to $[\clC^\op,\Set](P,N_F(D))$. We can thus denote $\clW(F,1)$ the functor $N_F : D\mapsto \lambda C.\clW(F C,D)$.
\end{proof}
Now, let's review the way in which a profunctorial analogue of \eqref{adjunzia} can be obtained: \autoref{nervereal} yields that a functor
\[ \fkR : \clA^\op\times \clB \to \Set \]
whose mate under the adjunction $\Qat(\clA^\op\times \clB ,\Set)\cong\Qat(\clB,[\clA^\op,\Set])$ is a functor
\[ \hat R : \clB \to \Qat(\clA^\op,\Set) \]
determines a NR paradigm, and thus gives rise to a pair of adjoint functor
\[ \Lan_{\yon_\clB} \hat R : \Qat(\clB^\op,\Set) \leftrightarrows \Qat(\clA^\op,\Set) : [\clA^\op,\Set](\hat R,1). \]
\begin{remark}\label{ciu}
	Note that, given a functor $F : \clA \to \clB$, the functor $N_F =\clB(F,1)$ coincides with the lower image of $F$ into $\Prof$, described in \autoref{upper_n_lower}
\end{remark}
We have just laid down all the terminology needed to prove that
\begin{proposition}\label{equ_prof_cocont}
	There is an equivalence of categories between $\Prof(\clA,\clB)$ and the category of colimit preserving functors $\Qat(\clB^\op,\Set) \to \Qat(\clA^\op,\Set)$.
\end{proposition}

\section{Theories and models}
\label{sec:orge02f333}
In this section we exploit the terminology established before.
\begin{definition}[Theory]\label{teoria}
	A \emph{theory} $\clL$ is the syntactic category $\clT_L$ (cf. \cite{lambek1988introduction}) of a type theory $L$.
\end{definition}
The reader interested in how the construction of $\clT_L$ goes can take from \cite{lambek1988introduction} as standard reference, or \cite{abramskyno} for a shorter survey, in the simple case $L$ admits product and function types.

% From $\clT_L$ it is possible to carve a cartesian closed category, whose obects are denoted $\trn{B}$ as $B$ runs over the basic types of $L$. Then, we define recursively a \emph{semantic translation} operation from basic types $\trn{B} := B$ and
% \[\trn{T\times U}:=\trn{T}\times\trn{U} \,,\qd \trn{T\to U}:=\trn{T}\Rightarrow\trn{U}\,, \]
% and on typed terms as the set of rules in \cite[1.6.5]{abramskyno}.

% The resulting categorical semantics is \emph{sound}, i.e.
% \[ t=_\lambda u \implies \trn{t}=\trn{u} \]
% for every term $t,u$, and $=_\lambda$ is $\lambda$-reduction. It is however not \emph{complete}, in the sense that
% \[ \lsem t \rsem = \lsem u \rsem \text{\qd yet \qd} t\neq_\lambda u\,. \]
% It is possible to build a new category $\clC_\lambda$ providing a sound and complete term model of (a fragment of simply-typed) $\lambda$-calculus; yet, the specifics of this construction are of little interest for the present discussion.
\begin{definition}[World, Yuggoth]\label{mondo_yalda}
	A \emph{world} is a large category $\clW$; a \emph{Yuggoth}\footnote{\emph{Yuggoth} (also \emph{Iukkoth}, or {\yugg} in Chtuvian language) is an enormous trans-Neptunian planet whose orbit is perpendicular to the ecliptic plane of the solar system. A Yuggoth is a world so big to inspire a sense of unfathomable awe.} is a world that, as a category, admits all small colimits.
\end{definition}
\begin{definition}[Canvas, science]\label{canvas_scienza}
	Given a theory $\clL$ and a world $\clW$, a $\clL$-\emph{canvas} of $\clW$ is a functor
	\[\xymatrix{\clL \ar[r]^\phi & \clW.}\]
	
	A canvas $\phi : \clL \to \clW$ is a \emph{\science} if $\phi$ is a dense functor.
\end{definition}
	\begin{remark}\label{remark_yuggoth_1}
	The NR paradigm exposed in \autoref{nr_para} now entails that given a canvas $\phi : \clL \to \clW$
	\begin{itemize}
		\item If $\clW$ is a world, we obtain a \emph{representation} functor
		      \[ \xymatrix{\clW \ar[r] & [\clL^\op, \Set];} \label{mvndvs}\]
		      this means: given a canvas $\phi$ of the world, the latter leaves an image on the canvas.
		\item If in addition $\clW$ is a Yuggoth, we obtain a NR-adjunction
		      \[\xymatrix{\clW \ar@<3pt>[r] & \ar@<3pt>[l] [\clL^\op, \Set];}\]
		      this has to be interpreted as: if $\clW$ is sufficiently expressive, then models of the theory that explains $\clW$ through $\phi$ can be used to acquire a two-way knowledge. Phenomena have a theoretical counterpart in $[\clL^\op,\Set]$ via the nerve; theoretical objects strive to describe phenomena via their realisation.
		\item If an $\clL$-canvas $\phi : \clL \to \clW$ is a \science, `the world' is a full subcategory of the class of all modes in which `language' can create interpretation.
	\end{itemize}
\end{remark}
\begin{remark}\label{remark_yuggoth_2}
The terminology is chosen to inspire the following idea in the reader: science strives to define \emph{theories} that allow for the creation of world representations; said representations are descriptive when there is dialectic opposition between world and models; when such representation is faithful, we have reduced `the world' to a piece of the models created to represent it.

The tongue-in-cheek here is, a science (in the usual sense of the world) can never attain the status of a \science, if not potentially; attempts to generate scientific knowledge are the attempts of
\begin{itemize}
	\item recognizing the world $\clW$ as a sufficiently expressive object for it to contain phenomena and information;
	\item carve a language $L$, if necessary from a small subset of $\clC$, that is sufficiently `compact', but also sufficiently expressive for its syntactic category to admit a representation into the world;
	\item obtaining an \emph{adjunction} between $\clW$ and models of the worlds obtained as models of the syntactic theory $\clL$; this is meant to generate models starting from observed phenomena, and to predict new phenomena starting from models;
	\item obtaining that `language is a dense subset of the world', by this meaning that the adjunction outlined above is sufficiently well-behaved to describe the world as a fragment of the semantic interpretations obtained from~$\clL$.
\end{itemize}
It is evident that there is a tension between two opposite feature that $\clL$ must exhibit; it has to be not too large to remain tractable, but on the other hand it must be large enough in order to be able to speak about `everything' it aims to describe.
\end{remark}
Regarding our definition of \science, we can't help but admit we had this definition in mind \cite[2.1]{biologia}:
\begin{definition*}[\protect{\cite[2.1]{biologia}}]
	A \emph{scientific theory} $\clT$ consists of a formal structure $F$ and a class of interpretations $M_i$, shortly denoted as $\clT=\langle F,M_i\mid i\in I\rangle$. The structure $F$ consists on its won right of
	\begin{itemize}
		\item a language $\clL$, in which it is possible to formulate propositions. If $\clL$ is fully formalised, it will consist of a finite set of symbols, and a finite set of rules to determine which expressions are well-formed. This is commonly called \emph{technical language};
		\item A set $A$ of `axioms' or `postulates' in $\clL^\star$;
		\item A \emph{logical apparatus} $R$, whose elements are rules of inference and logical axioms, allowing to prove propositions.
	\end{itemize}
\end{definition*}
The language of category theory allows for a refined rephrasing of the previous definition: we say that a \emph{$\clS$-scientific theory} is the following arrangement of data:
\begin{enumtag}{st}
	\item a formal language $\clL$;
	\item the syntactic category $T_\clL$, obtained as in \cite{lambek1988introduction};
	\item the category of functors $[T_\clC, \clS]$, whose codomain is a Yuggoth.
\end{enumtag}
More than often, our theories will be $\Set$-scientific: in such case we just omit the specification of the semantic Yuggoth, and call them \emph{scientific theories}.

Since the category $[T_\clC, \Set]$ determines $\clL$ and $T_\clL$ completely, up to Cauchy-completion \cite{borceuso-cauchy}, we can see that the triple $(\clL, T_\clL, [T_\clL,\Set])$ can uniquely be recovered from its model category $[T_\clC, \Set]$. We thus comply to the additional abuse of notation to call `scientific theory' the category $[T_\clL,\Set]$ for some $T_\clL$.

So, a `coherent correspondence linking expressions of $\clF$ with semantic expressions' boils down to a functor; this is compatible with \cite[2.1]{biologia}, and in fact an improvement (the mass of results in category theory become readily available to speak about --scientific-- theories; not to mention that the concept of `formal structure' is never rigorously defined throughout \cite{biologia}).

%\subsection{The first rough attempt of some infant god}
Let us consider two categories $\clO,\clT$, respectively the \emph{observational} and the \emph{theoretical}. Even though their origin is never examined further, it is fruitful to think that $\clO,\clT\subseteq \clW$, i.e. that they are `carved' from the world, building respectively on the tangible experience (for $\clO$) and on a linguistic structure (for $\clL$).

If $\clW$ is a Yuggoth each pair of canvases
\[ \xymatrix{
	\clO \ar[r]^-\psi & \clW & \ar[l]_-\phi \clT
} \label{2_canvases}\] gives rise, according to \eqref{mvndvs}, to representations 
\[ \xymatrix{
	[\clO^\op,\Set] \ar@<.5em>[r] & \ar@<.5em>[l]^-{N_\psi} \clW \ar@<-.5em>[r]_-{N_\phi} \ar@{}[r]|-\perp \ar@{}[l]|-\perp & \ar@<-.5em>[l] [\clT^\op,\Set]
} \label{2_reps}\]
The leftmost category is the category we have experimental access, starting from the fragment $\clO \subseteq \clW$ we can observe. The rightmost category is the category of symbols we can speak of, trying to reproduce the observed behaviour. 
\begin{definition}
	We refine the terminology introduced above to speak of a \emph{theoretical} (resp., a \emph{observational}) \emph{science}, assuming that $\phi$ (resp., $\psi$) is a \science.
\end{definition}
Assuming that $\phi : \clT \to \clW$ is a theoretical \science, now, the representation functor $\clW \to [\clT^\op,\Set]$ above acquires a left adjoint.
\section{The tension between observational and theoretical}
\label{sec:orge11c3c4}
When working with categorified relations, it is unnatural and somewhat restrictive to take into account a two-element set for the possible values a proposition(al function) `$(a,b)\in R$' can assume; instead we would like to consider an entire \emph{space} of such values, or rather a type of proofs that $(a,b)\in R$ is true. Again, this idea is best appreciated when thinking that the same proposition 
\begin{center}
	\begin{minted}{haskell}
	(n : Nat) -> (m : Nat) ->  n + m = m + n 
	\end{minted}
\end{center}
when encoded in any (sufficiently strongly-typed) DSL, can be interpreted as either the \emph{proposition} `given $n$ and $m$ natural numbers, their sum is a commutative operation' or as the \emph{type} \mil{n + m} $\equiv$ \mil{m + n} whose elements are the proofs that $n+m$ is in fact equal to $m+n$.

This intuition is based on the well-known proportion
\begin{center}
	truth values : proposition = section : presheaf
\end{center}
inspired by the `proposition as types' paradigm. In simple terms, categorifying a proposition $P : X\to \{0,1\}$ that can or cannot hold for an element $x$ of a set $X$, we shall marry the constructive church and say that there is an entire \emph{type} $PC$, image of an object $C\in\clC$ under a functor $P : \clC \to \Set$, whose \emph{terms} are the \emph{proofs} that $PC$ holds true. This is nothing but the propositions-as-types philosophy, in (not so much) disguise: \cite{hottbook,wadler,martin1984intuitionistic}

The important point for us is that the dialectical tension between observational and theoretical can be faithfully represented through profunctor theory; one can think of propositional functions as relations $(x,y)\in R$ if and only if the pair $x,y$ renders $\phi$ true; we use this idea, suitably adapted to our purpose and categorified. This very natural extension of propositional calculus, pushed to its limit, yields the following reformulation of the `tension between observational and theoretical'.% of \cite{u,v,w}
\begin{definition}\label{11_ramsey}
	Let $\clT,\clO$ be two small categories, dubbed respectively the \emph{theoretical} and the \emph{observational} settings. A \emph{Ramsey map} is merely a profunctor
	\[\fkK : \clT^\op \pto \clO\]
	or, spelled out completely, a functor $\fkK : \clT\times \clO \to \Set$.
\end{definition} 
\begin{example}
	Every functor $F : \clA \to \clB$ gives rise to a profunctor $F_* := \clB(1,F) : \clB^\op\times \clA\to\Set$ and a profunctor $F^* := \clB(F,1) : \clA^\op\times\clB \to \Set$ as in \autoref{nervereal}; the two functors are mutually adjoint, $F^*\dashv F_*$, see \cite[6.2]{Bor2}. This yield an example of what we call \emph{representable} Ramsey maps.
\end{example}
\begin{definition}[Observational and theoretical nucleus]\label{nuclei}
	Let $\fkR : \clT^\op\times \clO \to\Set$ be a Ramsey map, and $\hat R : \clO \to [\clT^\op,\Set]$ the associated canvas. Let
	\[ \label{lalan} \Lan_{\yon_\clO}\hat R : [\clO^\op,\Set] \leftrightarrows [\clT^\op,\Set] : N_{\hat R} \]
	be the adjunction between presheaf categories determined by virtue of \autoref{equ_prof_cocont}. Let us consider the equivalence of categories between the fix-points of the monad $T = N_{\hat R}\circ\Lan_{\yon_\clO}\hat R$ and the comonad $S=\Lan_{\yon_\clO}\hat R\circ N_{\hat R}$.
	
	This is the equivalence between the \emph{observational nucleus} $Fix(T)\subseteq [\clO^\op,\Set]$ and the \emph{theoretical nucleus} $Fix(S)\subseteq [\clT^\op,\Set]$.
\end{definition}
\begin{remark}
	Observational nucleus and theoretical nucleus always form equivalent categories; the tension in creating a satisfying image of reality as it is observed oscillates between the desire to enlarge as much as possible the subcategory of $[\clO^\op,\Set]$ with which our theoretical model is equivalent, where we can have access to $\clT, [\clT^\op, \Set]$ only.
\end{remark}
The following remark shows how new structure comes `almost for free' when things are interpreted this way. 

Assume $\phi : \clT \to \clW$ and $\psi : \clO \to \clW$ are canvases, $\fkR$ is a Ramsey map, and $\Lan_{y_{\clO}}\hat R$ the functor corresponding to $\fkR$ under the construction in \eqref{lalan}; in this notation, we can state a tight condition of compatibility between the theory identified by $(\phi,\psi)$ and the Ramsey map $\fkR$. We employ freely the presence of adjunction 
\[L_\phi\dashv N_\phi \qquad 
L_\psi \dashv N_\psi \qquad 
L_{\hat R} \dashv N_{\hat R}.\]
\begin{remark}[Inducing an hermeneutics]\label{inducing_herme}
Consider the diagram
\[ \vcenter{\xymatrix{
	& \clW \ar[dl]_{N_\phi} \ar[dr]^{N_\psi} & \\ 
	[\clT^\op,\Set] \ar@<-.5em>[rr]_{N_{\hat R}} && \ar@<-.5em>[ll]_{L_{\hat R}} [\clO^\op,\Set]
}}	 \label{discuss_0}\]
given by the theoretical and observational nerves, plus the Ramsey adjunction mentioned above.
\end{remark}
We seek sufficient conditions in order for \eqref{discuss_0} to be filled by a suitable natural transformation $\omega : N_{\hat \fkR} \circ N_\phi \To N_\psi$: such a 2-cell will force a tameness property on the system described by the two canvases $(\phi,\psi)$: this is made precise by the following
\begin{definition}[Fundamental cell, Hermeneutics]\label{funcell_herme}
	In a display of categories like \eqref{2_canvases} we say that 
	\begin{itemize}
		\item A \emph{fundamental cell} is a natural transformation $\omega : N_{\hat \fkR} \circ N_\phi \To N_\psi$;
		\item we say that in the world $\clW$ \emph{hermeneutics is possible} if the right extension $\bk{\phi/\psi} := \Ran_{ N_\phi}N_\psi$ exists \emph{as a functor} (note that it always exists as a profunctor, but this might not be representable).
	\end{itemize}
	If hermeneutics is possible in $\clW$, and $R : \clT \pto \clO$ is a Ramsey map, any fundamental cell induces a natural transformation 
	\[ \varpi : N_{\hat R} \To \bk{\phi/\psi} \]
	obtained exploiting the universal property of $\bk{\phi/\psi}$. 
\end{definition}
If the right extension is representable in the sense above, this amounts to a higher type map (in the sense of the internal language of a closed category) comparing `generalised formulas' of kind $\fkR \To \clW(\phi,\psi)$.
\begin{remark}\label{herme_explained}
If we follow the customary practice to identify a morphism of a category as an entailment between sequents in a deductive system, it is easy to see that the condition that the possibility of hermeneutics captures is that we can embody sequents of the form $\llbracket T\vdash O\rrbracket \in \fkR(T,O)$ in the internal language of $\clW$; more precisely, if we think of $\fkR(T,O)$ as the type of all proofs that some theoretical terms describe an observational phenomenon, then the map $\varpi$ above can be represented as the higher order entailment relation between $\llbracket T\vdash O\rrbracket$ and the entailment $\phi(T) \to \psi(O)$ \emph{valid in $\clW$}:
\[ \begin{array}{c}
	\varpi_{T,O} : \fkR(T,O) \to \clW(\phi(T),\psi(O)) \\ \hline 
	\llbracket T\vdash O\rrbracket \Vdash (\Phi[T] \vdash \Psi[O])
\end{array} \] 
where $\Phi[T]$ is a shorthand for $\phi[\vec x/T]$, the context of premises saturated by the theoretical terms, and same for $\Psi[O]$, the context of deductions saturated by the observational terms. 
\end{remark}
All in all, the map $\varpi_{T,O}$ exhibits a \emph{witness} of the expressibility of the entailment `$\llbracket T\vdash o\rrbracket$' in the world $\clW$, through the Ramsey map.

More is true: the presence of a fundamental cell means that we can find a way to assert that the entailment $T\to O$ is actually embodied in the world by an entailment $\phi(T)\to \psi(O)$ in the internal language of $\clW$.

If after a computation we find that a cannonball will follow a parabolic trajectory, the cannonball fired in the actual world is to be found at the point we predicted, even though there is no such thing as `a parabola' in the physical world. (Parabolas, and for that matter all geometric figures, arise as abstractions of a bundle of recurrent perceptions)

Such assumptions imply that "hermeneutics is possible", in the very sense of the word: we can interpret linguistic facts about the world, and derivations in the former system correspond to variations in the latter.
\begin{remark}
	There is nothing in their mere syntactical presentation allowing to tell apart the observational and the theoretical category; this can be justified with the fact that the bicategory $\Prof$ of \autoref{def_profu} is endowed with a canonical self-involution, exchanging the r\^ole of domain and codomain of 1-cells, and thus of the theoretical and observational category $\clT,\clO$.
	
	This is perhaps of some help in solving the conundrum posed by the existence of `fictional objects'. Sherlock Holmes clearly is the object of a theoretical category. Gandhi is the object of an observational category. But as linguistic objects they can't be told apart completely; they can be at most separated by a profunctor embedding the former in a realistic counterpart of fictitious model (that is, for example, the Reichenbach falls), and representing the latter as part of a fictional model (for example, as part of a movie directed by R. Attenborough).

	We can surely discuss what is the ontological status of each such object; maybe even exploiting the model of \cite{catont1}. If it is clear that in the universe of Conan Doyle, an individual named Sherlock Holmes lives at 221b Baker Street, it is also clear that it `projects' its existence in the actual world $\clW^@$; undoubtedly there are relations between Conan Doyle's Sherlock Holmes and its shadow in $\clW^@$; it is possible to rephrase there relations in terms of the syntactic categories presenting/describing the two universes, in the way that we have sketched. For a related topic, see the notion of metakosmial accessibility between worlds \cite{} or modal semantics of the non-existents \cite{}; as interesting as the topic may seem, we refrain to go further in this analysis, leaving the stage open for future discussion.
	
	The question deserves a deeper analysis: Attenborough's Gandhi isn't exactly an object inside $\clW^@$, but instead of an accessible sub-world $\clU_\text{G} \subseteq \clW^@$ that works as canvas; it might be that many well-tested approaches to the theory of modal relations might become more streamlined when expressed in our language: fictional worlds are just \emph{particular ways} to build canvases and representations thereof. 
\end{remark}
% \begin{remark}\label{multiramsey}
% 	The notion of Ramsey map as given above is unnecessarily restrictive, and does not account for many sorts of configurations that can occur in practice:
% 	\begin{itemize}
% 		\item a single observational token $O$ can't be described by a single theoretical token $T_1$, but instead needs $T_1,\dots,T_r$;
% 		\item inverting the r\^oles, a single theoretical token describes not only $O$, but different $O_1,\dots,O_s$.
% 	\end{itemize}
% 	Thus we must admit \emph{multiple} arguments for the domain and codomain of a Ramsey map. This yields the notion of a \emph{$(n,m)$-ary Ramsey map}.
% \end{remark}
\begin{remark}\label{resoudre_la_tension}
	The clearest possible sense in which the profunctorial approach `resolves' the tension between observational and theoretical is that the Gro\-then\-dieck construction associated to a profunctor $\fkR : \clT \pto \clO$ yields a category where the two `worlds', one carved from perception, and the other concocted from language, live harmoniously together. All in all, said tension is just an incarnation of the tension between speakable and unspeakable: given a Ramsey map $\fkR : \clT \pto \clO$, the equivalence between its theoretical and observational nuclei is an equivalence between the speakable (subclass of $[\clT^\op,\Set]$), with the observable (subclass of $[\clO^\op,\Set]$); what lies outside this equivalence in the latter category is observable but `unspeakable' in the strongest possible sense.
\end{remark}

\fo{Vanno messi a posto tutti i domini qua}
\begin{remark}[Ramseyfication and translation functors]\label{carnap_translation_functors}
	Assume that there exists an adjunction 
	\[ 
		L : \clT \leftrightarrows \clO : F
	\]
	between the theoretical and the observable. Following Carnap, we might assume that $F : \clO \hookrightarrow \clT$, and thus $G$ is a right translation functor for $(\clT, \clO)$.

	In these assumptions, given a Ramsey map $\fkK : \clT \pto \clO$ the function term
	\[\lambda O\lambda X . \fkK(O, X)\]
	can be pre-composed with $F$ obtaining
	\[\lambda  O\lambda O'.\fkK( O, F O').\]
	We say that a translation adjunction $(L,F)$ is `$\fkK$-admissible' (denoted $L \dashv_\fkK F$) when there is a natural isomorphism $\fkK(L,1)\cong\fkK(1,F)$.
\end{remark}
The property of $\fkK$-admissibility for a pair of functors is in general difficult to assess; nevertheless, there are interesting properties for the relation $F\dashv_\fkK G$: for example
\begin{theorem}
	Let $F : \clA \leftrightarrows \clB: G$ be a pair of functors in opposite directions; let $\fkK : \clB \pto \clA$ be a profunctor; if $F\dashv_\fkK G$, then there is a `genuine' adjunction 
	\[ F^e : \clA\uplus_\fkK\clB \leftrightarrows \clA\uplus_\fkK\clB : G^e \]
	`extended' to the category of elements of $\fkK$.
\end{theorem}
\section{The tension between observational and theoretical}
\label{sec:orge11c3c4}
When working with categorified relations, it is unnatural and somewhat restrictive considerare lo spazio per i valori che la proposizione `$(a,b)\in R$' può assumere come avente solo due valori; instead we would like to consider the entire \emph{space} of values that a proposition can take, or rather the type of proofs that $(a,b)\in R$ is true.

This intuition is based on the proportion
\begin{center}
  truth values : proposition = section : presheaf
\end{center}
In simple terms, categorifying a proposition $P : X\to \{0,1\}$ that can or cannot hold for an element $x$ of a set $X$, we shall marry the constructive church and say that there is an entire \emph{type} $PC$, image of an object $C\in\clC$ under a functor $P : \clC \to \Set$, whose \emph{terms} are the \emph{proofs} that $PC$ holds true. This is nothing but the propositions-as-types philosophy, in (not so much) disguise: \cite{a,b,c}

The important point for us is that the dialectical tension between observational and theoretical can be faithfully represented through profunctor theory; one can think of propositional functions as relations $(x,y)\in R$ iff the pair $x,y$ renders $\phi$ true; we use this idea, suitably adapted to our purpose and categorified. This very natural extension of propositional calculus, pushed to its limit, yields the following reformulation of the `tension between observational and theoretical' of \cite{u,v,w}
\begin{definition}
  Let $\clT,\clO$ be two small categories, dubbed respectively the \emph{theoretical} and the \emph{observational} settings. A \emph{$(1,1)$-ary Ramsey map} is merely a profunctor 
  \[\fkK : \clT^\op \pto \clO\]
  or, spelled out completely, a functor $\fkK : \clT\times \clO \to \Set$.
\end{definition}
Particular $(1,1)$-ary Ramsey maps can be obtained by elementary means:
\begin{example}
 Every functor $F : \clA \to \clB$ gives rise to a profunctor $F_* := \clB(1,F) : \clB^\op\times \clA\to\Set$ and a profunctor $F^* := \clB(F,1) : \clA^\op\times\clB \to \Set$ as in \autoref{nervereal}; the two functors are mutually adjoint, $F^*\dashv F_*$, see \cite{}. This yield an example of what we call \emph{representable} Ramsey maps. \foo{Say more; also, why (1,1)-ary? Wait and see}
\end{example}
\begin{definition}[Observational and theoretical core]
  Let $\fkR : \clT^\op\times \clO \to\Set$ be a Ramsey map, and $\hat R$ the associated canvas. Let 
  \[ \Lan_{\yon_\clO}\hat R : [\clO^\op,\Set] \leftrightarrows [\clT^\op,\Set] : N_{\hat R} \]
  be the adjunction between presheaf categories determined by virtue of \autoref{equ_prof_cocont}. Let us consider the equivalence of categories between the fixpoints of the monad $T = N_{\hat R}\circ\Lan_{\yon_\clO}\hat R$ and the comonad $S=\Lan_{\yon_\clO}\hat R\circ N_{\hat R}$; this is the equivalence between the \emph{observational core} $Fix(T)\subseteq [\clO^\op,\Set]$ and the \emph{theoretical core} $Fix(S)\subseteq [\clT^\op,\Set]$.
\end{definition}
\begin{remark}
Observational core and theoretical core always form equivalent categories; the tension in creating a satisfying image of reality as it is observed oscillates between the desre to enlarge as much as possible the subcategory of $[\clO^\op,\Set]$ with which our theoretical model is equivalent, where we can have access to $\clT, [\clT^\op, \Set]$ only.
\end{remark}
The reader might have observed, now, that there is nothing in their mere syntactical presentation allowing to tell apart the observational and the theoretical category; this can be justified with the fact that the bicategory $\Prof$ of \autoref{def_profu} is endowed with a canonical self-involution, exchanging the r\^ole of domain and codomain of 1-cells, and thus of the theoretical and observational category $\clT,\clO$. 



This might help to solve the conundrum posed by the existence of `fictional objects'. Sherlock Holmes clearly is the object of a theoretical category. Gandhi is the object of an observational category. But as linguistic objects they can't be told apart completely; they can be at most separated by a profunctor embedding the former in a realistic but fictitious model (that is, for example, the Reichenbach falls), and representing the latter as part of a fictional model (for example, as part of a movie directed by R. Attenborough).

The notion of Ramsey map as given above is unnecessarily restrictive, and does not account for many sorts of configurations that can occur in practice:
\begin{itemize}
  \item a single observational token $O$ can't be described by a single theoretical token $T_1$, but instead needs $T_1,\dots,T_r$;
  \item inverting the r\^oles, a single theoretical token describes not only $O$, but different $O_1,\dots,O_s$.
\end{itemize}
Thus we must admit \emph{multiple} arguments for the domain and codomain of a Ramsey map. This yields the notion of a \emph{$(n,m)$-ary Ramsey map}.

\section{Ramseyfication and beyond: generalised profunctors}
\label{sec:org50db6c2}
We can generalise the definition above to encompass Ramsey sentences:
\begin{definition}\label{mn_ramsey}
	Let $\clT,\clO$ be two categories; a \emph{Ramsey map}, or a \emph{$(n,m)$-ary Ramsey map} is a profunctor $\fkK : \clT^n \pto \clO^m$; note that we allow $n,m$ to be zero; in that case, $\clA^0$ is understood to be the terminal category $\boldsymbol{1}$.
\end{definition}
The intuition behind this definition is as follows: given $\uT\in\clT^n, \uO\in\clO^m$, the set $\fkK(\uT, \uO)$ represents the type of proofs that the observational tuple $\uO$ admits a description in terms of the theoretical tuple $\uT$.

This formalism allows to speak about particular worlds, obtained as presheaf categories over observational $\clO$; if $\clT, \clO$ is a theoretical pair, we can instantiate \autoref{nervereal} above in the particular case where $\clW = [\clO^\op, \Set]$ (observe that in this case $\clW$ is a Yaldabaoth). We can thus address a certain number of questions, arising from the canonical adjunction obtained by virtue of \autoref{equ_prof_cocont}:% and \autoref{}:
\[
	\xymatrix{ [(\clO^m)^\op, \Set] \ar@<3pt>[r] & \ar@<3pt>[l] [(\clT^n)^\op, \Set];}
\]
It is worth to mention that since the diagram
\[
	\vcenter{\xymatrix{
			(\clO^m)^\op \ar[rr]\ar[dr]&& [(\clT^n)^\op, \Set] \ar[dl]\\
			& [(\clO^m)^\op, \Set]
		}}
\]
is pseudocommutative, the composition $L\circ y$ s equal to (the mate of) $\fkK$. This means: $\clO$-models, when interpreted inside $\clT$-models, carry representations corrisponding to the observational tokens interpreted in $\clT$-models; that is, the representation is coherent over observational tokens, that is\dots
\begin{remark}[Ramseyfication and translation functors]\label{carnap_translation_functors}
	Assume that there exists an adjunction 
	\[ 
		F : \clO \leftrightarrows \clT : G
	\]
	between the theoretical and the observable. Following Carnap, we might assume that $F : \clO \hookleftarrow \clT$, and thus $G$ is a right translation functor for $(\clT, \clO)$.

	In these assumptions, given a higher Ramsey map $\fkK : \clO \times \clT \to \Set$ the function term
	\[\lambda \uO\uX . \fkK(\uO, \uX)\]
	can be pre-composed with $G$ obtaining
	\[\lambda \uT.\fkK(F\uT, \uT)\]
	whenever there is an adjunction $F : \clO \leftrightarrows \clT : G$ between the theoretical and the observable. We say that a translation adjunction $(F,G)$ is `$\fkK$-admissible' when there is an isomorphism $\fkK(F\uT,\uT')\cong\fkK(\uT,G\uT')$.

	% This, together with the fact that $F\dashv G$ iff $F^*\cong G_*$ iff $G^*\cong F_*$, suggests a fruitful intuition: in presence of a `botched isomorphism' between observational and theoretical, witnessed by the adjunction $(F,G)$, we consider the theoretical trace left by the (image under $F$ of the) observational tokens, so that the dependency of $\fkK$ from $\uT$ is `eliminated' by means of the adjunction.

	% Clearly, the opposite procedure is possible: the adjunction $(F,G)$ allows to consider the observational trace left by the image of a theoretical token $\uT$ under $G$, so that
	% \[ \exists \uX . \fkK(\uO, \uX) \equiv \lambda \uO.\fkK(G\uT, \uT) \]
\end{remark}
\input{secs/section08.tex}

\end{document}