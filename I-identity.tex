\documentclass{amsart}

\makeatletter
\def\@settitle{\begin{center}%
  \baselineskip14\p@\relax
  \bfseries
  \uppercasenonmath\@title
  \@title
  \ifx\@subtitle\@empty\else
     \\[1ex]\uppercasenonmath\@subtitle
     \footnotesize\mdseries\@subtitle
  \fi
  \end{center}%
}
\def\subtitle#1{\gdef\@subtitle{#1}}
\def\@subtitle{}
\makeatother

\newcommand{\true}{\texttt{t}}
\def\id{\text{id}}
\author{Fouche and Denta}
\title{Categorical ontology I}
\subtitle{Identity and Yoneda lemma}
\usepackage{fouche}

\setlength{\epigraphwidth}{.6\textwidth}
\begin{document}
\maketitle
\begin{abstract}
  As it stands, Yoneda lemma allows for a decent solution of the identity problem in ontology. Identity is not a primitive concept; instead it is a relational, context-dependent notion. Identity and existence through time is then solved by topos theory. As both a test-bench for the language and proof of concept we offer a possible topos-theoretic solution of the famous paradox of the nine coins. In this perspective, there is no need to agglomerate the category of Being into a unique subject.
\end{abstract}
\section{Introduction}
\epigraph{\textsc{Eadem svnt qvorvm vnvm potest svbstitvi alteri per transmvtationem}}{Leibniz, if he knew category theory}
Serve un paragrafo che spieghi diffusamente dove prendiamo le categorie, e di conseguenza da dove deriviamo il metalinguaggio in cui ha senso la nozione di uguaglianza (metalinguistica).
\section{Preliminaries on variable set theory}
In some of our proofs it will be crucial to treat the category of functors $I \to \Set$ and the slice category $\Set/I$ (see \ref{}) on the same ground; once the following result is proved, we freely refer to each of these two categories as \emph{variable sets}.
\begin{proposition}
Let $I$ be a set, regarded as a discrete category; let $\Set/I$ the slice category. Then, there is an equivalence (actually, an isomorphism) between these two categories. 
\end{proposition}
\begin{proof}
  Let us give a very hands-on proof: consider an object $F_h : X\to I$ of $\Set/I$, and define a function as $i\mapsto h^\leftarrow(i)$; of course, since $I$, being a set, is a discrete category, this sets up a functor $I \to \Set$. 

  In the opposite direction, let $F : I \to \Set$ be a functor. This defines a function $h_F : \coprod_{i\in I}Fi \to I$, where $\coprod_{i\in I} Fi$ is the disjoint union of all the sets $Fi$.
  The claim follows from the fact that the correspondences $h\mapsto F_h$ and $F\mapsto h_F$ are mutually inverse.
  
  This is however obvious: the function $F_{h_F}$ sends $i\in I$ to the set $h_F^\leftarrow(i)=Fi$, and the function $h_{F_h} \in \Set/I$ has domain $\coprod_{i\in I}F_h(i) = \coprod_{i\in I}h^\leftarrow(i)=X$ (as $i$ runs over the set $I$, the disjoint union of all preimages $h^\leftarrow(i)$ equals the domain of $h$, i.e. the set $X$).
\end{proof}
\begin{remark}
  A more abstract look at this result regards it as a particular instance of the \emph{Grothendieck construction} (see \cite{}): for every small category $\clC$, the category of functors $\clC\to\Set$ is equivalent to the category of \emph{discrete fibrations} on $\clC$ (see \ref{}). In this case, the domain $\clC=I$ is a discrete category, so that all functors $\clE \to I$ are discrete fibrations.
\end{remark}
The next crucial step of our analysis is the observation that the category of variable sets is a topos: we break the result into the verification of the various axioms, as given in \cite{}.
\begin{proposition}
The category of variable sets is cartesian closed.
\end{proposition}
\begin{proof}
  We shall first show that the category of variable sets admits products: this is obvious in $\Set/I$, products are precisely pullbacks; note that theorem \ref{} gives an identification
  \[\vcenter{\scriptsize\xymatrix@!=1mm{
& X\times_I Y \ar[dd]^h \ar@[lightgray][dr]\ar@[lightgray][dl]&  \\
{\color{lightgray} X} \ar@[lightgray][dr]_{\color{lightgray} f}&& {\color{lightgray} Y} \ar@[lightgray][dl]^{\color{lightgray} g}\\
& I &
  }}\iff i\mapsto h^\leftarrow(i) = \Big\{(x,y) \in X\times_I Y \mid h(x,y)=i\Big\}\]
  and given the universal property of a pullback, this yields a canonical bijection $h^\leftarrow(i)\cong f^\leftarrow(i)\times g^\leftarrow(i)$. This is exactly the definition of the product of the two functors $F_f, F_g : I\to \Set$.

  Next, we shall show that each functor $\firstblank \times_I Y$ has a right adjoint $Y \pitchfork_I\firstblank$. The functor $\Set^I \to \Set^I : Z\mapsto Y\pitchfork_I Z$ where $Y\pitchfork_I Z : i \mapsto \Set(Y_i, Z_i)$ does the job. This sets up the bijection 
  \[\begin{array}{c}
    \xymatrix{X\times_I Y \ar[r] & Z} \\ \hline
    \xymatrix{X \ar[r] & Y\pitchfork_I Z}
  \end{array}\]
  and by a completely analogous argument (the functor $\firstblank\times_I\secondblank$ gives a symmetric monoidal structure to variable sets), 
  \[\begin{array}{c}
    \xymatrix{X\times_I Y \ar[r] & Z} \\ \hline
    \xymatrix{Y \ar[r] & X\pitchfork_I Z;}
  \end{array}\]
  this concludes the proof that the category of variable sets is cartesian closed.
\end{proof}
\begin{proposition}
The category of variable sets has a subobject classifier.
\end{proposition}
\begin{proof}
  Recall from \ref{} that we shall find a variable set $\Omega$ such that there is a bijection
  \[\begin{array}{c}
    \xymatrix{A \ar[r] & \Omega} \\ \hline
    \textsf{Sub}_I(A)
  \end{array}\]
  where $\textsf{Sub}_I(A)$ denotes the set of isomorphism classes of monomorphisms into $A$, in the category of variable sets.\footnote{A monomorphism into $A$ as an object of $\Set^I$ is nothing but a family of injections $s_i : S_i \to A_i$; a monomorphism in $\Set/I$ is a set $S$ in a commutative triangle 
  \[\scriptsize
  \xymatrix@!=1mm{S\ar[rr]\ar[dr]_s && A\ar[dl]^a \\ &I&}\]}

  In order to find such an object, we look at what shape shall $\Omega$ have, and what role its universal property plays in its characterization:
  \begin{itemize}
    \item 
    \item 
  \end{itemize}
  For the sake of simplicity, for the rest of the proof we fix as category of variable sets the slice $\Set/I$.

  From this we make the following guess: as an object of $\Set/I$, $\Omega$ is the object $\pi_I : I\times \{0,1\} \to I$. We are thus left with the verification that $\pi_I$ has the correct structure and universal property.

  First, we shall find a monomorphism $\true : * \to \Omega$ in $\Set/I$, i.e. an injective function $I\to \Omega$ that has $\pi_I$ as left inverse. This generalised element ``chooses the $\lceil$true$\rceil$ truth value'' in $\Omega$. (Evidently, the identity $\id_I : I \to I$ is the terminal object in $\Set/I$.)

  It turns out that the function $I \to I\times \{0,1\}$ plays the r\^ole of $\true$: indeed, given a monomorphism $S \hookrightarrow A$ the commutative square 
  \[
  \vcenter{\xymatrix{
    S\ar[d]\ar[r] & I \ar[d]^{\true}\\
    A \ar[r]_{\chi_S} & I\times \{0,1\}
  }}  
  \]
\end{proof}
\begin{proposition}
The category of variable sets is cocomplete and accessible.
\end{proposition}
\begin{proof}
  
\end{proof}
Accessibility is a corollary of Yoneda in the following form: every $F : I \to \Set$ is a colimit of representables
\[
F \cong \colim\Big(\clE(F) \xto{\Sigma} I \xto{y} \Set^I\Big)  
\]
($\clE(F)$ is small because in this case $\clE(F)\cong\coprod_{i\in I}Fi$).
\begin{corollary}
  The category of variable sets is a Grothendieck topos.
\end{corollary}
\section{The internal language of variable sets}
\epigraph{I am hard but I am fair; there is no racial bigotry here. [\dots\unkern] Here you are all equally worthless.}{GySgt Hartman}
\section{Nine copper coins}
\epigraph{Explicaron que una cosa es \emph{igualdad}, y otra \emph{identidad}, y formularon una especie de \emph{reductio ad absurdum}, o sea el caso hipotético de nueve hombres que en nueve sucesivas noches
padecen un vivo dolor. ¿No sería ridículo -interrogaron- pretender que ese dolor es el
mismo?}{JLB ---Tl\"on, Uqbar, Orbis Tertius}
\end{document}