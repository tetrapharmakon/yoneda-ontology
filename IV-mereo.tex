\documentclass{amsart}

\makeatletter
\def\@settitle{\begin{center}%
  \baselineskip14\p@\relax
  \bfseries
  \uppercasenonmath\@title
  \@title
  \ifx\@subtitle\@empty\else
     \\[1ex]\uppercasenonmath\@subtitle
     \footnotesize\mdseries\@subtitle
  \fi
  \end{center}%
}
\def\subtitle#1{\gdef\@subtitle{#1}}
\def\@subtitle{}
\makeatother

\author{Fouche and Denta}
\title{Categorical Ontology IV}
\subtitle{Mereology}
\usepackage{fouche}
\begin{document}
\maketitle
\begin{abstract}
  As it stands, mereology is just the internal language of the category of ordinals/cardinals.

  We outline a theory of ordinals/cardinals in every sufficiently structured category (e.g. in ever topos), to show how subtle the choice of axioms for a mereology can be, in terms of the internal and external properties of the ambient category (two-valued logic, being Boolean, IAC, having an infinite NNO\dots).

  This shall serve as a guidance to study the totality of mereological theories as a whole, instead of focusing on the study now of this, now of that explicit model, with peculiarities and limitations. Ideally, a ``mereological theory'' will be none other than the theory of ordinals in the internal language of a topos, and a ``mereotopological theory'' the theory of ordinals in the internal language of a geometric topos.
\end{abstract}
\section{}
\section{}
\section{}
\section{}
\end{document}
