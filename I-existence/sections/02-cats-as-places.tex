\section{Categories as places}\label{as_places}
\epigraph{\begin{CJK}{UTF8}{bsmi} 故有之以為利,無之以為用。 \end{CJK} \\ 
\footnotesize\emph{---Therefore, profit comes from what is there; usefulness from the empty around.}}{Laozi XI}
The present section has double, complementary purposes: we would like to narrow the discussion down to the particular flavour in which we interpret the word `category', but also to expand its meaning to encompass its r\^ole as a foundation for Mathematics. More or less, the idea is that a category is both an algebraic structure (a microcosm) and a meta\hyp{}structure in which \emph{all} other algebraic structures can be interpreted (a macrocosm).\footnote{As an aside, we shall at least mention the dangers of too much a naive approach towards the micro/macrocosm dichotomy: all known algebraic structures can be interpreted in a category; categories are themselves algebraic structures; there surely must be such a thing as the theory of categories \emph{internal} (i.e., interpreted internally) to a given one. Thus, large categories shall be thought as categories internal to the `meta-'category (unfortunate but unavoidable name) of `all' categories. There surely is a well-developed and expressive theory of internal categories (see \cite[Ch. 8]{Bor1}); but our reader surely has understood that the two `categories', albeit bearing the same name, shall be considered on totally different grounds: one merely is a `small structure'; the other is a foundational object for that, and others, structures.}

In Lawvere's idea, a certain type of category provides their users with a sound graphical representation of the defining operation of a certain type of structure $\fkT$ (see \autoref{unialg} below; we take the word \emph{universal} in the sense of \cite[XV.1]{grillet2007abstract}).

Such a perspective allows to concretely build an object representing a given (fragment of) a language $L$, and a topos obtained as a sort of universal semantic interpretation of $L$ as internal language. In this topos, the structure prescribed by $L$ can be retrieved as a category of `models' of $L$: this construction is a classical piece of categorical logic, and will not be recalled in detail.% (yet, some of the salient features are recalled in our Appendix): the reader is invited to consult \cite[II.12, 13, 14]{lambek1988introduction}, and in particular
\begin{quote}
	J. Lambek proposed to use the \emph{free topos} [on a type theory/language] as the ambient world to do Mathematics in; [\dots\unkern] Being syntactically constructed, but universally determined, with higher-order intuitionistic type theory as internal language, [Lambek] saw [this structure] as a reconciliation of the three classical schools of philosophy of Mathematics, namely formalism, Platonism, and intuitionism. \hfill\cite{free_topos}
\end{quote}
Interpretation, as defined in logical semantics \cite{gamut1991logic}, can be seen as a function $t: L^\star \to K$ that associates elements of a set $K$ to the free variables of a formula $\alpha$ in the language $L^\star$ generated by an alphabet $L$; along with the history of category theory, subsequent refinements of this fundamental idea led to revolutionary notions as that of functorial semantics and internal logic of a topos.
As an aside, it shall be noted that the impulse towards this research was somewhat motivated by the refusal of set-theoretic foundations, as opposed to more type-theoretic flavoured ones.

In the following subsection, we give a more fine-grained presentation of the philosophical consequences that a `meta\hyp{}theoretical structural' perspective has on mathematical ontology.
\subsection{Theories and their models}
In \cite{lajolla} the author W. Lawvere builds a formal language \theory{ETAC} encompassing `elementary' category theory, and a theory \theory{ETCC} for the category of all categories, yielding a model for \theory{ETAC}. In this perspective category theory has a syntax in which categories are just terms. Besides, we are provided with a \emph{meta}theory, in which we can consider categories of categories, etc.:
\begin{quote}
	If $\Phi$ is any theorem of the elementary theory of abstract categories, then $\forall \clA (\clA \models \Phi)$ is a theorem of the basic theory of the category of all categories. \hfill \cite{lajolla}
\end{quote}
After this, the author makes the rather ambiguous statement that `\textit{every object in a world described by basic theory is, at least, a category}'. This is a key observation: what is the world described by \theory{ETAC}, what are its elements?

We posit that the statement shall be interpreted as follows: categories in Mathematics carry a double nature. They surely are the structures in which the entities we are interested to describe organise themselves; but on the other hand, they inhabit a single, big (meta)category of all categories. Such a big structure is fixed once and for all, at the outset of our discussion, and it is the \emph{place} in which we can provide concrete models for `small' categories.

In other words categories live on different, almost opposite, grounds: as -small- syntactic objects, that can be used to model \emph{language}, and as -big- semantic objects, that can be used to model \emph{meaning}.

To fix ideas with a particular example: we posit that there surely is such a thing as `the category of groups'. But on the other hand, groups are just very specific kinds of sets, so groups are but a substructure of `the only category that exists'.

Sure, such an approach is quite unsatisfactory from a structural perspective. It bestows the category of sets with a privileged role that it does not have: sets are just \emph{one} of the possible choices for a foundation of Mathematics. Instead, we would like to disengage the (purely syntactic) notion of structure from the (semantic) notion of interpretation.

This is where Lawvere's intuition comes into play: the `categories as places' philosophy now provides such a disengagement, to approach the foundation of Mathematics agnostically: whatever the semantic universe $\clC$ is, it is just a parameter in our general theory of all possible semantic\emph{s}.

Various research tracks in categorical algebra, \cite{Janelidze2004}, functorial semantics \cite{lawvere1963functorial,hyland2007category}, categorical logic \cite{lambek1988introduction}, and topos theory \cite{JohnstonePT} that characterised the last sixty years of research in category theory fit into this perspective.

Our scope here is to recall the fundamental features of Lawvere's approach, returning, in \autoref{are_universes}, to a careful analysis of the philosophical implications of the `categories as places' principle.

Lawvere's \emph{functorial semantics} (LFS) was introduced in the author's PhD thesis \cite{lawvere1963functorial} to provide a categorical axiomatisation of universal algebra, the part of mathematical logic whose subject is the abstract notion of mathematical structure: a semi-classical reference for universal algebra, mingled with a structuralist perspective, is \cite{manes2012algebraic}; see also \cite{sankappanavar}. Our approach here is classical, but the reader can find modern treatment of LFS in \cite{hyland2007category,curiennone}.%For the sake of completeness, a slightly more technical presentation of the basic ideas of LFS is given in our \autoref{funsemanzi} below; here we aim neither at completeness nor at self-containment.

Everything starts with the following definition:
\begin{definition}\label{unialg}
	A \emph{type $\fkT$ of universal algebra} is a pair $(T,\underline{\alpha})$ where $T$ is a set called the (\emph{algebraic}) \emph{signature} of the theory, and $\underline\alpha$ a function $T \to \bbN$ that assigns to every element $t: T$ a natural number $n_t: \bbN$ called the \emph{arity} of the function symbol $t$.
\end{definition}
\begin{definition}
	A (\emph{universal}) \emph{algebra} of type $\fkT$ is a pair $(A,f^A)$ where $A$ is a set and $f^A$ is a function that sends every function symbol $t: T$ to a function $f^A_t: A^{n_t} \to A$; $f^A_t$ is called the $n_t$-ary operation on $A$ associated to the function symbol $t: T$.
\end{definition}
We could evidently have replaces $\Set$ with another category $\clC$ of our choice, provided the object $A^n: \clC$ still has a meaning for every $n: \bbN$ (to this end, it suffices that $\clC$ has finite products; we call such a $\clC$ a \emph{Cartesian} category). A universal algebra of type $\fkT$ in $\clC$ is now a pair $(A,f^A)$ where $A: \clC$ and $f^A: \prod_{t: T} \clC(A^{n_t},A )$; it is however possible to go even further, enlarging the notion of `type of algebra' even more.

The abstract structure we are trying to classify is a \emph{sketch} (the terminology is neither new nor inexplicable: see \cite{ehresmann1968esquisses,coppey1984leccons, Bor2}) representing the most general arrangement of operations $f^A: A^n \to A$ and properties thereof\footnote{Examples of such properties are (left) alternativity: for all $x,y,z$, one has $f^A(x,f^A(x,y)) = f^A(f^A(x,x),y)$; associativity: $f^A(x,f^A(y,z)) = f^A(f^A(x,y),z)$; commutativity: $f^A(x,y)=f^A(y,x)$; and so on.} that coexist on an object $A$; such a sketch is pictorially represented as a (rooted and directed) graph, modeling arities of the various function symbols determining a given type of algebra $\fkT$ (see also \cite[XV.3]{grillet2007abstract} for the definition of \emph{variety of algebras}).

The main intuition of Lawvere's \cite{lawvere1963functorial} was that these algebras of type $\fkT$ can be described through category theory, by means of a syntax-VS-semantics dialectic opposition: to every algebraic theory $\fkT$ we can attach a category that realises the set of abstract operations $t\in T$ as a certain graph, and consequently as a category. The category $\clL_\fkT$ is the \emph{theory} associated to $\fkT$; in a suitable sense, $\fkT$ is generated by a single object $[1]$ and its iterated powers $[n] := [1]^n$.

To this category one can attach a category of \emph{models}, that realise every possible way in which the abstract structure of $\clL_\fkT$ can be interpreted in a concrete set: a model for $\clL_\fkT$ is just a functor
\[
	M : \clL_\fkT \to \Set
\]
that strictly preserves products, and thus is completely determined by its action on $[1]$: each $M[n]$ is indeed $M[1]^n$.

% Given the `theory' $\fkT$ and the graph $G_{\fkT}$ that it represents, the category $\mathcal{L}_{\fkT}$ generated by $G_\fkT$ `is' the theory we aimed to study, and every functor $A: \clL_{\fkT} \to \Set$ with the property that $A([n+m]) \cong A[n] \times A[m]$ concretely realises via its image a \emph{representation} of $\clL_\fkT$ (and thus of $\fkT$) in $\Set$.
\begin{remark}\label{rmk_explicit_theoer}
	More concretely, there is a `theory of groups'. Such a theory determines a graph $G_{\cate{Grp}}$ built in such a way to generate a category $\clL=\clL_{\cate{Grp}}$ with finite products. \emph{Models} of the theory of groups are functors $\clL \to \Set$ uniquely determined by the image of the `generating object' $[1]$ (the set $G=G[1]$ is the underlying set, or the \emph{carrier} of the algebraic structure in study; in our \autoref{unialg} the carrier is just the first member of the pair $(A,f^A)$); the request that $G$ is a product preserving functor entails that if $\clL$ is a theory and $G: \clL \to \Set$ one of its models, we must have $G[n]=G^n = G \times G \times\dots\times G$, and thus each function symbol $f: [n]\to [1]$ describing an abstract operation on $G$ receives an interpretation as a concrete function $f: G^n \to G$.
\end{remark}
Until now, we interpreted our theory $\clL$ in sets; but we could have chosen a different category $\clC$ at no additional cost, provided $\clC$ was endowed with finite products, to speak of the object $A^n = A\times \cdots\times A$ for all $n: \bbN$. In this fashion, we obtain the $\clC$-models of $\clL$, instead of its $\Set$-models: formally, and conceptually, the difference is all there.

Yet, the freedom to disengage language and meaning visibly has deep consequences: suddenly, and quite miraculously, we are allowed to speak of groups internal to the category of sets, i.e. functors $\clL_{\cate{Grp}} \to \Set$, topological groups, i.e. functors $\clL_{\cate{Grp}} \to \cate{Top}$ (so multiplication and inversion are continuous maps \emph{by this very choice}, without additional requests); we can treat monoids in the category of $R$-modules, i.e. $R$-algebras \cite[IV]{book337527}, and monoids in the category of posets (i.e. quantales \cite{Paseka2000}) all on the same conceptual ground.

It is evident that $\clL_{\fkT}$ is the same in all our choices; all that changes is the semantic universe in which the theory acquires meaning. This context-dependent definition of `structure' inspires our stance towards ontology, as the reader will notice throughout our \autoref{sec_coins} and \autoref{vistas}.

\subsection{The r\^ole of toposes}
Among many different Cartesian categories in which we can interpret a given theory $\clL$, toposes play a special r\^ole; this is mostly because the \emph{internal language} every topos carries (in the sense of \autoref{da_lang}) is quite expressive.

To every theory $\clL$ one can associate a category, called the \emph{free topos} $\clE(\clL)$ on the theory (see \cite{lambek1988introduction}), such that there is a natural bijection between the $\clF$-models of $\clL$, in the sense of  \autoref{rmk_explicit_theoer}, and (a suitable choice of) morphisms of toposes\footnote{We refrain to enter the details of the definition of a morphism of toposes, but we glimpse at the definition: given two toposes $\clE,\clF$ a morphism $(f^*,f_*): \clE \to \clF$ consists of a pair of \emph{adjoint} functors (see \cite[3]{Bor1}) $f^*: \clE \leftrightarrows \clF: f_*$ with the property that $f^*$ commutes with finite limits (see \cite[2.8.2]{Bor1}).} $\clE(\clL) \to \clF$:
\[\cate{Mod}(\clL, \clF) \cong \hom(\clE(\clL), \clF).\]
In the present subsection we analyse how the construction of models of $\clL$ behaves when the semantics takes value in a category of presheaves.% (cf. \cite[??]{Bor1}).

Let's start stating a plain tautology, that still works as blatant motivation for our interest in toposes opposed as more general categories for our semantics. Sets can be canonically identified with the category $[1,\Set]$, so models of $\clL$ are tautologically identified to its $[1,\Set]$-models. It is then quite natural to wonder what $\clL$-models become when the semantics is taken in more general functor categories like $[\clC,\Set]$. This generalisation is compelling to our discussion: if $\clC=I$ is a discrete category, we get back the well-known category of variable sets $\Set/I$ of \autoref{variabbo_set}.

Now, it turns out that $[\clC,\Set]$-model for an algebraic theory $\clL$, defined as functors $\clL \to [\clC,\Set]$ preserving finite products, correspond precisely to functors $\clC\to \Set$ such that each $FC$ is a $\clL$-model: this gives rise to the following `commutative property' for semantic interpretation:
\begin{quote}
	$\clL$-models in $[\clC,\Set]$ are precisely those models $\clC \to \Set$ that take value in the subcategory $\cate{Mod}_{\clL}(\Set)$ of models for $\clL$. In other words we can `shift' the $\cate{Mod}(-)$ construction in and out $[\clC,\Set]$ at our will:
	\[
		\cate{Mod}_{\clL}([C,\Set]) \cong [C, \cate{Mod}_{\clL}(\Set)]
	\]
\end{quote}
As the reader can see, the procedure of interpreting a given `theory' inside an abstract finitely complete category $\clK$ is something that is only possible when the theory is interpreted as a category, and when a model of the theory as a functor. This part of Mathematics goes under many names: the one we will employ, i.e. \emph{categorical}, or \emph{functorial}, semantics \cite{lawvere1963functorial}, but also \emph{internalisation} of structures, \emph{categorical algebra}.

The internalisation paradigm sketched above suggests how `small' mathematicians often happily develop their Mathematics without ever exiting a single (large) finitely complete category $\clK$, without even suspecting the presence of models for their theories outside $\clK$. To a category theorist, `groups' as abstract structures behave similarly to the disciples of the sect of the Phoenix \cite{fenix}: `the name by which they are known to the world is not the same as the one they pronounce for themselves.' They are a different, deeper structure than the one intended by their users.

By leaving the somewhat unsatisfying picture that `all categories are small' and by fixing a semantic universe like $\Set$), every large category works as a world in which one can speak mathematical language (i.e. `study models for the theory of $\Omega$-structures' as long as $\Omega$ is one of our abstract theories).

So, categories truly exhibit a double nature: they are the theories we want to study, but they also are the \emph{places} where we want to realise those theories; looking from high enough, there is plenty of other places where one can move, other than the category $\Set$ of sets and functions (whose existence is, at the best of our knowledge, the consequence of a postulate; we similarly posit that there exists a model for \theory{ETAC}). Small categories model theories, they are \emph{syntax}, in that they describe a relational structure using compositionality; but large categories offer a way to interpret the syntax, so they are a \emph{semantics}. Our stance is that a large relational structure is fixed once and for all, lying on the background, and allows for all other relational structures to be interpreted.

It is nearly impossible to underestimate the importance of this disengagement: syntax and semantics, once separated and given a limited ground of action, acquire their meaning.\footnote{Of course, this is dialectical opposition at its pinnacle, and certainly not a sterile approach to category theory; see \cite{lawvere1996unity} for a visionary account of how `dialectical philosophy can be modelled mathematically'.}

More technically, in our \autoref{da_lang} we recall how this perspective allows interpreting different kinds of logic in different kinds of categories: such an approach leads very far, to the purported equivalence between different flavours of logic\emph{s} and different classes of categories; the particular shape of semantics that you can interpret in $\clK$ is but a reflection of $\clK$'s nice categorical properties (e.g., having finite co/limits, nice choices of factorisation systems \cite[5.5]{Bor1}, \cite{FK}, a subobject classifier; or the property of the posets $\cate{Sub}(A)$ of subobjects of an object $A$ of being a complete, modular, distributive lattice\dots).

In light of \autoref{da_lang}, this last property has to do with the internal logic of the category: propositions are the set, or rather the \emph{type}, of `elements' for which they are `true'; and in nice cases (like e.g. in toposes), they are also arrows with codomain a suitable \emph{type of truth values} $\Omega$).
\subsection{Categories are universes of discourse}\label{are_universes}
Somehow, the previous section postulates that category theory as a whole is `bigger than Mathematics itself', and it works as one of its foundations; a category is just (!) a totality where all Mathematics can be re-enacted; in this perspective, \theory{ETAC} works as a meta\hyp{}language in which we develop our approach. This is not very far from current mathematical practice, and in particular from Mac Lane's point of view expressed in
\begin{quote}
	We [can think] `ordinary' Mathematics as carried out exclusively within [a universe] $U$ (i.e. on elements of $U$) while $U$ itself and sets formed from $U$ are to be used for the construction of the desired large categories.\hfill \cite[I.6]{McL}
\end{quote}
since the unique large category we posit is `the universe'. But this approach goes further, as it posits that \emph{mathematical theories are in themselves mathematical objects}, and as such, subject to the same analysis we perform on the object of which those theories speak about.

Such an approach has many advantages:
\begin{itemize}
	\item If ontologies are mathematical objects, the discipline can be approachable with a \emph{problem-solving} attitude: answers to its problems are at least to some extent quantitatively determined; cf. our \autoref{sec_coins} for a tentative roundup of examples in this direction. There is an undeniable \emph{computational} content is such discussion, that is absent from ``usual'' treatments of this matter. The only explanation is that when a more expressive, objective language is adopted, ontology acquires a quantitative content.
	\item The possibility of reading theories in terms of relations allows to suspend ontological commitment on the nature of objects. They are given, but just as embedded in a relational structure; this relational structure, a category modelling the ontology in study, is the subject of our discourse, and not the entities themselves.

	      This perspective is sketched in our \autoref{vistas}. This allows a sharper, more precise, and less time-consuming conceptualisation process for ontology's subjects of study; if it is regarded as (the internal language of) a category, ontology suddenly becomes a \emph{context-dependent} entity, i.e. dependent on the relational structure in which it is located. There are many categories, each of which describes the web of relations to which the objects thereof are subject.\footnote{The tongue-in-cheek is inevitable here, but also explicative: shifting our attention from objects as monadic particles to the whole category they are embedded into, relations become more important than relata: \emph{objects} become \emph{subject} to mutual interdependence.}
	\item Weak structuralism, i.e. a structuralist view kept at an informal, meta\hyp{}theoretical level, is more than enough to appreciate the merits of seeing objects as interacting \emph{relata} and not just as monadic particles.

	      To put it simply, one isn't forced to unconditionally adopt a hardcore\hyp{}structuralist foundation such as e.g. \emph{structural realism} \cite{bain2013category,eva2016category} to profit from category theoretic arguments; a similar attitude towards foundations is widespread in mathematical practice: many people do \textit{structural Mathematics} without even knowing, or worrying about, the subtleties of what a purely structuralist foundation implies (cf. \autoref{weak_structuralism}). And this for the simple reason that structural Mathematics `just \emph{works}': structural theories are easier to grok, they are modular, resilient to change, concise, easier to explain, they outline hidden patterns.
\end{itemize}
Our main point is that the majority of, if not all, theories of existence seem burdened by the impossibility to disentangle existence from a `context of existence' bounding \emph{in what sense}, and \emph{relative to what} things `are'.

But now, the super-exponential progress of Mathematics along the 20$^\text{th}$ century can only be explained by the general, collective promotion of the belief that things do not `exist' in themselves; instead, they do as long as they are meaningful bricks of relational structures, defining, and defined by, said building blocks.

As extremist as it may seem, our position fits into an already developed open debate; see for example the work of J.P. Marquis:
\begin{quote}
	[...] \emph{to be} is \emph{to be related}, and the `essence' of an `entity' is given by its relations to its `environment'.
	\hfill \cite{Marquis1997}
\end{quote}
See also
\begin{quote}
	this reproach is empty and one tries to explain the clearer by the more obscure when giving priority to ontology in such situations [...]. Structure occurs in dealing with something and does not exist independently of this dealing. \cite{kromer2007tool}
\end{quote}
Sure, structuralism is just a single philosophical current among many others: but if you do Mathematics, if you study its shape, its history and its foundations, it is `more right than others' since it profoundly reshaped, alone, the face of the discipline, unraveling it as the most prominent, most efficient factory of epistemic knowledge humankind has ever produced.

In our system, ontology does not pose absolute questions, and is contempt with equally partial answers, not as a defeat: relativity of existence is just inevitable to avoid incurring in annoying paradoxes coming from the universalist desire to speak about `all that there is'.
In this perspective something certainly is lost: as Solomonic as the claim might seem, we posit what is lost is just a confusing tangle of misunderstandings, caused by the habit of blurring natural and formal languages. Again, a fortunate analogy still involves foundations of Mathematics: bounding the existence of sets to a hierarchy of universes undoubtedly loses something. But what is this something? How can it be probed? In no way: and so, in what sense it `exists', if it is outside the hierarchy of probe-able entities?

Sure, positing that `there is just what hits my probes' might seem as a defeat, and the question `what is there?' remains a fertile area of discussion. We aim at adding but a grain of sand to this infinite seaside of ideas.

% The conceptual tools of category theory helped mathematical practice and has inadvertently inspired an epistemological and ontological quest about its objects of study. Someone might object that we should first know what is a structure before working with it; Kr\"omer's reply seems particularly compelling:
% \begin{quote}
%     this reproach is empty and one tries to explain the clearer by the more obscure when giving priority to ontology in such situations [...]. Structure occurs in dealing with something and does not exist independently of this dealing. \cite{kromer2007tool}
% \end{quote}
Category theory was invented without worrying about foundational issues; they were addressed just later. It effectively and irredeemably changed the face of Mathematics; we now claim it provides a `practical foundation' for meta\hyp{}physical problems.% For pragmatism and what it means in category theory, see
% \begin{quote}
%     that structural Mathematics is characterized as an activity by treatment of things as if one were dealing with structures. From the pragmatist viewpoint, we do not know much more about structures than how to deal with them, after all. \hfill \cite{kromer2007tool}
% \end{quote}
\subsection{Existence: persistence of identity?} \label{existence}
If, following Quine, ontology studies `what there is', it then appears natural to define at the outset of our system the notion of \emph{existence}.

Many approaches are possible to tackle this complicated and fundamental matter:
\begin{itemize}
	\item in the Fregean perspective \cite{Frege} `$x$ exists' if and only if `$x$ is identical to something [e.g., to itself]', whereas
	\item for Quine \cite{Qui53} `being is being the value of a [bounded] variable'.
\end{itemize}
The first approach is affine to conceptual realism, but it has scarce informative power (what is equality? Frege defines existence as self-identity, an equally perilous concept, and even more, this is not the right track: identity isn't a thing; it is a \emph{judgment}. More: every identity is the result of an identification process, every equality is an equivalence relation performed \emph{onto} a pre-existing `thing' --of course, regarding existents as the result of a computation relies on the notion of Bishop set \cite{Bis67,hofmann2012extensional}): before the identification, a Bishop set $(S,\sim)$ has elements that are morally more than the elements of the quotient $S/\!\sim$. Evidently, turning this suggestion into a precise statement is complex enough a topic to dedicate it a separate chapter of the present polyptych (in fact, two: \cite{black,homot}). We will attribute to the topic due importance, in due course.

Quine's approach inspired the notion of \emph{ontological commitment} (the class of objects that `we admit to exist' when we talk about ontology, or we participate in it) and the subsequent definition of ontology (of a theory) as the domain of objects on which logical quantifiers vary (cf. \cite{Qui53}): each ontological theory is committed to the entities on which the quantifiers of its statements vary.

We will see in \autoref{metaon} that the Quinean conception fits in our categorical ontology perspective as a consequence of a technical procedure called \emph{internalization} of theories.

There is a third way, less common in literature but seemingly less questionable: define existence through \emph{persistence over time}.

A \emph{time-frame} is a pair $\la T, <\ra$ where $T$ is a inhabited type whose terms are `instants' and $<$ a binary partial order relation. Every `temporal type' $X$ is endowed with a relation $\inplus$, defined as $X\times T \to\Omega : (x, t) \mapsto x \mathrel \inplus t $ that defines when a term $x$ `exists in the time-frame $T$': the predicate $x\mathrel\inplus t$ is constantly true as $t:T$ varies.

This definition lends itself to all sorts of specialisations ($x$ has always existed; $x$ ceased to exist; $x$ esisted before/after $t_0$\dots), and it captures well one of the intuitive aspects of existence, relying on both identity (with all the problems this might raise) and time, thus framed in an appropriate temporal logic $\sf *TL$ in which entities `persist'. (It's relatively easy to write down what the relationship `$x$ exists in $T$' means in terms of linear temporal logic, $\sf LTL$.)

One of the results of our work is that we can define existence in an equally intuitive way, but without appealing neither to an identity principle nor to a time-frame reference; in fact providing a more general concept within which others can also be retrieved.

We will phrase the question as follows: what is variable relative to $x$ is the degree, or \emph{strength} of its existence; `classical' existence can be considered existence `at the highest degree' in the internal language of the category we live in (in our \autoref{sec_coins}, a Borgesian universe); there we will be able to indicate the strength of existence of objects, without assuming to move through instants of time (as in the example of the nine copper coins of \cite{tlonEN} might suggest, cf. \autoref{bla}) or points in space (such as \emph{hr\"onir}, cf. \autoref{blu}).

Depending on the structure of the domain, we can choose the logic that is most suited to the context for the intuition we have of the universe in natural language. Persistence over time is therefore not removed from the description, or denied; rather, it becomes an implicit part of it.

Existence in this conception is nothing more than the `mode of presence' of the objects within a relational model. It is therefore literally defined by what we can \emph{do} with objects, how we can fit them in a sensible web of relations.

This is not just an operational notion of existence, close to our common sense: things exist if we can touch them, see them, indirectly measure them if invisible, based on interactions they have with observable entities, describe them, count them, use them; and this is independent of \emph{how} they exist; rather, our possibility to probe them defines \emph{that} they exist.

So, things exist since they belong to a category; based on our meta\hyp{}theoretical and foundational choices, existence concerns how things relate to each other. Here is the advantage of categorical thinking in a nutshell: to exploit structuralism to be able to bind not only our world but distant realities such as \tlon's, which is but an example of an element in an \emph{ontology scheme}.

The notion of existence on \tlon will presumably be different from our own:
\begin{quote}
	Things became duplicated in \tlon; they also tend to become effaced and lose their details when they are forgotten. \hfill\cite{Borges1963}
\end{quote}
such obvious and powerful intuition has often been belittled by ontologists. But once the idea is properly framed, it is possible to measure this difference precisely. Changing ontology changes the concept of \emph{what is there} and \emph{how it's there}; we believe our work explicitly clarifies to what extent this happens, through mathematical language.

Mathematical language happens to be able to quantify how much Earth and \tlon are different, what is the parameter in the `furniture' of their respective worlds that has been hacked in order to admit the existence of impossible entities like group of coins that persist even though no one is observing them.

In the next section, we begin to lay the foundation of mathematical language needed in §\ref{sec_coins}.