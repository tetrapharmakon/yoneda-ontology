\section{A universal notion of theory} 
%\color{blue!40}

%\subsection{The Dummett-Plantinga problem}
%La manfrina su semantica pura e applicata, applicazione concreta del framework, problema delle modalities, trattare semantica applicata (tipo Lewis) come \emph{teoria} nel senso funtoriale evitando ontological committment bla bla bla (qualche riferimento su sta cosa, remember)

%\subsection{Naturalizing Epistemology}
%cenni alla questione della incommensurabilità delle teorie, sia tesi Duhem-Quine che accezione radicale di Feyerabend, e come questo framework la risolve (creazione di un linguaggio-ter che non è nè quello del "testo" di partenza nè quello del "testo" di arrivo).
%\color{black}
Come si è visto in \autoref{resoudre_la_tension} l'approccio profuntoriale all'epistemologia altro non è che un approccio relazionale: permette di guardare alle teorie scientifiche in termini di aggiunzioni con la parte di mondo (non necessariamente fisico) sulla quale vorrebbero 'doing theory' (fare predizioni etc). Le teorie non si distinguono le une dalle altre per via degli oggetti di cui trattano ma per le relazioni che intercorrono con le altre categorie sintattiche con le quali si confrontano. 

Questo comporta che:
\begin{itemize}
    \item Non c'è sostanziale distinzione tra categorie sintattiche come $\clT$ e $\clO$, cioè tra mondo e teorie, o meglio tra termini 'osservativi' e 'puri'. Questo non è un passo indietro perchè è un ostacolo nel quale ci si era imbattuti in precendenza, come visto in Section 1. I tentativi primonovecenteschi di 'naturalizzare l'epistemologia', originariamente addirittura \emph{nella} sintassi, come Ramsey, si scontravano con la consapevolezza di non poter mai uscire dalla teoria, cioè dal linguaggio. Il che non implica che non si stia parlando del mondo o della realtà (qualunque cosa sia) ma che parlare di teorie, e fare metateoria su teorie e mondo, è una operazione innanzitutto linguistica (sintattica e semantica)
    \item La categoria osservazionale non ha un accesso privilegiato alla categoria mondo. I due vocabolari che formano $\clW$ sono interscambiabili (e arbitrari in certa misura). Non solo $\clO$ 'dice' di $\clT$ tante cose quanto il viceversa (as usual in scientific practice) ma è anche vero che la categoria osservazionale può essere un'altra categoria 'teorica': un mondo possibile, come in narratologia, o un fenomeno inosservabile, come può avvenire egualmente in fisica o in teologia. Una teoria può voler parlare di una porzione di $\clW$ che non rientra nella classe delle tradizionali osservazioni empiriche, privilegiate dagli studiosi che per primi, per ragioni storiche, hanno avviato questi studi. 
    \item Una 'teoria delle teorie' non sarà object-dependent ma context-dependent, dove context è l'insieme delle aggiunzioni che si stabiliscono tra le categorie dentro le quali scegliamo di muoverci. 
\end{itemize}

Una teoria è qualcosa, perciò, di più largo, per noi, di quello che risulta essere di norma in epistemologia. è un oggetto sintattico descrittivo e predittivo su qualunque porzione di mondo ci si voglia occupare.

Stabilire una demarcazione tra teorie scientifiche e non in questo contesto è vano: una teoria biologica non è necessariamente più 'predittiva' o 'reale' di una ontologica \footnote{O, ancora meglio, non ha senso porsi la domanda in questi termini}: dipende da cosa avviene dentro $\clW$. 

La 'scientificità' di una teoria non è qualcosa di indagabile dal punto di vista della concezione semantica, ammesso che sia una questione indagabile e non un reperto storico. Ma ragionare in a categorical perspective aiuta a dotarsi di una struttura universale, che non gerarchizza le teorie o le loro sentence-versions.   
