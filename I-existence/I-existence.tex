\documentclass{amsart}

\usepackage{fouche}
% \usepackage{braket}
\makeatletter
\def\@settitle{\begin{center}%
  \baselineskip14\p@\relax
  \bfseries
  \uppercasenonmath\@title
  \@title
  \ifx\@subtitle\@empty\else
     \\[1ex]\uppercasenonmath\@subtitle
     \footnotesize\mdseries\@subtitle
  \fi
  \end{center}%
}
\def\subtitle#1{\gdef\@subtitle{#1}}
\def\@subtitle{}
\makeatother

\newcommand{\var}[3][]{
  \left[\begin{smallmatrix} #2 \\
  #1\downarrow \\ #3
  \end{smallmatrix}\right]}
\newcommand{\cvar}[3]{
  \begin{xsmallmatrix}{0em}
  & #1 \\ #2 & \downarrow \\ & #3
  \end{xsmallmatrix}}

\def\la{\langle}
\def\ra{\rangle}
\def\lr#1#2{\la #1,#2\ra}
\def\tr{\textsf{t}}
\newcommand{\true}{\texttt{t}}
\def\id{\text{id}}
\author{Dario Dentamaro}
\address{Dario \textsc{Dentamaro}: }
\email{dio@cane.it}
\author{Fosco Loregian}
\address{%
  Fosco \textsc{Loregian}: %
  Tallinn University of Technology, %
  Institute of Cybernetics, Akadeemia tee 15/2,%
  12618 Tallinn, Estonia}
\email{fosco.loregian@taltech.ee}
\email{fosco.loregian@gmail.com}
\title{Categorical ontology I}
\subtitle{Existence}
\usepackage{proof}

\usepackage{minted}
\def\mil#1{\mintinline{haskell}{#1}}
\newcommand{\po}[1][dr]{\save*!/#1+1.5pc/#1:(1,-1)@^{|-}\restore}
\newcommand{\pb}[1][dr]{\save*!/#1-1.5pc/#1:(-1,1)@^{|-}\restore}


\setlength{\epigraphwidth}{.6\textwidth}
\setcounter{tocdepth}{1}

\usepackage{fontawesome}

\newenvironment{fo}[1]{\color{green!60!black} \textbf{Fouche said}: #1}{}
\newenvironment{de}[1]{\color{green!60!black} \textbf{Denta said}: #1}{}

\newcommand{\topicRule}{\todo[cyan, inline]{}}
\usepackage{booktabs}
\begin{document}
\begin{abstract}
  The present paper is the first piece of a series whose aim is to develop an approach to ontology and metaontology through category theory. We exploit the theory of \emph{elementary toposes} to purport the claim that a satisfying ``theory of existence'', and more at large ontology itself, can be obtained appealing to category theory. In this perspective, an ontology is a mathematical object: a category, the universe of discourse in which our mathematics (intended at large, as a theory of knowledge) can be deployed. The \emph{internal language} that all categories possess prescribes the modes of existence for the objects of a fixed ontology/category.

  Such internal language allows to re-enact propositional calculus: all first-order properties of an object can be modeled by propositions $p : U \to\Omega$ valued in an object whose ``elements'' are the truth values that $p$ can assume. Each ontology $\clE$ now yields its own $\Omega_\clE$, and thus its own set of truth values, and thus its own ``theory of meaning''. This approach resembles, but is more general than, fuzzy logics, as most choices of $\clE$ and thus of $\Omega_\clE$ yield nonclassical, many-valued logics.

  Framed this way, ontology suddenly becomes more mathematical: a solid corpus of techniques can be used to backup philosophical intuition with a useful, modular language, suitable for a practical foundation. As both a test-bench for our theory, and a literary \emph{divertissement}, we propose a possible category-theoretic solution of Borges' famous paradoxes of Tl\"on's ``nine copper coins'', and of other seemingly paradoxical construction in his literary work. We then delve into the topic with some vistas on our future works.
\end{abstract}
\maketitle
  \tableofcontents
  \section{Introduction}\label{sec_intro}
\epigraph{El mundo, desgraciadamente, es real.}{\cite{confutacion}}
This is the first chapter of a series of works aiming to touch a pretty wide range of topics.

Its purpose is to adopt a wide-ranging approach to a fragment of elementary problems in a certain branch of contemporary philosophy of Mathematics. More in detail, we attempt at laying a foundation for solving a number of problems in ontology employing pure Mathematics; in particular, using the branch of Mathematics known as \emph{category theory}.

As authors, we are aware that such an ambitious statement of purpose must be adequately motivated, bounded to a realistic goal, and properly framed in the current state of the art on the matter. This is the scope of the initial section of the present first manuscript.
\subsection{What is this series}
Since forever, Mathematics studies three fundamental indefinite terms: \emph{form}, \emph{measure}, and \emph{inference}. Apperception makes us recognise that there are extended entities in space, persisting in time. From this, the necessity to measure how much these entities are extended, and to build a web of conceptual relations between them, explaining how they arrange `logically'.
Contamination between these three archetypal processes is certainly possible and common: in fact, Mathematics happens exactly at the crossroad where algebra, geometry and logic intersect.

We can even say more: Mathematics is a language engineered to systematically infer properties of the three mentioned indefinite; so, meta-Mathematics done through Mathematics (if such a thing even exists) exhibits the features of a \emph{ur-language}, a generative scheme for `all' possible languages. It is a language whose elements are the rules to give oneself a language, convey information to other selves, and allow deduction. It is a meta-object, a scheme of rules to generate objects/languages.

Taken this tentative definition, Mathematics (not its history, not its philosophy, but its \emph{practice}) serves as a powerful tool to tackle the essential questions of ontology: what things are cogently, what makes them what they are and not different.

Quantitative thinking is a consistent (some would say `honest') way of approaching the deep problem of the cogency of entities; yet, it is undeniable that a certain philosophical debate has become indifferent, or even overtly hostile, to mathematical language. A tear in the veil that occurred a long time ago, due to different purposes and different specific vocabulary, can not be repaired by two people only. If, however, the reader of these notes asks for an extended motivation for our work, a wide-ranging project in which it fits, a long-term goal, in short, a \emph{program}, they will find it now: there is a piece of Mathematics whose purpose is to solve philosophical problems, in the same sense certain Mathematics `solves' the motion of celestial bodies.

It does not annihilate the question: it proposes models within which one can reformulate it; it highlights what is a trivial consequence of the axioms of that model and what instead is not, and requires the language to be expanded, modified, sharpened. We aim to approach this never-mentioned discipline as mathematicians. But we do it without elaborating `new' theorems; we draw connections between the modern mathematical practice to use them in the context of philosophical research.

Sure, solving once and for all the problems posited by ontology would be megalomaniac; we do not claim to have reached such an ambitious objective. More modestly, we propose a starting point unhinging some well-established beliefs;\footnote{Above all else, the belief that ontology is too general to be approached quantitatively, and that it contains mathematical language as a proper subclass: it is instead the exact opposite, as the central idea of our work is that ontologies -there are many- are mathematical objects.} we humbly point the finger at some problems that prose is unable to notice, because it lacks a specific and technical language; we suggest that only in such a language, when words mean precise things and are tools of epistemic research instead of mere magic spells, a few essential questions of recent ontology dissolve in a thin thread of smoke, and others simply become `the wrong question': not false, just meaningless; meaningless as the endeavour to bestow a concrete attribute as a temperature to abstract concept like consciousness or justice.

We shall say at the outset that the ur-language we are tackling is not Mathematics. Yet, mathematical language has an enormous potential to hint at what the elementary particles of the \emph{characteristica universalis} we're looking for should be made of.

It may also seem suspicious to employ Mathematics to tackle questions that traditionally pertain to philosophy: some philosophers believe they are debating about problems more general than Mathematics, from a higher point of observation on the fundamental questions of Being. By proposing the `ontology as categories' point of view, our work aims to dismantle such a false belief.%: each and every debate must appear in a language.

In doing so, we believe we can provide a more adequate language, taken from Mathematics, within which to frame some deep ontological questions, whose potential is often lost in the threads of ill-posed questions and clumsy answers. The reader will allow a tongue-in-cheek here, subsuming our position: in ontology, it is not a matter of making a \emph{correct use of language}, but rather a matter of \emph{using the correct language}.

This correct language is inherently mathematical, as only Mathematics proved to be able to substantiate a qualitative analysis in quantitative terms. This language `must be' \emph{category theory}, as only category theory has the power to speak about a totality of structures of a given kind in a compelling way, treating mathematical \emph{theories} as mathematical \emph{objects}.

As of now, our work unravels in three different chapters, and it will attempt to cover a variety of topics: 
\begin{itemize}
    \item the present manuscript, \emph{Existence}, provides the tools to build a sufficiently expressive `theory of existence' inside a category. This first chapter has a distinctly foundational r\^ole; its scope to build the fundamentals of our tool-set (category theory and categorical logic, as developed in \cite{mac1992sheaves,JohnstonePT,lambek1988introduction}). 
    
    As both a test-bench for our theory and a literary \emph{divertissement}, we propose a category-theoretic solution of Borges' paradoxes present in \cite{Borges1963}. In our final section, we relate our framework to more classical ancient and modern philosophers; we link topos theory to Berkeley's instantaneism and internal category theory to Quine's definition of the [domain of] existence of an entity as a domain of validity of quantifiers (intended as propositional functions, i.e. functions whose codomain is a space of truth values).
    \item A second chapter \cite{black}, currently in preparation, addresses the problem of \emph{identity}, and in particular its context-dependent nature. Our proof of concept here consists of a rephrasing of Black's classical `two spheres' paradox \cite{papear_di_black} in the elementary terms of invariance under a group of admissible transformations; this time the solution is provided by Klein's famous \emph{Erlangen program} group-theoretic foundation for geometry: the two interlocutors of Black's imaginary dialogue respectively live in an Euclidean and an affine world: this difference, not perceived by means of language, affects their understanding of the `two' spheres, and irredeemably prevents them from mutual intelligence.
    \item A third chapter \cite{homot}, currently in preparation, addresses again the problem of identity, but this time through the lens of algebraic topology, a branch of Mathematics that in recent years defied well\hyp{}established ontological assumptions ; the many commonalities between category theory and homotopy theory suggest that `identity' is not a primitive concept, but instead depends on our concrete representation of mathematical entities. This can be formalised in various ways, among many the \emph{Homotopy Type Theory} foundation of \cite{hottbook,cwp}. \cite{homot} aims to be an introduction to the fundamental principles of HoTT \emph{ad usum delphini}: we investigate how when $X,Y : \clC$ are objects in a category, there often is a class of equivalences $W \subseteq \hom(\clC)$ prescribing that $X,Y$ shouldn't be distinguished in the associated ontology; equality is then defined \emph{ex post} in terms of $W$, varying as the ambient category $\clC$ does; this yields a $W$-parametric notion of identity $\equiv_W$, allowing categories to be categorified versions of \emph{Bishop sets}, i.e. pairs $(S,\rho)$ where $\rho$ is an equivalence relation on $S$ prescribing a $\rho$-equality.
\end{itemize}
Our main tenet in the present chapter is that ontolog\emph{ies} are mathematical objects: each ontology is a certain category $\clO$, inside which `Being' unravels as the sum of all statements that the internal language (see \ref{da_lang}) of $\clO$ can concoct. 

Of course, the more expressive is this language, the more expressive the resulting theory of existence will turn out to be. Our presupposition here is that trying to let ontology speak about `\emph{all} that there is' (the accent is on the adverb, on the famous quote of Quine \cite{quine1948there}) can lead to annoying paradoxes and foxholes.

Instead research shall concentrate on clarifying what the verb means: in what sense, `what there is' \emph{is}? What is is-ness? As category theorists, our -perhaps simplistic- answer is that, again paraphrasing Quine, 
\begin{quote}
being is \emph{being the object of a category}.
\end{quote}
Explaining why this is exactly Quine's motto, just shifted one universe higher, is the content of our §\ref{metaon}.
\subsubsection{Structure of the paper}
The remaining part of the first section draws a picture as accurate as possible, of the wheres and whys of structural Mathematics; its implications are the subject of several essays on the philosophy of Mathematics, like \cite{kromer2007tool,Marquis1997,marquis2010category,marquis2008geometrical}. This section has several different purposes: it provides an explicit statement of purposes for the entire polyptych \cite{black,homot}; it declares our stance on the foundation we choose, clarifying assumptions that we feel are usually neglected on essays on the topic (`where are the objects that Mathematics aims to describe? Where is the language by means of which this description is possible?') -of course without claiming to have solved the matter once and for all; we provide pointers as specific as possible in order to help the reader navigate the relevant literature.

The second section delves into the first major point of our presentation: large categories are universes where `Being', intended as the sum of information acquired from the perceptual bundle we experience, that language organises and conceptualises. `Language' here is a shortcut to denote the power of a fixed large category $\clC$ to express well-formed formulas of (a certain fragment of) logic. Objects and morphisms of $\clC$ shall be considered respectively as types and terms of a language, \emph{the} internal language of $\clC$ (see \autoref{da_lang}); now, the richer $\clC$ is, the more it is able to faithfully represent the cosmos we're thrown into. Among many possible choices for $\clC$, we take \emph{toposes} as the class of categories harbouring `set theories': the internal language of a topos is powerful enough to re-enact set theory, and subsequently propositional logic.

The third section deals with a specific example of a topos, useful for later examination: the categories of objects parameterised by a fixed `space of parameters' $I$. The `slice' category $\Set/I$ is a topos, and its internal language, namely its internal logic, is tightly linked to set-theoretic properties of the slicing set $I$. The logic we obtain in $\Set/I$ by casting the general definitions of subobject classifier \ref{eletop} and internal language is genuinely non-classical.

The fourth section contains a careful analysis of the internal language of $\Set/I$.

The fifth section contains an application of the tools we exposed until now: the seemingly paradoxical `nine copper coins' problem exposed in Jorge Luis Borges' \cite{Borges1963}, far from being paradoxical, admits a natural interpretation as a statement in $\Set/I$, for a suitable choice of $I$, and thus of the induced internal logic. We propose other examples of seemingly paradoxical statements in Borges' literary work that instead are admissible statements in the internal logic of \emph{some} topos: on Tl\"on entities may disappear if neglected: this means that $I$ is linearly ordered; Babylon's chaotic lottery resembles, with their obscure, impenetrable purposes, the chaotic behaviour of a dynamical system: this means that $I$ carries a semigroup action; Tl\"on's instantaneism, mimicking/mocking Berkeley, can be obtained assuming $I$ is a discrete, uncountable set.

We close the paper with a section on future development, vistas for future applications, and with a wrap-up of the discussion as we have unraveled so far. Ideally, §\ref{sec_prelim} and §\ref{int_lang} shall be skipped by readers already having some acquaintance with category theory; the second half of the paper makes however heavy use of the notation established before.
\subsection{On the choice of a meta-theory and a foundation}
Along the 20$^\text{th}$ century, the discipline of Mathematics divided into different sub-classes, each with their specific problems and its specific language, just to find, soon after, unification under a single notion of \emph{structure}, through the notion of abstract category \cite{gtone}. This process led to an epistemological revision of Mathematics and has inspired, parallel to the development of operative tools, a revision of both the foundations of Mathematics and the purposes of its research. 

According to many, it is undeniable that
\begin{quote}
    [the] mathematical uses of the tool `category theory' and epistemological considerations having category theory as their object cannot be separated, neither historically nor philosophically. \cite{kromer2007tool}
\end{quote}
Structural-mathematical practice, i.e. the practice of everyday Mathematics directed by structural meta-principles, produced a `natural' choice for the underlying meta-ontology of Mathematics\footnote{Perhaps improperly, the locution \emph{meta-ontology of Mathematics} is used here to refer to the totality of operative beliefs inspiring the ergonomic of mathematical objects. Some of these principles are: objects not enjoying a universal property shall be discarded; definitions that are isomorphism-invariant shall be preferred over those who are not; both these commandments are based on the idea that classes of mathematical objects arrange in coherent conglomerates exhibiting more structure than the mere aggregation of their elements: the requests of universality and isomorphism-invariance are meant not to destroy such additional structure. Examples of these meta-principles can be found in various other areas of Mathematics.} which, later, felt the need to be characterized more precisely. Similarly to what Carnap\footnote{Some words that philosophers should keep in mind, on the lawfulness of the use of abstract entities (specifically mathematical) in semantic reflection, also valid in ontology:
    \begin{quote}
        we take the position that the introduction of the new ways of speaking does not need any theoretical justification because it does not imply any assertion of reality [...].  it is a practical, not a theoretical question; it is the question of whether or not to accept the new linguistic forms. The acceptance cannot be judged as being either true or false because it is not an assertion. It can only be judged as being more or less expedient, fruitful, conducive to the aim for which the language is intended. Judgments of this kind supply the motivation for the decision of accepting or rejecting the kind of entities. \hfill \cite{carnap1956meaning}
    \end{quote}} suggested regarding semantics,
\begin{quote}
    mathematicians creating their discipline were not seeking to justify the constitution of the objects studied by making assumptions as to their ontology.\hfill  \cite{kromer2007tool}
\end{quote}
Beyond the attempts (above all, those of Bourbaki group: but see \cite{McL}, historical notes on Ch. 4, for a hint that Bourbaki didn't really get the point of structural Mathematics), what matters is that the habit of reasoning in terms of structures has suggested implicit epistemological and ontological attitudes. This matter would deserve an exhausting independent inquiry.

For our objectives it's enough to declare a differentiation that Kr\"omer
elaborated, inspired by \cite{Cor96}: the difference between \emph{structuralism} and \emph{structural Mathematics}:
\begin{enumtag}{s}
    \item \label{s:uno} Structuralism: the philosophical position regarding structures as the subject matter of Mathematics;
    \item \label{s:due} Structural Mathematics: the methodological approach to look in a given problem for the structure itself.
\end{enumtag}
Of course, \ref{s:uno} implies \ref{s:due} but the opposite is not always true:
\begin{remark} \label{weak_structuralism}
    That is, one can do structural Mathematics without being a structuralist and taking different, or even opposite, positions concerning structuralism itself. 

    Nevertheless, the use of \CT as meta-language, despite the historical link with structuralism, doesn't make automatic the transition from \ref{s:due} to \ref{s:uno}; it just suggests that the ontology is not only dependent on the `ideology' (in a Quinean sense) of the theory, but it is instead influenced by the epistemological model inspired by formal language.
\end{remark}
Kr\"omer's distinction, however, has another virtue: instead of stumbling in a possibly not ambiguous definition of \textit{structure} (with the unwanted consequences that could arise in the operational practice), \ref{s:uno} can be reduced to (or can redefine) \ref{s:due}, saying that:
\begin{quote}
    \emph{structuralism is the claim that Mathematics
        is essentially structural Mathematics} \cite{kromer2007tool}
\end{quote}
This is the same thing as saying: the structural practice already is its philosophy.

Attempts to explain the term `structure' by Bourbaki in the years following the publication of the \textit{Elements des Mathématiques}, led to the first systematic elaboration of a philosophy that we could appropriately call \textit{structural Mathematics}. Its target is to `\textit{assembling all possible ways in which given set can be endowed with certain structure}' \cite{kromer2007tool}, and elaborate, in the programmatic paper \textit{The Architecture of Mathematics} (written by Dieudonné alone and published in 1950), a formal strategy. While specifying that `\textit{this definition is not sufficiently general for the needs of Mathematics}' \cite{Bourb50}, the author encoded a series of operational steps through which a structure on a collection is assembled set-theoretically. Adopting therefore a reductionist perspective in which
\begin{quote}
    the structure-less sets are the raw material of structure building which in Bourbaki’s analysis is `unearthed' in a quasi-archaeological, reverse manner; they are the most general objects which can, in a rewriting from scratch of Mathematics, successively be endowed with ever more special and richer structures.\hfill  \cite{kromer2007tool}
\end{quote}
On balance, in Bourbaki's structuralism, the notion of set doesn't disappear definitively in front of the notion of structure. Times were not ripe to abandon set theory; the path towards an `integral' structuralism was still long, and culminated years after, with Lawvere's attempt at a foundation \theory{ETCS} of set theory first \cite{lawvere1964elementary} and \theory{ETCC} of category theory (and as a consequence, `of all Mathematics') after \cite{lajolla}, through structuralism.

To appreciate the depth and breadth of such an impressive piece of work, however, the word `foundation' must be taken in the particular sense intended by mathematicians:
\begin{quote}
    [\dots\unkern] a single system of first-order axioms in which all usual mathematical objects can be defined and all their usual properties proved.
\end{quote}
Such a position sounds at the same time a bit cryptic to unravel, and unsatisfactory; Lawvere's (and others') stance on the matter is that a foundation of Mathematics is \emph{de facto} just a set $\clL$ of first-order axioms organised in a Gentzen-like deductive system. The deductive system so generated reproduces Mathematics as we know and practice it, providing a formalisation for something that already exists and needs no further explanation, and that we call `Mathematics'.

It is not a vacuous truth that $\clL$ exists somewhere: point is, the fact that the theory so determined has a nontrivial model, i.e the fact that it can be interpreted inside a given familiar structure, is at the same time the key assumption we make, and the less relevant aspect of the construction itself.

Showing that $\clL$ `has a model' is --although slightly improperly-- meant to ensure that, \emph{assuming the existence of a naive set theory} (i.e., assuming the prior existence of structures called `sets'), axioms of $\clL$ can be satisfied by a naive set. Alternatively, and more crudely: assuming the existence of a model of \theory{ZFC}, $\clL$ has a model \emph{inside that model of \theory{ZFC}}.\footnote{It shall be made clear, ensuring that a given theory has a model isn't driven by psychological purposes only: on the one hand, purely syntactic Mathematics would be very difficult to parse, as opposed to the more colloquial practice of mathematical development; on the other hand (and this is more important), the only things syntax can see are equality and truth. To prove that a given statement is false, one either has to check all possible syntactic derivations leading to $\varphi$, finding none --this is unpractical, to say the least-- or to \emph{find a model} where $\lnot\varphi$ holds.}

\subsection{Our foundation, at last.} A series of works attempting to unhinge some aspects of ontology through category theory should at least try to tackle such a simple and yet diabolic question as `where' are the symbols forming the first-order theory \theory{ETCC}. And yet, everyone just believes in -some flavour of- sets and solves the issue of `where' they are with a leap of faith from which all else must follow.

This might appear somewhat circular: aren't sets in themselves already a mathematical object? How can they be a piece of the theory they aim to be a foundation of? In his \cite{lolli1977categorie} the author addresses the problem as follows:\footnote{Authors' translation: \emph{When mathematicians talk about models they do not have the impression to have exited a set-theoretic foundation. This impression is correct, and justified by the possibility to represent formal languages through set theory, to study the relations among structures and symbolic representations. When mathematicians turn their attention to models of a set theory, some philosophical questions can however no longer be avoided. The natural question of what is the relation between the sets that are models of a theory, and the sets of which the theory talks about is nothing but a question about the relation between the semantic, set-theoretic meta-theory and the `object theory' we want to talk about. The two theories can coincide, and in fact the meta-theory can even be a proper sub-theory of the object theory.\\
\indent We choose a set whose elements, usually finite sets, represent the symbols of the language we want to study, and then with a concatenation operation that can be the `ordered pair' construction we define the set of words $L$ of the well-formed formulas, of terms, and the operation that associates a free variable with a formula, and so on for all syntactic notions.}}
\begin{quote}
    Quando un matematico parla di modelli non ha [\dots\unkern] l'impressione di uscire dall'ambito insiemistico. Questa impressione, che è corretta, è giustificata dalla possibilità di rappresentare i linguaggi formali con gli oggetti della teoria degli insiemi, di studiare in essa le relazioni tra le strutture e le rappresentazioni dei simboli. Quando l'attenzione è rivolta ai modelli di una teoria degli insiemi, certe questioni sofistiche non possono però essere più evitate. La domanda spontanea sulla relazione che intercorre tra gli insiemi che so­no modelli di una teoria e gli insiemi di cui parla la teoria non è altro che una domanda sulla relazione tra la metateoria semantica insiemi­stica e la teoria in esame, o teoria oggetto. Le due teorie possono coincidere, anzi la metateoria può essere anche una sottoteoria pro­pria della teoria oggetto. [\dots\unkern]

    Si sceglie un insieme i cui elementi, di solito insiemi finiti, rappresentano i simboli del linguaggio che si vuole studiare, quindi con una operazione di concatenazione che può essere la coppia ordinata si definisce l'insieme delle parole $L$ delle for­mule ben formate, dei termini, l'operazione che a una formula asso­cia le sue variabili libere, e così via per tutte le nozioni sintattiche.
\end{quote}
The idea that a subtheory $L'\subset L$ of the object theory can play the r\^ole of meta-theory might appear baffling; in practice, the choice is to rely on one among two possible solutions. Pure Platonism assumes the existence of a hierarchy of universes harbouring the object theory; pure syntacticism exploits G\"odel's completeness theorem: every proof is a finite object, and every theorem proved in the meta-theory is just a finite string of symbols. No need for a model.

Platonism has limits: in a fixed a class theory $\sfC$ ($\sf MK$, Morse-Kelley(-Mostowski); or $\sf NBG$, Von Neumann-Bernays-G\"odel), there's an object $V$ that plays the r\^ole of the universe of sets; in $V$, all Mathematics can be enacted. Of course, consistency of $\sfC$ is only granted by an act of faith.

Syntacticism has limits: following it, one abjures any universality Mathematics might claim. But syntacticism also has merits: undeniably (disgracefully, luckily) the World is real. And reality is complex enough to contain languages as purely syntactical objects; the percussion of a log with oxen bones, rather than prophecies over the entrails of a lamb, or intuitionistic type theory, all have the same purpose: intersubjective convection of meaning, deduced by a bundle of perceptions, so to gain an advantage, \emph{id est} some predictive power, over said perceptions. Of course: intuitionistic type theory is just \emph{slightly} more effective than hepatomancy.

Knowledge is obtained by collision and retro-propagation between Reality and the perceptual bundle it generates.

Accepting this, the urge to define seemingly abstract concepts like learning, conscience, and knowledge, together with precious continuous feedback coming from real objects, evidently determine an undeniable primacy of quantitative thinking, this time intended as machine learning and artificial intelligence, that sets (or should set if only more philosophers knew linear algebra) a new bar for research in philosophy of mind.

However, we refrain from entering such a deep rabbit-hole, as it would have catastrophic consequences on the quality, length, and depth of our exposition.

The usual choice for mathematicians is to assume that, wherever and whatever they are, these symbols `are', and our r\^ole in unveiling Mathematics is \emph{descriptive} rather than generative.\footnote{Inside (say) a constructivist foundation it is not legitimate to posit that axioms `create' mathematical objects; from this, the legitimacy of the question of where they are, and the equally legitimate answer `nowhere'. The only thing we can say is that they `make precise, albeit implicitly, the \emph{meaning} of mathematical objects' \cite{Agzz} (it seems to us that in Mathematics as well as in philosophy of language, meaning and denotation are safely kept separate). We take this principle -that the world/metamodel exists and we can just attempt at describing it by means of the language/model- as evident to anyone in a healthy state of mind, and we leverage it without further question.}

This state of affairs has, to the best of our moderate knowledge on the subject, various possible explanations:
\begin{itemize}
    \item On one hand, it constitutes the heritage of Bourbaki's authoritarian stance on formalism in pure Mathematics;
    \item on the other hand, a different position would result in barely any difference for the `working-class'; mathematicians are irreducible pragmatists, somewhat blind to the consequences of their philosophical stances.
\end{itemize}
So, symbols and letters do not exist outside of the Gentzen-like deductive system we specified together with $\clL$.

As arid as it may seem, this perspective proved itself to be quite useful in working Mathematics; consider for example the type declaration rules of a typed functional programming language: such a concise declaration as
\begin{lstlisting}[ language=Haskell
                  , basicstyle=\ttfamily\small
                  , keywordstyle=\color{blue!75}
                  , morekeywords={S,Z}
                  ]
  data Nat = Z | S Nat
\end{lstlisting}
makes no assumption on `what' \verb|Z| and \verb|S:: Nat -> Nat| are; instead, it treats these constructors as meaningful formally (in terms of the admissible derivations a well-formed expression is subject to) and intuitively (in terms of the fact that they model natural numbers: every data structure that has those two constructors must be the type $\bbN$ of natural numbers -provided data constructors like \verb|S| are all injective).

Taken as an operative rule, this reveals exactly what is our stance towards foundations: we are `structuralist in the meta-theory', meaning that we treat the symbols of a first-order theory or the constructors of a type system regardless of their origin, provided the same relation occur between criptomorphic collections of labelled atoms.

In this precise sense, we are thus structuralists in the meta-theory, and yet we do so with a grain of salt, maintaining a transparent approach to the consequences and limits of this partialisation. On the one hand, pragmatism works; it generates rules of evaluation for the truth of sentences. On the other hand, this sounds like a Munchhausen-like explanation of its the value, in terms of itself. Yet there seems to be no way to do better: answering the initial question `where are the letters of \theory{ETCC}?' would result in no less than a foundation of language.

And this for no other reason than `our' meta-theory is something near to a structuralist theory of language; thus, a foundation for such a meta-theory shall inhabit a meta-meta-theory\dots{} and so on.

Thus, rather than trying to revert this state of affairs we silently comply to it as everyone else does; but we feel contempt after a brief and honest declaration of intents towards where our meta-theory lives. Such a meta-theory hinges again on work of Lawvere, and especially on his series of works on functorial semantics.

  \section{Preliminaries on variable set theory}
In some of our proofs it will be crucial to blur the distinction between the category of functors $I \to \Set$ and the slice category $\Set/I$ (see \cite[??]{Bor1}); once the following result is proved, we freely refer to any of these two categories as the category of \emph{variable sets} (indexed by $I$).
\begin{proposition}\label{variabbo_set}
	Let $I$ be a set, regarded as a discrete category, and let $\Set^I$ be the category of functors $F : I \to \Set$; moreover, let $\Set/I$ the slice category. Then, there is an equivalence (actually, an isomorphism) between $\Set^I$ and $\Set/I$.
\end{proposition}
\begin{proof}
	Let us give a very hands-on proof, based on the fact that the category $\Set^I$ coincides on its own right with the category of $I$-indexed families of objects, i.e. with the category whose objects are $(\underline X)_I := \{X_i\mid i\in I\}$, and morphisms $(\underline X)_I\to (\underline Y)_I$ the families $\{f_i : X_i \to Y_i\mid i \in I\}$.

	Consider an object $h : X\to I$ of $\Set/I$, and define a function as $i\mapsto h^\leftarrow(i)$; of course, $(X(h))_I := \{h^\leftarrow(i) \mid i \in I\}$ is a $I$-indexed family, and since $I$ can be regarded as a discrete category, this is sufficient to define a functor $F_h : I \to \Set$.

	Let us define a functor in the opposite direction: let $F : I \to \Set$ be a functor. This defines a function $h_F : \coprod_{i\in I}Fi \to I$, where $\coprod_{i\in I} Fi$ is the disjoint union of all the sets $Fi$.

	The claim now follows if we show that the correspondences $h\mapsto F_h$ and $F\mapsto h_F$ are mutually inverse.

	This is however easy to verify: the function $F_{h_F}$ sends $i\in I$ to the set $h_F^\leftarrow(i)=Fi$, and the function $h_{F_h} \in \Set/I$ has domain $\coprod_{i\in I}F_h(i) = \coprod_{i\in I}h^\leftarrow(i)=X$ (as $i$ runs over the set $I$, the disjoint union of all preimages $h^\leftarrow(i)$ equals the domain of $h$, i.e. the set $X$).
\end{proof}
\begin{remark}
	A more abstract look at this result regards the equivalence $\Set/I\cong \Set^I$ as a particular instance of the \emph{Grothendieck construction} (see \cite[1.1]{Leinster2004}): for every small category $\clC$, the category of functors $\clC\to\Set$ is equivalent to the category of \emph{discrete fibrations} on $\clC$ (see \cite[1.1]{Leinster2004}). In this case, the domain $\clC=I$ is a discrete category, hence all functors $\clE \to I$ are, trivially, discrete fibrations.
\end{remark}
The next crucial step of our analysis is the observation that the category of variable sets is a topos: we break the result into the verification of the various axioms, as explained in \autoref{eletop} and \autoref{grotop}.

\begin{proposition}
	The category of variable sets is Cartesian closed.
\end{proposition}
\begin{proof}
	We shall first show that the category of variable sets admits products: this is obvious in $\Set/I$, products are precisely pullbacks; note that \autoref{variabbo_set} gives an identification
	\[\vcenter{\scriptsize\xymatrix@!=1mm{
		& X\times_I Y \ar[dd]^h \ar@[lightgray][dr]\ar@[lightgray][dl]&  \\
		{\color{lightgray} X} \ar@[lightgray][dr]_{\color{lightgray} f}&& {\color{lightgray} Y} \ar@[lightgray][dl]^{\color{lightgray} g}\\
		& I &
		}}\iff i\mapsto h^\leftarrow(i) = \Big\{(x,y) \in X\times_I Y \mid h(x,y)=i\Big\}\]
	and given the universal property of a pullback, this yields a canonical bijection $h^\leftarrow(i)\cong f^\leftarrow(i)\times g^\leftarrow(i)$. This is exactly the definition of the product of the two functors $F_f, F_g : I\to \Set$.

	Next, we shall show that each functor $\firstblank \times_I Y$ has a right adjoint $Y \pitchfork_I\firstblank$. The functor $\Set^I \to \Set^I : Z\mapsto Y\pitchfork_I Z$ where $Y\pitchfork_I Z : i \mapsto \Set(Y_i, Z_i)$ does the job. This sets up the bijection
	\[\begin{array}{c}
			\xymatrix{X\times_I Y \ar[r] & Z}               \\ \hline
			\xymatrix{X \ar[r]           & Y\pitchfork_I Z}
		\end{array}\]
	and by a completely analogous argument (the functor $\firstblank\times_I\secondblank$ gives a symmetric monoidal structure to variable sets),
	\[\begin{array}{c}
			\xymatrix{X\times_I Y \ar[r] & Z}                \\ \hline
			\xymatrix{Y \ar[r]           & X\pitchfork_I Z;}
		\end{array}\]
	this concludes the proof that the category of variable sets is Cartesian closed.
\end{proof}
\begin{proposition}\label{variable_sets_have_omega}
	The category of variable sets has a subobject classifier.
\end{proposition} 
\begin{proof}
	From \autoref{eletop} we know that we shall find a variable set $\Omega$ such that there is a bijection
	\[\begin{array}{c}
			\xymatrix{A \ar[r] & \Omega} \\ \hline
			\textsf{Sub}_I(A)
		\end{array}\]
	where $\textsf{Sub}_I(A)$ denotes the set of isomorphism classes of monomorphisms into $A$, in the category of variable sets.\footnote{A monomorphism into $A$ as an object of $\Set^I$ is nothing but a family of injections $s_i : S_i \to A_i$; a monomorphism in $\Set/I$ is a set $S$ in a commutative triangle
	\[\scriptsize
		\xymatrix@!=1mm{S\ar[rr]\ar[dr]_s && A\ar[dl]^a \\ &I.&}\]}

	In order to find such an object, we look at what shape shall $\Omega$ have, and what role its universal property plays in its characterization:
	\begin{itemize}
		\item
		\item
	\end{itemize}
	For the sake of simplicity, for the rest of the proof we fix as category of variable sets the slice $\Set/I$.

	From this we make the following guess: as an object of $\Set/I$, $\Omega$ is the object $\pi_I : I\times \{0,1\} \to I$. We are thus left with the verification that $\pi_I$ has the correct structure and universal property.

	First, we shall find a monomorphism $\true : * \to \Omega$ in $\Set/I$, i.e. an injective function $I\to \Omega$ that has $\pi_I$ as left inverse. This generalised element ``chooses the $\lceil$true$\rceil$ truth value'' in $\Omega$. (Evidently, the identity $\id_I : I \to I$ is the terminal object in $\Set/I$.)

	It turns out that the function $I \to I\times \{0,1\}$ plays the r\^ole of $\true$: indeed, given a monomorphism $S \hookrightarrow A$ the commutative square
	\[
		\vcenter{\xymatrix{
				S\ar[d]\ar[r] & I \ar[d]^{\true}\\
				A \ar[r]_{\chi_S} & I\times \{0,1\}
			}}
	\]
\end{proof}
\begin{proposition}
	The category of variable sets is cocomplete and accessible.
\end{proposition}
\begin{proof}

\end{proof}
Accessibility is a corollary of Yoneda in the following form: every $F : I \to \Set$ is a colimit of representables
\[
	F \cong \colim\Big(\clE(F) \xto{\Sigma} I \xto{y} \Set^I\Big)
\]
($\clE(F)$ is small because in this case $\clE(F)\cong\coprod_{i\in I}Fi$).
\begin{corollary}
	The category of variable sets is a Grothendieck topos.
\end{corollary}

  \section{The internal language of variable sets}
\epigraph{I am hard but I am fair; there is no racial bigotry here. [\dots\unkern] Here you are all equally worthless.}{GySgt Hartman}
\begin{definition}\label{da_lang}
The internal language of a topos $\clE$ is a formal language defined by \emph{types} and \emph{terms}; suitable terms form the class of variables. Other terms form the class of \emph{formul\ae}.
\begin{itemize}
	\item \emph{Types} are the objects of $\clE$
	\item \emph{Terms} of type $X$ are morphisms of codomain $X$, usually denoted $\alpha,\beta,\sigma,\tau : U \to X$.
	      \begin{itemize}
		      \item Suitable terms are variables: the identity arrow of $X\in\clE$ is the variable  $x : X \to X$. For technical reasons we shall keep a countable number of variables of the same type distinguished:\footnote{These technical reasons lie on the evident necessity to be free to refer to the same free variable an unbounded number of times. This can be formalised in various ways: we refer the reader to \cite[]{lambekscott} and \cite[]{johnstopos}.} $x,x',x'',\dots : X \to X$ are all interpreted as $1_X$.
	      \end{itemize}
	\item Generic terms may depend on multiple variables; the domain of a term of type $X$ is the \emph{domain of definition} of a term.
\end{itemize}
A number of inductive clauses define the other terms of the language:
\begin{itemize}
	\item the identity arrow of an object $X\in\clE$ is a term of type $X$;
	\item given terms $\sigma : U \to X$ and $\tau :  V\to Y$ there exists a term $\lr{\sigma}{\tau}$ of type $X\times Y$ obtained from the pullback
	      \[\xymatrix{
		      W \ar[d]\ar[r]\ar[dr]|{\lr{\sigma}{\tau}} & X \times V \ar[d]\\
		      U\times Y \ar[r]& X\times Y
		      }\]
	\item Given terms $\sigma : U \to X, \tau : V \to X$ of the same type $X$, there is a term $[\sigma = \tau] : W \xto{\lr{\sigma}{\tau}} X\times X \xto{\delta_X} \Omega$, where $\delta_X : X\times X \to \Omega$ is defined as the classifying map of the mono $X \hookrightarrow X\times X$.
	\item Given a term $\sigma : U \to X$ and a term $f : X \to Y$, tere is a term $f[\sigma] := f\circ\sigma : U \to Y$.
	\item Given terms $\theta :  V \to Y^X$ and $\sigma : U\to X$, there is a term
	      \[
		      W\lr{\theta}{\sigma} \xto{}Y^X\times X \xto{\text{ev}} Y
	      \]
	\item In the particular case $Y=\Omega$, the term above is denoted
	      \[[\sigma\in\theta] : W\lr{\theta}{\sigma} \to \Omega\]
	\item If $x$ is a variable of type $X$, and $\sigma : X\times U \to Z$, there is a term
	      \[\lambda x.\sigma : U \xto{\eta} (X\times U)^X \xto{\sigma^X} Z^X\]
	      obtained as the mate of $\sigma$.
\end{itemize}
These rules can of course be also presented as the formation rules for a Gentzen-like deductive system: let us rewrite them in this formalism.
\[ \begin{array}{cc}
		\infer{1_X : X \to X}{}                                                              &
		\infer{\lr{\sigma}{\tau} : W\lr{\theta}{\sigma} \to X \times Y }{\sigma : U \to X    &   & \tau : V \to Y}             \\[1em]
		\infer{[\sigma=\tau] : W \to \Omega}{\sigma : U \to X                                &   & \tau : V \to X}           &
		\infer{f[\sigma] : U \to Y}{\sigma : U \to X                                         &   & f : X \to Y}                \\[1em]
		\infer{W\lr{\theta}{\sigma} \xto{}Y^X\times X \xto{\text{ev}} Y}{\theta :  V \to Y^X &   & \sigma : U\to X}          &
		\infer{\lambda x.\sigma = \sigma^X\circ\eta : U \to (X\times U)^X \to Z^X}{ x: X     &   & \sigma : X\times U \to Z}
	\end{array}\]
To formulas of the language of $\clE$ we apply the usual operations and rules of first-order logic: logical connectives are induced by the structure of internal Heyting algebra of $\Omega$: given formulas $\varphi,\psi$ we define
\begin{itemize}
	\item $\varphi\lor \psi$ is the formula $W\lr{\varphi}{\psi} \to \Omega\times \Omega \xto{\lor} \Omega$;
	\item $\varphi\land\psi$ is the formula $W\lr{\varphi}{\psi} \to \Omega\times \Omega \xto{\land} \Omega$;
	\item $\varphi\Rightarrow\psi$ is the formula $W\lr{\varphi}{\psi} \to \Omega\times \Omega \xto{\Rightarrow} \Omega$;
	\item $\lnot\varphi$ is the formula $U \to \Omega \xto{\lnot} \Omega$.
\end{itemize} 
\end{definition}
\todo[inline]{universal quantifiers}
Each formula $\varphi : U \to \Omega$ defines a subobject $\{x\mid \varphi\} \subseteq U$ of its domain of definition; this is the subobject classified by $\varphi$, and must be thought as the subobject where ``$\varphi$ is true''.

If $\varphi : U\to\Omega$ is a formula, we say that $\varphi$ is \emph{universally valid} if $\{x\mid\varphi\}\cong U$. If $\varphi$ is universally valid in $\clE$, we write ``$\clE\Vdash \varphi$'' (read: ``$\clE$ believes in $\varphi$'').

Examples of universally valid formulas:
\begin{itemize}
	\item $\clE\Vdash [x=x]$
	\item $\clE\Vdash [(x \in_X \{x\mid\varphi\}) \iff \varphi]$
	\item $\clE\Vdash \varphi$ if and only if $\clE \Vdash \forall x.\varphi$
	\item $\clE\Vdash [\varphi \Rightarrow \lnot\lnot\varphi]$
\end{itemize}
\todo[inline]{Ora facciamo delle considerazioni sul lingo interno di $\Set/I$}
Chi sono tipi e termini; chi sono le proposizioni e come si scrive il calcolo proposizionale in $\Set/I$; i quantificatori, in dettaglio pornografico.
\begin{definition}
	Types and terms of $\clL(\Set/I)$.
\end{definition}
\begin{definition}
	Propositional calculus, quantifiers.
\end{definition}
\begin{remark}
	Like every other Grothendieck topos, the category $\Set/I$ has a \emph{natural number object} (see \cite[??]{mac1992sheaves}); here we shall outline its construction. It is a general fact that such a natural number object in the category of variable sets, consists of the constant functor on $\bbN : \Set$, when we realise variable sets as functors $I \to \Set$: thus, in fibered terms, the natural number object is just $\pi_I : \bbN \times I \to I$.

	A natural number object provides the category $\clE$ it lives in with a notion of \emph{recursion} and with a notion of $\clE$-induction principle: namely, we can interpret the sentence
	\[\textstyle\big(Q0\land \bigwedge_{i\le n} Qi\Rightarrow Q(i+1)\big)\Rightarrow \bigwedge_{n : \bbN} Qn.\]
	In the category of variable sets, the universal property of $\pi_I : \bbN\times I \to I$ amounts to the following fact: given any diagram of solid arrows 
	\[
	\xymatrix{
		I \ar[r]^0 \ar@{=}[d] & \bbN\times I\ar@{.>}[d]^u \ar[r]^{s\times I} & \bbN \times I\ar@{.>}[d]^u \\ 
		I \ar[r]_x & X \ar[r]_f & X
	}	
	\]
	where every arrow carry a structure of morphism over $I$ (and $0 : i \mapsto (0,i)$, $s\times I : (n,i) \mapsto (n+1,i)$), there is a unique way to complete it with the dotted arrow, i.e. with a function $u : \bbN \times I \to X$ such that 
	\[u \circ (s\times I) = f \circ u.\]
	Clearly, $u$ must be defined by induction: if it exists, the commutativity of the left square amounts to the request that $u(0,i)=x(i)$ for every $i : I$. Given this, the inductive step is 
	\[ 
		u(s(n,i)) = u(n+1,i) = f(u(n,i))
	\]
\end{remark}

  \section{Nine copper coins, and other toposes}
\epigraph{
	Explicaron que una cosa es \emph{igualdad}, y otra \emph{identidad}, y formularon una especie de \emph{reductio ad absurdum}, o sea el caso hipotético de nueve hombres que en nueve sucesivas noches padecen un vivo dolor. ¿No sería ridículo -interrogaron- pretender que ese dolor es el mismo?
}{JLB ---Tl\"on, Uqbar, Orbis Tertius}

According to our description of the Mitchell-Bénabou language in the category of variable sets, \emph{propositions} are morpisms of the form
\[p : U \to \Omega_I\]

where $\Omega_I$ is the subobject classifier of $\Set/I$ described in \ref{}; now, recall that
\begin{itemize}
	\item the object $\Omega_I = \{0,1\}\times I \to I$ becomes an object of $\Set/I$ when endowed with the projection $\pi_I : \Omega_I \to I$ on the second factor of its domain;
	\item the universal monic $\tr : I \to \Omega_I$ consists of a section of $\pi_I$, precisely the one that sends $i : I$ to the pair $(i,1) : \Omega_I$;
	\item every subobject $U \hookrightarrow A$ of an object $A$ results as a pullback (in $\Set/I$) along $\tr$:
	      \[\xymatrix@R=5mm@C=5mm{
		      U\ar[dr]^u\ar[rr]^u\ar[dd]_m&& I\ar[dd]^{\tr} \ar@{=}[dl]\\
		      &I& \\
		      A \ar[ur]\ar[rr]_{\chi_U}&& \ar[ul]\Omega_I
		      }\]
	      (see \ref{} for a complete proof)
\end{itemize}
The set $I$ in this context acts as a \emph{multiplier} of truth values, in that every proposition can have a pair $(\epsilon, i)$ as truth value. We introduce the following notation: a proposition $p : U \to \Omega_I$ is \emph{true}, in context $x :U$, with \emph{strength} $t$, if $p(x) =(1,t)$ (resp., $p(x)=(0,t)$).

So, a proposition is a morphism of the following kind: a function $p : U \to \Omega_I$, defined on a certain domain, and such that
\[
	\xymatrix{
		U\ar[d]_u\ar[r]^-p  & \{0,1\}\times I \ar[d]^{\pi_I}\\
		I \ar@{=}[r]& I
	}
\]
(it must be a morphism of variable sets!) This means that $\pi p(x : U) = u(x : U)$, so that $p(x) = (\epsilon_x, u(x))$ for $\epsilon_x =0,1$ and $u$ is uniquely determined by the "variable domain" $U$. This is an important observation: the strength with which $p$ is true/false is completely determined by the structure of its domain, in the form of the function $u : U \to I$ that renders the pair $(U,u)$ an object of $\Set/I$.

To get a grip of the different roles of various classes of propositions, and given that our interest will be limited to a certain class of particular propositions that we will construct \emph{ex nihilo}, it is now convenient to discuss what constraints we have to put on the structure of $I$: of course, the richest this structure is, the better will the category $\Set/I$ behave: it is for example possible to equip $I$ with an order structure, or a natural topology. Among different choices of truth multiplier, yielding different categories of variable sets, and different kinds of internal logic therein, we will privilege those that make $I$ behave like a space of strengths: a dense, linear order with LUP, thus not really far from being a closed, bounded subset of the real line.

The main result of the present section is a roundup of examples showing that it is possible to concoct categories of variable sets where some seemingly paradoxical constructions coming from J.L. Borges' literary world have, instead, a perfectly ``classical'' behaviour when looked with the lenses of the logic of variable set theory.

Each of the examples in our roundup \ref{bla} \ref{bli} \ref{blu} is organised as follows: we recall the shape of a paradoxical statement in Borges' literary world. Then, we show in which topos this reduces to an intuitive statement expressed in the syntax of a variable set category.

More than often, we use $I=[0,1]$ as base of variable sets; as already said, there are different reasons for this choice: the most intuitive is that if a truth value is given with a \emph{strength} $t\in I$ it is a natural request to be able to \emph{compare} elements in this set; in particular, it should always be possible to assess what truth value is stronger.

For this reason, even if this assumption is never strictly necessary (the only constraint is that $I$ is totally, or partially, ordered set by a relation $\le$), a natural choice for $I$ is a \emph{continuum} (=a dense total order with LUP --see \cite{moschovakis2009descriptive}). An alternative choice drops the density assumption: in that case the (unique) finite total order $\Delta[n] = \{0 < 1 <\dots < n\}$, or the countable total order $I=\omega = \bigcup_n \Delta[n]$ are all pretty natural choices for $I$ (although it is way more natural for $I$ to have a minimum \emph{and} a maximum element).\footnote{We're oly interested in the notion of an abstract interval here: a continuum $X$ endowed with an operation $X \to X \vee X$ of ``zooming'', uniquely defined by this property. In a famous paper Freyd characterises ``the interval'' as the terminal interval coalgebra: see \cite[§1]{freyd2008algebraic}; for our purposes, note that $[0,1]$ is a natural choice: it is a frame, thus a Heyting algebra $\fkH =([0,1],\land,\lor,\Rightarrow)$ with respect to the pseudo-complement operation given by $(x \Rightarrow z) := \bigvee_{x\land y \le z} y$ (it is immediate that $x \land a \le b$ if and only if $a \le x \Rightarrow b$ for every $a,b\in [0,1]$).} In each of these cases ``classical'' logic is recovered as a projection: propositions $p$ can be true or false with strength $1$,\footnote{Here $I$ is represented as an interval whose minimal and maximal element are respectively $0$ and $1$; of course these are just placeholders, but it is harmless for the reader to visualise $I$ as the interval $[0,1]$.} the maximum element of $I$:
\begin{center}
	\begin{tikzpicture}
		\draw[gray!70,dashed] (2,1.25) -- (2,-.25) node[below] {$\{\perp,\top\}$};
		\draw[fill] (0,0) circle (1pt) node[left] {$0$};
		\draw[fill] (2,0) circle (1pt) node[right] (dis) {$1$};
		\draw (0,0) -- (2,0);
		\begin{scope}[yshift=1cm]
			\draw[fill] (0,0) circle (1pt) node[left] {$0$};
			\draw[fill] (2,0) circle (1pt) node[right] (dat) {$1$};
			\draw (0,0) -- (2,0);
			\node[right of=dis] {$\{\top\}\times I$};
			\node[right of=dat] {$\{\perp\}\times I$};
		\end{scope}
	\end{tikzpicture}
\end{center}
In order to aid the reader understand the explicit way in which $I$ ``multiplies'' truth values, we spell out explicitly the structure of the subobject classifier in $\Set/\Delta[2]$. In order to keep calling the minimum and maximum of $I$ respectively $0$ and $1$ we call $\frac{1}{2}$ the intermediate point of $\Delta[2]$.
\begin{remark}
	The subobject classifier of $\Set/\Delta[2]$ consists of the partially ordered set $\Delta[1]\times\Delta[2]$ that we can represent pictorially as a rectangle
	\[\begin{tikzpicture}[xscale=2]
			\foreach \i/\name in {0/0,.5/{\frac{1}{2}},1/0}{
			\foreach \j/\pos in {0/below,1/above}
			\fill (\i,\j) ellipse (1pt and 2pt) node[\pos] {\tiny $(\name,\j)$};
			}
			\draw (0,0) rectangle (1,1);
			\draw (.5,1) -- (.5,0);
		\end{tikzpicture}\]
	endowed with the product order. The resulting poset is partially ordered, and in fact a Heyting algebra, because it results as the product of two Heyting algebras: the Boole algebra $\{0<1\}$ and the frame of open subsets of the Sierpinski space $\{a,b\}$ (the topology is $\tau_S = \{\varnothing, \{a\}, \{a,b\}\}$).
\end{remark}
\begin{remark}
	Siccome il caso $I=[0,1]$ con la topologia euclidea è quello piu naturale per diversi motivi, definiamo alcuni insiemi di interesse per una data proposizione $p : U \to \Omega_I$ per questa scelta di $I$:
	\begin{itemize}
		\item $A^\top = \{x : U \mid p(x) = (1,t_x), t_x > 0\} = p^\leftarrow(\{1\}\times (0,1])$ e $A^\perp = \{x : U \mid p(x) = (0,t_x), t_x > 0\}$; cose vere (risp., false) con forza maggiore di zero. Sono le funzioni $u : U \to I$ tali che $u^\leftarrow 0 = \varnothing$.
		\item $B^\top = \{x : U \mid p(x) = (1,1)\} = p^\leftarrow((1,1))$ e $B^\perp = \{x : U \mid p(x) = (0,1)\}$ cose vere (risp.,false).
		\item $E_t^\top = \{ x : U \mid p(x)=(1,t)\}$ e $E_t^\perp = \{ x : U \mid p(x)=(0,t)\}$; cose vere (risp., false) con forza $t$.
	\end{itemize}
\end{remark}
Last but not least, a crucial assumption will be that the strength of $p$ depends continuously, or not, on the variables on its domain of definition. Without such continuous dependence, small changes in context $x : U$ might drastically change the truth value $p(x)$.\footnote{There is no a priori reason to maintain that $p$ is a continuous proposition; one might argue that discontinuous changes in truth value of $p$ happen all the time in ``real life''; see the family of paradoxes based on so-called \emph{separating instants}: how well-defined the notion of ``time of death'' is? How well-defined the notion of ``instant in time''?}
\subsection{The unimaginable topos theory hidden in Borges' library}
Jorge Luis Borges' literary work is well-known for being made by paradoxical worlds; oftentimes, seemingly absurd consequences follow by stretching to their limit ideas from logic and mathematics: time, infinite, self-referentiality, duplication, recursion, the relativity of time, the illusory nature of our perceptions, the limits of language, its capacity to generate worlds.

In the present section we choose \emph{Fictions}, Borges' famous collection of novels, as source of inspiration for possible and impossible worlds and their ontology.

Usualmente la costruzione di un ``mondo impossibile'' va circa come segue: \dots ; noi rovesciamo tale prospettiva, e invece di depennare dal computo degli universi i mondi le cui caratteristiche do not comply with sensorial experience, or imply paradoxical entities/constructs, we accept their existence for bizarre that it may seem, and we try to deduce what kind of logic can consistently generate such statements.

The interest in such a literary calembour is manifold, and the results are surprising:
\begin{itemize}
	\item we unravel how a mathematically deep universe Borges has inadvertently created: of the many compromises we had to take in order to reconcile literature and the uniderlying mathematics,\footnote{See \ref{} below: these ``compromises'' mainly amount to assumptions on the behaviour of space-time on Tl\"on and Babylon.} we believe no one is particularly far-fetched one;
	\item we unravel how \emph{relative} ontological assumptions are; they are not given: using category theory, ontology, far from being the presupposition on which it is based, is a byproduct of language itself. The more expressive language is, the more ontology; the fuzzier its capacity to assert truth, the fuzzier existence becomes;
	\item ``Fuzziness'' of existence, i.e. the fact \emph{entia} exist less than completely, is hard-coded in the language (in the sense of \ref{}) of the category we decide to work in from time to time;
	\item ...
\end{itemize}
To sum up, readers willing to find an original result in this paper, might find it precisely here: we underline how Borges' alternative worlds (Babylon, Tl\"on \ dots) are mathematically consistent places, worthy of existence as much as our world, just based on a different internal logic. And they are so just thanks to a base-sensitive theory of existence --ontology breaks in a spectrum of ontolog\emph{ies}, one for each category/world.

The first paradox we aim to frame in the right topos is the famous nine copper coins argument, used by the philosophers of Tlön to construct a paradoxical object whose existence persists over time, in absence of a consciousness continually perceiving it and maintaining in a state of being.
\begin{example}[Nine copper coins]\label{bla}
	First, we recall the exact statement of the paradox from \cite{}:\footnote{The translation we employ is classical and comes from \cite{tlonEN}:
		\begin{quote}
			\hspace{.5em} Tuesday, $X$ crosses a deserted road and loses nine copper coins. On Thursday, $Y$ finds in the road four coins, somewhat rusted by Wednesday's rain. On Friday, $Z$ discovers three coins in the road. On Friday morning, $X$ finds two coins in the corridor of his house. The heresiarch would deduce from this story the reality - i.e., the continuity - of the nine coins which were recovered.

			\hspace{.5em} It is absurd (he affirmed) to imagine that four of the coins have not existed between Tuesday and Thursday, three between Tuesday and Friday afternoon, two between Tuesday and Friday morning. It is logical to think that they have existed - at least in some secret way, hidden from the comprehension of men - at every moment of those three periods.
		\end{quote}}
	\begin{quote}
		El martes, $X$ atraviesa un camino desierto y pierde nueve monedas de cobre.
		El jueves, $Y$ encuentra en el camino cuatro monedas, algo herrumbradas por la lluvia del miércoles. El viernes, $Z$ descubre tres monedas en el camino. El viernes de mañana, $X$ encuentra dos monedas en el corredor de su casa. El  quería deducir de esa historia la realidad -id est la continuidad- de las nueve monedas recuperadas.

		Es absurdo (afirmaba) imaginar que  cuatro de las monedas no han existido entre el martes y el jueves, tres entre el martes y la tarde del viernes, dos entre el martes y la madrugada del viernes. Es lógico pensar que han existido -siquiera de algún modo secreto, de comprensión vedada a los hombres- en todos los momentos de esos tres plazos.
	\end{quote}
	Before going on with our analysis, two remarks are in order:
	\begin{itemize}
		\item the paradox appears in a primitive version in \cite{}, where instead of nine copper coins, a single arrow, shot by an anonymous archer, disappears among the woods. The text appears in a hard-to-find edition of \emph{Inquisiciones} \cite{}; in the last chapter, we read the ``arrow avatar'' (\emph{avatar de la flecha}):
		      \begin{quote}
			      $X$ scocca una freccia da un arco, ed essa si perde fra gli alberi.

			      $X$ la cerca e riesce a ritrovarla.

			      E' assurdo immaginare che la freccia non sia esistita durante il periodo fra i momenti in cui $X$ l'ha persa di vista e l'ha ritrovata.

			      E' logico pensare che essa sia esistita - anche se in un certo modo segreto, di comprensione vietata agli uomini - in tutti i momenti di questo periodo.
		      \end{quote}
		\item There is one and only one reason why the paradox of the nine copper coins is invalid: copper does not rust.
	\end{itemize}
	It is obvious that both constructions leverage on the same argument to build an efficient aporia: the mysterious persistence of things through time without a perceivent consciousness. We concentrate on the copper coins dilemma, per il semplice fatto che Finzioni è raggiungibile molto piu facilmente ai nostri lettori; incidentally, we happen to be able to rectify the ``rust counterargument'' without appealing to the assumption that copper can rust on Tl\"on due to a difference in Tl\"onian chemistry.

	Expressed in natural language, our solution to the paradox goes more or less as follows: $X$ loses their coins on Tuesday, and the force $\varphi$ with which they ``exist'' lowers; it grows back in the following days, going back to a maximum value when $X$ retrieves two of their coins on the front door. $Y$ findings of other coins raises their existence force to a maximum. The coins that $Y$ has found rusted (or rather, the surface copper oxidized: this is possible, but water is rarely sufficient to ignite the process alone --certainly not in the space of a few hours.\footnote{in una sorta di principio di wormhole, eventi indipendenti sulla terra sono dipendenti su Tlön, perché l'evento A influenza, in uno spazio pluridimensionale di scelte di z di verità, l'evento B in modi che gli sarebbero vietati se fosse sulla Terra.}

	In this perspective, Tl\"on classifier of truth values can be taken as $\Omega_I = \{0<1\}\times I$, where $I$ is any set with more than one element; a minimal example can be $I=\{N,S\}$ (justifying this choice from inside Tl\"on is easy: the planet is subdivided into two emispheres; each of which now has its own logic ``line'' independent from the other), but as explained in \ref{} a more natural choice for our purposes is the closed real interval $I=[0,1]$.

	This allows for a continuum of possible forces with which a truth value can be true or false;  it is tobe noted that $[0,1]$ is also the most natural place on which to interpret fuzzy logic, albeit the interest for $[0,1]$ therein can be easily and better motivated starting from probability theory. (But see \cite{} for an interesting perspective on how to develop basic measure teory out of $[0,1]$.)

	We now start to formalise properly what we said until now.

	To set our basic assumptions straight, we proceed as follows:
	\begin{itemize}
		\item senza perdita di generalità possiamo supporre l'insieme $C = \{c_1,\dots,c_9\}$ delle monete totalmente ordinato e partizionato in modo tale che le prime due monete siano quelle ritrovate da $X$ il martedì, le seconde quattro quelle che $Y$ ritrova sul cammino, e le ultime 3 quelle viste da $Z$. Allora
		      \[C = C_X \sqcup C_Y \sqcup C_Z.\]
		      and $C_X = \{c_{X1}, c_{X2}\}$, $C_Y = \{
			      c_{Y1},c_{Y2},c_{Y3},c_{Y4}\}$, $C_Z= \{c_{Z1}, c_{Z2}, c_{Z3}\}$ As already said, the truth multiplier $I$ is the closed interval $[0,1]$ with its canonical order --so with its canonical structure of Heyting algebra, and if needed, endowed with the usual topology inherited by the real line.
		\item Propositions of interest for us are of the following form:
		      \[\lambda gcd.p(g, c, d) : \{X,Y,Z\}\times C\times W \to \Omega_I\]
		      where $W$ is a set of days, that for the sake of explicitness can be taken equal to the set of weekdays $S,M,T$ (strictly speaking, the paradox involves just the interval $[Tu,Fri]$). $p(g,c,d)$ has to be read as ``in $g$'s frame of existence the coin $c$ exists at day $d$ with strength $p(g,c,d)$''
	\end{itemize}
	Definiamo ora \emph{ammissibile} una configurazione tale che le condizioni seguenti sono rispettate: for all day $d$ and coin $c$, we have
	\[
		\sum_{u\in \{X,Y,Z\}} p(u,c,d) = (\top, 1)
	\]
	where we denote as ``sum'' the logical conjunction in $\Omega_I$: this means that day by day, the \emph{global} existence of the group of coins constantly attains the maximum; it is the \emph{local} existence that lowers when the initial conglomerate of coins is partitioned. Moreover,
	\[
		\begin{cases}
			\sum_{c_X\in C_X} p(X,c_X,V) = (\top,1) \\
			\sum_{c_Y\in C_Y} p(Y,c_Y,G) = (\top,1) \\
			\sum_{c_Z\in C_Z} p(Z,c_Z,V) = (\top,1)
		\end{cases}
	\]
	In an admissible configuration the subsets $ C_X, C_Y, C_Z $ can only attain an existence $p(g,c,d) \lneq (\top,1)$; that is no coin completely exists \emph{locally}. But for an hypothetical external observer, capable of adding up the forces with which the various parts of $C$ exist, the coins \emph{globally} exist `` in some secret way, of understanding forbidden to men'' (or rather, to $ X, Y, Z $).
	\begin{center}
		\begin{figure}
			\begin{tikzpicture}[xscale=6, yscale=4]
				\coordinate (1) at (0,0);
				\foreach \i/\j in {1/W,2/Th,3/Fr}{
						\draw[gray!90] (\i/4,0) node[below] {\tiny \j} -- (\i/4,1);
					}
				\node[below,gray!90] at (0,0) {\tiny Tu};
				\node[below,gray!90] at (1,0) {\tiny S};
				\coordinate (1) at (0,0);
				\coordinate (2) at (1/2, 1/5);
				\coordinate (3) at (3/4,1);
				\draw[ultra thick,red] (1) -- (2) node[right] {\scriptsize $1/5$} -- (3);
				\draw[thin,yellow] (1) -- (2) -- (3);
				\draw[blue] (1) -- (1/2,1) node[above] {\scriptsize $p(Y, c_Y)=(\top,1)$} -- (3/4,1/4) node[right] {\scriptsize $1/4$};
				\node[fill=gray!60] (cloud) at (1/4,1/2) {\large\color{black} \faCloud};
				\draw[thick] (0,0) rectangle (1,1);
			\end{tikzpicture}
			\caption{A pictorial representation of the truth forces of coins in different days, a piecewise linear model. $X$ is red, $Z$ is yellow, $Y$ is blue. Time is considered as a continuum marked at weekdays.}
		\end{figure}
	\end{center}
\end{example}
\begin{remark}
  L'aritmetica di Tlon; proposizioni a forza additiva; parallelismi tra la nave di Teseo e le nove monete.
\end{remark}
\begin{remark}[Continuity for a proposition]\label{continuiti}
	Let $p : U \to \Omega_I$ be a proposition; here we investigate what does it mean for $p$ to be (globally) continuous with respect to the Euclidean topology on $I=[0,1]$, in the assumption that its domain of definition $U$ is metrizable (this is true for example when $U$ is a subset of space-time). The condition is that
	\[ \forall \epsilon > 0,\,\exists \delta > 0 : |x-y|< \delta \To |px-py| < \epsilon \]
	Da questo segue immediatamente che quando $p$ è continua nel suo dominio, i valori di verità di $p$ in configurazioni ``vicine'' in un senso opportuno sono dati con forze allo stesso modo vicine (chiaramente questa è una descrizione spannometrica della nozione di continuità\dots).

	All elementary topology results apply to such a proposition: the set of forces with which $p$ is true or false is a connected subset of $\Omega_I$, compact if $U$ was compact.
\end{remark}
\begin{example}[Discontinuity, Lo zaffiro di Taprobana]\label{bli}
	la lotteria a Babilonia come in \cite{babil}: proposizioni $p : U \to \Omega_{[0,1]}$ possono essere fortemente discontinue nelle variabili/contesto da cui dipendono: tali proposizioni descrivono eventi apparentemente caotici, innescati come termine finale di una catena di eventi tra loro disconnessi e paradossali;\footnote{Assumendo una base reale per lo spazio dei parametri da cui $p$ dipende, la sua dipendenza continua è la proprietà enunciata in \ref{continuiti}; ciò significa che eventi vicini --nello spazio o nella consequenzialità temporale-- non possono avere valori di verità diversi, e forze ``vicine'' in un senso opportuno.} un modello di questo unvierso si trova probabilmente nella Babilonia di \cite{babil}, e nei ``sorteggi impersonali, di proposito indefinito'' che caratterizzano l'operato della Compagnia, azioni apparentemente scorrelate tra loro (scagliare ``nelle acque dell'Eufrate uno zaffiro di Taprobana''; sciogliere ``dal tetto d'una torre [\dots\unkern] un uccello''; togliere (o aggiungere) ``un granello di rena ai grani innumerevoli della spiaggia''; queste azioni hanno ``conseguenze, a volte, tremende''.
\end{example}
\begin{example}[Continuity: a few birds, a horse]\label{blu}
	Per quanto riguarda le proposizioni che sono continue nelle proprie variabili, invece, esempio canonico sono le ``tigri di cristallo'' e le ``torri di sangue'' di Tl\"{o}n: oggetti ed entità usuali, diversi da quelli ``classici'' per un dettaglio solo (il colore, la consistenza, il materiale di cui sono composte): la loro esistenza è sfumata, forte meno del massimo, cosicché tigri di carne e una torre di pietra sono tali che $p=(\top,1)$, le loro controparti su Tl\"on esistono con meno forza.

	Un altro esempio illustre è fatto da ogetti la cui forza di esistenza dipende in maniera \emph{monotona} e continua dai loro parametri: per esempio una proposizione $p$ può essere tanto più vera quanta più gente la osserva, perché ``le cose, su Tlön, si duplicano; ma tendono anche a cancellarsi e a  perdere i dettagli quando la gente le dimentichi. È classico l'esempio di  un'antica soglia, che perdurò finchè un mendicante venne a visitarla, e che alla  morte di colui fu perduta di vista. Talvolta pochi uccelli, un cavallo, salvarono le  rovine di un anfiteatro. ''

	In questa situazione, poniamo ad esempio che la forza di esistenza di alcune rovine --modellate come è ingenuo fare, come un corpo rigido $R$ nello spazio, dipenda dal numero dei suoi osservatori:
	\[\textstyle p(R, n) = \big(\top, 1-\frac{1}{n}\big)\]
\end{example}
Succedono cose interessanti anche a cambiare topologia su $I$: per esempio, su $[0,1]$ possiamo mettere brutalmente la topologia discreta; in questo modo $I$ è l'unione disgiunta dei suoi punti $\{ \{t\} \mid t\in [0,1]\}$, e il classifo è l'unione disgiunta di $[0,1]$ copie di $\{0,1\}$. (See \ref{fig:berkeley} for a picture.)
\begin{example}[La campagna incendiata] 
  Il Berkeley idealista degli infiniti istanti di tempo continuo, disconnessi e incomunicabili: $\Omega_I = \coprod_{t : [0,1]} \{ 0 < 1 \}$. E' evidente come questa particolare struttura logica influenzi il linguaggio piegandolo a diventare l'istantaneismo berkeleyano: i termini sono costruiti per accrezione istantanea, per somma disgiunta dei costituenti e delle loro proprietà: ``aereo-chiaro sopra scuro rotondo''; oggetti determinati dalla loro simultaneità, e non da una dipendenza logica: il significato si costruisce per accrezione di istanti simultanei, e non per sequenzialità temporale (vedi \ref{} per un legame tra questo principio, la natura additiva della forza di verità in Tlon, e la particolare forma dell'aritmetica Tloniana). 
  
  Il rifiuto della consequenzialità temporale per gli abitanti di Tlon sta nel passo 
  \begin{quote}
    \hspace{.5em} Spinoza attribuisce alla sua inesauribile divinità i modi del pensiero e dell'estensione; su Tlön, nessuno comprenderebbe la giustapposizione del  secondo (che caratterizza solo alcuni stati) e del primo, che è un sinonimo  perfetto del cosmo. In altre parole: non concepiscono che lo spaziale perduri  nel tempo. La percezione di una fumata all'orizzonte, e poi della campagna  incendiata, e poi della sigaretta mal spenta che provocò l'incendio, è  considerata un esempio di associazione di idee.
  \end{quote} 
  Ciò si lega anche al passo ``L'universo è paragonabile a quelle crittografie in cui non tutti i segni hanno un valore, e che solo è vero ciò che accade ogni trecento notti'': un mondo dove ogni trecento notti $p(x) =(\top,1)$, e per le successive 299 notti $p$ ha forza $<1$.
	\begin{center}
		\begin{figure}
			\begin{tikzpicture}[xscale=4, yscale=2]
				\fill[gray!30] (0,0) rectangle (1,1);
				\foreach \i in {.1,1,...,10}
				\draw[xshift=\i, yshift=-\i, ultra thin, fill=gray!30, opacity=.5] (0,0) rectangle (1,1);
				\draw[->, >=stealth, thin] (-.1,0) -- (-.1,1.5);
				\draw[->, >=stealth, thin] (-.1,0) -- (.5,-3/5);
				\draw[gray!60, xshift=.5cm, yshift=.25cm] (0,0) .. controls (.5,0) and (0,-.5) .. (.5,0);
			\end{tikzpicture}
      \caption{Il tempo come sequenza infinita, e infitamente suddivisibile, di istanti distinti: il paradosso Berkeleyano.}
      \label{fig:berkeley}
		\end{figure}
	\end{center}
\end{example}
\begin{example}

\end{example}

  \section{Vistas on ontologies}
\todo[inline]{Qui cosa mettiamo?}

  \appendix
  \section{Category theory}
\epigraph{El atanor está apagado -repitió- y están llenos de polvo los alambiques. En este tramo
	de mi larga jornada uso de otros instrumentos.}{\cite{arena}}
\subsection{Fundamentals of CT}
Throughout the paper we employ standard basic category-theoretic terminology, and thus we refrain from giving a self contained exposition of elementary definitions. Instead, we rely on famous and wide-spread sources like \cite{Bor1,Bor2,McL,riehlcontext,leinster2014basic,simmons2011introduction}.

Precise references for the basic definitions can be found
\begin{itemize}
	\item for the definition of category, functor, and natural transformation, in \cite[1.2.1]{Bor1}, \cite[I.2]{McL}, \cite[1.2.2]{Bor1}, \cite[I.3]{McL}, \cite[1.3.1]{Bor1}.
	\item The Yoneda lemma is stated as \cite[1.3.3]{Bor1}, \cite[III.2]{McL}.
	\item For the definition of co/limit and adjunction, in \cite[2.6.2]{Bor1}, \cite[III.3]{McL}, \cite[2.6.6]{Bor1}, \cite[III.4]{McL} (consider in particular the definitions of \emph{pullback}, \emph{product}, \emph{terminal object}).
	\item For the definition of accessible and locally presentable category in \cite[5.3.1]{Bor2}, \cite[5.2.1]{Bor2}, \cite{Adamek1994}.%, \cite[]{}.
	\item Basic facts about ordinal and cardinal numbers can be found in \cite{kunen}; another comprehensive reference on basic and non-basic set theory is \cite{jech2013set}.
	\item The standard source for Lawvere functorial semantics is Lawvere's PhD thesis \cite{lawvere1963functorial}; more modern accounts are \cite{hyland2007category}.
	\item Standard references for topos theory are \cite{mac1992sheaves,JohnstonePT}. See in particular \cite[VI.5]{mac1992sheaves} and \cite[5.4]{JohnstonePT} for what concerns the Mitchell-Bénabou language of a topos.
\end{itemize}
\subsection{Toposes}\leavevmode
For us, an \emph{ordinal number} will be an isomorphism class of well\hyp{}ordered sets, and a \emph{cardinal number} is any ordinal which is not in bijection with a smaller ordinal. Every set $X$ admits a unique \emph{cardinality}, i.e. a least ordinal $\kappa$ with a bijection $\kappa \cong X$ such that there are no bijections from a smaller ordinal. We freely employ results that depend on the axiom of choice when needed. A cardinal $\kappa$ is \emph{regular} if no set of cardinality $\kappa$ is the union of fewer than $\kappa$ sets of cardinality less than $\kappa$; all cardinals in the following subsection are assumed regular without further mention.

Let $\kappa$ be a cardinal; we say that a category $\clA$ is $\kappa$\hyp{}filtered if for every category $\clJ\in\Qat_{<\kappa}$ with less than $\kappa$ objects, $\clA$ is injective with respect to the cone completion $\clJ\to \clJ^\rhd$; this means that every diagram
\[
	\vcenter{\xymatrix{
			\clJ\ar[d]\ar[r]^D & \clA \\
			\clJ^\rhd\ar@{.>}[ur]_{\bar D}
		}}
\]
has a dotted filler $\bar D : \clJ^\rhd \to \clA$.

We say that a category $\clC$ admits filtered colimits if for every filtered category $\clA$ and every diagram $D : \clA \to \clC$, the colimit $\colim D$ exists as an object of $\clC$. Of course, whenever an ordinal $\alpha$ is regarded as a category, it is a filtered category, so a category that admits all $\kappa$\hyp{}filtered colimits admits all colimits of chains
\[
	C_0 \to C_1 \to \cdots \to C_\alpha \to\cdots
\]
with less than $\kappa$ terms. A useful, completely elementary result is that the existence of colimits over all ordinals less than $\kappa$ implies the existence of $\kappa$\hyp{}filtered colimits; this relies on the fact that every filtered category $\clA$ admits a cofinal functor (see \cite{Bor1}) from an ordinal $\alpha_\clA$.

We say that a functor $F : \clA \to \clB$ \emph{commutes with} or \emph{preserves} filtered colimits if whenever $\clJ$ is a filtered category, $D : \clJ \to \clA$ is a diagram with colimit $L=\colim_\clJ D_j$, then $F(L)$ is the colimit of the composition $F\circ D$. Anoter common name for such an $F$ is a \emph{finitary} functor, or a functor \emph{with rank $\omega$}.
\begin{definition}\label{accepre}
	Let $\clC$ be a category;
	\begin{itemize}
		\item We say that $\clC$ is \emph{$\kappa$\hyp{}accessible} if it admits $\kappa$\hyp{}filtered colimits, and if it has a \emph{small} subcategory $\clS\subset \clA$ of $\kappa$\hyp{}presentable objects such that every $A\in\clA$ is a $\kappa$\hyp{}filtered colimit of objects in $\clS$.
		\item We say that $\clC$ is \emph{(locally) $\kappa$\hyp{}presentable} if it is accessible and cocomplete.
	\end{itemize}
	The theory of presentable and accessible categories is a cornerstone of \emph{categorical logic}, i.e. of the translation of model theory into the language of category theory.

	Accessible and presentable categories admit \emph{representation theorems}:
	\begin{itemize}
		\item A category $\clC$ is accessible if and only if it is equivalent to the ind\hyp{}completion $\text{Ind}_\kappa(\clS)$ of a small category, i.e. to the completion of a small category $\clS$ under  $\kappa$\hyp{}filtered colimits;
		\item A category $\clC$ is presentable if and only if it is a full reflective subcategory of a category of presheaves $i : \clC \to \Qat(\clS^\op,\Set)$, such that the embedding functor $i$ commutes with $\kappa$\hyp{}filtered colimts.
	\end{itemize}
\end{definition}
All categories of usual algebraic structures are accessible, and they are locally presentable as soon as they are cocomplete; an example of a category which is $\aleph_1$\hyp{}presentable but not $\aleph_0$\hyp{}presentable: the category of metric spaces and short maps.

We now glance at \emph{topos theory}:
\begin{definition}\label{eletop}
	An \emph{elementary topos} is a category $\clE$
	\begin{itemize}
		\item which is \emph{Cartesian closed}, i.e. each functor $\firstblank\times A$ has a right adjoint $[A, \firstblank]$;
		\item having a \emph{subobject classifier}, i.e. an object $\Omega\in\clE$ such that the functor $\text{Sub} : \clE^\op\to \Set$ sending $A$ into the set of isomorphism classes of monomorphisms $\var{U}{A}$ is representable by the object $\Omega$.
	\end{itemize}
	The natural bijection $\clE(A,\Omega)\cong\text{Sub}(A)$ is obtained pulling back the monomorphism $U\subseteq A$ along a \emph{universal arrow} $t : 1\to \Omega$, as in the diagram
	\[
		\vcenter{\xymatrix{
				U \pb\ar[r]\ar[d]& 1\ar[d]^t \\
				A \ar[r]_{\chi_U}& \Omega
			}}
	\]
	so, the bijection is induced by the map $\var{U}{A}\mapsto \chi_U$.
\end{definition}
\begin{definition}\label{grotop}
	A \emph{Grothendieck topos} is an elementary topos that, in addition, is locally presentable.
\end{definition}
The well-known \emph{Giraud theorem} gives a proof for the difficult implication of the following \emph{recognition principle} for Grothendieck toposes:
\begin{theorem}
	Let $\clE$ be a category; then $\clE$ is a Grothendieck topos if and only if it is a left exact reflection of a category $\Qat(\clA^\op,\Set)$ of presheaves on a small category $\clA$.
\end{theorem}
(recall that a \emph{left exact reflection} of $\clC$ is a reflective subcategory $\clR\hookrightarrow \clC$ such that the reflector $r : \clC \to \clR$ preserves finite limits. It is a reasonably easy exercise to prove that a left exact reflection of a Grothendieck topos is again a Grothendieck topos; Giraud proved that all Grothendieck toposes arise this way.)
\subsection{A little primer on algebraic theories}\label{funsemanzi}
The scope of this short subsection is to collect a reasonably self-contained account of functorial semantics. It is unrealistic to aim at such a big target as providing a complete account of it in a single appendix; the reader is warmly invited to parallel their study with more classical references as \cite{lawvere1963functorial}.
\begin{definition}[Lawvere theory]\label{lo_tiori}
	A \emph{Lawvere theory} is a category having objects the natural numbers, and where the sum on natural numbers has the universal property of a categorical product, as defined e.g. in \cite[2.1.4]{Bor1}.
\end{definition}
Let us denote $[n]$ the typical object of $\clL$. Unwinding the definition, we deduce that in a Lawvere theory $\clL$ the sum of natural numbers $[n+m]$ is equipped with two morphisms $[n] \leftarrow [n+m] \to [m]$ exhibiting the universal property of the product.

Every Lawvere theory comes equipped with a functor $p : \cate{Fin}^\op \to \clC$ that is the identity on objects and preserves finite products. A convenient shorthand to refer to the Lawvere theory $\clL$ is thus as the functor $p$, or as the pair $(p,\clL)$.
\begin{definition}
	The category $\cate{Law}$ of Lawvere theories has objects the Lawvere theories, understood as functors $p : \cate{Fin}^\op \to \clL$, and morphisms the functors $h :  p\to q$ such that the triangle
	\[\xymatrix{
			& \cate{Fin}^\op \ar[dr]^q \ar[dl]_p & \\
			\clL \ar[rr]_h && \clM
		}\]
	is commutative. It is evident that $\cate{Law}$ is the subcategory of the undercategory $\cate{Fin^\op}/\Qat$ (see e.g. \cite[I.6]{McL}for a precise definition) made by those functors that preserve finite products.
\end{definition}
\begin{remark}
	The category $\textsf{Law}$ has no nonidentity 2-cells; this is a consequence of the fact that a natural transformation $\alpha : h \To k$ that makes the triangle ``commute'', i.e. $\alpha * p = \id_q$ must be the identity on all objects.
\end{remark}
\begin{example}[The trivial theories]
	The category $\cate{Fin}^\op$, opposite to the category of finite sets and functions, is the initial object in the category  $\cate{Law}$; the terminal object is constructed as follows: the category $\clT$ has objects the natural numbers, and $\clT([n],[m])=\{*\}$ for every $n,m: \bbN$. It is evident that given this definition, there is a unique identity-on-objects functor $\clL \to \clT$ for every other Lawvere theory $(p,\clL)$.
\end{example}
\begin{definition}[Model of a Lawvere theory]
	A \emph{model} for a Lawvere theory $(p ,\clL)$ consists of a product-preserving functor $L : \clL \to \Set$. The subcategory $[\clL,\Set]_\times \subset [\clL, \Set]$ of models of the theory $\clL$ is \emph{full}, i.e. a morphism of models $L \to L'$ consists of a natural transformation $\alpha : L \Rightarrow L'$ between the two functors.
\end{definition}
Observe that the mere request that $\alpha : L \to L'$ is a natural transformation between product preserving functors means that $\alpha_{[n]} : L[n] \to L'[n]$ coincides with the product $(\alpha_{[1]})^n : L[1]^n \to L'[1]^n$.
\begin{proposition}
	Let $p : \cate{Fin}^\op\to \clL$ be a Lawvere theory. Then, the following conditions are equivalent for a functor $L : \clL \to \Set$:
	\begin{itemize}
		\item $L$ is a model for the Lawvere theory $(p,\clL)$;
		\item the composition $L\circ p : \cate{Fin}^\op \to \Set$ preserves finite products;
		\item there exists a set $A$ such that $L\circ p = \Set(j[n],A)$.
	\end{itemize}
\end{proposition}
\begin{corollary}\label{da_pull}
	The square
	\[
		\xymatrix{
			\cate{Mod}(p,\clL) \ar[d]_u \ar[r]^r & [\clL,\Set]\ar[d]^{p^*} \\
			\Set \ar[r]_{N_j} & [\cate{Fin}^\op,\Set]
		}
	\]
	is a pullback of categories. The functor $u$ is completely determined by the fact that $u(L) = L[1]$, $r$ is an inclusion, and $N_j(A) = \lambda F.\Set(F,A)$ is the functor induced by the inclusion $j : \cate{Fin} \subset \Set$.
\end{corollary}
\begin{corollary}
	The category of models $\cate{Mod}(p,\clL)$ of a Lawvere theory is a locally presentable, accessibly embedded, complete and cocomplete subcategory of $[\clL,\Set]$. Moreover, the forgetful functor $u : \cate{Mod}(p,\clL) \to \Set$ of \autoref{da_pull} is \emph{monadic} in the sense of \cite[4.4.1]{Bor2}. A complete proof of all these facts is in \cite[3.4.5]{Bor2}, \cite[3.9.1]{Bor2}, \cite[5.2.2.a]{Bor2}. A terse argument goes as follows: the functors $p^*, N_j$ are accessible right adjoints between locally presentable categories; therefore, so is the pullback diagram: $r$ is a fully faithful, accessible right adjoint, and $u$ is an accessible right adjoint, that moreover reflects isomorphisms. It can be directly proved that it preserves the colimits of split coequalizers, and thus the adjunction $f \dashv u$ is monadic by \cite[4.4.4]{Bor2}.
\end{corollary}
The last technical remark that we collect sheds a light on the discorso prolisso in \autoref{as_places}: the models of a thelory $\clL$ interpreted in the category of models of a theory $M$ correspond to the models of a theory $\clL \otimes \clM$, defined by a suitable universal property:
\[
	\cate{Mod}(\clL\otimes \clM,\Set)  \cong
	\cate{Mod}(\clL, \cate{Mod}(\clM,\Set))  \cong
	\cate{Mod}(\clM, \cate{Mod}(\clL,\Set)).
\]
\begin{definition}
	Given two theories $\clL$ and $\clM$ it is possible to construct a new theory called the \emph{tensor product} $\clL \otimes \clM$; this new theory can be characterized by the following universal property: the models of $\clL \otimes \clM$ consist of the category of $\clL$-models interpreted in the category of $\clM$-models or, equivalently (and this is remarkable) of $\clM$-models interpreted in the category of $\clL$-models.
\end{definition}
\begin{theorem}
	(\cite[4.6.2]{Bor2}) There is an equivalence between the following two categories:
	\begin{itemize}
		\item $\cate{Law}$, regarded as a non-full subcategory of the category $\textsf{Fin}^\op/\Qat$, i.e. where a morphism of Lawvere theories consists of a functor $h : \clL \to \clM$ that preserves finite products;
		\item \emph{finitary} monads, i.e. those monads that preserve filtered colimits, and morphisms of monads in the sense of \cite[4.5.8]{Bor2}.
	\end{itemize}
\end{theorem}
\begin{proof}
	The proof goes as follows: given a Lawvere theory $p : \textsf{Fin}^\op \to \clL$, we have shown that the functor $u : \cate{Mod}(p) \to \Set$ in the pullback square \autoref{da_pull} has a left adjoint $f : \Set \to \cate{Mod}(p)$; the composition $uf$ is thus a monad on $\Set$. This is functorial, when a morphism of monads is defined
\end{proof}
\begin{definition}
	\label{internista}
	Let  $\clC$ be a category with finite limits. An \emph{internal category} $\underline{A}$ consists of a pair $(A_0, A_1)$ of objects of $\clC$, endowed with morphisms $s,t,c,i$ as in the following diagram
	\[
		\xymatrix{
		A_1 \times_{A_0} A_1 \ar[r]|-c \ar@<6pt>[r]^-{p_0}  \ar@<-6pt>[r]_-{p_1} & A_1 \ar@<6pt>[r]^s  \ar@<-6pt>[r]_t & A_0 \ar[l]|i
		}
	\]
	where $A_1 \times_{A_0} A_1$ is obtained pulling back $s,t$. These data must satisfy the following axioms, resembling the category axioms:
	\begin{enumtag}{ic}
		\item $i$ equalises the pair $(s,t)$, i.e. $s\circ i= t\circ i$, and this composition makes the identity $1_{A_0}$;
		\item $t\circ p_1 = t\circ c$ and $s\circ p_0 = s \circ c$;
		\item $c\circ\la 1_{A_1},i\circ s\ra = c\circ\la i\circ t, 1_{A_1}\ra$, and this composition makes the identity $1_{A_1}$;
		\item $c$ is \emph{associative}, i.e. $c \circ (1_{A_1}\times_{A_0} c)= c \circ (c \times_{A_0} 1_{A_1})$.
	\end{enumtag}
\end{definition}
These axioms are meant to re-enact the definition of category, as an abstract ``object of objects'' and ``object of morphisms'', endowed with maps $s,t: A_1 \to A_0$ sending every morphism to its \emph{source} and \emph{target}, with an \emph{identity selector} map $i : A_0 \to A_1$ and with a composition partial binary operation $c : A_1 \times A_1 \to A_1$ whose domain is the object of \emph{composable} arrows. Cf. \cite[8]{Bor1} for a torough discussion.
It has to be noted however that the theory of categories is \emph{not} an algebraic theory in the sense of our \autoref{lo_tiori}, and this just because the composition operation is only \emph{partially} defined over the domain of composable arrows, i.e. on the pullback $A_1\times_{A_0} A_1$. Theories whose syntax allows for a number of partially defined relation and function symbol are termed \emph{essentially algebraic}. See \cite{Adamek1994} for a torough discussion of the topic.

  \bibliography{allofthem}{}
  \bibliographystyle{amsalpha}
\end{document}
