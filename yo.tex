\documentclass{amsart}

\usepackage{fouche}
\begin{document}
\section{} 
Ti risparmio una disamina completa delle varie nozioni di identità esistenti in letteratura analitica, anche se è di per sè interessante. Basti sapere alcune questioni sulle quali credo un approccio categorial-omotopico abbia qualcosa da dire: 

Il dibattito Black-Lowe-Wiggins-Lewis (cfr. bibliografia) sulla legge di Leibniz è stato proficuo nella misura in cui ha chiarito cosa non va nel criterio di identità da essa implementato; lasciando da parte un attimo Black, su cui torneremo alla fine, - e al quale interessa far notare come la diversa locazione spaziale di 2 oggetti con le stesse proprietà non è condizione sufficiente a distinguerli, e il criterio fallisce proprio con oggetti astratti come una/due sfera/e -, la questione che si pongono più o meno tutti è cosa caratterizza l'identità rispetto ad altre proprietà, e se è davvero una nozione così banale e intuitiva come sembra, dotata solo di banali proprietà formali (a=a). 

La risposta è ovviamente "NO". Non è una nozione "primitiva", nè realmente intuitiva nè necessaria (può essere ridotta a relazioni più astratte, come la composizione ristretta (assioma della GEM) o la I-predicabilità); questo mi sembra concordare con la dissoluzione della nozione attuata dall'avvento dell'omotopia.

Inoltre si rileva l'assurdità di una caratterizzazione "assoluta" dell'identità, laddove ha senso - se ha senso parlarne - solo relativamente a una ontologia/potenza espressiva di una teoria T. 

[la distinzione di Lowe tra insieme collettivo (semplice somma di parti) e aggregato (e oggetto integrale), utile a catturare alcune intuitive proprietà dell'identità di oggetti materiali, è una questione mereologica, informa una serie di problemi specifici della mereologia formale e a mio parere (più sotto se ne parla meglio) va affrontata adeguatamente con strumenti topologici, peraltro con la possibilità (ma cavoli se sarà dura dimostrarlo) di mantenere la cosiddetta "mereological innocence", cioè l'assenza di ontological committment nei confronti di interi che non siano aggregati. Questione niente affatto da trascurare e che, peraltro, homotopy type theory + lavorone su ROSR "categoriale" (ebbene sì, i lavori possono essere collegati...ma già lo sai) chiarirebbero definitivamente, o comunque con una certa precisione]

[(Lewis, 1991) non mi convince appieno nell'argomentazione ma le conclusioni a cui arriva sì. Ho fatto lunghi audio di riflessione su questa roba, e riguarda i rapporti tra identità e composizione ristretta. Anch'essi a mio parere riceverebbero un trattamento migliore in matematica]

Altra roba interessante rilevata nel dibattito classico è la nozione di I-predicabile in Geach (1967). Mentre in Lewis si caratterizza l'identità sotto un concetto sortale (che si può dimostrare corrispondere alla condizione Q nella composizione ristretta della GEM) in Geach l'identità tra n oggetti è un'abbreviazione per un'espressione che contiene un nome sottinteso. 
I-PREDICABLE: è un predicato (lasciamo stare la sua distinzione tra predicato e predicabile) a 2 posti in una data teoria T che soddisfa lo schema:
\[\tag{$\star$}
\vdash F(a) \iff Ex (F(x) \& x=a)	\text{per tutte le espressioni costruibili di } T
\]
(esempio: se in $G(\xi, \eta, \zeta)$ sostituisco con variabili nominali tranne in un posto, $G(v,a,w)$,  voglio che $G(v,a,w) \iff Ex (G(v,x,w) \& x=a)$ risulti valida in base a (*)).

Un i-predicabile è dipendente dalla teoria T e non è detto che esprima l'identità stretta (relativamente a T), ma più semplicemente esprime il fatto che 2 oggetti sono indiscernibili in base ai predicati di T.

Insomma l'identità (o una relazione equivalente) è dipendente dalla ontologia di T e dalla sua "ideologia" (per dirla alla Quine) ovvero - su per giù - le regole di formazione delle ffbf di T (giusto per tradurre un po' di lessico filosofico). 

Perchè tutta questa noia assoluta? Sono stanco ma credo di poter dire che l'utilità sta nell'aver parzialmente risposto alla domanda "in quale corrente filosofica si inserisce il motto no identity without homotopy". In questo dibattito sufficientemente serio, che può in gran parte essere riformulato e dissolto omotopicamente, e che solleva alcune questioni interessanti (gli aggregati, l'innocenza mereologica, la dipendenza dall'ontologia della nozione di identità o suoi sostituti eventuali). 

ah, poi, ovvio, l'inadeguatezza del criterio di Leibniz. Ma c'è un modo più specifico per confrontarcisi, vediamo dopo.

per ora credo che la domanda da porsi e la direzione su cui lavorare, stante il punto (1), sia: "La nozione di omotopia è generalizzabile a sufficienza (immagino che qui categorie a gogò) da catturare le differenti nozioni di identità, le sue lacune (alla fine non parliamo di identità stretta o assoluta, e qui è stato significativo 'sto dibattito filo), magari evitando ulteriori impegni ontologici?" (qui anticipo che a me pare che, al di fuori delle relazioni, non ci si debba impegnare su nulla, e forse nemmeno sulle relazioni, stante lavoro adeguato su ROS moderato). E, aggiungo: "cattura essa il "ventaglio" fuzzy di relazioni di identità comprese tra quella leibniziana stretta (opportunamente riformulata) e la "distinzione" stretta?"

\section{}
DUE QUESTIONI SERIE sulle quali il possibile paper si imbatterà e che richiederebbero una trattazione adeguata, una più matematica, una più filosofica.

2.1)

per affrontare i vari problemi che si incontrano quando si caratterizza l'identità, o la si rifiuta, per una gamma ampia di oggetti (materiali, astratti, etc) (cfr. Varzi, 2007) è necessario che l'omotopia, nel trattarli, sia sensibile alle modificazioni temporali, oltre che spaziali. Per cui quello su cui bisogna lavorare (e c'è poca roba in giro su questo) è una seria mereotopologia formale. 

Credo che sia sufficiente introdurre nel sistema un operatore di "posizione" Pa alla Rescher e Garson (1968) + magari operatore $\xi$ di "posizione privilegiata" e far quadrare tutto con una logica temporale (lo fa già Rescher) e una mereologia non minimale [la GEM, perché è come avere a che fare con la counterpart theory ma senza il realismo modale estremo; e in più in essa il ruolo dell'identità è svolto da altre relazioni, credo tutte reinterpretabili omotopicamente] (e questo non lo ha fatto credo nessuno).

2.2)

serissima ma forse inaffrontabile per ora: credo che un lavoro del genere, così "totale" a livello di trattamento degli oggetti di "competenza" dell'ontologia, costringa a chiedersi se per caso non si ripropone qua la distinzione tra semantica algebrica ed applicata (Plantinga, 1978), ma in termini di ontologia algebrica/applicata. Cioè il problema se e quanto sia lecito passare dal trattamento formale di un dominio di oggetti all'interpretazione del modello inteso (nel nostro caso immagino i diversi tipi di oggetti di cui tratterebbe questa nozione "allargata" di omotopia) con una giustificazione, in un certo qual modo, extra-logica. L'analisi delle proposizioni della semantica algebrica che fanno Plantinga e Lewis per giustificare il collegamento tra quell'oggetto formale "V" e l'effettiva verità della proposizione, (se non la conosci ne parliamo), si può probabilmente "tradurre" in termini ontologici, ma credo vada ripetuta per i diversi domini di oggetti ogni volta. (si può riformulare dicendo che va giustificato il passaggio dalla ideologia alla ontologia di una data T?).

[sembra una roba astrusa ma a un certo punto non  si può far finta di niente, se uno vuol fare veramente ontologia con strumenti diversi.

peraltro torno a ripetere che la questione è collegata col ROSR (non col ROSR in sè ma con la tipologia di lavoro che lì desideri fare), soprattutto se si vanno a vedere i suoi "equivalenti ontologici" (teoria dei fasci, teoria dei tropi in una particolare accezione etc). Ma boh su questo mi riservo il dubbio di star saltando troppo di palo in frasca]
\section{}
il problema specifico che a mio parere si dovrebbe affrontare per primo
(e allora perchè lo metti per ultimo? non lo so)

assumiamo che sia certo che tutte le nozioni di identità elaborate dalla tradizione analitica siano "isomorfe" alla legge di Leibniz. Ci dovremmo ridurre a osservare i problemi di quest'ultima e verificare se la candidata "omotopia" li risolve.

L'articolo notorio di Black (1952) è la miglior confutazione del criterio che ci sia. O meglio presenta un valido controesempio alla implicazione da sinistra a destra del criterio ($x=y \rightarrow F(x) \leftrightarrow F(y)$):
\begin{quote}
si supponga che l'universo sia costituito solo da due sfere perfettamente identiche in tutto e per tutto, con medesime proprietà formali e materiali. Esse risulterebbero assolutamente indistinguibili. Per determinarne l'identità numerica dovrebbe bastare l'unica proprietà che presumibilmente non hanno in comune, cioè la locazione spaziale. Tuttavia:

"[A:] Ognuna delle due sfere sarà certamente diversa dall'altra per essere a una certa distanza da quell'altra, ma a distanza nulla da sè stessa; vale a dire essa avrà almeno una relazione con sè stessa - l'essere a distanza nulla da o l'essere nello stesso luogo di - che non ha con l'altra. (...)   

B: (...) Ognuna avrà la caratteristica relazionale di essere a una distanza di 2 miglia, diciamo, dal centro di una sfera di un diametro, ecc. E ognuna avrà la caratteristica relazionale (...) di essere nello stesso luogo di se stessa. le due sono simili per questo riguardo come per chiunque altro. 

A: Ma ogni sfera occupa un luogo diverso; e questo varrà a distinguerle

B: Ciò suona come se voi pensaste che i luoghi abbiano una qualche esistenza indipendente (...). [qui il ragionamento sui luoghi è di natura relazionale, e credo sia inevitabile nella nostra prospettiva ragionare così] Dire che due sfere sono in luoghi diversi equivale appunto a dire che c'è una certa distanza tra 2 sfere e abbiamo visto che questo non varrà a distinguerle. Ognuna è a una certa distanza - invero la stessa distanza - dall'altra"
\end{quote}
Ci sono altre specifiche nel dialogo ma lo scheletro dell'esempio è questo. La nozione classica non riesce a distinguere tra le due sfere dell'esperimento mentale. (Il mio sospetto è che parte della problematicità sia nel fatto che le sfere sono oggetti astratti. Bisognerà porsi il problema a lungo termine. Basterebbe forse un lavoro puramente formale: il miglior trattamento degli AO, a livello di ontologia formale, è quello di Linsky e Zalta, e si può prendere quella assiomatizzazione e vedere come lavorarci con la meretopologia formale finita). 
\section{}
COSA BISOGNA FARE, insomma.
omotopia e esempio di Black: lo "risolve"? lo precisa? lo dissolve? lo aiuta a confutare definitivamente il criterio di Leibniz?

omotopia e tutte le altre implicazioni delle diverse nozioni di identità (punto (1)): sostituisce adeguatamente i-predicabilità, composizione alla Lewis, somma di aggregati et altre nozioni equivalenti, e meno rigorose? preserva le relazioni mereologiche e la loro "innocenza"? appiana e/o precisa i dissidi scaturiti dal rifiuto della nozione classica?

affrontare di petto le robe serie matematiche prima (mereotopologia che sia anche logica topologica nel senso di Rescher (cioè anche temporale), magari il tutto in linguaggio categoriale) e filosofiche poi (analisi delle proposizioni per giustificare il passaggio da ontologia pura/algebrica ad ontologia applicata). 

CAPTATIO BENEVOLENTIAE: 
mi scuso per l'estremo ritardo, volevo studiare un po' e mettere in campo tutte le cose rilevanti (e quindi ho tagliato parecchio). Non so se sia utile tutto ciò, riflessioni in libertà (o quasi) ma era per dare una idea delle direzioni verso le quali a mio parere ci si potrà muovere. Se ne parlerà a voce e mi dirai cosa ne pensi, limando i miei deliri come sai ben fare.  
BIBLIOGRAFIA:
(Black, 1952): Max Black, "The Identity of Indiscernibles"
(Geach, 1967): Peter Geach, "Identity"
(Lewis, 1991): David Lewis, "Composition as Identity", in Parts of Classes
(Lowe, 1989): Jonathan Lowe, "Parts and Wholes", in Kinds of Being
(Plantinga, 1978): Alvin Plantinga, The Nature of Necessity, (in particolare cap. VII.4)
(Rescher, 1968): Nicholas Rescher and James Garson, "Topological Logic"
(Varzi, 2007): Achille Varzi, "La natura e l'identità degli oggetti materiali", in A. Coliva, Filosofia Analitica
(Wiggins, 1968): David Wiggins, "On Being in the Same Place at the Same Time"
\end{document}