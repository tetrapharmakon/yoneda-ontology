\section{Semantical conception of theories}
During the XXth century it was considered necessary to develop a formal treatment of scientific theories. The Wiener Kreis verificationist paradigm/account, and the Neurath theory of `protocollar statements', was the input to elaborate a completely semantical framework for working with scientific theories, and the clue of a pan-linguistic vision of philosophy of science.

For the sake of strictness the formal account in which Carnap and associates provide their notion of `theory' is known in literature as \emph{syntactical conception of theories} \cite{?} while the introduction of term `semantic' is due to later developments. But the field of epistemology that the logical neopositivism started one can call `semantics of theories', because some characteristics, and above all the underlying ideology, are the same from Carnap to Beth to Suppes, up to the recent canonical uses of physical handbooks.

	[scrivere quali sono queste caratteristiche]

	[sintesi delle varie concezioni; le teorie come classi di modelli $\clK$;
		le teorie come oggetti formali]

semantica non standard per teorie empiriche in cui le teorie sono sistemi formali e tutte le nozioni diventano oggetti matematici; più propriamente una \emph{teoria} diventa una struttura $(F_\clL, \clK)$ dove $F_\clL$ è il vero sistema formale e $\clK$ è la classe di tutti i suoi modelli. La nostra strategia è separare ulteriormente $F_\clL$ in due `vocabolari' (per noi, le categorie sintattiche di teorie al primo ordine), uno $P_{F_\clL}$ che rappresenta i termini puri (nel senso di Plantinga) e uno $A_{F_\clL}$ che rappresenta i termini \emph{applicati}. Ergo una teoria $\mathbf{T}$ sarà una particolare tripla $\bk{(P_{F_\clL},A_{F_\clL}), \clK}$ in cui la prima coppia configura una logica (uno `spazio degli stati' che configura una logica).

La coppia $(P_{F_\clL},A_{F_\clL})$ è poi soggetta a una ulteriore condizione di ammissibilità, cf. \ref{}, chiedendo che esista un profuntore tra le due categorie sintattiche $P_{F_\clL},A_{F_\clL}$.

La specificazione del dominio di $A_{F_\clL}$ determina il tipo di teoria che stiamo considerando (scientifica, strettamente empirica, logico-matematica, metafisica).

Dire che $A_{F_\clL}$ determina le \emph{tipizzazioni} della teoria significa
dire che svolge lo stesso ruolo della legge $\beta$ nella semantica dello spazio degli
stati, mentre la classe $\clK$ è isomorfa all'insieme $\mathcal{M}$ dello spazio
degli stati. Il tipo di $\beta$ determina il tipo di $\mathcal{M}$ che determina il
tipo di $\mathbf{T} = (\mathcal{M}, \beta)$. Idem nel nostro approccio: $A_{F_\clL}
	= \{\alpha_1,\dots,\alpha_n\}$ determina il tipo, che implementa una logica che determina
la classe $\clK$.
%si può generalizzare tutto ciò dicendo appunto che a seconda di come è strutturata la classe $A_{\mathcal{F_L}}$ noi possiamo distinguere le teorie anche al di fuori dell'ambito strettamente scientifico. Questa considerazione va a scomparire proprio con ciò che emerge in sezione 6: eliminando questa distinzione ottieni un account più generale di trattamento delle teorie%

\subsection{The Two Dictionaries}
Nella concezione neopositivistica la distinzione tra legge teorica e legge empirica non è dovuta alla natura ipotetica della prima (anche una legge empirica può esserlo) quanto dal fatto che i due tipi di legge contengono tipi differenti di termini \cite{?}. La distinzione è quindi formale, e indica una approccio prettamente linguistico a questioni epistemologiche.
\begin{remark}\label{hint_at_collage}
	Anche in questa visione `sintattica' \cite{ } una teoria è sempre una struttura che contiene un sistema formale $\mathcal{F_L}$ e la classe $\clK$ dei suoi modelli. La strategia carnapiana per rendere conto della presenza di entità `osservazionali' e quindi, a rigore, non formalizzabili, all'interno di teorie scientifiche è quella di considerare due diversi dizionari: $\mathcal{V_T}$ che contiene \emph{termini teorici} e $\mathcal{V_O}$ che contiene \emph{termini osservativi}. 
	%Intuitivamente $\mathcal{F_L} = \mathcal{V_T} \cup \mathcal{V_O}$, ma piu precisamente $\mathcal{F_L} = \mathcal{V_T} \uplus_\varphi \mathcal{V_O}$.
\end{remark}


\begin{definition} [Wiener Theory]
	A theory $T$ consists in a formal system $\mathcal{F_L}$ and a class of models $\{\mathcal{K}_n | n \in \N \}$. The formal system consists in:
	\begin{itemize}
		\item A language $\mathcal{L}$
		\item Two vocabularies: $\mathcal{V_T}$ for theoretical terms and $\mathcal{V_O}$ for observational terms 
		\item Given $\mathsf{L}$ a first-order logic $\mathcal{V}_{\mathcal{T}}^L = \mathcal{V_T} \cup \mathsf{L}$ is a logical vocabulary of $T$
	\end{itemize}    
\end{definition}

The vocabularies exhaust all the terms of formal system: $\mathcal{F_L} = \mathcal{V_T} \cup \mathcal{V_O} \cup \mathcal{V}_{\mathcal{T}}^L$.

In carnapian, and in general neopositivistic, account a theory is expressible as a sentence formed by terms $\tau_1, \dots, \tau_k$ taken from one of the two dictionaries \cite{?}. 

Non è chiaro cosa siano i termini di $\mathcal{V_O}$ nell'epistemologia viennese: Carnap ritiene che nella sentence-form della teoria debbano intervenire delle correspondance rules che associno ad ogni termine di $\mathcal{V_O}$, o O-term, il proprio corrispondente elemento di $\mathcal{V_T}$ o T-term. [nota sull'ambiguità di tutto ciò considerando che bla bla subseteq bla bla].

Noi possiamo mantenere la stessa idea usando, più agevolmente, una \emph{translate function} $\varphi$ che "traduca" gli O-terms in T-terms. 

So, a Wiener definition of a theory is 

\[  T = \{ \pi_i, \varphi (\omega_j)\}_{\pi_i \in \mathcal{V_T}, \omega_j \in \mathcal{V_O}}
\] 
where $\varphi: \mathcal{V_O} \to \mathcal{V_T}$ such that $\omega_j \mapsto \varphi (\omega_j)$ for all $\omega_j \in \mathcal{V_O}$. In this way it is trivial that all the terms of a theory are in the first dictionary.

L'ulteriore idea neopositivistica è quella di costruire una versione "osservazionale" della teoria $T$ secondo un procedimento ispirato da un lavoro di Ramsey \cite{?}.


Più semplicemente, usando def 2.1, e considerando che $\mathcal{V_O} \subset \mathcal{V_T}$, ci pare che la ramseyfication, non potendo nessun termine $\tau_k \in T$ uscire dal vocabolario teorico, sia un ri-tradurre nel linguaggio di $\mathcal{V_O}$ il contenuto della teoria, sostituendo i termini teorici "puri" con delle variabili \cite{?}.
\begin{definition}
	Data una \emph{funzione di applicazione} $\psi: \mathcal{V_T} \to \mathcal{V_O}$ tale che $\pi_i \mapsto \psi (\pi_i) $ per ogni $\pi_i \in \mathcal{V_T}$, a Ramsey-sentence of a theory $T$ is 
	\[ \overline{T} = \{[\pi_i/x_i], \psi (\varphi (\omega_j))\} 
	\]
\end{definition}



In questa semantica si dice allora che $\overline{T}$ rappresenta il \emph{contenuto osservativo} della teoria e in generale vale che: $T \cong \overline{T}$. 



Non ci pare, al di là degli intenti riduzionistici del Wiener Kreis \cite{Weinb}, che ci siano motivi epistemologicamente validi per eseguire un tale procedimento, che del resto non ha avuto molto seguito nella storia successiva, ma la distinzione carnapiana ripresa nel nostro framework, per trarre alcune conclusioni on relations between theoretical and observational core.
%spiegare meglio perché la Ramsey-sentence da sola è un fallimento del riduzionismo viennese%

Già agli albori dell'applicazione di metodi formali in filosofia della scienza è evidente la difficoltà di ignorare la tesi duhemeana secondo la quale "tutta l'osservazione [in fisica] è carica di teoria". 

Le ambiguità carnapiane sul dominio di oggetti sui quali verterebbe la definizione viennese noi, coerenti col testo, le risolviamo indicando come dominio $\mathcal{V_T}$. Cosa sia il mondo puro delle osservazioni al quale farebbe riferimento la Ramsey-version della teoria non è chiaro, se non un altro sotto-dizionario teorico, per l'appunto. Non a caso il neopositivismo da un iniziale fisicalismo approda ad un'ottica convenzionalista \cite{?}, in seguito ai falliti tentativi di formalizzazione di un contesto osservazionale extra-teorico. 

Seguendo \cite{psillos} il realismo strutturalista che ispira questi attempts si riduce ad affermare che una teoria $T$ is logically equivalent to the conjunction
\[\overline{T} \land (\overline{T} \rightarrow T)
\] where the second member is the \emph{meaning postulate}:
\begin{quotation}
	Carnap notes that this conditional has no factual content and takes it to be a meaning postulates \cite{?}
\end{quotation}
[spiegare con un esempio tipo paper Psillos].

This kind of realism is an if-then form: non è il mondo l'oggetto della scientific knowledge ma le condizioni che, date certe premesse, si verificano nel mondo strutturale di cui parlano le teorie. [esempio del tipo: non è più "la terra gira intorno al sole" ma "dati la terra e il sole, se si verificano (nella realtà) le condizioni $x_1,x_2,x_3$ allora sarà vero il fenomeno della rotazione terrestre". Da cui emerge anche la natura predittiva]. 

Il meaning postulate non era altro che the Wiener Kreis' version of the famous demarcation problem between science and metaphysics: the secular attempt to elaborate a precise criterion to distinguish a proposition belonging to empirical sciences from a metaphysical (or, at large, not scientifical) proposition. Un criterio che, a seconda dei contesti, è sempre risultato troppo stretto o troppo largo. 

Tolti i due estremi, gli enunciati protocollari da un lato \cite{?}, i discorsi di Heidegger dall'altro \cite{?}, esistono tutti i casi intermedi per i quali a meaning criterion (o una operazione come la drastica traduzione dei costrutti teorici in "referti osservativi puri" tramite ramseyfication) impedisce concretamente di individuare la demarcazione. Classical objections range from Popper \cite{?} to more recent sociology of science.

Mostreremo, in section 5-6, come nel nostro framework si chiariscano i rapporti tra i due dizionari e quanto sia inutile il tentativo di stabilire il primato di uno dei due sull'altro (remark 5.3). Una nozione larga di \emph{teoria} richiede una efficace descrizione degli interscambi, vale a dire delle aggiunzioni, tra i due dizionari, cioè tra le due categorie sintattiche; il privilegio di una a favore dell'altra lascerebbe fuori le molte teorie scientifiche che non si occupano d'altro che di oggetti teorici oppure le altrettante che si identificano quasi totalmente con l'interpretazione di una serie di dati. 


%Per derivare una legge empirica da una teorica Carnap introduce delle \emph{correspondance rules} ma senza definirle adeguatamente. Possiamo analogamente fornire il framework `viennese' di una \emph{funzione di traduzione} $\varphi: \mathcal{V_O} \to \mathcal{V_T}$ tale che $\omega_j \mapsto \varphi (\omega_j)$ \footnote{In generale Carnap sembra assumere che $\mathcal{V_O} \subset \mathcal{V_T}$ ma specifica comunque che è errato dire che gli O-terms siano esempi di T-terms.}.
%qui va aggiunta la funzione di "applicazione" $\psi: \mathcal{V_T} \to \mathcal{V_O}$ che fa il percorso inverso di quella di traduzione. è un modo migliore per rappresentare il dibattito di inizio 900: quale delle due funziona meglio? Non esiste la risposta corretta (per i neopositivisti presumibilmente la Ramsey version della def 2.1 coinvolge $\psi$ e ha come oggetti elementi di $\mathcal{V_O}$). La risposta storicamente più accurata è "entrambe alternativamente". Bene, altra cosa risolta dalla sezione 6% 

%\begin{definition} [Wiener Definition]
%	Una teoria $\mathbf{T}$ è una coppia $\bk{\tau_i, \varphi (\omega_j)}$ dove $\tau_i$, $\varphi(\omega_j) \in \mathcal{V_T}$.
%\end{definition}



%[\cite{} Carnap da pag. 299; importante la 314]


