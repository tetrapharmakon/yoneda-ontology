\documentclass{amsart}


\makeatletter
\def\@settitle{\begin{center}%
  \baselineskip14\p@\relax
  \bfseries
  \uppercasenonmath\@title
  \@title
  \ifx\@subtitle\@empty\else
     \\[1ex]\uppercasenonmath\@subtitle
     \footnotesize\mdseries\@subtitle
  \fi
  \end{center}%
}
\def\subtitle#1{\gdef\@subtitle{#1}}
\def\@subtitle{}
\makeatother

\newcommand{\var}[3][]{
  \left[\begin{smallmatrix} #2 \\
  #1\downarrow \\ #3
  \end{smallmatrix}\right]}
\newcommand{\cvar}[3]{
  \begin{xsmallmatrix}{0em}
  & #1 \\ #2 & \downarrow \\ & #3
  \end{xsmallmatrix}}

\def\la{\langle}
\def\ra{\rangle}
\def\lr#1#2{\la #1,#2\ra}

\newcommand{\true}{\texttt{t}}
\def\id{\text{id}}
\author{Fouche and Denta}
\title{Categorical ontology I}
\subtitle{Identity and Yoneda lemma}
\usepackage{fouche}
\usepackage{proof}

\usepackage{minted}
\def\mil#1{\mintinline{haskell}{#1}}
\newcommand{\po}[1][dr]{\save*!/#1+1.5pc/#1:(1,-1)@^{|-}\restore}
\newcommand{\pb}[1][dr]{\save*!/#1-1.5pc/#1:(-1,1)@^{|-}\restore}


\setlength{\epigraphwidth}{.6\textwidth}
\begin{document}
\maketitle
\begin{abstract}
  As it stands, Yoneda lemma allows for a definitive solution of the identity problem in ontology. Identity shall not be seen as a primitive concept; it is instead a relational, and context-dependent, notion. We put the philosophical problems of \emph{identity} and \emph{existence through time} through a structuralist lens, and we solve them by means of topos theory. In our perspective, there is no need to agglomerate the category of Being into a unique subject.

  We offer a possible topos-theoretic solution of the famous paradox of the nine coins, as both a test-bench for our language, and a proof of concept of two Borges' fanatics.
\end{abstract}
\section{Introduction}
\epigraph{\textsc{Eadem svnt qvorvm vnvm potest svbstitvi alteri per transmvtationem}}{Leibniz, if only he knew category theory}
\subsection{What is this series}
[...]
\subsection{On our choice of metatheory and foundation}
In his influential paper \cite{} William Lawvere proposes a foundation of mathematics based on category theory. To appreciate the depth and breadth of such an impressive piece of work, however, the word ``foundation'' must be taken in the particular sense intended by mathematicians:
\begin{quote}
  [\dots\unkern] a single system of first-order axioms in which all usual mathematical objects can be defined and all their usual properties proved.
\end{quote}
Such a position sounds at the same time a bit cryptic to unravel, and unsatisfactory; Lawvere's (and others') stance on the matter is that a foundation of mathematics is \emph{de facto} just a set $\clL$ of first order axioms organised in a Gentzen-like deduction system. The deductive system so generated reproduces mathematics as we know and practice it, i.e. provides a formalisation for something that already exists and needs no further explanation, and that we call ``mathematics''.

It is not a vacuous truth that $\clL$ exists somewhere: the fact that the theory so determined has a nontrivial model, i.e the fact that it can be interpreted inside a given familiar structure, is both the key assumption we make, and the less relevant aspect of the construction itself; showing that $\clL$ ``has a model'' is --although slightly improperly-- meant to ensure that, \emph{assuming the existence of a naive set theory}, axioms of $\clL$ can be satisfied by a naive set. Alternatively, and more crudely, assuming the existence of a model of ZFC, $\clL$ has a model inside that model of ZFC.

A series of works attempting to unhinge some aspects of ontology through category theory should at least try to tackle such a simple and yet diabolic question as ``where'' are the symbols forming the first-order theory of ETCC. And yet, everyone just believes in sets, and solves the issue of ``where'' they are with a leap of faith from which all else follows.

This might appear somewhat circular: aren't sets in themselves already a mathematical object? How can they be a piece of the theory they aim to be a foundation of?

Following this path would have, however, catastrophic consequences on the quality and depth of our exposition. The usual choice is thus to assume that, wherever and whatever they are, these symbols ``are'', and our r\^ole in unveiling mathematics is \emph{descriptive} rather than generative.

This state of affairs has, to the best of our moderate knowledge on the subject, various possible explanations:
\begin{itemize}
  \item On one hand, it constitutes the heritage of Bourbaki's authoritarian stance on formalism in pure mathematics;
  \item on the other hand, a different position would result in barely no difference for the ``working class''; mathematicians are irreducible pragmatists, somewhat blind to the consequences of their philosophical stances.
\end{itemize}
So, symbols and letters do not exist outside of the Gentzen-like deductive system we specified together with $\clL$.

As arid as it may seem, this perspective proved itself to be quite useful in working mathematics; consider for example the type declaration rules of a typed functional programming language: such a concise declaration as
\begin{minted}{haskell}
  data Nat = Z | S Nat
\end{minted}
makes no assumption on ``what'' \mil{Z} and \mil{S :: Nat -> Nat} are; instead, it treats these constructors as meaningful formally (in terms of the admissible derivations a well-formed expression is subject to) and intuitively (in terms of the fact that they model natural numbers: every data structure that has those two constructors must be the type $\bbN$ of natural numbers).

Taken as an operative rule, this reveals exactly what is our stance towards foundations: we are ``structuralist in the metatheory'', meaning that we treat the symbols of a first-order theory or the constructors of a type system irregardless of their origin, provided the same relation occur between criptomorphic collections of labeled atoms.

In this precise sense we are thus structuralists in the metatheory, and yet we do so with a grain of salt, maintaining a transparent approach to the consequences and limits of this partialisation. On the one hand, pragmatism works; it generates rules of evaluation for the truth of sentences. On the other hand, this sounds like a Munchhausen-like explanation of its the value, in terms of itself. Yet there seems to be no way to do better: answering the initial question would give no less than a foundation of language.

And this for no other reason that ``our'' metatheory is something near to a structuralist theory of language; thus, a foundation for such a metatheory shall inhabit a meta-metatheory\dots{} and so on.

Thus, rather than trying to revert this state of affairs we silently comply to it as everyone else does; but we feel contempt after a brief and honest declaration of intents towards where our metatheory lives. Such a metatheory hinges again on work of Lawvere, and especially on the series of works on functorial semantics.
\subsection{Categories as places}
[...]
\subsection{Identity}
Ontology rests upon the principl of identity. It is this very principle that here we aim to unhinge.

Cosa significa che \emph{due cose sono, invece, una} è un problema che ci arrovella fin da quando otteniamo la ragione e la parola; ciò perché il problema è tanto elementare quanto sfuggente: l'unica maniera in cui possiamo esibire ragionamento certo è il calcolo; del resto, se la sintassi non vede che l'uguaglianza in senso più stretto possibile, la prassi deve diventare in fretta capace di una maggiore elasticità: per un istante ho postulato che ci fossero ``due'' cose, non una. E non è forse questo a renderle due? E questa terza cosa che le distingue, è davvero diversa da entrambe?
\section{Preliminaries on variable set theory}
In some of our proofs it will be crucial to blur the distinction between the category of functors $I \to \Set$ and the slice category $\Set/I$ (see \ref{}); once the following result is proved, we freely refer to any of these two categories as the category of \emph{variable sets} (indexed by $I$).
\begin{proposition}
  Let $I$ be a set, regarded as a discrete category, and let $\Set^I$ be the category of functors $F : I \to \Set$; moreover, let $\Set/I$ the slice category. Then, there is an equivalence (actually, an isomorphism) between $\Set^I$ and $\Set/I$.
\end{proposition}
\begin{proof}
  Let us give a very hands-on proof, based on the fact that the category $\Set^I$ coincides on its own right with the category of $I$-indexed families of objects, i.e. with the category whose objects are $(\underline X)_I := \{X_i\mid i\in I\}$, and morphisms $(\underline X)_I\to (\underline Y)_I$ the families $\{f_i : X_i \to Y_i\mid i \in I\}$.

  Consider an object $h : X\to I$ of $\Set/I$, and define a function as $i\mapsto h^\leftarrow(i)$; of course, $(X(h))_I := \{h^\leftarrow(i) \mid i \in I\}$ is a $I$-indexed family, and since $I$ can be regarded as a discrete category, this is sufficient to define a functor $F_h : I \to \Set$.

  Let us define a functor in the opposite direction: let $F : I \to \Set$ be a functor. This defines a function $h_F : \coprod_{i\in I}Fi \to I$, where $\coprod_{i\in I} Fi$ is the disjoint union of all the sets $Fi$.

  The claim now follows if we show that the correspondences $h\mapsto F_h$ and $F\mapsto h_F$ are mutually inverse.

  This is however easy to verify: the function $F_{h_F}$ sends $i\in I$ to the set $h_F^\leftarrow(i)=Fi$, and the function $h_{F_h} \in \Set/I$ has domain $\coprod_{i\in I}F_h(i) = \coprod_{i\in I}h^\leftarrow(i)=X$ (as $i$ runs over the set $I$, the disjoint union of all preimages $h^\leftarrow(i)$ equals the domain of $h$, i.e. the set $X$).
\end{proof}
\begin{remark}
  A more abstract look at this result regards the equivalence $\Set/I\cong \Set^I$ as a particular instance of the \emph{Grothendieck construction} (see \cite{}): for every small category $\clC$, the category of functors $\clC\to\Set$ is equivalent to the category of \emph{discrete fibrations} on $\clC$ (see \ref{}). In this case, the domain $\clC=I$ is a discrete category, hence all functors $\clE \to I$ are, trivially, discrete fibrations.
\end{remark}
The next crucial step of our analysis is the observation that the category of variable sets is a topos: we break the result into the verification of the various axioms, as given in \cite{}.

First, we recall a few basic facts and definitions on topos theory.
\begin{definition}
  An \emph{elementary topos} is a category $\clE$
  \begin{itemize}
    \item which is \emph{cartesian closed}, i.e. each functor $\firstblank\times A$ has a right adjoint $[A, \firstblank]$;
    \item having a \emph{subobject classifier}, i.e. an object $\Omega\in\clE$ such that the functor $\text{Sub} : \clE^\text{op}\to \Set$ sending $A$ into the set of isomorphism classes of monomorphisms $\var{U}{A}$ is representable by the object $\Omega$.
          \index{Subobject classifier}
  \end{itemize}
  The natural bijection $\clE(A,\Omega)\cong\text{Sub}(A)$ is obtained pulling back the monomorphism $U\subseteq A$ along a \emph{universal arrow} $t : 1\to \Omega$, as in the diagram
  \[
    \vcenter{\xymatrix{
        U \pb\ar[r]\ar[d]& 1\ar[d]^t \\
        A \ar[r]_{\chi_U}& \Omega
      }}
  \]
  so, the bijection is induced by the map $\var{U}{A}\mapsto \chi_U$.
\end{definition}
\begin{definition}\label{grotop}\index{Topos!Grothendieck ---}
  A \emph{Grothendieck topos} is an elementary topos that, in addition, is locally finitely presentable.
\end{definition}
Whenever we spoke about sheaves on a topological space or a Grothendieck site, we wer secretly talking about topos theory; the notion of Grothendieck topos is intimately connected with co\fshyp{}end calculus, as we have seen all along chapter 3, and especially in \ref{giraudo}.

In fact, Giraud theorem gives a proof for the difficult implication of the following \emph{recognition principle} for Grothendieck toposes:
\begin{theorem}
  Let $\clE$ be a category; then $\clE$ is a Grothendieck topos if and only if it is a left exact reflection of a category $\Cat(\clA^\text{op},\Set)$ of presheaves on a small category $\clA$.
\end{theorem}
(recall that a \emph{left exact reflection} of $\clC$ is a reflective subcategory $\cate{R}\hookrightarrow \clC$ such that the reflector $r : \clC \to \cate{R}$ preserves finite limits. It is a reasonably easy exercise to prove that a left exact reflection of a Grothendieck topos is again a Grothendieck topos; Giraud proved that all Grothendieck toposes arise this way.)

\begin{proposition}
  The category of variable sets is cartesian closed.
\end{proposition}
\begin{proof}
  We shall first show that the category of variable sets admits products: this is obvious in $\Set/I$, products are precisely pullbacks; note that theorem \ref{} gives an identification
  \[\vcenter{\scriptsize\xymatrix@!=1mm{
    & X\times_I Y \ar[dd]^h \ar@[lightgray][dr]\ar@[lightgray][dl]&  \\
    {\color{lightgray} X} \ar@[lightgray][dr]_{\color{lightgray} f}&& {\color{lightgray} Y} \ar@[lightgray][dl]^{\color{lightgray} g}\\
    & I &
    }}\iff i\mapsto h^\leftarrow(i) = \Big\{(x,y) \in X\times_I Y \mid h(x,y)=i\Big\}\]
  and given the universal property of a pullback, this yields a canonical bijection $h^\leftarrow(i)\cong f^\leftarrow(i)\times g^\leftarrow(i)$. This is exactly the definition of the product of the two functors $F_f, F_g : I\to \Set$.

  Next, we shall show that each functor $\firstblank \times_I Y$ has a right adjoint $Y \pitchfork_I\firstblank$. The functor $\Set^I \to \Set^I : Z\mapsto Y\pitchfork_I Z$ where $Y\pitchfork_I Z : i \mapsto \Set(Y_i, Z_i)$ does the job. This sets up the bijection
  \[\begin{array}{c}
      \xymatrix{X\times_I Y \ar[r] & Z}               \\ \hline
      \xymatrix{X \ar[r]           & Y\pitchfork_I Z}
    \end{array}\]
  and by a completely analogous argument (the functor $\firstblank\times_I\secondblank$ gives a symmetric monoidal structure to variable sets),
  \[\begin{array}{c}
      \xymatrix{X\times_I Y \ar[r] & Z}                \\ \hline
      \xymatrix{Y \ar[r]           & X\pitchfork_I Z;}
    \end{array}\]
  this concludes the proof that the category of variable sets is cartesian closed.
\end{proof}
\begin{proposition}
  The category of variable sets has a subobject classifier.
\end{proposition}
\begin{proof}
  Recall from \ref{} that we shall find a variable set $\Omega$ such that there is a bijection
  \[\begin{array}{c}
      \xymatrix{A \ar[r] & \Omega} \\ \hline
      \textsf{Sub}_I(A)
    \end{array}\]
  where $\textsf{Sub}_I(A)$ denotes the set of isomorphism classes of monomorphisms into $A$, in the category of variable sets.\footnote{A monomorphism into $A$ as an object of $\Set^I$ is nothing but a family of injections $s_i : S_i \to A_i$; a monomorphism in $\Set/I$ is a set $S$ in a commutative triangle
  \[\scriptsize
    \xymatrix@!=1mm{S\ar[rr]\ar[dr]_s && A\ar[dl]^a \\ &I&}\]}

  In order to find such an object, we look at what shape shall $\Omega$ have, and what role its universal property plays in its characterization:
  \begin{itemize}
    \item
    \item
  \end{itemize}
  For the sake of simplicity, for the rest of the proof we fix as category of variable sets the slice $\Set/I$.

  From this we make the following guess: as an object of $\Set/I$, $\Omega$ is the object $\pi_I : I\times \{0,1\} \to I$. We are thus left with the verification that $\pi_I$ has the correct structure and universal property.

  First, we shall find a monomorphism $\true : * \to \Omega$ in $\Set/I$, i.e. an injective function $I\to \Omega$ that has $\pi_I$ as left inverse. This generalised element ``chooses the $\lceil$true$\rceil$ truth value'' in $\Omega$. (Evidently, the identity $\id_I : I \to I$ is the terminal object in $\Set/I$.)

  It turns out that the function $I \to I\times \{0,1\}$ plays the r\^ole of $\true$: indeed, given a monomorphism $S \hookrightarrow A$ the commutative square
  \[
    \vcenter{\xymatrix{
        S\ar[d]\ar[r] & I \ar[d]^{\true}\\
        A \ar[r]_{\chi_S} & I\times \{0,1\}
      }}
  \]
\end{proof}
\begin{proposition}
  The category of variable sets is cocomplete and accessible.
\end{proposition}
\begin{proof}

\end{proof}
Accessibility is a corollary of Yoneda in the following form: every $F : I \to \Set$ is a colimit of representables
\[
  F \cong \colim\Big(\clE(F) \xto{\Sigma} I \xto{y} \Set^I\Big)
\]
($\clE(F)$ is small because in this case $\clE(F)\cong\coprod_{i\in I}Fi$).
\begin{corollary}
  The category of variable sets is a Grothendieck topos.
\end{corollary}
\section{The internal language of variable sets}
\epigraph{I am hard but I am fair; there is no racial bigotry here. [\dots\unkern] Here you are all equally worthless.}{GySgt Hartman}
The internal language of a topos $\clE$ is a formal language defined by \emph{types} and \emph{terms}; suitable terms form the class of variables. Other terms form the class of \emph{formul\ae}.
\begin{itemize}
  \item \emph{Types} are the objects of $\clE$
  \item \emph{Terms} of type $X$ are morphisms of codomain $X$, usually denoted $\alpha,\beta,\sigma,\tau : U \to X$.
        \begin{itemize}
          \item Suitable terms are variables: the identity arrow of $X\in\clE$ is the variable  $x : X \to X$. For technical reasons we shall keep a countable number of variables of the same type distinguished: $x,x',x'',\dots : X \to X$ are all interpreted as $1_X$.
        \end{itemize}
  \item Generic terms may depend on multiple variables; the domain of a term of type $X$ is the \emph{domain of definition} of a term.
\end{itemize}
A number of inductive clauses define the other terms of the language:
\begin{itemize}
  \item the identity arrow of an object $X\in\clE$ is a term of type $X$;
  \item given terms $\sigma : U \to X$ and $\tau :  V\to Y$ there exists a term $\lr{\sigma}{\tau}$ of type $X\times Y$ obtained from the pullback
        \[\xymatrix{
          W \ar[d]\ar[r]\ar[dr]|{\lr{\sigma}{\tau}} & X \times V \ar[d]\\
          U\times Y \ar[r]& X\times Y
          }\]
  \item Given terms $\sigma : U \to X, \tau : V \to X$ of the same type $X$, there is a term $[\sigma = \tau] : W \xto{\lr{\sigma}{\tau}} X\times X \xto{\delta_X} \Omega$, where $\delta_X : X\times X \to \Omega$ is defined as the classifying map of the mono $X \hookrightarrow X\times X$.
  \item Given a term $\sigma : U \to X$ and a term $f : X \to Y$, tere is a term $f[\sigma] := f\circ\sigma : U \to Y$.
  \item Given terms $\theta :  V \to Y^X$ and $\sigma : U\to X$, there is a term
        \[
          W\lr{\theta}{\sigma} \xto{}Y^X\times X \xto{\text{ev}} Y
        \]
  \item In the particular case $Y=\Omega$, the term above is denoted
        \[[\sigma\in\theta] : W\lr{\theta}{\sigma} \to \Omega\]
  \item If $x$ is a variable of type $X$, and $\sigma : X\times U \to Z$, there is a term
        \[\lambda x.\sigma : U \xto{\eta} (X\times U)^X \xto{\sigma^X} Z^X\]
        obtained as the mate of $\sigma$.
\end{itemize}
These rules can of course be also presented as the formation rules for a Gentzen-like deductive system: let us rewrite them in this formalism.
\[ \begin{array}{cc}
    \infer{1_X : X \to X}{}                                                              &
    \infer{\lr{\sigma}{\tau} : W\lr{\theta}{\sigma} \to X \times Y }{\sigma : U \to X    &   & \tau : V \to Y}             \\[1em]
    \infer{[\sigma=\tau] : W \to \Omega}{\sigma : U \to X                                &   & \tau : V \to X}           &
    \infer{f[\sigma] : U \to Y}{\sigma : U \to X                                         &   & f : X \to Y}                \\[1em]
    \infer{W\lr{\theta}{\sigma} \xto{}Y^X\times X \xto{\text{ev}} Y}{\theta :  V \to Y^X &   & \sigma : U\to X}          &
    \infer{\lambda x.\sigma = \sigma^X\circ\eta : U \to (X\times U)^X \to Z^X}{ x: X     &   & \sigma : X\times U \to Z}
  \end{array}\]
To formulas of the language of $\clE$ we apply the usual operations and rules of first-order logic: logical connectives are induced by the structure of internal Heyting algebra of $\Omega$: given formulas $\varphi,\psi$ we define
\begin{itemize}
  \item $\varphi\lor \psi$ is the formula $W\lr{\varphi}{\psi} \to \Omega\times \Omega \xto{\lor} \Omega$;
  \item $\varphi\land\psi$ is the formula $W\lr{\varphi}{\psi} \to \Omega\times \Omega \xto{\land} \Omega$;
  \item $\varphi\Rightarrow\psi$ is the formula $W\lr{\varphi}{\psi} \to \Omega\times \Omega \xto{\Rightarrow} \Omega$;
  \item $\lnot\varphi$ is the formula $U \to \Omega \xto{\lnot} \Omega$.
\end{itemize}
\todo[inline]{universal quantifiers}
Each formula $\varphi : U \to \Omega$ defines a subobject $\{x\mid \varphi\} \subseteq U$ of its domain of definition; this is the subobject classified by $\varphi$, and must be thought as the subobject where ``$\varphi$ is true''.

If $\varphi : U\to\Omega$ is a formula, we say that $\varphi$ is \emph{universally valid} if $\{x\mid\varphi\}\cong U$. If $\varphi$ is universally valid in $\clE$, we write ``$\clE\Vdash \varphi$'' (read: ``$\clE$ believes in $\varphi$'').

Examples of universally valid formulas:
\begin{itemize}
  \item $\clE\Vdash [x=x]$
  \item $\clE\Vdash [(x \in_X \{x\mid\varphi\}) \iff \varphi]$
  \item $\clE\Vdash \varphi$ if and only if $\clE \Vdash \forall x.\varphi$
  \item $\clE\Vdash [\varphi \Rightarrow \lnot\lnot\varphi]$
\end{itemize}
\todo[inline]{Ora facciamo delle considerazioni sul lingo interno di $\Set/I$}
Chi sono tipi e termini; chi sono le proposizioni e come si scrive il calcolo proposizionale in $\Set/I$; i quantificatori, in dettaglio pornografico.

L'oggetto dei numeri naturali e il principio di induzione nel topos degli insiemi variabili.
\section{Nine copper coins}
\epigraph{Explicaron que una cosa es \emph{igualdad}, y otra \emph{identidad}, y formularon una especie de \emph{reductio ad absurdum}, o sea el caso hipotético de nueve hombres que en nueve sucesivas noches padecen un vivo dolor. ¿No sería ridículo -interrogaron- pretender que ese dolor es el mismo?}{JLB ---Tl\"on, Uqbar, Orbis Tertius}
According to our description of the internal language of variable sets, \emph{propositions} in $\Set/I$ are morpisms of the form
\[p : U \to \Omega_I\]

where $\Omega_I$ is the subobject classifier of \ref{}; recall that
\begin{itemize}
  \item the object $\Omega_I = \{0,1\}\times I \to I$ becomes a variable set in a canonical way with the projection $\pi_I : \Omega_I \to I$ on the second factor;
  \item the universal monic $I \to \Omega_I$ consists of a section of $\pi_I$, precisely the one that sends $i : I$ to the pair $(i,1) : \Omega_I$;
  \item every subobject $U \hookrightarrow A$ of an object $A$ results as a pullback (in $\Set/I$) along true:
        \[here be diagram\]
\end{itemize}
So, a proposition is a morphism of the following kind: it is a function $p : U \to \Omega_I$, defined on a certain domain, and such that
$$
  \begin{array}{ccc}
    U           & \overset{p}\to & \{0,1\}\times I \\
    u\downarrow &                & \downarrow\pi   \\
    I           & =              & I
  \end{array}
$$
(it mustbe a morphism of variable sets!) This means that $\pi p(x : U) = u(x : U)$, so that $p(x) = (\epsilon_x, u(x))$ for $\epsilon_x =0,1$ and $u$ is uniquely determined by the "variable domain" $U$.

To get a grip dei vari ruoli di diverse classi di proposizioni, e dato che noi ci concentreremo -costruendola e nihilo- su una proposizione particolare, ossia quella che asserisce l'esistenza e il perdurare nell'esistenza di una freccia scoccata da un arco e persa tra gli arbusti, consideriamo la seguente proposizione:
\[p : \mathbb R \to \Omega_I\]
"al tempo $t : \mathbb R$ la freccia esiste" (per ragioni psicologiche, qui $I$ è un intervallo di $R$, diciamo $[0,1]$ per fissare le idee e perché è una scelta standard di logica sfumata). Also, La nozione di "persistenza nel tempo" è collegata a quella di identità laddove si può dire che a persiste se è identica a se stessa in ogni frame blablabla.

$A = \{x : U \mid p(x) = (1,t_x), t_x > 0\} = p^\leftarrow(\{1\}\times (0,1])$ tempi in cui la freccia esiste almeno un po';
$B = \{x : U \mid p(x) = (1,1)\} = p^\leftarrow((1,1))$ tempi in cui la freccia esiste che più non si può;
$E_t = \{ x : U \mid p(x)=(1,t)\}$ tempi in cui la freccia esiste un dato po'.

Ogni sorta di osservazione si può ora fare: la logica classica si recupera così; e in effetti se ne recupera una per ogni fetta di $I$ ad un punto specifico; l'esistenza ora è "vera con forza" o "falsa con forza", e la forza è data dal parametro $i\in I$; è quindi naturale assumere $I$ totalmente ordinato e denso (o forse no: bisogna capire se e come qui stiamo smanettando con un modello di logica temporale)

Generica notazione: $p(x)=(1,t)$ si legge "$p$ è vera con forza $t$"; idem per $p(x)=(0,t)$: $p$ è falsa con forza $t$: chiaramente, se $I=[0,1]$ possiamo prenderlo col suo ordine solito (cita il mitico paper di Freyd su una caratterizzazione completamente algebrica dell'intervallo) e avere un poset dei valori di verità; è anche un'algebra di Heyting? (E' facile controllare; solo, ora non ho veramente voglia di scrivere)

Il paradosso di Borges allora paradosso non è: è una sentenza perfettamente lecita, che non ha niente di paradossale nel linguaggio interno degli insiemi variabili, e che appare invece paradossale per gli abitanti della copia di $\Set$ che sta dentro gli insiemi variabili."Perdurare nell'esistenza" al massimo grado possibile è un attributo divino.

Da notare anche questi passaggi:

1) A cento anni dall’enunciazione del problema, un pensatore non meno brillante dell'eresiarca, ma di tradizione ortodossa, formulò un'ipotesi molto audace.  Secondo  questa  felice  congettura,  -  v'è  un solo  soggetto:  questo soggetto indivisibile è ciascuno degli esseri dell'universo, i quali sono organi e maschere della divinità. X è Y ed è Z. Z scopre tre monete perché ricorda che X le ha perdute.
2) Il processo è periodico: il hrön di dodicesimo grado comincia già di nuovo a decadere.
3) Più strano e più  puro di ogni hrön è talvolta l'ur la cosa prodotta per suggestione, l’oggetto evocato dalla speranza. La gran maschera d’oro cui ho accennato ne è un illustre esempio

1 e 2 dicono qualcosa su chi è I in Set/I; secondo 2) la struttura di I è ciclica (e contata in base 12, chiaramente);  1) è solo un modo letterario di riferirsi a un oggetto universale dentro Set/I; potremmo divertirci a pensare qual è (l'oggetto terminale? Il rappresentante di un altro funtore particolare?)
3 ci devo pensare: gli ur sono particolari hrönir. Potrebbero essere un po' di tutto, funzioni derivabili verso I, funzioni che sono "vere 1" su un dominio denso, funzioni che sono non-false su un dominio denso

$p(x) = (0,1/2)$

$p(x) = (1,1/2)$


$p(x) = (0,1/3)$

$p(x) = (0,1/\pi)$

$p(x) = \text{la freccia esiste}$
$p$ è super vera
$p(x)=(1,1)$

$p$ è vera un po', non tanto
$p(x) =(1,t<1)$

$p$ è falsa un po'
$p(x) =(0,t<1)$

$\pi p$ non deve essere continua; quando è continua la freccia perdura nell'esistenza, e questi oggetti divini si chiamano hrön
$\pi_I \circ p : U \to \{0,1\}\times I \to I$
\appendix
\section{Category theory}
\todo[inline]{E' probabilmente saggio mettere qui anche la digressione con le definizioni di topos}
For us, an \emph{ordinal number} will be any well\hyp{}ordered set, and a \emph{cardinal number} is any ordinal which is not in bijection with a smaller ordinal. Every set $X$ has a unique \emph{cardinality}, i.e. a cardinal $\kappa$ with a bijection $\kappa \cong X$ such that there are no bijections from a smaller ordinal. We freely employ results that depend on the axiom of choice when needed. A cardinal $\kappa$ is \emph{regular} if no set of cardinality $\kappa$ is the union of fewer than $\kappa$ sets of cardinality less than $\kappa$; all cardinals in the following subsection are assumed regular without further mention.

\index{Category!filtered ---}\index{Filtered category}
Let $\kappa$ be a cardinal; we say that a category $\clA$ is $\kappa$\hyp{}filtered if for every category $\clJ\in\Cat_{<\kappa}$ with less than $\kappa$ objects, $\clA$ is injective with respect to the cone completion $\clJ\to \clJ^\rhd$; this means that every diagram
\[
  \vcenter{\xymatrix{
      \clJ\ar[d]\ar[r]^D & \clA \\
      \clJ^\rhd\ar@{.>}[ur]_{\bar D}
    }}
\]
has a dotted filler $\bar D : \clJ^\rhd \to \clA$.

We say that a category $\clC$ admits filtered colimits if for every filtered category $\clA$ and every diagram $D : \clA \to \clC$, the colimit $\colim D$ exists as an object of $\clC$. Of course, whenever an ordinal $\alpha$ is regarded as a category, it is a filtered category, so a category that admits all $\kappa$\hyp{}filtered colimits admits all colimits of chains
\[
  C_0 \to C_1 \to \cdots \to C_\alpha \to\cdots
\]
with less than $\kappa$ terms. A useful, completely elementary result is that the existence of colimits over all ordinals less than $\kappa$ implies the existence of $\kappa$\hyp{}filtered colimits; this relies on the fact that every filtered category $\clA$ admits a cofinal functor (see \ref{finalfunctor}) from an ordinal $\alpha_\clA$.
\begin{definition}\label{accepre}\index{Category!accessible ---}\index{Accessible category}\index{Category!presentable ---}\index{Presentable category}
  Let $\clC$ be a category;
  \begin{itemize}
    \item We say that $\clC$ is \emph{$\kappa$\hyp{}accessible} if it admits $\kappa$\hyp{}filtered colimits, and if it has a \emph{small} subcategory $\cate{S}\subset \clA$ of $\kappa$\hyp{}presentable objects such that every $A\in\clA$ is a $\kappa$\hyp{}filtered colimit of objects in $\cate{S}$.
    \item We say that $\clC$ is \emph{(locally) $\kappa$\hyp{}presentable} if it is accessible and cocomplete.
  \end{itemize}
  The theory of presentable and accessible categories is a cornerstone of \emph{categorical logic}, i.e. of the translation of model theory into the language of category theory.

  Accessible and presentable categories admit \emph{representation theorems}:
  \begin{itemize}
    \item A category $\clC$ is accessible if and only if it is equivalent to the ind\hyp{}completion $\text{Ind}_\kappa(\cate{S})$ of a small category, i.e. to the completion of a small category $\cate{S}$ under  $\kappa$\hyp{}filtered colimits;
    \item A category $\clC$ is presentable if and only if it is a full reflective subcategory of a category of presheaves $i : \clC \to \Cat(\cate{S}^\op,\Set)$, such that the embedding functor $i$ commutes with $\kappa$\hyp{}filtered colimts.
  \end{itemize}
\end{definition}
All categories of usual algebraic structures are (finitely) accessible, and they are locally (finitely) presentable as soon as they are cocomplete; an example of a category which is $\aleph_1$\hyp{}presentable but not $\aleph_0$\hyp{}presentable: the category of metric spaces and short maps.

\end{document}
