\section{Introduction: two exactly similar spheres}
\epigraph{%
  Questo è il mio fucile. Ce ne sono tanti come lui, ma questo è il mio.
}{\emph{Il credo del fuciliere}}

La cosiddetta \emph{aporia delle due sfere identiche} è stata enunciata per la prima volta da M. Black in \cite{foo} col fine di criticare la maniera in cui il principio di identità degli indiscernibili viene adottato in ontologia.

Tale ``paradosso'' si riassume circa come segue: due interlocutori (chiamati per brevità $A$ e $B$) discutono delle conseguenze della supposizione che l'universo sia costituito unicamente da due sfere `perfettamente identiche', ovvero dotate delle medesime proprietà formali e materiali (lo stesso raggio; la stessa composizione chimica; la stessa densità, eccetera). Esse risulterebbero, proprio in forza di questo postulato, impossibili da distinguere. A determinarne l'alterità dovrebbe essere sufficiente l'unica proprietà che presumibilmente non hanno in comune, essendo appunto \emph{due} sfere, ossia la loro localizzazione spaziale: una è \emph{qui}, l'altra è \emph{lì} --qualsiasi cosa questo dualismo significhi. Tuttavia, $B$ continua a negare che esista un modo di distinguere veramente due sfere identiche, proprio in forza della loro identità: il loro dialogo va circa come segue.

"A: Ognuna delle due sfere sarà certamente diversa dall'altra per essere a una certa distanza da quell'altra, ma a distanza nulla da sé stessa; vale a dire essa avrà almeno una relazione con sé stessa --l'essere a distanza nulla da o l'essere nello stesso luogo di-- che non ha con l'altra. (...)

B: (...) Ognuna avrà la caratteristica relazionale di essere a una distanza di 2 miglia, diciamo, dal centro di una sfera di un diametro, ecc. E ognuna avrà la caratteristica relazionale (...) di essere nello stesso luogo di se stessa. le due sono simili per questo riguardo come per qualsiasi altra proprietà.

A: Ma ogni sfera occupa un luogo diverso; e questo varrà a distinguerle.

B: Ciò suona come se voi pensaste che i luoghi abbiano una qualche esistenza indipendente (...). [qui il ragionamento sui luoghi é di natura relazionale, e credo sia inevitabile nella nostra prospettiva ragionare cos\'e.] Dire che due sfere sono in luoghi diversi equivale appunto a dire che c'é una certa distanza tra due sfere, e abbiamo visto che questo non varrà a distinguerle. Ognuna é a una certa distanza --in effetti la stessa distanza-- dall'altra.

Dopo una discussione sofferta, e non povera di argomenti, i due interlocutori non arrivano ad un accordo; il nostro lettore ora è invitato a notare che si possono produrre molti argomenti in sfavore della buona posizione di questa aporia, dai più ingenui --e perciò praticamente inconfutabili: se marchio una delle due sfere con un segno rosso, sull'altra ne appare uno identico? Se sì, c'è una sola sfera, checché ne dica la mia percezione; se no, due), ai più elaborati (l'ipotesi di partenza, che l'universo sia composto `unicamente da due sfere' è impossibile da realizzare, ma ancor prima da formalizzare; la lista di `tutte' le proprietà che le sfere hanno in comune è imposibile da stilare compiutamente --il problema è nella nozione di proprietà, ma anche in quel \emph{tutte}, aggettivo pericoloso; il materiale che le compone dovrebbe essere una sorta di \emph{mithril}, refrattario a ogni anisotropia --e un tale materiale, chiaramente, non esiste-- o altrimenti potremmo misurare delle disomogeneità macro- o meso-scopiche dei pezzi di materiale che le compongono --e il fatto stesso che questa misurazione sia possibile implica che $B$ abbia torto; la proprietà $P(X)$=`$Y$ crede che la sfera $s_1$ sia uguale alla sfera $X$' è, praticamente per costruzione, una proprietà di una sola delle due sfere --incidentalmente, questo propone un meta-problema: \emph{in quale logica} va interpretata l'identità delle sfere?).

Un argomento tanto ingenuo, già disassemblato in molti modi \cite{foobar,baz} non ha certo bisogno urgente di una ulteriore confutazione per poter essere derubricato a un puro (e inelegante) interesse storiografico; lo scopo di questa nota è di elaborare una sua confutazione moderna, basata sull'idea permeante della matematica del XX secolo.
\begin{quote}
  L'identità non è una nozione assoluta, indipendente da un contesto di riferimento, ma --all'esatto contrario-- non è possibile stabilire se due enti siano `identici' fino a quando non sia stato specificato \emph{in quale senso} e \emph{relativamente a quale concetto di `uguaglianza'}, scelto tra tanti, desideriamo compararli.
\end{quote}
Affermare che la pratica matematica sia intrisa fino alle fondamenta di questo principio operativo è un blando eufemismo; ogni singolo mattone dell'edificio costruito durante il XX secolo è conseguenza di questa maniera, peculiare ma infinitamente più malleabile ed espressiva, di intendere la nozione di uguaglianza tra enti matematici. Esiste una quantità sterminata di letteratura atta a stabilire quale sia il punto d'inizio di questa rivoluzione del pensiero matematico; \emph{in nuce} l'idea era già nel programma di Erlangen di Klein: ridurre la `geometria' allo studio delle orbite di un insieme di figure sotto l'azione di un gruppo di trasformazioni.

L'idea che nell'attività di classificazione di enti geometrici (delle coniche \cite{}; delle cubiche \cite{}; delle superfici compatte \cite{}) tutte le figure trasportabili l'una nell'altra con una trasfomazione ammissibile contino come una sola, è semplice e fruttuosa: l'algebra lineare moderna, la geometria proiettiva, la teoria delle categorie, sono alcuni dei frutti di questa idea.

Questo lavoro si basa sull'ultimo di questi frutti (si veda \cite[??]{foo} per una chiara connessione tra il programma di Klein e la teoria delle categorie); l'uso delle categorie rende evidente uno degli errori di ingenuità commesso dai due interlocutori del paradosso di Black. Essi parlano di `due sfere' che appaiono ad $A$ come oggetti di una categoria $\clA$, e a $B$ come oggetti di una categoria $\clB$: l'origine del fraintendimento è tutta qui: è illecito comparare oggetti di categorie distinte, benché nominalmente identici.

Più precisamente, nella sua formulazione classica l'interlocutore, $A$ pensa le sfere immerse in uno spazio euclideo; invece $B$ le pensa in uno spazio affine. Le categorie degli spazi euclidei ed affini sono --dimostrabilmente-- tra loro distinte, ed ecco trovato il busillis: la scarsa conoscenza matematica di $A$ e di $B$ `dimentica' la presenza di un maggior numero di morfismi in $\clB$, la cui azione trasformativa è capace di identificare le sfere per $B$. Assenti quei morfismi da $\clA$, $A$ è destinato ad apprezzare le innumerevoli differenze tra $S_1$ ed $S_2$.

La conseguenza interessante di questa linea di pensiero non è la sua capacità di confutare Black; invece, affinando leggermente questa analisi, comprensibile da uno studente medio esposto a una moderata quantità di algebra lineare, otteniamo uno strumento generale di analisi; ad ogni `spazio geometrico' $\bbS$ si può associare un \emph{gruppoide d'azione}, una particolare categoria, ottenuta a partire da quelli che sono gli omomorfismi di struttura per $\bbS$, che prescrive una scelta canonica di identificazioni `ammissibili' a decidere se due oggetti di $\bbS$ siano `uguali' (di più: questo \emph{definisce} il significato della locuzione \emph{$U,V$ sono uguali in $\bbS$}). L'insieme delle orbite per l'azione di $\clG$ su $\bbS$ forma lo \emph{scheletro} $\text{sk}(\bbS)$ di $\bbS$. All'interno di $\text{sk}(\bbS)$, oggetti uguali in $\bbS$ sono strettamente uguali.

\color{blue}
In particolare, $A$ vede diverse cose che $B$ vede uguali, perché $A$ opera un quoziente rispetto a una relazione di equivalenza più fine di quella che opera $B$. Questo esempio da solo introduce praticamente tutti gli strumenti di cui rendiamo conto in questo lavoro:
\begin{itemize}
	\item rende lampante il fatto che una categoria è determinata dai suoi morfismi molto più che dai suoi oggetti; $A$ e $B$ sono convinti di parlare delle stesse cose, perché $|\mathcal C_A|=|\mathcal C_B|$, ma la seconda ha molti più morfismi, e quindi può far collassare molti più oggetti rispetto alla sua nozione di ``uguaglianza categoriale'' (dobbiamo stabilire una volta per tutte dei nomi per le cose cui vogliamo riferirci).
	\item mette in chiaro in un esempio specifico il fatto che la relazione di identità non è primitiva; è contestuale; è indotta dalla nozione di uguaglianza che ogni categoria si porta dietro. Ed è binaria, perché è un caso particolare di una nozione di omotopia (meglio: è la nozione di isomorfismo in una categoria dell'omotopia, che resta indotta dalla nozione di omotopia di cui $\mathcal C$ era dotata. No identity without homotopy, appunto).
	\item Questo è né più né meno che il contenuto strutturale del programma di Klein: le figure dello spazio restano sempre le stesse, raccolte nella classe $\mathbb S$; ciò che cambia è il gruppo(ide) $\mathcal G$ che facciamo agire sulle figure e di cui prendiamo la categoria delle azioni $\mathbb S /\!\!/\mathcal G$; ``il mondo'' con la sua nozione di identità è l'insieme di Bishop
	      \[\Big\{ \pi_0(\mathbb S/\!\!/\mathcal G) \mid [x] \equiv [y] \iff \exists \varphi : x \to y\Big\}\]
	\item Permette di prevedere cosa penserebbe del problema un ipotetico interlocutore $C$ che vivesse in un mondo proiettivo (una sfera e un'iperbole sono lo stesso oggetto) o in uno topologico (una sfera e un cubo sono lo stesso oggetto), etc. Ovviamente si può ragionare anche viceversa: basta trovare un contesto dove due oggetti spazialmente coincidenti non sono ``lo stesso'' oggetto. La sfida è che bisogna uscire da fondazione insiemistica (l'assioma di estensionalità è precisamente l'asserto per cui due cose che hanno gli stessi punti sono una cosa sola); invece di impelagarsi nella costruzione precisa di un controesempio (ce ne sono, in fondazioni type-teoretiche: per esempio i tipi $\mathbb N : \sf Ord$ e $\mathbb N : \sf Mon$ sono diversi -hanno diverse proprietà universali), penso sia meglio dire semplicemente: ``ecco, vedete? Il principio di identità è  a tutti gli effetti l'assioma di estensionalità: voi trovate assurdo che a dichiarare uguali due enti non sia sufficiente dire che hanno gli stessi atomi. Eppure questo è possibile: voi classici non avete la nozione di identità più forte in assoluto, state nel mezzo; e la metateoria risultante dal prendere una nozione di identità più stringente è semmai ancor piu interessante di quella cui siete abituati. Quindi la nozione di uguaglianza non è \emph{immutabile, scolpita nel tempo}. E' un assioma: se lo vuoi lo prendi, altrimenti non lo prendi. E se non lo prendi si apre un mondo, perché ammetti che due cose sono uguali anche quando non hanno gli stessi punti, oppure che \emph{non basta} avere gli stessi punti per essere uguali, nello stesso senso in cui non è sufficiente che due insiemi siano in biiezione affinché siano omeomorfi, o isomorfi come gruppi, etc.'' Chiaramente, questo lastrica la strada al lemma di Yoneda, che ``è l'assioma di estensionalità in CT'' (lo introdurrei esattamente con queste parole, e spenderei una parte congrua del lavoro a darne una introduzione \emph{ad usum delphini} che invece che di trasformazioni naturali e gruppianellicampi parli di ontologia e di estensionalità).
\end{itemize}
Qual è il punto di tutto questo? Che una sfera è molte cose: è uno spazio topologico, è una varietà algebrica, è una varietà differenziale, è un gruppo di Lie (non in dimensione 2, ma per esempio in dimensione 1 o 3), è una superficie di Riemann, è questo ed è quell'altro. $A$ e $B$ nell'esempio di Black parlano uno di una sfera $S^2_A$ che appartiene a una categoria $A$, e l'altro di una sfera $S^2_B$ che appartiene a una categoria $B$. Il paradosso nasce quindi da un uso scorretto e impreciso del formalismo.