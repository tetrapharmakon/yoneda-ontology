\section{Vistas on ontologies}

\subsection{Dialogo con l'idealista} 
Let us reconsider \autoref{bla}; although in passing, we mentioned famous Berkeley's view of perception as a bundle of stimulation incapable to cohere. Let's say clearly that endeavouring on such a wide ground as classical idealism isn't the purpose of our work; yet, in his novel Borges regards Tl\"on's language and philosophy as a concrete realisation of Berkeley's theory of knowledge. %\footnote{cerchiamo di dire che non ci frega molto dell'idealista in sè. Il paper non è una tentativo di confutazione dell'idealismo. Ma, Borges docet, la metafisica di Berkeley è un caso estremo per far valere un discorso più ampio, che viene chiarito nel resto della sezione}. All'idealista, il berkeleyano puro, che volesse dedurre, da questa "esistenza ad intermittenza" delle monete, ch'esse in realtà non sussistano, potremmo obiettare che sta parlando male. 

Within topos theory, è scorretto dire che $p$ è falsa in "assoluto", ma che semmai è "falsa con forza $t$" dove $t \in I$. Il berkeleyano vorrebbe ottenere (accettato che le monete esistano solo un po' in alcuni punti) che $\sum_u p(u,c,d) = (\bot,1)$. Non può per costruzione, ma se volesse deve fornire un modello alternativo del paradosso che sia altrettando valido (ci sembra sia difficile comunque non parlare in termini di "forze" o loro equivalenti, in "istanti" singoli).  

Ad essere ancora più pignoli, alla domanda se le monete esistano o meno la risposta corretta è a sua volta una domanda: "In quale $I$ sei?" ed essendo $I$ totalmente ordinato "In quale suo punto vuoi calcolare la forza?". [Il punto può essere un riferimento temporale, spaziale o di qualunque altro genere (a seconda della struttura di $U$), l'importante è che l'insieme abbia le caratteristiche da noi indicate]. [mi dai i dati (osservatore, subset di C, giorno della settimana) e assegno un valore di verità (\{0,1\} e forza); dipendenza funzionale delle proposizioni dalle configurazioni]. 

Sostituire nozioni classiche (identità, persistenza-nel-tempo) con nozioni precise suggerite dal linguaggio che proponiamo, vuol dire perciò modificare anche il linguaggio naturale. Quando l'idealista afferma che negli istanti in cui chiude gli occhi la sua casa non esiste più, sta anch'egli parlando di "forza" nel senso qui definito: lo deve solo esplicitare. Si mette in un universo (un topos) "non-classico" con più valori di verità e deve semplicemente calcolare \emph{qual è} il valore di $p$. 

L'ampliamento dei valori di verità si traduce in un ampliamento delle risposte possibili da dare alla domanda sulle monete. Non indichiamo quella "corretta" ma il topos in cui trovarla, il contesto di discorso dal quale non si può prescindere, intuitivamente un contesto fuzzy, che evita le sterili dicotomie in cui era impantanato il dibattito da qualche secolo [In realtà la cosa è questa: $\{0,1\}$ è "fisso"; la varietà di risposte è data dalla scelta di $I$ e dalla struttura di $U$].

Da notare (come anticipato in 1.4) che abbiamo potuto modellizzare il paradosso senza impiegare un framework di logica temporale; è lecito interpretare gli elementi di $I$ come istanti, ma non necessario. La descrizione in termini temporali è un sottocaso del modello più generale che abbiamo fornito. Persino un presentista o l'idealista coerente di Borges (che dalle tesi berkeleyane deduce l'inesistenza del tempo) può stabilire la verità di $p$ senza abiurare alla sua posizione ontologica, ma deve poi accettare il risultato del calcolo. 

\todo[inline]{accordo intersoggettivo in Tl\"on}

La nostra sembra una posizione intermedia tra empirismo e idealismo. La critica dell'idealista al metodo scientifico sono le prove indirette di esistenze (come le dimostrazioni non costruttive che non piacciono agli intuizionisti). La nostra è una visione fuzzy dell'esistenza: sei libero di assumere che la barca invisibile che nessuno vede, se non perché l'acqua si sposta, non esista, o assumere che la barca ci sia ma con una forza minore del massimo. (è un po' un tentativo di risoluzione epistemica del paradosso. cfr quando si parla di "maggiore è il numero di osservatori più intenso è il grado di esistenza delle monete"). Se si usa questo linguaggio non ti basta più dire che la barca non esiste perché non la vedi, cosa che non spiega lo spostamento dell'acqua \footnote{Anche il metafisico che non assume avvenga il processo di causazione deve spiegare la relazione (qualunque essa sia, se c'è) o comunque la correlazione tra il fenomeno "la barca si muove" (che presuppone l'esistenza della barca) e il fenomeno "l'acqua si sposta", non collegato, ad esempio, al fenomeno dell'esistenza delle nuvole}. In più con la possibilità del compromesso dato dalla densità di $I$.
\todo[inline]{discorso su scollamento linguaggio formale/naturale} %la questione è epistemologica, credo; ci ho riflettuto e si era già accennato in salotto #2; eventuali critiche e dibattiti non possono avvenire nel linguaggio naturale%

\subsection{Metaontology} Gli osservatori di Tl\"on non riescono a dire nulla sulle monete degli altri, neanche con strength. Il nostro modello è inutilizzabile \emph{dentro} Tl\"on; ed è dovuto alla diversità di linguaggio dell'abitante di Tlon. Sospendiamo il giudizio su cosa venga "prima", se l'ontologia o il linguaggio (gli autori hanno posizioni divergenti ma la ricerca assume pragmaticamente il secondo) ma senza dubbio essi sono strettamente correlati.

Possiamo imprecisamente definire l'ontologia come "ciò che gli oggetti \emph{sono} \footnote{cioè come si comportano nel modello} all'interno di un linguaggio".

 Possiamo dire, figurativamente, che l'ontologia è il modo che usiamo per "apparecchiare" il mondo dentro un determinato linguaggio. Questo è ciò che gli ontologi possono dire, il luogo nel quale lavorare.
 Dove sia il linguaggio e "qual è il principio che ispira la scelta di un determinato linguaggio (e quindi di un'ontologia)" ci sembra una domanda di interesse antropologico, o forse neurofisiologico, ma è fuori dalla portata di ciò che l'ontologia operativa può dire \footnote{\de{Dovrebbe essere una domanda metaontologica ma qui la posizione che assumo è che la nostra metaontologia dice che la risposta non rientra nel campo delle sue possibilità}\endde} (se non ricascando nelle pastoie dell'impostazione classica, priva degli "ambienti" adatti dentro i quali parlare delle cose). 
 
 Rapporto tra "esistenza" e "comportamento nel modello": rinunciamo alla caratterizzazione classica di esistenza? qualunque essa sia (presumibilmente quella orribile quineana). Sembra di sì. Quindi 
 \begin{center}
 	"$x$ esiste (nel modello $\mathcal{M}$)" := "insieme dei comportamenti di $x$ in $\mathcal{M}$"
 \end{center}
  \de{va precisata meglio ma è una def relativizzata ad un contesto (come promettevamo in 1.4) e ci permette di parlare di esistenza senza ricorrere a identità e persistenza nel tempo. Molto più agevole}\endde]. 
 
 Questo è quello che intendiamo con "esistenza" a livello metaontologico. La nozione poi si specifica a seconda del mondo (topos) che analizziamo in quel momento. Ad esempio in Tl\"on esistere significa avere una forza strettamente maggiore di 0.  

Prendiamo lo stato dell'arte dell'ontologia in filosofia analitica: l'apostolo quineano di turno ti dice che "ontologia" è il dominio di oggetti su cui variano i quantificatori \cite{?}. Questa def sufficientemente operativa dice "di meno" di quello che vorrebbe l'impostazione classica (così come la nostra; ma, come sempre, è l'unico modo per essere precisi); ma dice "di meno" anche della nostra. Se ontologia è il comportamento degli oggetti nel modello, è "irrilevante" cosa io ritenga di poter inserire nel dominio (struttura di $U$) (il realista concettuale dice che sono oggetti anche $\top$ e $\bot$ \footnote{O meglio sono oggetti ciò che $\top$ e $\bot$ denotano} mentre l'empirista puro no) (e così via di chiacchiere su "ce lo metto o no il tempo nel mio modello?"): il dibattito può rimanere sul modello. In questa maniera abbiamo un agone intersoggettivo di attribuzione dei significati (vabbè, si fa per ridere: cfr. Umberto Eco, ovunque) dentro il quale \emph{fare} ontologia (anzi, ontologie).   
