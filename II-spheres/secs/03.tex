\section{Two exactly similar spheres, again}
\hrulefill

\subsubsection{Existence: Persistence of Identity?}
 Ontology rests upon the principle of identity: it is this very principle that our category-theoretic approach aims to unhinge.
 \begin{italian}
 E tuttavia formalizzare il concetto intuitivo di identità si rivela una questione estremamente spinosa: cosa significa che \emph{due cose sono, invece, una} è un problema che ci arrovella fin da quando otteniamo la ragione e la parola; ciò perché il problema è tanto elementare quanto sfuggente: l'unica maniera in cui possiamo esibire ragionamento certo è il calcolo; del resto, se la sintassi non vede che l'uguaglianza in senso più stretto possibile, la prassi deve diventare in fretta capace di una maggiore elasticità: per un istante ho postulato che ci fossero ``due'' cose, non una. E non è forse questo a renderle due? E questa terza cosa che le distingue, è davvero diversa da entrambe?

 Usciti dalle nebbie delle speculazioni tradizionali, i filosofi a cavallo tra '800 e '900 si sono posti questo complicato problema: due vie sono possibili: la risposta fregeana \cite{} per cui ''$x$ esiste'' se e solo se ''$x$ è identico a qualcosa [banalmente, a sè stesso]'' alla soluzione logica quineana per cui ''essere è essere il valore di una variabile [vincolata]''.

 Il primo approccio si rivela comodo solo se si è realisti concettuali, ma è scarsamente informativo (cosa è l'uguaglianza tra due oggetti apparentemente diversi? Quel qualcosa che ci ha fatto sospettare lo fossero non è abbastanza a renderli tali?); coinvolge la nozione di identità, ed anzi scarica su questa l'onere di definire esistenza; questa non è la strada giusta: l'identità di fatto non esiste, perché ogni identità è un'identificazione, e ogni uguaglianza una relazione di equivalenza; la questione è tuttavia complessa abbastanza da dedicarvi un lavoro a parte (in effetti, due \cite{,}) di questo polittico.

 Il secondo tipo di approccio ha ispirato l'interesse per la nozione di \emph{ontological committment} (l'insieme di assunti che ``si danno per scontati'' quando si parla di ontologia, o se ne partecipa) e per la conseguente definizione di ontologia (di una teoria) come ''dominio di oggetti su cui variano i quantificatori'' (cf. \cite{}): namely una teoria qualsiasi è impegnata sulle entità su cui variano i quantificatori dei suoi enunciati.

 	Vedremo in \autoref{metaon} che la concezione Quineana fitta nella nostra visione di ontologia categoriale come conseguenza di una internalizzazione.

 	Esiste una terza via, meno diffusa in letteratura ma decisamente meno opinabile: definire l'esistenza tramite la persistenza nel tempo. Diciamo che ''$x$ esiste'' se e solo se ''$x$ è identico a sè stesso in ogni frame temporale $\la T,<\ra$'', dove $T$ è un insieme non vuoto di istanti e $<$ una relazione binaria in $T$ (e la relazione di esistenza in $T$ è allora una relazione $(x,t)\mapsto x\mathrel{\tilde\in} t$; ``$x$ esiste in $T$'' se per ogni $t : T$ si ha $x\mathrel{\tilde\in} t$).

 	Come si vede questa definizione cattura una nozione intuitiva di esistenza, impiegando sia l'identità (con tutti i problemi che essa comporta) che il tempo, o meglio una opportuna logica temporale nella quale far "persistere" le entità. (E' facile scrivere cosa significa la relazione ``$x$ esiste in $T$'' in termini di (L)TL)

	 Uno dei risultati di questo paper è che possiamo definire l'esistenza in maniera altrettanto intuitiva, senza riferirsi all'identità, né a un frame temporale; in effetti, fornendo un concetto più generale, dentro il quale si troveranno anche gli altri.

	 Porremo la questione nel seguente modo: ciò che è variabile relativo ad $x$ è il grado, o \emph{forza} della sua esistenza; l'esistenza ``classica'' è esistenza in massimo grado nel linguaggio interno del topos che fa da Universo (per noi, un universo borgesiano); lì saremo in grado di indicare il ''grado'' di esistenza degli oggetti che lo abitano, senza presupporre di muoverci attraverso istanti di tempo (come potrebbe suggerire l'esempio delle monete) o punti dello spazio (come la freccia).

	 A seconda della struttura del dominio possiamo scegliere la logica da utilizzare e così il ''contesto'' più adatto all'intuizione che abbiamo dell'universo nel linguaggio naturale.

 	La persistenza nel tempo non è perciò rimossa dalla descrizione, o negata; piuttosto, inglobata. E' un sottocaso del modello generale, precisamente quello in cui la proposizione sull'esistenza dell'oggetto è vera con forza 1 i tutti gli istanti (cfr. \autoref{}, nota 16).

 	Va da sè che, molto informalmente, l'esistenza in this conception non è altro che la modalità di "presenza" degli oggetti all'interno di un modello. E' quindi letteralmente ciò che possiamo \emph{farci} con gli oggetti, come possiamo porci rispetto a essi.

 	Non è solo una nozione operativa di esistenza, vicina peraltro al nostro senso comune: per noi le cose esistono se possiamo toccarle, vederle, postularle (quando invisibili) in base a ipotesi su rapporti di causazione che hanno con entità osservabili, descriverle, contarle, utilizzarle; e ciò è indipendente dal \emph{come} esse esistano. E di conseguenza è anche una visione epistemica.

 	Bypassata la domanda ''se le cose esistono'', in base alle nostre scelte metateoriche e fondazionali, l'esistenza riguarda i modi tramite i quali le cose entrano in relazione l'una con l'altra. Questo ci permette, ecco i vantaggi dell'ontologia categoriale, di sfruttare la visione strutturalista e poter descrivere e render cogenti non solo il nostro mondo ma realtà distanti, come Tl\"on, fornite di un'ontologia diversa.

 	In linguaggio matematico questo "modo di comportarsi delle cose" non è altro che lo studio delle relazioni tra gli oggetti di una categoria.

 	Infine sarà, l'esistenza, anche context-dependent; varierà a seconda del linguaggio interno della teoria (cioè della categoria) nella quale operiamo. E questo permette di formalizzare la banale intuizione, spesso sfuggente agli occhi degli ontologi, per cui l'esistenza in un mondo come Tl\"on sarà presumibilmente diversa dalla nostra. L'ovvia constatazione che cambiando ontologia cambia il concetto di ''esistere'' diventa qui una cosneguenza automatica dell'uso di un linguaggio matematico.
 \end{italian}