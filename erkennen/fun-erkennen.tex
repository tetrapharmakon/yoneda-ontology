\documentclass[a4paper]{../birkjour}

\usepackage{../fouche-copy}
\makeatletter
\def\@settitle{\begin{center}%
  \baselineskip14\p@\relax
  \bfseries
  \uppercasenonmath\@title
  \@title
  \ifx\@subtitle\@empty\else
     \\[1ex]\uppercasenonmath\@subtitle
     \footnotesize\mdseries\@subtitle
  \fi
  \end{center}%
}
\def\subtitle#1{\gdef\@subtitle{#1}}
\def\@subtitle{}
\makeatother

\newcommand{\var}[3][]{
  \left[\begin{smallmatrix} #2 \\
  #1\downarrow \\ #3
  \end{smallmatrix}\right]}
\newcommand{\cvar}[3]{
  \begin{xsmallmatrix}{0em}
  & #1 \\ #2 & \downarrow \\ & #3
  \end{xsmallmatrix}}

\def\la{\langle}
\def\ra{\rangle}
\def\lr#1#2{\la #1,#2\ra}
\def\tr{\textsf{t}}
\newcommand{\true}{\texttt{t}}
\def\id{\text{id}}
\author{Dario Dentamaro}
\address{Dario \textsc{Dentamaro}: }
\email{dio@cane.it}
\author{Fosco Loregian}
\address{%
  Fosco \textsc{Loregian}: %
  Tallinn University of Technology, %
  Institute of Cybernetics, Akadeemia tee 15/2,%
  12618 Tallinn, Estonia}
\email{fosco.loregian@taltech.ee}
\email{fosco.loregian@gmail.com}
\title{Categorical ontology I}
\subtitle{Existence}
\usepackage{proof}

\usepackage{minted}
\def\mil#1{\mintinline{haskell}{#1}}
\newcommand{\po}[1][dr]{\save*!/#1+1.5pc/#1:(1,-1)@^{|-}\restore}
\newcommand{\pb}[1][dr]{\save*!/#1-1.5pc/#1:(-1,1)@^{|-}\restore}


\setlength{\epigraphwidth}{.6\textwidth}
\setcounter{tocdepth}{1}

\usepackage{fontawesome}
\title{Categorical Ontology I$\frac{1}{2}$: Functorial erkennen}
%== authors' info
% \author{Blinded authors}
%
%
%
\author{Dario \textsc{Dentamaro}}
\address{Università degli Studi di Firenze,\newline Dipartimento di Matematica e Informatica
        }
\email{dario.dentamaro@stud.unifi.it}
% \email{dariodentamaro26@gmail.com}
%
\author{Fosco \textsc{Loregian}}
\address{ Tallinn University of Technology,\newline %
          Institute of Cybernetics, Akadeemia tee 15/2, \newline %
          12618 Tallinn, Estonia }
\email{fosco.loregian@taltech.ee}
% \email{fosco.loregian@gmail.com}


\usepackage{booktabs}
\usepackage{braket}
\begin{document}

\maketitle
\tableofcontents

\section{Introduction}
Qui spiegare il senso del lavoro: fornire gli strumenti per una semantica funtoriale delle teorie, che non siano solo scientifiche. Il lavoro si inserisce parzialmente nel nostro tentativo di fornire strumenti matematici adeguati per trattare problemi di natura filosofica. In questo caso si prende una delle parti meglio sviluppate della filosofia della scienza e la si adegua anche a teorie metafisiche o ontologiche, in più migliorando l'approccio agli oggetti di cui normalmente ci si occupa in questo campo (teorie fisiche e biologiche). Partirei con frase a effetto sul problema del rapporto tra teoria e mondo.

\section{Semantical conception of theories}

During the XXth century it was considered necessary to develop a formal treatment of scientific theories. The Wiener Kreis verificationist paradigm/account, and the Neurath theory of "protocollar statements", was the input to elaborate a completely semantical framework for working with scientific theories, and the clue of a pan-linguistic vision of philosophy of science. 

For the sake of strictness the formal account in which Carnap and associates provide their notion of "theory" is known in literature as \emph{syntactical conception of theories} \cite{.} while the introduction of term "semantic" is due to later developments. But the field of epistemology that the logical neopositivism started one can call "semantics of theories", because some characteristics, and above all the underlying ideology, are the same from Carnap to Beth to Suppes, up to the recent canonical uses of physical handbooks. 

[scrivere quali sono queste caratteristiche]

[sintesi delle varie concezioni; le teorie come classi di modelli $\mathcal{K}$; le teorie come oggetti formali]

semantica non standard per teorie empiriche in cui le teorie sono sistemi formali e tutte le nozioni diventano oggetti matematici; più propriamente una \emph{teoria} diventa una struttura $(F_{\mathcal{L}}, \mathcal{K})$ dove $F_{\mathcal{L}}$ è il vero sistema formale e $\mathcal{K}$ è la classe di tutti i suoi modelli. La nostra strategia è separare ulteriormente $F_{\mathcal{L}}$ in due "vocabolari" (categorie sintattiche) $P_{F_{\mathcal{L}}}$ che rappresenta i termini puri (nel senso di Plantinga) e $A_{F_{\mathcal{L}}}$ che rappresenta i termini \emph{applicati}. Ergo una teoria $\mathbf{T}$ sarà una tripla $\braket{(P_{F_{\mathcal{L}}},A_{F_{\mathcal{L}}}), \mathcal{K}}$ in cui la prima coppia configura una logica (o meglio configura uno spazio degli stati che configura una logica). La specificazione del dominio di $A_{F_{\mathcal{L}}}$ determina il tipo di teoria che stiamo considerando (scientifica, strettamente empirica, logico-matematica, metafisica).

Dire che $A_{F_{\mathcal{L}}}$ determina le \emph{tipizzazioni} della teoria significa dire che svolge lo stesso ruolo della legge $\beta$ nella semantica dello spazio degli stati, mentre la classe $\mathcal{K}$ è isomorfa all'insieme $\mathcal{M}$ dello spazio degli stati. Il tipo di $\beta$ determina il tipo di $\mathcal{M}$ che determina il tipo di $\mathbf{T} = (\mathcal{M}, \beta)$. Idem nel nostro approccio: $A_{F_{\mathcal{L}}} = \{\alpha_1,\dots,\alpha_n\}$ determina il tipo, che implementa una logica che determina la classe $\mathcal{K}$.



\subsection{The Two Dictionaries}

Nella concezione neopositivistica la distinzione tra legge teorica e legge empirica non è dovuta alla natura ipotetica della prima (anche una legge empirica può esserlo) quanto dal fatto che i due tipi di legge contengono tipi differenti di termini \cite{}. La distinzione è quindi formale, e indica una approccio prettamente linguistico a questioni epistemologiche. 

Anche in questa visione "sintattica" \cite{ } una teoria è sempre una struttura che contiene un sistema formale $\mathcal{F_L}$ e la classe $\mathcal{K}$ dei suoi modelli. La strategia carnapiana per rendere conto della presenza di entità "osservazionali" e quindi, a rigore, non formalizzabili, all'interno di teorie scientifiche è quella di considerare due diversi dizionari: $\mathcal{V_T}$ che contiene \emph{termini teorici} e $\mathcal{V_O}$ che contiene \emph{termini osservativi}. Intuitivamente $\mathcal{F_L} = \mathcal{V_T} \cup \mathcal{V_O}$.

Per derivare una legge empirica da una teorica Carnap introduce delle \emph{correspondance rules} ma senza definirle adeguatamente. Possiamo analogamente fornire il framework "viennese" di una \emph{funzione di traduzione} $\varphi: \mathcal{V_O} \to \mathcal{V_T}$ tale che ogni termine osservativo $\omega_j$ viene sostituito da un corrispondente teorico $\varphi (\omega_j)$ \footnote{In generale Carnap sembra assumere che $\mathcal{V_O} \subset \mathcal{V_T}$ ma specifica comunque che è errato dire che gli O-terms siano esempi di T-terms.}. 

\begin{definition} [Wiener Definition]
	Una teoria $\mathbf{T}$ è una coppia $\braket{\tau_i, \varphi (\omega_j)}$ dove $\tau_i \in \mathcal{V_T}$ e $\omega_j \in \mathcal{V_O}$
\end{definition}



[\cite{} Carnap da pag. 299; importante la 314]

\subsection{\emph{Was Sind und was sollen die Erkennen?}}

La strategia carnapiana è figlia della distinzione di Moritz Schlick \cite{.} tra \emph{kennen} e \emph{erkennen} ... [spiegare la manfrina e la nostra "traduzione"]

In questo paragrafo parlerei della questione "sì ma cosa sono gli "osservativi" nella nostra semantica funtoriale?", dell'arbitrarietà della divisione in due categorie sintattiche, per comodità nel trattamento di determinate teorie, e introdurrei alla tensione tra teorico e osservazionale che si sviluppa formalmente in seguito (cenno storico in nota al perchè i neopositivisti fanno la ramseyfication e perchè a noi non interessa (citare lo Weinberg)).  

\section{Profunctors / Grothendieck construction}
\label{sec:org7dd09e1}
Sezione tecnica con solo robe di CT.
\section{Nerve and realisations}
\label{sec:org1a423df}
Sezione tecnica con solo robe di CT.
\section{Theories and models}
\label{sec:orge02f333}
Qui esploitiamo il linguaggio introdotto nelle precedenti due sezioni;
\begin{definition}[Theory]
  A \emph{theory} $\clL$ is the syntactic category $\clT_L$ (cf. \cite{lambek1988introduction}) of a first-order, finitely axiomatisable language $L$.
\end{definition}
\begin{definition}
  A \emph{world} is a large category $\clW$; a \emph{universe} is a world that, as a category, admits all small colimits.
\end{definition}
Given a theory $\clL$ and a world $\clW$, a $\clL$-\emph{canvas} of $\clW$ is a functor 
\[\xymatrix{\clL \ar[r]^\phi & \clW.}\]

A canvas $\phi : \clL \to \clW$ is a \emph{science} if $\phi$ is a dense functor.
\begin{remark}
  The NR paradigm exposed in \autoref{} now entails that 
  \begin{itemize}
    \item If $\clW$ is a world, we obtain a \emph{representation} functor 
    \[ \xymatrix{\clW \ar[r] & [\clL^\op, \Set];} \]
    this means: given a canvas $\phi$ of the world, the latter leaves an image on the canvas.
    \item If $\clW$ is a universe, we obtain a NR-adjunction
    \[\xymatrix{\clW \ar@<3pt>[r] & \ar@<3pt>[l] [\clL^\op, \Set];}\]
    this means: if $\clW$ is sufficiently expressive, then models of the theory that explains $\clW$ through $\phi$ can be used to acquire a two-way knowledge. Phenomena have a theoretical counterpart in $[\clL^\op,\Set]$ via the nerve; theoretical objects strive to describe phenomena via their realisation.
    \item If an $\clL$-canvas $\phi : \clL \to \clW$ is a science, `the world' is a full subcategory of the modes in which `language' can create interpretation.
  \end{itemize}
\end{remark}
The terminology is chosen to inspire the following idea in the reader: science strives to define \emph{theories} that allow for the creation of representations of the world; said representations are descriptive when there is dialectic opposition between world and models; when such representation is faithful, we have reduced `the world' to a piece of the models created to represent it.

The tongue-in-cheek here is, la scienza (nel senso usuale) non è una scienza (nel senso della definizione \autoref{}), se non in potenza; i tentativi di generare pensiero scientifico sono i tentativi di 
\begin{itemize}
  \item Riconoscere un mondo $\clW$ come un oggetto sufficientemente espressivo da contenere fenomeni e informazione;
  \item Creare un linguaggio $L$, sufficientemente `compatto', la cui categoria sintattica permette di rappresentare nel mondo;
  \item Ottenere una aggiunzione tra $\clW$ e modelli del mondo $[\clL^\op,\Set]$ ottenuti dal linguaggio $\clL$, per generare modelli a partire da fenomeni, e per prevedere fenomeni a partire da modelli;
  \item Ottenere che `il linguaggio sia un sottospazio denso del mondo', con ciò intendendo che l'aggiunzione del punto precedente è sufficientemente well-behaved da descrivere il mondo come un frammento delle rappresentazioni semantiche del linguaggio $L$.
\end{itemize}
Evidentemente, la tensione qui è tra due opposte qualità che $L$ deve avere: non deve essere troppo esteso, per essere trattabile; non deve essere troppo ristretto, per parlare di ``tutto'' il mondo che si prefigge di descrivere.

Cf. with this definition in mind \cite{biologia}:
\begin{quote}
  Lorem ipsum dolor sit amet
\end{quote}
We say that una \emph{teoria S-scientifica} è il dato di

1. L un linguaggio formale
2. TL la categoria sintattica di quel linguaggio
3. La categoria dei funtori [TL, S]

Dal momento che la categoria [TL,S] caratterizza completamente (up to Cauchy completion) $L$ e $TL$, facciamo una crasi e chiamiamo teoria $S$-scientifica semplicemente $[TL,S]$. Ecco allora he "una corrispondenza coerente che collega espressioni di F con espressioni semantiche" è semplicemente un funtore.
\section{The tension between observational and theoretical}
\label{sec:orge11c3c4}
All based on the proportion

truth values : proposition = section : presheaf

The tension between observational and theoretical can be faithfully represented through profunctor theory;
\begin{remark}
  One can think of propositional functions as relations $(x,y)\in R$ iff the pair $x,y$ renders $phi$ true; we use this idea, suitably adapted to our purpose and categorified.
\end{remark}
Yadda yadda on how questo è facile da categorificare; consideriamo la relazione $R : T \times O \to \{0,1\}$ per cui $tRo \iff \dots$
\begin{definition}
  Let $\clT,\clO$ be two categories, respectively the \emph{theoretical} and the \emph{observational} one. A \emph{$(1,1)$-ary Ramsey map} is a profunctor $\fkk : \clT \pto \clO$ (maybe ops have to be added for the sake of convention).
\end{definition}
\begin{remark}
  Particolar 1,1-ary Ramsey maps si ottengono mediante aggiunzioni; definiamo
  \begin{itemize}
    \item l'aggiunzione in Prof indotta dall'aggiunzione in Cat;
    \item il ``core osservazionale'' e il ``core teoretico'' come le subcat di dominio e codominio che sono rese equivalenti dall'aggiunzia col trucco dei punti fissi;
    \item core oss e core teo sono sempre equivalenti; come sopra, il fatto è che si cerca un modo di allargare quanto piu possibile l'equcat.
  \end{itemize}
\end{remark}
There is nothing, in their mere syntactical presentation, that allows to tell the observational and the theoretical category apart; justify with the self-involution of $\Prof$. Also, questo dà conto dell'``esistenza degli oggetti di finzione''. Sherlock Holmes è l'oggetto di una categoria teorica. Gandhi è l'oggetto di una categoria osservazionale. In quanto oggetti linguistici essi non possono essere distinti, la differenza tra loro che li etichetta uno come oggetto di finzione e l'altro come personaggio storico è la specifica di un profuntore che permette di embeddare Sherlock Holmes nella realtà (di postulare, cioè, come esso compia azioni quotidiane nella Londra vittoriana), e di rappresentare un personaggio storico in un modello finzionale (scrivendo, ad esempio, un libro sulla vita di Gandhi).

A limitation of the above definition is that in practice all sorts of configurations are possible:
\begin{itemize}
  \item una singola $O$ si lascia descrivere da due $T$, e non meno
  \item una stesso $T$ descrive due $O$ diverse
  \item ogni altra combinazione numerica possibile
\end{itemize}
Thus we have to admit multiple arguments in domain and codomain.
\section{Ramseyfication and beyond: generalised profunctors}
\label{sec:org50db6c2}
We can generalise the definition above to encompass Ramsey sentences:
\begin{definition}
  Let $\clT,\clO$ be two categories; a \emph{Ramsey map}, or a \emph{$(n,m)$-ary Ramsey map} is a profunctor $\fkK : \clT^n \pto \clO^m$
\end{definition}
The set $\fkk(\uT, \uO)$ represents the type of proofs that the observational tuple $\uO$ admits a description in terms of the theoretical tuple $\uT$.

This formalism allows to speak about particular worlds, obtained as presheaf categories over observational $\clO$; if $\clT, \clO$ is a theoretic pair, we can instantiate \autoref{} above in the particular case where $\clW = [\clO^\op, \Set]$; observe that $\clW$ is a universe! We can thus address a certain number of questions, arising from the canonical adjunction obtained by virtue of \autoref{} and \ref{}:
\[
  \xymatrix{ [(\clO^m)^\op, \Set] \ar@<3pt>[r] & \ar@<3pt>[l] [(\clT^n)^\op, \Set];}
\]
Vale la pena notare che siccome il triangolo
\[
\vcenter{\xymatrix{
  (\clO^m)^\op \ar[rr]\ar[dr]&& [(\clT^n)^\op, \Set] \ar[dl]\\
  & [(\clO^m)^\op, \Set]
}}
\]
pseudocommuta, allora la composizione $L\circ y$ fa esattamente (mate di) $\fkK$. Ciò significa: gli $\clO$-modelli, interpretati nei $\clT$-modelli, hanno rappresentazioni corrispondenti ai termini osservazionali interpretati nei $\clT$-modelli; ovvero, la rappresentazione è `coerente' sui generatori dei modelli osservazionali, ovvero\dots

Può essere che l'operazione 
\[\exists \uX . \fkk(\uO, \uX)\]
si traduca come 
\[\lambda \uO.\fkk(\uO, F\uO)\]
quando c'è una aggiunzione $F : \clO \leftrightarrows \clT : G$? Ossia, invece di saturare i termini teorici in una ipotetica tupla (operazione priva di senso senza una specificazione di condizione sulla tupla quantificata esistenzialmente) si sta considerando solo la controparte osservazionale ma trasportata nel modello mediante $F$ (ed è tutto da vedere se $F$ è fedele) cosicché la dipendenza di $\fkk$ da $T$ viene `eliminata' per mezzo di una aggiunzione (cf. `elimination of imaginaries')
\subsection{On the bicategory of generalised profunctors}
Puta caso questo è un oggetto interessante di per sé.

\section{Concluding remarks}

\subsection{The Dummett-Plantinga problem}
La manfrina su semantica pura e applicata, applicazione concreta del framework, problema delle modalities, trattare semantica applicata (tipo Lewis) come \emph{teoria} nel senso funtoriale evitando ontological committment bla bla bla (qualche riferimento su sta cosa, remember)

\subsection{Naturalizing Epistemology}
Lotta Dura Contro Natura (cit. Movimento di Liberazione Omosessuale)

Come questo lavoro chiarisce alcuni problemi aperti in quello che è il problema fondamentale della teoria della conoscenza: il rapporto mondo-teorie. 
 
\bibliography{../allofthem}{}
\bibliographystyle{amsplain}
\end{document}
