

\section{The actual work}
% Da dove iniziamo allora; per esempio dal ricostruire il nucleo del programma di Erlangen. Sicché
% 	\begin{itemize}
% 		\item Definizione di gruppo(ide);
% 		\item Definizione di azione di un gruppo(ide) su un insieme/categoria;
% 		\item Definzione dell'action groupoid e delle sue componenti connesse: il $\pi_0$ dell'action groupoid di una certa azione è l'oggetto di studio della geometria (breve digressione su come questa geometria sia una geometria discreta, i.e. algebra lineare senza topologia, o su campi finiti, così come continua, i.e. geodiff su varietà, e azioni di gruppi di Lie che sono gruppi strutturali di fibrati);
% 		\item Esempi vari dalla geometria classica: qual è il gruppo, qual è l'insieme su cui agisce, nei vari tipi di geometria:
% 		      \[proiettiva \supset affine \supset lineare \supset euclidea\]
% 		\item Come questo, a partire dalla geometria, diventa una teoria ontologica: l'identità è sempre una identità nel $\pi_0$ di un action groupoid.
% 		\item Allora, l'esempio delle sfere di Black è quello di due bambini che litigano per un giocattolo che chiamano in modi diversi.
% 	\end{itemize}
\begin{definition}[Groupoid]
A \emph{groupoid} is a category where all morphisms are invertible; a \emph{group} is a groupoid with a single object $\star$. Endomorphisms of $G$ are the \emph{elements} of the group; the set of elements of $G$ becomes a monoid under composition.
\end{definition}
\begin{definition}[Actegory]
Let $\clC$ be a category, and $\clG$ be a groupoid; a (left) actegory structure on $\clC$ consists of a functor
\[
\lambda : \clG \times \clC \to \clC
\]
such that
\begin{itemize}
  \item
\end{itemize}
This amounts to the choice of an \emph{action} of the groupoid $\clG$ over the categry $\clC$, i.e. of a strong monoidal functor $\lambda : \clG \to [\clC,\clC]$.
\end{definition}
\begin{remark}
  A \emph{right} actegory structure consists not of a functor $\rho : \clC \times \clG \to \clC$ such that axioms ... are satisfied; instead it consists of a left actegory structure with respect to the groupoid $\clG^\op$. This complies with the fact that in classical algebra a right group action is a left action of the opposite group $G^\op$.
\end{remark}
\begin{notation}
  We denote a left $\clG$-actegory structure on $\clC$ as $\lhd_\clG$ or simply $\lhd$; so, the action of $G \in\clG$ on $C \in \clC$ is denoted $G\lhd C$, and similarly for $\varphi\lhd f$. In particular, the action of a group $G$ on $\clC$ is completely determined by a homomorphism from $G$ to the group of self-equivalences of $\clC$.
\end{notation}
\begin{definition}[Action groupoid of an actegory]
Let $(\clC,\lhd_\clG)$ be an actegory; we define the \emph{action groupoid} of $(\clC,\lhd_\clG)$ as the category having the same objects of $\clC$, and where there is a morphism $X\to G\lhd X$ for every $X\in\clC, G\in\clG$.
\end{definition}
\begin{example}
un po' di esempi
\end{example}
\begin{example}
di action groupoid
\end{example}
\begin{example}
nella natura
\end{example}
\begin{definition}[Connected components functor]
Every directed graph $\underline G$, with set of vertices $V$ and set of edges $E$, defines a quotient set of $V$ by the equivalence relation generated by $A\approx B$ if there is an arrow $A\to B$; the symmetric and transitive closure of this relation yields a quotient $V/R$ that is usually denoted as the set of \emph{connected components} $\pi_0(\underline G)$. This is a functor ${\sf Gph} \to \Set$, and if we now regard a category $\clC$ as a mere directed graph we obtain a well-defined set $\pi_0(\clC)$ of connected components of a category.
\end{definition}
\subsection{Klein program, with groupoids}
\begin{remark}[In the discrete as well as in the continuous]
Breve digressione su come questa geometria sia una geometria discreta, i.e. algebra lineare senza topologia, o su campi finiti, così come continua, i.e. geodiff su varietà, e azioni di gruppi di Lie che sono gruppi strutturali di fibrati
\end{remark}
\begin{example}
  A roundup of examples from all geometry
\[proiettiva \supset affine \supset lineare \supset euclidea\]
\end{example}
Effects on ontology: identity always is an identity in the $\pi_0$ of a certain action groupoid.