\documentclass[a4paper]{../birkjour}

\usepackage{../fouche-copy}
\makeatletter
\def\@settitle{\begin{center}%
  \baselineskip14\p@\relax
  \bfseries
  \uppercasenonmath\@title
  \@title
  \ifx\@subtitle\@empty\else
     \\[1ex]\uppercasenonmath\@subtitle
     \footnotesize\mdseries\@subtitle
  \fi
  \end{center}%
}
\def\subtitle#1{\gdef\@subtitle{#1}}
\def\@subtitle{}
\makeatother

\newcommand{\var}[3][]{%
\left[%
\begin{smallmatrix}
  #2            \\%
  #1 \downarrow \\%
  #3%
\end{smallmatrix}%
\right]%
}
%
\newcommand{\cvar}[3]{%
\begin{xsmallmatrix}{0em}%
  & #1         \\ 
  #2 & \downarrow \\ 
  & #3%
\end{xsmallmatrix}%
}

\newcommand{\po}[1][dr]{\save*!/#1+1.5pc/#1:(1,-1)@^{|-}\restore}
\newcommand{\pb}[1][dr]{\save*!/#1-1.5pc/#1:(-1,1)@^{|-}\restore}
\def\theory#1{\textsf{#1}}
\def\CT{\theory{CT}\@\xspace}
\setlength{\epigraphwidth}{.6\textwidth}
\setcounter{tocdepth}{1}
%
\def\ra{\rangle}
\def\la{\langle}
\def\lr#1#2{\la #1,#2\ra}
\def\tr{\textsf{t}}
\newcommand{\true}{\texttt{t}}
\def\id{\text{id}}

%== authors' info
\author{Dario Dentamaro}
\address{ Dario \textsc{Dentamaro}: 
        }
\email{dio@cane.it}
%
\author{Fosco Loregian}
\address{ Fosco \textsc{Loregian}:%
          Tallinn University of Technology,%
          Institute of Cybernetics, Akadeemia tee 15/2,%
          12618 Tallinn, Estonia%
        }
\email{fosco.loregian@taltech.ee}
\email{fosco.loregian@gmail.com}

%== metadata
\title{Categorical ontology I}
\subtitle{Existence}

\usepackage{ xspace
           , proof
           , fontawesome
           , listings
           }

\renewcommand*{\ttdefault}{cmvtt}
\title{Categorical Ontology I$\frac{1}{2}$: Erkennen}
%== authors' info
% \author{Blinded authors}
%
%
%
\author{Dario \textsc{Dentamaro}}
\address{Università degli Studi di Firenze,\newline Dipartimento di Matematica e Informatica
        }
\email{dario.dentamaro@stud.unifi.it}
% \email{dariodentamaro26@gmail.com}
%
\author{Fosco \textsc{Loregian}}
\address{ Tallinn University of Technology,\newline %
          Institute of Cybernetics, Akadeemia tee 15/2, \newline %
          12618 Tallinn, Estonia }
\email{fosco.loregian@taltech.ee}
% \email{fosco.loregian@gmail.com}


\usepackage{booktabs}
\begin{document}

\maketitle
\tableofcontents


\section{Profunctors / Grothendieck construction}
\label{sec:org7dd09e1}
Sezione tecnica con solo robe di CT.
\section{Nerve and realisations}
\label{sec:org1a423df}
Sezione tecnica con solo robe di CT.
\section{Theories and models}
\label{sec:orge02f333}
Qui esploitiamo il linguaggio introdotto nelle precedenti due sezioni;
\begin{definition}[Theory]
  A \emph{theory} $\clL$ is the syntactic category $\clT_L$ (cf. \cite{}) of a first-order, finitely axiomatisable language $L$.
\end{definition}
\begin{definition}
  A \emph{world} is a large category $\clW$; a \emph{universe} is a world that, as a category, admits all small colimits.
\end{definition}
Given a theory $\clL$ and a world $\clW$, a $\clL$-\emph{canvas} of $\clW$ is a functor 
\[\xymatrix{\clL \ar[r]^\phi & \clW.}\]

A canvas $\phi : \clL \to \clW$ is a \emph{science} if $\phi$ is a dense functor.
\begin{remark}
  The NR paradigm exposed in \autoref{} now entails that 
  \begin{itemize}
    \item If $\clW$ is a world, we obtain a \emph{representation} functor 
    \[ \xymatrix{\clW \ar[r] & [\clL^\op, \Set];} \]
    this means: given a canvas $\phi$ of the world, the latter leaves an image on the canvas.
    \item If $\clW$ is a universe, we obtain a NR-adjunction
    \[\xymatrix{\clW \ar@<3pt>[r] & \ar@<3pt>[l] [\clL^\op, \Set];}\]
    this means: if $\clW$ is sufficiently expressive, then models of the theory that explains $\clW$ through $\phi$ can be used to acquire a two-way knowledge. Phenomena have a theoretical counterpart in $[\clL^\op,\Set]$ via the nerve; theoretical objects strive to describe phenomena via their realisation.
    \item If an $\clL$-canvas $\phi : \clL \to \clW$ is a science, `the world' is a full subcategory of the modes in which `language' can create interpretation.
  \end{itemize}
\end{remark}
The terminology is chosen to inspire the following idea in the reader: science strives to define \emph{theories} that allow for the creation of representations of the world; said representations are descriptive when there is dialectic opposition between world and models; when such representation is faithful, we have reduced `the world' to a piece of the models created to represent it.

The tongue-in-cheek here is, la scienza (nel senso usuale) non è una scienza (nel senso della definizione \autoref{}), se non in potenza; i tentativi di generare pensiero scientifico sono i tentativi di 
\begin{itemize}
  \item Riconoscere un mondo $\clW$ come un oggetto sufficientemente espressivo da contenere fenomeni e informazione;
  \item Creare un linguaggio $L$, sufficientemente `compatto', la cui categoria sintattica permette di rappresentare nel mondo;
  \item Ottenere una aggiunzione tra $\clW$ e modelli del mondo $[\clL^\op,\Set]$ ottenuti dal linguaggio $\clL$, per generare modelli a partire da fenomeni, e per prevedere fenomeni a partire da modelli;
  \item Ottenere che `il linguaggio sia un sottospazio denso del mondo', con ciò intendendo che l'aggiunzione del punto precedente è sufficientemente well-behaved da descrivere il mondo come un frammento delle rappresentazioni semantiche del linguaggio $L$.
\end{itemize}
Evidentemente, la tensione qui è tra due opposte qualità che $L$ deve avere: non deve essere troppo esteso, per essere trattabile; non deve essere troppo ristretto, per parlare di ``tutto'' il mondo che si prefigge di descrivere.
\section{The tension between observational and theoretical}
\label{sec:orge11c3c4}
All based on the proportion

truth values : proposition = section : presheaf

The tension between observational and theoretical can be faithfully represented through profunctor theory;
\begin{definition}
  Let $\clT,\clO$ be two categories, respectively the \emph{theoretical} and the \emph{observational} one. A \emph{$(1,1)$-ary Ramsey map} is a profunctor $\fkk : \clT \pto \clO$ (maybe ops have to be added for the sake of convention).
\end{definition}
There is nothing, in their mere syntactical presentation, that allows to tell the observational and the theoretical category apart; justify with the self-involution of $\Prof$.

The set $\fkk(\underline T, \underline O)$ represents the type of proofs that the observational tuple $\underline O$ admits a description in terms of the theoretical tuple $\underline T$.

A limitation of the above definition is that in practice all sorts of configurations are possible:
\begin{itemize}
  \item una singola $O$ si lascia descrivere da due $T$, e non meno
  \item una stesso $T$ descrive due $O$ diverse
  \item etc
  \item etc
\end{itemize}
Thus we have to admit multiple arguments in domain and codomain.
\section{Ramseyfication and beyond: generalised profunctors}
\label{sec:org50db6c2}
We can generalise the definition above to encompass Ramsey sentences:
\begin{definition}
  Let $\clT,\clO$ be two categories; a \emph{Ramsey map}, or a \emph{$(n,m)$-ary Ramsey map} is a profunctor $\fkK : \clT^n \pto \clO^m$
\end{definition}
This formalism allows to speak about particular worlds, obtained as presheaf categories over observational $\clO$; if $\clT, \clO$ is a theoretic pair, we can instantiate \autoref{} above in the particular case where $\clW = [\clO^\op, \Set]$; observe that $\clW$ is a universe!
\end{document}
