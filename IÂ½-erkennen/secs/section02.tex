\section{Semantical conception of theories}
During the XXth century it was considered necessary to develop a formal treatment of scientific theories. The Wiener Kreis verificationist paradigm/account, and the Neurath theory of `protocollar statements', was the input to elaborate a completely semantical framework for working with scientific theories, and the clue of a pan-linguistic vision of philosophy of science. 

For the sake of strictness the formal account in which Carnap and associates provide their notion of `theory' is known in literature as \emph{syntactical conception of theories} \cite{.} while the introduction of term `semantic' is due to later developments. But the field of epistemology that the logical neopositivism started one can call `semantics of theories', because some characteristics, and above all the underlying ideology, are the same from Carnap to Beth to Suppes, up to the recent canonical uses of physical handbooks. 

[scrivere quali sono queste caratteristiche]

[sintesi delle varie concezioni; le teorie come classi di modelli $\clK$; 
le teorie come oggetti formali]

semantica non standard per teorie empiriche in cui le teorie sono sistemi formali e tutte le nozioni diventano oggetti matematici; più propriamente una \emph{teoria} diventa una struttura $(F_\clL, \clK)$ dove $F_\clL$ è il vero sistema formale e $\clK$ è la classe di tutti i suoi modelli. La nostra strategia è separare ulteriormente $F_\clL$ in due `vocabolari' (per noi, le categorie sintattiche di teorie al primo ordine), uno $P_{F_\clL}$ che rappresenta i termini puri (nel senso di Plantinga) e uno $A_{F_\clL}$ che rappresenta i termini \emph{applicati}. Ergo una teoria $\mathbf{T}$ sarà una particolare tripla $\bk{(P_{F_\clL},A_{F_\clL}), \clK}$ in cui la prima coppia configura una logica (uno `spazio degli stati' che configura una logica). 

La coppia $(P_{F_\clL},A_{F_\clL})$ è poi soggetta a una ulteriore condizione di ammissibilità, cf. \ref{}, chiedendo che esista un profuntore tra le due categorie sintattiche $P_{F_\clL},A_{F_\clL}$.

La specificazione del dominio di $A_{F_\clL}$ determina il tipo di teoria che stiamo considerando (scientifica, strettamente empirica, logico-matematica, metafisica).

Dire che $A_{F_\clL}$ determina le \emph{tipizzazioni} della teoria significa 
dire che svolge lo stesso ruolo della legge $\beta$ nella semantica dello spazio degli 
stati, mentre la classe $\clK$ è isomorfa all'insieme $\mathcal{M}$ dello spazio 
degli stati. Il tipo di $\beta$ determina il tipo di $\mathcal{M}$ che determina il 
tipo di $\mathbf{T} = (\mathcal{M}, \beta)$. Idem nel nostro approccio: $A_{F_\clL} 
= \{\alpha_1,\dots,\alpha_n\}$ determina il tipo, che implementa una logica che determina 
la classe $\clK$. 
%si può generalizzare tutto ciò dicendo appunto che a seconda di come è strutturata la classe $A_{\mathcal{F_L}}$ noi possiamo distinguere le teorie anche al di fuori dell'ambito strettamente scientifico. Questa considerazione va a scomparire proprio con ciò che emerge in sezione 6: eliminando questa distinzione ottieni un account più generale di trattamento delle teorie%

\subsection{The Two Dictionaries}
Nella concezione neopositivistica la distinzione tra legge teorica e legge empirica non è dovuta alla natura ipotetica della prima (anche una legge empirica può esserlo) quanto dal fatto che i due tipi di legge contengono tipi differenti di termini \cite{}. La distinzione è quindi formale, e indica una approccio prettamente linguistico a questioni epistemologiche. 

Anche in questa visione `sintattica' \cite{ } una teoria è sempre una struttura che contiene un sistema formale $\mathcal{F_L}$ e la classe $\clK$ dei suoi modelli. La strategia carnapiana per rendere conto della presenza di entità `osservazionali' e quindi, a rigore, non formalizzabili, all'interno di teorie scientifiche è quella di considerare due diversi dizionari: $\mathcal{V_T}$ che contiene \emph{termini teorici} e $\mathcal{V_O}$ che contiene \emph{termini osservativi}. Intuitivamente $\mathcal{F_L} = \mathcal{V_T} \cup \mathcal{V_O}$, ma piu precisamente $\mathcal{F_L} = \mathcal{V_T} \uplus_\varphi \mathcal{V_O}$.

Per derivare una legge empirica da una teorica Carnap introduce delle \emph{correspondance rules} ma senza definirle adeguatamente. Possiamo analogamente fornire il framework `viennese' di una \emph{funzione di traduzione} $\varphi: \mathcal{V_O} \to \mathcal{V_T}$ tale che $\omega_j \mapsto \varphi (\omega_j)$ \footnote{In generale Carnap sembra assumere che $\mathcal{V_O} \subset \mathcal{V_T}$ ma specifica comunque che è errato dire che gli O-terms siano esempi di T-terms.}.
%qui va aggiunta la funzione di "applicazione" $\psi: \mathcal{V_T} \to \mathcal{V_O}$ che fa il percorso inverso di quella di traduzione. è un modo migliore per rappresentare il dibattito di inizio 900: quale delle due funziona meglio? Non esiste la risposta corretta (per i neopositivisti presumibilmente la Ramsey version della def 2.1 coinvolge $\psi$ e ha come oggetti elementi di $\mathcal{V_O}$). La risposta storicamente più accurata è "entrambe alternativamente". Bene, altra cosa risolta dalla sezione 6% 

\begin{definition} [Wiener Definition]
	Una teoria $\mathbf{T}$ è una coppia $\bk{\tau_i, \varphi (\omega_j)}$ dove $\tau_i$, $\varphi(\omega_j) \in \mathcal{V_T}$. 
\end{definition}



[\cite{} Carnap da pag. 299; importante la 314]

\subsection{\emph{Was Sind und was sollen die Erkennen?}}

La strategia carnapiana è figlia della distinzione di Moritz Schlick \cite{.} tra \emph{kennen} e \emph{erkennen} ... [spiegare la manfrina e la nostra `traduzione']

In questo paragrafo parlerei della questione `sì ma cosa sono gli `osservativi' nella nostra semantica funtoriale?', dell'arbitrarietà della divisione in due categorie sintattiche, per comodità nel trattamento di determinate teorie, e introdurrei alla tensione tra teorico e osservazionale che si sviluppa formalmente in seguito (cenno storico in nota al perchè i neopositivisti fanno la ramseyfication e perchè a noi non interessa (citare lo Weinberg)).  
%forse tutta sta parte confluisce in section 6%

