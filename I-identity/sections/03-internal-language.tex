\section{The internal language of variable sets}\label{int_lang}
\epigraph{I am hard but I am fair; there is no racial bigotry here. [\dots\unkern] Here you are all equally worthless.}{GySgt Hartman}
\begin{definition}\label{da_lang}
The internal language of a topos $\clE$ is a formal language defined by \emph{types} and \emph{terms}; suitable terms form the class of variables. Other terms form the class of \emph{formul\ae}.
\begin{itemize}
	\item \emph{Types} are the objects of $\clE$
	\item \emph{Terms} of type $X$ are morphisms of codomain $X$, usually denoted $\alpha,\beta,\sigma,\tau : U \to X$.
	      \begin{itemize}
		      \item Suitable terms are variables: the identity arrow of $X\in\clE$ is the variable  $x : X \to X$. For technical reasons we shall keep a countable number of variables of the same type distinguished:\footnote{These technical reasons lie on the evident necessity to be free to refer to the same free variable an unbounded number of times. This can be formalised in various ways: we refer the reader to \cite[]{lambekscott} and \cite[]{johnstopos}.} $x,x',x'',\dots : X \to X$ are all interpreted as $1_X$.
	      \end{itemize}
	\item Generic terms may depend on multiple variables; the domain of a term of type $X$ is the \emph{domain of definition} of a term.
\end{itemize}
A number of inductive clauses define the other terms of the language:
\begin{itemize}
	\item the identity arrow of an object $X\in\clE$ is a term of type $X$;
	\item given terms $\sigma : U \to X$ and $\tau :  V\to Y$ there exists a term $\lr{\sigma}{\tau}$ of type $X\times Y$ obtained from the pullback
	      \[\xymatrix{
		      W \ar[d]\ar[r]\ar[dr]|{\lr{\sigma}{\tau}} & X \times V \ar[d]\\
		      U\times Y \ar[r]& X\times Y
		      }\]
	\item Given terms $\sigma : U \to X, \tau : V \to X$ of the same type $X$, there is a term $[\sigma = \tau] : W \xto{\lr{\sigma}{\tau}} X\times X \xto{\delta_X} \Omega$, where $\delta_X : X\times X \to \Omega$ is defined as the classifying map of the mono $X \hookrightarrow X\times X$.
	\item Given a term $\sigma : U \to X$ and a term $f : X \to Y$, tere is a term $f[\sigma] := f\circ\sigma : U \to Y$.
	\item Given terms $\theta :  V \to Y^X$ and $\sigma : U\to X$, there is a term
	      \[
		      W\lr{\theta}{\sigma} \xto{}Y^X\times X \xto{\text{ev}} Y
	      \]
	\item In the particular case $Y=\Omega$, the term above is denoted
	      \[[\sigma\in\theta] : W\lr{\theta}{\sigma} \to \Omega\]
	\item If $x$ is a variable of type $X$, and $\sigma : X\times U \to Z$, there is a term
	      \[\lambda x.\sigma : U \xto{\eta} (X\times U)^X \xto{\sigma^X} Z^X\]
	      obtained as the mate of $\sigma$.
\end{itemize}
These rules can of course be also presented as the formation rules for a Gentzen-like deductive system: let us rewrite them in this formalism.
\[ \begin{array}{cc}
		\infer{1_X : X \to X}{}                                                              &
		\infer{\lr{\sigma}{\tau} : W\lr{\theta}{\sigma} \to X \times Y }{\sigma : U \to X    &   & \tau : V \to Y}             \\[1em]
		\infer{[\sigma=\tau] : W \to \Omega}{\sigma : U \to X                                &   & \tau : V \to X}           &
		\infer{f[\sigma] : U \to Y}{\sigma : U \to X                                         &   & f : X \to Y}                \\[1em]
		\infer{W\lr{\theta}{\sigma} \xto{}Y^X\times X \xto{\text{ev}} Y}{\theta :  V \to Y^X &   & \sigma : U\to X}          &
		\infer{\lambda x.\sigma = \sigma^X\circ\eta : U \to (X\times U)^X \to Z^X}{ x: X     &   & \sigma : X\times U \to Z}
	\end{array}\]
To formulas of the language of $\clE$ we apply the usual operations and rules of first-order logic: logical connectives are induced by the structure of internal Heyting algebra of $\Omega$: given formulas $\varphi,\psi$ we define
\begin{itemize}
	\item $\varphi\lor \psi$ is the formula $W\lr{\varphi}{\psi} \to \Omega\times \Omega \xto{\lor} \Omega$;
	\item $\varphi\land\psi$ is the formula $W\lr{\varphi}{\psi} \to \Omega\times \Omega \xto{\land} \Omega$;
	\item $\varphi\Rightarrow\psi$ is the formula $W\lr{\varphi}{\psi} \to \Omega\times \Omega \xto{\Rightarrow} \Omega$;
	\item $\lnot\varphi$ is the formula $U \to \Omega \xto{\lnot} \Omega$.
\end{itemize} 
\end{definition}
\todo[inline]{universal quantifiers}
Each formula $\varphi : U \to \Omega$ defines a subobject $\{x\mid \varphi\} \subseteq U$ of its domain of definition; this is the subobject classified by $\varphi$, and must be thought as the subobject where ``$\varphi$ is true''.

If $\varphi : U\to\Omega$ is a formula, we say that $\varphi$ is \emph{universally valid} if $\{x\mid\varphi\}\cong U$. If $\varphi$ is universally valid in $\clE$, we write ``$\clE\Vdash \varphi$'' (read: ``$\clE$ believes in $\varphi$'').

Examples of universally valid formulas:
\begin{itemize}
	\item $\clE\Vdash [x=x]$
	\item $\clE\Vdash [(x \in_X \{x\mid\varphi\}) \iff \varphi]$
	\item $\clE\Vdash \varphi$ if and only if $\clE \Vdash \forall x.\varphi$
	\item $\clE\Vdash [\varphi \Rightarrow \lnot\lnot\varphi]$
\end{itemize}
\todo[inline]{Ora facciamo delle considerazioni sul lingo interno di $\Set/I$}
Chi sono tipi e termini; chi sono le proposizioni e come si scrive il calcolo proposizionale in $\Set/I$; i quantificatori, in dettaglio pornografico.
\begin{definition}
	Types and terms of $\clL(\Set/I)$.
\end{definition}
\begin{definition}
	Propositional calculus, quantifiers.
\end{definition}
\begin{remark}
	Like every other Grothendieck topos, the category $\Set/I$ has a \emph{natural number object} (see \cite[??]{mac1992sheaves}); here we shall outline its construction. It is a general fact that such a natural number object in the category of variable sets, consists of the constant functor on $\bbN : \Set$, when we realise variable sets as functors $I \to \Set$: thus, in fibered terms, the natural number object is just $\pi_I : \bbN \times I \to I$.

	A natural number object provides the category $\clE$ it lives in with a notion of \emph{recursion} and with a notion of $\clE$-induction principle: namely, we can interpret the sentence
	\[\textstyle\big(Q0\land \bigwedge_{i\le n} Qi\Rightarrow Q(i+1)\big)\Rightarrow \bigwedge_{n : \bbN} Qn\]
	for every $Q : \bbN \to \Omega_I$.
	
	In the category of variable sets, the universal property of $\pi_I : \bbN\times I \to I$ amounts to the following fact: given any diagram of solid arrows 
	\[
	\xymatrix{
		I \ar[r]^0 \ar@{=}[d] & \bbN\times I\ar@{.>}[d]^u \ar[r]^{s\times I} & \bbN \times I\ar@{.>}[d]^u \\ 
		I \ar[r]_x & X \ar[r]_f & X
	}	
	\]
	where every arrow carry a structure of morphism over $I$ (and $0 : i \mapsto (0,i)$, $s\times I : (n,i) \mapsto (n+1,i)$), there is a unique way to complete it with the dotted arrow, i.e. with a function $u : \bbN \times I \to X$ such that 
	\[u \circ (s\times I) = f \circ u.\]
	Clearly, $u$ must be defined by induction: if it exists, the commutativity of the left square amounts to the request that $u(0,i)=x(i)$ for every $i : I$. Given this, the inductive step is 
	\[ 
		u(s(n,i)) = u(n+1,i) = f(u(n,i)).
	\]
	This recursively defines a function with the desired properties; it is clear that these requests uniquely determine $u$.
\end{remark}
Such a terse exposition obviously does not exhaust such a vast topic as recursion theory conducted with category-theoretic tools. The interested reader shall consult \cite{jacobs1997tutorial} for a crystal-clear introductory account, and \cite{cockett2008introduction,cockett2014total} for more recent and modern development of recursion theory.

The object of natural numbers of $\Set/I$ is easily seen to match the definition of the \emph{initial object} \cite[]{Bor1} of the category $\cate{Dyn}/I$ so defined:
\begin{itemize}
	\item the objects of $\cate{Dyn}/I$ are \emph{dynamical systems} in $\Set/I$, i.e. the triples $(x,X,f)$, where $X : \Set/I$ (say, with structure map $\xi : X \to I$), $x : (I,\id_I) \to (X,\xi)$ and $f : X \to X$ is an endo-morphism of variable sets;
	\item given two dynamical systems $(x,X,f)$ and $(y,Y,g)$ a \emph{morphism} between them is a function $u  :X \to Y$ such that the diagram
	\[yadda yadda\]
	commutes in all its parts.
\end{itemize}