\begin{abstract}
  The present paper approaches ontology and metaontology through mathematics, and more precisely through category theory. We exploit the theory of \emph{elementary toposes} to claim that a satisfying ``theory of existence'', and more at large ontology itself, can both be obtained through category theory. In this perspective, an \emph{ontology} is a mathematical object: it is a category, the universe of discourse in which our mathematics (intended at large, as a theory of knowledge) can be deployed. The \emph{internal language} that all categories possess prescribes the modes of existence for the objects of a fixed ontology/category.

  This approach resembles, but is more general than, fuzzy logics, as most choices of $\clE$ and thus of $\Omega_\clE$ yield nonclassical, many-valued logics.
  
  Framed this way, ontology suddenly becomes more mathematical: a solid corpus of techniques can be used to backup philosophical intuition with a useful, modular language, suitable for a practical foundation. As both a test-bench for our theory, and a literary \emph{divertissement}, we propose a possible category-theoretic solution of Borges' famous paradoxes of Tl\"on's ``nine copper coins'', and of other seemingly paradoxical construction in his literary work. We then delve into the topic with some vistas on our future works.
\end{abstract}