\epigraph{[\dots\unkern] Le rôle essentiel [du travail de mathématicien] c'est cette transition qu'il y a entre ce quelque chose qui est écrit, qui parait incompréhensible, et les images mentales que l'on crée.}{A. Connes}
\section{Semantic conception of theories}\label{sec_1_intro}
The present work approaches a well-established problem in epistemology: what is the way in which we build representations of the world from perception? What is, if any, the relation between the two `worlds', one depicted in our minds, and one on our fingertips? Sometimes, such a representation results in a faithful image of the perceived world (we call it `science'); sometimes it doesn't (we call it `superstition').

We hereby propose a sense in which science and superstition can be told apart using a mathematical theory, or even a mathematical object.
\subsection{A convincing notion of theory: two dictionaries}
Along the XXth century there have been many attempts towards a formal definition of a scientific theory.

Examples are the \emph{Wiener Kreis}' verificationist paradigm, and Neurath's theory of `protocollar statements', that gave an initial input towards the elaboration of a semantic framework for scientific theories, and spurred the search for a pan-linguistic vision of philosophy of science \cite{Weinb}.

The formal account in which --among others-- Carnap \cite{carnapfound} provided his notion of `theory' is known in the literature as \emph{syntactical conception of theories} or `received view' \cite{krause-foundation,krause2011axiomatization,giunti2016}. Albeit the term `semantic' is due to later developments, the field of epistemology that logical neopositivism started can legitimately be labeled a `semantics of theories', because some of its features, if not the underlying ideology, are the same throughout the work of Carnap \cite{carnap56,carnapfound},  Beth \cite{beth1961semantics}, and Suppe \cite{suppe89}.

Thus, a `Wiener Kreis' theory' is understood as a structure $(F_\clL, \clK)$ where $F_\clL$ is a formal language, and $\clK$ the totality of all its interpretations, or \emph{models}.

The idea to separate further $F_\clL$ into two `vocabularies' $(\clT,\clO)$ (intended, in modern terms, as two syntactic categories carved from two first order theories) first appears in Carnap; these are respectively the \emph{pure} or $\clT$\emph{heoretical} terms, and the \emph{applied} or $\clO$\emph{bservational} terms \cite{carnap56}.

It is a commonly accepted belief --albeit rarely formalised-- that scientific theories arise from some kind of tension between the theoretical and the observational world. Our aim here is to try and `resolve the tension', acknowledging $\clT,\clO$ and their mutual relations as concrete mathematical object, rooted in category theory \cite{McL,pedicchiofoundations,riehlcontext,leinster2014basic}.

As elementary as it may seem, this idea seems fruitful to us: building on Carnap,
\begin{remark*}
	A reasonable notion of `scientific theory' is a triple $\bk{(\clT,\clO), \clK}$ whose first two elements form the `underlying logic' $F_\clL=(\clT,\clO)$ and where $\clK$ is a (possibly large) category of models or `interpretations'.
\end{remark*}
This is in fact a familiar old idea for mathematicians, as the habit of identifying a sort of mathematical structure in a way that is independent from the cohort of its syntactic presentations, permeates classical universal algebra since the early work of Lawvere \cite{lawvere1963functorial,lawvere1996unity} (see also \cite{abramskyno,Borceux1994,makkai1989accessible} for applications to logic and other disciplines).

In the Carnapian --and in general the neopositivistic-- account a theory can be expressed as a sentence formed by terms $\tau_1, \dots, \tau_k$ taken from both the dictionaries of $F_\clL$.

In the Wiener Kreis paradigm, the formal specification of $\clO$ is left unclear; Carnap \cite{carnapfound} posits the existence of \emph{correspondence rules} between $\clO$ and $\clT$, associating to each term $o$ of $\clO$, or O-term, its companion in $\clT$, or the T-term $\tau$ derived from $o$. In general, Carnap holds that $\clO \subset \clT$, but at the same time he blurs the features of this identification of observational terms as `types of T-terms'.

We can maintain a similar idea, just phrased in a slightly more precise way: we posit the existence of a function $\varphi$ that translates O-terms into T-terms. So, a Wiener definition of a theory is a suitable set of pairs $\{\bk{\tau,\varphi(\tau)} \mid \tau\in\clT\}\subseteq \clT\times\clO$, where $\varphi: \clT \to \clO$ is called a \emph{translation function}. In this way it is trivially true that all the terms of a theory are in the first dictionary.
\begin{remark*}
	A reasonable notion of scientific theory must take into account `meaningful relations' between the observable world $\clO$ and the theoretical world $\clT$; the Carnapian request that there is a functional correspondence between the two is, however, too restrictive when $\clT,\clO$ are thought as categories.
\end{remark*}
In fact, the set of pairs of a Carnap translation function $\varphi$ is precisely the \emph{graph} of $\varphi$, and not by chance: cf. our \autoref{da_collage}.\footnote{In passing, it is worth to notice that Carnap's intuition fits even more nicely in our functorial framework: assume $\clT, \clO$ exhibit some kind of structure, and that $i : \clO\subseteq\clT$ \emph{as substructures} (e.g., assume that they are some sort of ordered sets, and that the order on $\clO$ is induced by the inclusion); then, a \emph{left} (resp., \emph{right}) \emph{translation function} $\varphi_L$ (resp., $\varphi_R$), is a left (resp., right) adjoint for the inclusion $i : \clO\hookrightarrow \clT$. We will not expand further on this idea, but see \autoref{carnap_translation_functors}.}

Now, the neopositivistic current of epistemologists was the first to observe that one can build an observational version of a theory $T$ following a procedure first outlined by Ramsey \cite{ramsey1931foundations} and colloquially called \emph{Ramseyfication} of a theory; the nature of this operation seems quite elusive to those approaching it: colloquially, it can be thought as the process of replacement of each observational term of a theory with a `corresponding' theoretical term. The nature of this replacement, the syntactic domain of terms, and the sense in which the process makes a theory $F_\clL$ and its `Ramseyfied' analogue $F_\clL^\text{Ra}$ equivalent are however quite elusive.

In the language of category theory --and especially through our profunctorial approach-- instead things become clearer: under mild assumptions on a diagram
\[ \vcenter{\xymatrix{
	& \clW \ar[dl]_{N_\phi} \ar[dr]^{N_\psi} & \\
	[\clT^\op,\Set] \ar[rr]_{N_{\hat R}} && [\clO^\op,\Set]
	}} \notag\]
of categories and profunctors, if a `deduction' entails a certain interaction or a certain behaviour for the theoretical category over the observational one, then there exists a particularly well behaved natural transformation
\[ \varpi : N_{\hat R} \To \bk{\phi/\psi}\notag \]
filling the triangle above; in non\hyp{}mathematical terms, this means that the entailment of a theoretical prediction into an observed system (i.e. a term $\tau$ of type $\fkR(T,O)$) yields an entailment $\varphi(T) \to \psi(O)$ \emph{in the world}. The details of this construction, that we consider the heart of the paper, are contained in \autoref{inducing_herme}, \autoref{funcell_herme}, and heavily rely on the terminology introduced in Sections \ref{sec_2_profu} and \ref{sec_3_nervi}.

The next subsection offers a birds-eye view of the structure of the paper.
\subsection{Our contribution}
The first remarks that we made in the introductory subsection motivate at least our tentative definition for a `pre-scientific' theory: it is some sort of correspondence of categories $R : \clT \pto \clO$, between an observational and a theoretical category.

Another important point throughout the above discussion however is that from a neo-positivistic stance the distinction between theoretical and empirical is purely formal.

This is not due to the hypothetical nature of the former (empirical laws can be hypothetical), but to the fact that the two kinds of law contain different types of terms, as first observed in \cite{carnap56}. This purports a purely linguistic approach to epistemological issues, that we want to take at the extreme.

In fact, our work pushes in this direction even more: the profunctorial formulation of scientific theories deletes even more forcefully any intrinsic distinction that might be between the observational and the theoretical\fshyp{}linguistic structure of a theory.

In profunctorial terms, thanks to \autoref{da_collage} and standard category\hyp{}theoretic arguments, there is a direct counterpart \emph{in the mathematical model} for the vanishing of the distinction between observational and theoretical.

First of all, the bicategory defined in \autoref{def_profu}, taking from \cite{benabou2000distributors}, is self-dual; this means that every profunctor $\fkR : \clT \pto \clO$ admits a `mirror image' $\fkR^\op : \clO \pto \clT$;\footnote{This is reminiscent of the fact that, as observed in \autoref{sec:org7dd09e1}, a relation has not a privileged domain of definition; clearly, the category $\mathsf{Cat}$ has a nontrivial involution given by $\op$ing a category, and this renders the auto-duality slightly more visible in the case of categorified relations (i.e., profunctors).} second, and certainly more decisive a comment towards our thesis, as outlined in \autoref{resoudre_la_tension} a generic profunctor $\fkR : \clT \pto\clO$ yields the `collage' of the observational and theoretical categories $\clT,\clO$ `glued along' $\fkR$; in simple terms, the collage of $\clT,\clO$ along $\fkR$ is a new category $\clT\uplus_p\clO$, fitting in a span
\[ \vcenter{\xymatrix{
			& \clT\uplus_p \clO \ar[dr]\ar[dl]& \\
			\clT  && \clO
		}} \] having suitable fibrational properties (cf. \autoref{def:dfib} and in particular \autoref{collage_explaned}) allowing to recover the theoretical and observational terms as `lying over' $(T,O)\in\clT\times\clO$.

From this perspective, it seems obvious why we require the pair $(\clT,\clO)$ to admit a profunctor in either direction; profunctors categorify the notion of `meaningful relation' between structured high-level systems, i.e. two syntactic categories $\clT,\clO$ `modeling' the environment to which we have access.

Proposing the fundamental features of a `general theory of scientific theories' stated in terms of profunctors is the main contribution of the present work.

\medskip
We conclude this introductory section with a paragraph discussing about the `nature'' of the categories $\clT,\clO$, while surveying on the main arguments of the paper.
As we already observed in a previous work \cite{catont1}, the problem of locating the syntactic objects embodying a linguistic theory can be easily solved from an esperientialist stance: the world undeniably exists, and it is a sufficiently complex structure to contain the concrete building blocks of a formal system. We derive the primitive symbols of language from a portion of the world, complex enough to offer expressive power.

This problem, and its proposed solution, reflect unavoidably on the way in which the categories $\clT,\clO$ are built. In our model the world is a (possibly large) category $\clW$, unfathomable and given since the beginning of time, to which we can only access through \emph{probe maps} (functors) $\phi : \clL \to \clW$ (cf. \autoref{canvas_scienza}) representing small `accessible' categories construed from parts of $\clW$ that we can experience.

The request that $\clW$ is `sufficiently expressive' now translates into the request that as a category $\clW$ contains enough traces of functors like $\phi$; this (cf. \autoref{mondo_yalda}) translates formally in the request that any such $\phi$ admits a \emph{colimit} (cf. \cite[Ch. 2]{Bor1}) in $\clW$.

When things are put in this perspective, a few remarks are in order:
\begin{itemize}
	\item This perspective allows to close the circle over the problem of representation of a world $\clW$ in terms of a portion $\clT$ to which we have hermeneutical access, and from which we have carved a language.

	      In fact, such a representation happens through `canvas functors' $\phi : \clL \to \clW$ that, thanks to the cocompleteness property of $\clW$, extend uniquely to representation functors $[\clL^\op,\Set] \leftrightarrows \clW$.
	\item On the other hand, `the world' as a whole is unknowable: instead of $\clW$, we can access to an observational fragment $\clO$, from which we recover, exploiting the cocompleteness of $[\clO^\op,\Set]$, a further representation $[\clL^\op,\Set] \leftrightarrows [\clO^\op,\Set]$. In general, this is all that can be said; such a picture is already capable of determining, by elementary means, an equivalence of categories (i.e., an equivalence of models) between the observational and the theoretical \emph{nuclei} of $[\clT^\op,\Set] \leftrightarrows [\clO^\op,\Set]$: we discuss the matter in \autoref{nuclei}, and \autoref{resoudre_la_tension}.
	\item Additional assumptions on the canvas $\phi : \clL\to \clW$, however, can refine our analysis: we can infer that the totality of models $[\clL^\op,\Set]$ \emph{contains a copy} of the world $\clW$. In this precise sense, assuming what is outlined in the definition of \science in \autoref{canvas_scienza}, language prevails: the unfathomable world is a full subcategory of the class of all modes in which the language of $\clT$ can be interpreted.
	\item Under very mild assumptions on the arrangement of functors
	      \[\notag\vcenter{\xymatrix{
		      & \clW \ar[dr]^{N_\psi}\ar[dl]_{N_\phi} & \\
		      [\clT^\op,\Set] \ar[rr]_{N_{\hat R}} && [\clO^\op,\Set]
		      }}\]
	      (cf. \autoref{nervereal}) where $\phi,\psi$ are two canvases, respectively on the theoretical and observational side, we can find a natural 2-cell filling the triangle; this amounts to a `concretisation' of the canvases (see \autoref{funcell_herme} and \autoref{herme_explained}) into an implication between (a trace that) the theoretical terms (left in the world via $\phi$) and the observational terms (to which we have experimental access) in $\clW$. This last sentence is `the Ramsey sentence' that the canvases carve into the world, expressed in the internal language of $\clW$.
\end{itemize}
% As bold a statement as it might seem, this has fruitful consequences: see for example \autoref{remark_yuggoth_1}, \autoref{remark_yuggoth_2}.
\subsubsection*{Structure of the paper}
Section \ref{sec_2_profu} and \ref{sec_3_nervi} outline the mathematical background we need throughout the work; the focus is not on proofs, but we refrain from delivering a terse account of the mathematical paraphernalia without any intuition. Section \ref{sec_4_theories} introduces our main notions: a canvas, i.e. a functor $\phi : \clL \to \clW$ representing a small category in a `world', a big category $\clW$; a theory, and a science, i.e. a well-behaved canvas. Sections \ref{sec_5_tension} and \ref{sec_6_universal} will conclude the discussion; we propose some vistas for future investigation.