\section{Introduction}\label{sec_intro}
\epigraph{El mundo, desgraciadamente, es real.\\[2mm]
\footnotesize\emph{---Disgracefully, the world is real.}
}{\cite{confutacion}}
\subsection{Aim of the work}
To a mathematician living in the $21^\text{st}$ century, a more refined answer to Quine's question \cite{} ``what is there?'' is ``a category of things''. 

This accounts for the fact that the ``Everything'' in Quine's answer is not a disorganised mass of stuff, but instead, a hierarchical, structured entity, that we approach through rational thinking as well as through the structuralist's point of view (by which, in the present work, we mean its mathematical incarnation, \emph{category theory}).

We draw a series of consequences from this basic assumption, mostly aimed at showing that as extreme as it could be, it has interesting outcomes. The mathematical language employed is well-known in the circle of modern mathematicians, but we believe it deserves more attention from the philosophers' community due to its intrinsic descriptive power. If anything, this is an outreach piece of writing, attempting to propose \emph{novel} ways to employ \emph{old} ideas in a field that, at least apparently, is impervious to quantitative thinking. 

It may in fact seem suspicious to employ Mathematics to tackle questions that traditionally pertain to philosophy. By proposing the `ontology as categories' point of view, our work aims to dismantle the false belief that deep philosophical questions are too vague and fundamental to be approached quantitatively.

The reader will allow a tongue-in-cheek here, subsuming our position: in ontology, it is not a matter of making a \emph{correct use of language}, but rather a matter of \emph{using the correct language}.

This language `must be' \emph{category theory}, as only category theory has the power to speak about a totality of structures of a given kind in a compelling way, treating mathematical \emph{theories} as mathematical \emph{objects}.

%===

\subsection{Outline of the work}

The blanket assumption we make throughout the present work is that whatever it is, `the universe' is a category $\clE$ whose properties prescribe properties of the world through the \emph{internal language} (or `Mitchell-Bénabou' language) of $\clE$. 

The proposal to emply category theory as a foundation of mathematics isn't new, dating back to the pioneering work of Lawvere \cite{lawvere1964elementary,lajolla,lawvere1969adjointness,lawvere1963functorial}: in \cite{lajolla} Lawvere builds a formal language \theory{ETAC} encompassing `elementary' category theory, and a model \theory{ETCC} for the category of all categories, ultimately yielding a model for \theory{ETAC}. In this perspective category theory has a syntax in which categories are just terms. Besides, we are provided with a \emph{meta}theory, in which we can consider categories of more structured categories, etc.:
\begin{quote}
	If $\Phi$ is any theorem of the elementary theory of abstract categories, then $\forall \clA (\clA \models \Phi)$ is a theorem of the basic theory of the category of all categories. \hfill \cite{lajolla}
\end{quote}
After this, the author makes the rather ambiguous statement that `\textit{every object in a world described by the basic theory is, at least, a category}'. We posit that the statement shall be interpreted as follows: categories in Mathematics carry a double nature. They surely are the structures in which the entities we are interested to describe organise themselves; but on the other hand, they inhabit a single, big (meta)category of all categories. Such a big structure is fixed once and for all, at the outset of our discussion, and it is the \emph{place} in which we can provide concrete models for `small' categories.

In other words categories live on different, almost opposite, grounds: as -small- syntactic objects, that can be used to model \emph{language}, and as -big- semantic objects, that can be used to model \emph{meaning}. A big category $\clC$ -say, a cumulative hierarchy of sets- is given once and for all, in the background. Small categories can be used to \emph{probe} $\clC$ and draw pictures of it in terms of the traces the probes leave.

For example: there surely is such a thing as `the category of groups'. But on the other hand, groups are just very specific kinds of sets, so groups are but a substructure of `the only category that exists'.

Sure, such an approach is quite unsatisfactory from a structural perspective. It bestows the category of sets with a privileged role that it does not have: sets are just \emph{one} of the possible choices for a foundation of Mathematics. Instead, we would like to disengage the (purely syntactic) notion of structure from the (semantic) notion of interpretation.

This is where Lawvere's intuition comes into play: the `categories as places' philosophy now provides such a disengagement, to approach the foundation of Mathematics agnostically: whatever the semantic universe $\clC$ is, it is just a parameter in our general theory of all possible semantic\emph{s}.

Various research tracks in categorical algebra, \cite{Janelidze2004}, functorial semantics \cite{lawvere1963functorial,hyland2007category}, categorical logic \cite{lambek1988introduction}, and topos theory \cite{JohnstonePT} that characterised the last sixty years of research in category theoryorm this circle of ideas.

We try to push the methodological principle of structural thinking to its limit, in order to defend the following tripartite claim: 
\begin{quote}
  There are many different ontologies; each ontology is a mathematical object; the study of ontologies can thus be fruitfully approached mathematically.
\end{quote}
To wit, each ontology is a certain category $\clO$, inside which `existence' unravels as the sum of all statements that the internal language of $\clO$ can concoct.

Evidently, the more expressive is this language the more expressive the resulting theory of existence will turn out to be. Our presupposition here is that trying to let ontology speak about `\emph{all} that there is' (the accent is on the adverb, on this famous quote of Quine \cite{quine1948there}) can lead to annoying paradoxes and ambiguities. 

Instead, research shall concentrate on clarifying what the \emph{verb} means: in what sense, `what there is' \emph{is}? \emph{What is is-ness?} As category theorists, our answer --again mimicking Quine-- is that
\begin{quote}
	being is \emph{being the object of a category}.
\end{quote}
Explaining why this is exactly Quine's motto, just shifted one universe higher, is the content of our §\ref{metaon}.
\subsubsection{Structure of the paper}
% The remaining part of the first section draws a picture as accurate as possible, of the wheres and whys of structural Mathematics; its implications are the subject of several essays on the philosophy of Mathematics, like \cite{kromer2007tool,Marquis1997,marquis2010category,marquis2008geometrical}. This section has several different purposes: it provides an explicit statement of purposes for the entire polyptych \cite{black,homot}; it declares our stance on the foundation we choose, clarifying assumptions that we feel are usually neglected on essays on the topic (`where are the objects that Mathematics aims to describe? Where is the language by means of which this description is possible?') -of course without claiming to have solved the matter once and for all; we provide pointers as specific as possible in order to help the reader navigate the relevant literature.

% The second section delves into the first major point of our presentation: large categories are universes where `Existence', intended as the sum of information acquired from the perceptual bundle we experience, that language organises and conceptualises. `Language' here is a shortcut to denote the power of a fixed large category $\clC$ to express well-formed formulas of (a certain fragment of) logic. Objects and morphisms of $\clC$ shall be considered respectively as types and terms of a language, \emph{the} internal language of $\clC$; now, the richer $\clC$ is, the more it is able to faithfully represent the cosmos we're thrown into. Among many possible choices for $\clC$, we take \emph{toposes} as the class of categories harbouring `set theories': the internal language of a topos is powerful enough to re-enact set theory, and subsequently propositional logic.

After this introductory section, we deal with a specific example of a topos, useful for later examination: the categories of objects parameterised by a fixed `space of parameters' $I$. The `slice' category $\Set/I$ is a topos, and its internal language, namely its internal logic, is tightly linked to set-theoretic properties of the slicing set $I$. The logic we obtain in $\Set/I$ by casting the general definitions of subobject classifier (and internal language) is genuinely non-classical.

The subsequent section contains a careful analysis of the internal language of $\Set/I$.

The subsequent section contains an application of the tools we exposed until now: the seemingly paradoxical `nine copper coins' problem exposed in Jorge Luis Borges' \cite{Borges1963}, far from being paradoxical, admits a natural interpretation as a statement in $\Set/I$, for a suitable choice of $I$, and thus of the induced internal logic. We propose other examples of seemingly paradoxical statements in Borges' literary work that instead are admissible statements in the internal logic of \emph{some} topos: on Tl\"on entities may disappear if neglected: this means that $I$ is linearly ordered; Babylon's chaotic lottery resembles, with their obscure, impenetrable purposes, the chaotic behaviour of a dynamical system: this means that $I$ carries a semigroup action; Tl\"on's instantaneism, mimicking/mocking Berkeley, can be obtained assuming $I$ is a discrete, uncountable set.

We close the paper with a section on future development, vistas for future applications, and with a wrap-up of the discussion as we have unraveled so far. Ideally, §\ref{sec_prelim} and §\ref{int_lang} shall be skipped by readers already having some acquaintance with category theory; the second half of the paper makes however heavy use of the notation established before.