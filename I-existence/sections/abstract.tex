\begin{abstract}
  The present paper approaches ontology and metaontology through mathematics, and more precisely through category theory. We exploit the theory of \emph{elementary toposes} to claim that a satisfying ``theory of existence'', and more at large ontology itself, can both be obtained by means of category theory. For us, an \emph{ontology} is a mathematical object: it is a category $\clE$, the universe of discourse in which our mathematics (intended at large, as a theory of knowledge) can be deployed. The \emph{internal language} that all categories possess, in the particular case of an elementary topos, is induced by the presence of an object $\Omega_\clE$ parametrizing the truth values of the internal propositional calculus; such pair $(\clE,\Omega_\clE)$ prescribes the modes of existence for the objects of a fixed ontology/category.

  This approach resembles, but is more general than, the one leading to \emph{fuzzy logics}, as most choices of $\clE$ and thus of $\Omega_\clE$ yield nonclassical, many-valued logics.
  
  Framed this way, ontology suddenly becomes more mathematical: a solid corpus of techniques can be used to backup philosophical intuition with a useful, modular language, suitable for a practical foundation. 
  
  As both a test-bench for our theory, and a literary \emph{divertissement}, we propose a possible category-theoretic solution of the famous Tl\"on's ``nine copper coins'' paradox, and of other seemingly paradoxical construction in Jorge Luis Borges' literary work. 
  
  We conclude with some vistas on the most promising applications of our future work.
\end{abstract}