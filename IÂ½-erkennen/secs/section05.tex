\section{The tension between observational and theoretical}
\label{sec:orge11c3c4}
When working with categorified relations, it is unnatural and somewhat restrictive to take into account a two-element set for the possible values a proposition(al function) `$(a,b)\in R$' can assume; instead we would like to consider an entire \emph{space} of such values, or rather a type of proofs that $(a,b)\in R$ is true. Again, this idea is best appreciated when thinking that the same proposition 
\begin{center}
	\begin{minted}{haskell}
	(n : Nat) -> (m : Nat) ->  n + m = m + n 
	\end{minted}
\end{center}
when encoded in any (sufficiently strongly-typed) DSL, can be interpreted as either the \emph{proposition} `given $n$ and $m$ natural numbers, their sum is a commutative operation' or as the \emph{type} \mil{n + m} $\equiv$ \mil{m + n} whose elements are the proofs that $n+m$ is in fact equal to $m+n$.

This intuition is based on the well-known proportion
\begin{center}
	truth values : proposition = section : presheaf
\end{center}
inspired by the `proposition as types' paradigm. In simple terms, categorifying a proposition $P : X\to \{0,1\}$ that can or cannot hold for an element $x$ of a set $X$, we shall marry the constructive church and say that there is an entire \emph{type} $PC$, image of an object $C\in\clC$ under a functor $P : \clC \to \Set$, whose \emph{terms} are the \emph{proofs} that $PC$ holds true. This is nothing but the propositions-as-types philosophy, in (not so much) disguise: \cite{hottbook,wadler,martin1984intuitionistic}

The important point for us is that the dialectical tension between observational and theoretical can be faithfully represented through profunctor theory; one can think of propositional functions as relations $(x,y)\in R$ iff the pair $x,y$ renders $\phi$ true; we use this idea, suitably adapted to our purpose and categorified. This very natural extension of propositional calculus, pushed to its limit, yields the following reformulation of the `tension between observational and theoretical'.% of \cite{u,v,w}
\begin{definition}\label{11_ramsey}
	Let $\clT,\clO$ be two small categories, dubbed respectively the \emph{theoretical} and the \emph{observational} settings. A \emph{$(1,1)$-ary Ramsey map} is merely a profunctor
	\[\fkK : \clT^\op \pto \clO\]
	or, spelled out completely, a functor $\fkK : \clT\times \clO \to \Set$.
\end{definition}
By accepting the prefix ``$(1,1)$-'' we ask the reader the leap of faith that \autoref{11_ramsey} is but a particular instance of a more general definition, \autoref{mn_ramsey}. 

Before delving into this generalisation in \autoref{multiramsey}, however, a more elementary remark is in order: there is a particularly elementary recipe to generate $(1,1)$-ary Ramsey maps.
\begin{example}
	Every functor $F : \clA \to \clB$ gives rise to a profunctor $F_* := \clB(1,F) : \clB^\op\times \clA\to\Set$ and a profunctor $F^* := \clB(F,1) : \clA^\op\times\clB \to \Set$ as in \autoref{nervereal}; the two functors are mutually adjoint, $F^*\dashv F_*$, see \cite[6.2]{Bor2}. This yield an example of what we call \emph{representable} Ramsey maps.
\end{example}
\begin{definition}[Observational and theoretical nucleus]\label{nuclei}
	Let $\fkR : \clT^\op\times \clO \to\Set$ be a Ramsey map, and $\hat R : \clO \to [\clT^\op,\Set]$ the associated canvas. Let
	\[ \label{lalan} \Lan_{\yon_\clO}\hat R : [\clO^\op,\Set] \leftrightarrows [\clT^\op,\Set] : N_{\hat R} \]
	be the adjunction between presheaf categories determined by virtue of \autoref{equ_prof_cocont}. Let us consider the equivalence of categories between the fixpoints of the monad $T = N_{\hat R}\circ\Lan_{\yon_\clO}\hat R$ and the comonad $S=\Lan_{\yon_\clO}\hat R\circ N_{\hat R}$.
	
	This is the equivalence between the \emph{observational nucleus} $Fix(T)\subseteq [\clO^\op,\Set]$ and the \emph{theoretical nucleus} $Fix(S)\subseteq [\clT^\op,\Set]$.
\end{definition}
\begin{remark}
	Observational nucleus and theoretical nucleus always form equivalent categories; the tension in creating a satisfying image of reality as it is observed oscillates between the desire to enlarge as much as possible the subcategory of $[\clO^\op,\Set]$ with which our theoretical model is equivalent, where we can have access to $\clT, [\clT^\op, \Set]$ only.
\end{remark}
{\color{blue} Tutta sta roba va rivista}

\color{red}
The following remark shows how new structure comes `almost for free' when things are interpreted this way. 

Assume $\phi : \clT \to \clW$ and $\psi : \clO \to \clW$ are canvases, $\fkR$ is a Ramsey map, and $\Lan_{y_{\clO}}\hat R$ the functor corresponding to $\fkR$ under the construction in \eqref{lalan}; in this notation, we can state a tight condition of compatibility between the theory identified by $(\phi,\psi)$ and te Ramsey map $\fkR$. We employ freely the presence of adjunction 
\[L_\phi\dashv N_\phi \qquad 
L_\psi \dashv N_\psi \qquad 
L_{\hat R} \dashv N_{\hat R}.\]
\begin{remark}[Inducing an hermeneutics]\label{inducing_herme}
Consider the diagram
\[ \vcenter{\xymatrix{
	& \clW \ar[dl]_{N_\phi} \ar[dr]^{N_\psi} & \\ 
	[\clT^\op,\Set] \ar@<-.5em>[rr]_{N_{\hat R}} && \ar@<-.5em>[ll]_{L_{\hat R}} [\clO^\op,\Set]
}}	 \label{discuss_0}\]
given by the theoretical and observational nerves, plus the Ramsey adjunction mentioned above.

We seek sufficient conditions in order for \eqref{discuss_0} to be filled by a suitable natural transformation $N_{\hat \fkR} \circ N_\phi \To N_\psi$: it is well-known from basic proper of the mating operation, and standard results on adjoint functors, that this is a way to fill at the same time the triangles on the left and on the right here: 
\[oinoguao\]
\end{remark}
The possibility to find a well-behaved 2-cell $N_{\hat \fkR} \circ N_\phi \To N_\psi$ amounts to a tameness property of the system: this is made precise by the following
\begin{definition}[Fundamental cell, Hermeneutics]
	In a display of categories like \eqref{2_canvases} we say that 
	\begin{itemize}
		\item A \emph{fundamental cell} is a natural transformation $R_*\circ N_\psi \To N_\phi$;
		\item we say that in the world $\clW$ \emph{hermeneutics is possible} if the right extension $\bk{\phi,\psi} := \Ran_{ N_\phi}N_\psi$ exists \emph{as a functor} (note that it always exists as a profunctor, but this might not be representable).
	\end{itemize}
	If hermeneutics is possible in $\clW$, and $R : \clO \pto \clT$ is a Ramsey map, any fundamental cell induces a natural transformation 
	\[ \varpi : N_{\hat R} \To \bk{\phi/\psi} \]
	obtained exploiting the universal property of $\bk{\phi/\psi}$. 
\end{definition}
If the right extension is representable in the sense above, this amounts to a higher type map (in the sense of the internal language of a closed category) comparing `generalised formulas' of kind $\fkR \To \clW(\phi,\psi)$.
\color{black}
\begin{remark}
	The reader might have observed, now, that there is nothing in their mere syntactical presentation allowing to tell apart the observational and the theoretical category; this can be justified with the fact that the bicategory $\Prof$ of \autoref{def_profu} is endowed with a canonical self-involution, exchanging the r\^ole of domain and codomain of 1-cells, and thus of the theoretical and observational category $\clT,\clO$.
	
	This is perhaps of some help in solving the conundrum posed by the existence of `fictional objects'. Sherlock Holmes clearly is the object of a theoretical category. Gandhi is the object of an observational category. But as linguistic objects they can't be told apart completely; they can be at most separated by a profunctor embedding the former in a realistic counterpart of fictitious model (that is, for example, the Reichenbach falls), and representing the latter as part of a fictional model (for example, as part of a movie directed by R. Attenborough).

	Si può indagare lo statuto ontologico di questi oggetti (magari nel modello fornito in \cite{catont1}). Ma è chiaro che, ad esempio, nel mondo 'sherlockiano' $\clW^{Sh}$ Sherlock debba esistere, per qualche nozione di esistenza interna al mondo, altrimenti non potrebbe morire nelle Reichenbach falls; esse sono anche un oggetto del nostro mondo $\clW^@$ e quindi è utile considerare le aggiunzioni tra le due categorie sintattiche equivalentemente a come il problema è studiato in termini di accessibilità tra mondi nella semantica modale dei non-esistenti \cite{} \footnote{Nel secondo esempio la questione è più sottile perché riguarda la nozione di rappresentazione: il Gandhi di Attenborough non è propriamente un oggetto di $\clW^@$ ma quello di un sub-mondo che si vuole isomorfo a esso. Anche qui esistono approcci collaudati in letteratura, in termini di modal relations}. L'intero dibattito è quindi profunctionalizzabile: dopotutto i mondi finzionali o narrativi non sono altro che teorie su strutture di mondi. 
\end{remark}
\begin{remark}\label{multiramsey}
	The notion of Ramsey map as given above is unnecessarily restrictive, and does not account for many sorts of configurations that can occur in practice:
	\begin{itemize}
		\item a single observational token $O$ can't be described by a single theoretical token $T_1$, but instead needs $T_1,\dots,T_r$;
		\item inverting the r\^oles, a single theoretical token describes not only $O$, but different $O_1,\dots,O_s$.
	\end{itemize}
	Thus we must admit \emph{multiple} arguments for the domain and codomain of a Ramsey map. This yields the notion of a \emph{$(n,m)$-ary Ramsey map}.
\end{remark}
\begin{remark}\label{resoudre_la_tension}
	The clearest possible sense in which the profunctorial approach `resolves' the tension between observational and theoretical is that the Gro\-then\-dieck construction associated to a profunctor $\fkR : \clT \pto \clO$ yields a category where the two `worlds', one carved from perception, and the other concocted from language, live harmoniously together. All in all, said tension is just an incarnation of the tension between speakable and unspeakable: given a Ramsey map $\fkR : \clT \pto \clO$, the equivalence between its theoretical and observational nuclei is an equivalence between the speakable (subclass of $[\clT^\op,\Set]$), with the observable (subclass of $[\clO^\op,\Set]$); what lies outside this equivalence in the latter category is observable but `unspeakable' in the strongest possible sense.
\end{remark}
