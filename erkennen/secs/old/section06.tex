\section{Ramseyfication and beyond: generalised profunctors}
\label{sec:org50db6c2}
We can generalise the definition above to encompass Ramsey sentences:
\begin{definition}\label{mn_ramsey}
	Let $\clT,\clO$ be two categories; a \emph{Ramsey map}, or a \emph{$(n,m)$-ary Ramsey map} is a profunctor $\fkK : \clT^n \pto \clO^m$; note that we allow $n,m$ to be zero; in that case, $\clA^0$ is understood to be the terminal category $\boldsymbol{1}$.
\end{definition}
The intuition behind this definition is as follows: given $\uT\in\clT^n, \uO\in\clO^m$, the set $\fkK(\uT, \uO)$ represents the type of proofs that the observational tuple $\uO$ admits a description in terms of the theoretical tuple $\uT$.

This formalism allows to speak about particular worlds, obtained as presheaf categories over observational $\clO$; if $\clT, \clO$ is a theoretical pair, we can instantiate \autoref{nervereal} above in the particular case where $\clW = [\clO^\op, \Set]$ (observe that in this case $\clW$ is a Yuggoth). We can thus address a certain number of questions, arising from the canonical adjunction obtained by virtue of \autoref{equ_prof_cocont}:% and \autoref{}:
\[
	\xymatrix{ [(\clO^m)^\op, \Set] \ar@<3pt>[r] & \ar@<3pt>[l] [(\clT^n)^\op, \Set];}
\]
It is worth to mention that since the diagram
\[
	\vcenter{\xymatrix{
			(\clO^m)^\op \ar[rr]\ar[dr]&& [(\clT^n)^\op, \Set] \ar[dl]\\
			& [(\clO^m)^\op, \Set]
		}}
\]
is pseudocommutative, the composition $L\circ y$ s equal to (the mate of) $\fkK$. This means: $\clO$-models, when interpreted inside $\clT$-models, carry representations corrisponding to the observational tokens interpreted in $\clT$-models; that is, the representation is coherent over observational tokens, that is\dots

% \fo{Vanno messi a posto tutti i domini qua}
% \begin{remark}[Ramseyfication and translation functors]\label{carnap_translation_functors}
% 	Assume that there exists an adjunction 
% 	\[ 
% 		L : \clT \leftrightarrows \clO : F
% 	\]
% 	between the theoretical and the observable. Following Carnap, we might assume that $F : \clO \hookrightarrow \clT$, and thus $G$ is a right translation functor for $(\clT, \clO)$.

% 	In these assumptions, given a Ramsey map $\fkK : \clT \pto \clO$ the function term
% 	\[\lambda \uO\lambda\uX . \fkK(\uO, \uX)\]
% 	can be pre-composed with $F$ obtaining
% 	\[\lambda \uO\lambda\uO'.\fkK(\uO, F\uO').\]
% 	We say that a translation adjunction $(L,F)$ is `$\fkK$-admissible' (denoted $L \dashv_\fkK F$) when there is a natural isomorphism $\fkK(L,1)\cong\fkK(1,F)$.
% \end{remark}
% The property of $\fkK$-admissibility for a pair of functors is in general difficult to assess; nevertheless, there are interesting properties for the relation $F\dashv_\fkK G$: for example
% \begin{theorem}
% 	Let $F : \clA \leftrightarrows \clB: G$ be a pair of functors in opposite directions; let $\fkK : \clB \pto \clA$ be a profunctor; if $F\dashv_\fkK G$, then there is a `genuine' adjunction 
% 	\[ F^e : \clA\uplus_\fkK\clB \leftrightarrows \clA\uplus_\fkK\clB : G^e \]
% 	`extended' to the category of elements of $\fkK$.
% \end{theorem}