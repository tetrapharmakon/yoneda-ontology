\section{Introduction}\label{sec_intro}
\epigraph{El mundo, desgraciadamente, es real.\\[2mm]
\footnotesize\emph{---Disgracefully, the world is real.}
}{\cite{confutacion}}
The purpose of this work is to adopt a wide-ranging approach to a fragment of elementary problems in a certain branch of contemporary philosophy of Mathematics. More in detail, we attempt at laying a foundation for solving a number of problems in ontology employing pure Mathematics; in particular, using the branch of Mathematics known as \emph{category theory}.

As authors, we are aware that such an ambitious statement of purpose must be adequately motivated, bounded to a realistic goal, and properly framed in the current state of the art on the matter. This is the scope of the initial section of the present first manuscript.
\subsection{What is this series}
Since forever, Mathematics studies three fundamental indefinite terms: \emph{form}, \emph{measure}, and \emph{inference}. Apperception lets us recognise that there are extended entities in space, persisting in time. From this, the necessity to measure how much these entities are extended, and to build a web of conceptual relations between them, explaining how they arrange `logically'.
Contamination between these three archetypal processes is certainly possible, common, and desirable.

We can even say more: Mathematics is a language engineered to systematically infer properties of the three mentioned indefinite; so, meta\hyp{}Mathematics done through Mathematics (if such a thing even exists) exhibits the features of a \emph{ur-language}, a generative scheme for `all' possible languages. It is a language whose elements are the rules to give oneself a language, convey information to other selves, and allow deduction. It is a meta-object, to generate objects/languages.

Taken this tentative definition, Mathematics (not its history, not its philosophy, but its \emph{practice}) shall serve as a powerful tool to tackle the essential questions of philosophy, and even more, of ontology: what things are, what cogently makes them what they are and not different.

Yet, it is undeniable that a certain philosophical debate (even when `formal') is foreign to mathematical language. A tear in the veil that occurred a long time ago, due to different purposes and different specific vocabulary, can not be repaired by two people only. If, however, the reader of these notes asks for an extended motivation for our work, a wide-ranging project in which it fits, a long-term goal, in short, a \emph{program}, they will find it now: there is a piece of Mathematics whose purpose is to solve philosophical problems, in the same sense certain Mathematics `solves' the motion of celestial bodies. It does not annihilate the question: it proposes models within which one can reformulate it; it highlights what is a trivial consequence of the axioms of that model and what instead is not, and requires the language to be expanded, modified, sharpened. 

We aim to approach this never-mentioned discipline as mathematicians. Without elaborating `new' theorems, we nevertheless draw connections with the modern mathematical practice to use it in the context of philosophical research.

Sure, solving once and for all the problems posited by ontology sounds like an ambitious objective. 

More modestly, we propose a starting point unhinging some well\hyp{}established beliefs. Above all else, the belief that ontology is too general to be approached quantitatively, and that it contains mathematical language as a proper subclass: it is instead the exact opposite, as the central idea of our work is that ontologies -there are many- are mathematical objects. we humbly point the finger at some problems that prose is unable to notice, because it lacks a specific and technical language; we suggest that only in such a language, when words mean precise things and are tools of epistemic research, a few essential questions of recent ontology dissolve, and others simply become `the wrong question': not false, just meaningless.

It may also seem suspicious to employ Mathematics to tackle questions that traditionally pertain to philosophy. By proposing the `ontology as categories' point of view, our work aims to dismantle such a false belief.%: each and every debate must appear in a language.

In doing so, we believe we can provide a more adequate language, taken from Mathematics, within which to frame some deep ontological questions. The reader will allow a tongue-in-cheek here, subsuming our position: in ontology, it is not a matter of making a \emph{correct use of language}, but rather a matter of \emph{using the correct language}.

 This language `must be' \emph{category theory}, as only category theory has the power to speak about a totality of structures of a given kind in a compelling way, treating mathematical \emph{theories} as mathematical \emph{objects}.

As of now, our work unravels in three different chapters, and it will attempt to cover a variety of topics:
\begin{itemize}
	\item the present manuscript, \emph{Existence}, provides the tools to build a sufficiently expressive `theory of existence' inside a category. This first chapter has a distinctly foundational r\^ole; its scope to build the fundamentals of our tool-set (category theory and categorical logic, as developed in \cite{mac1992sheaves,JohnstonePT,lambek1988introduction}).

	      As both a test-bench for our theory and a literary \emph{divertissement}, we propose a category-theoretic solution of Borges' paradoxes present in \cite{Borges1963}. In our final section, we relate our framework to more classical ancient and modern philosophers; we link topos theory to Berkeley's instantaneism and internal category theory to Quine's definition of the [domain of] existence of an entity as a domain of validity of quantifiers (intended as propositional functions, i.e. functions whose codomain is a space of truth values).
	\item A second chapter \cite{black}, currently in preparation, addresses the problem of \emph{identity}, and in particular its context-dependent nature. Our proof of concept here consists of a rephrasing of Black's classical `two spheres' paradox \cite{papear_di_black} in the elementary terms of invariance under a group of admissible transformations; this time the solution is provided by Klein's famous \emph{Erlangen program} group-theoretic foundation for geometry: the two interlocutors of Black's imaginary dialogue respectively live in an Euclidean and an affine world: this difference, not perceived by means of language, affects their understanding of the `two' spheres, and irredeemably prevents them from mutual intelligence.
	\item A third chapter \cite{homot}, currently in preparation, addresses again the problem of identity, but this time through the lens of algebraic topology, a branch of Mathematics that in recent years defied well\hyp{}established ontological assumptions; the many commonalities between category theory and homotopy theory suggest that `identity' is not a primitive concept, but instead depends on our concrete representation of mathematical entities. This can be formalised in various ways, among many the \emph{Homotopy Type Theory} foundation of \cite{hottbook,cwp}. 
\end{itemize}
Our main tenet in the present chapter is that ontolog\emph{ies} are mathematical objects: each ontology is a certain category $\clO$, inside which `Exist' unravels as the sum of all statements that the internal language of $\clO$ can concoct.

Of course, the more expressive is this language, the more expressive the resulting theory of existence will turn out to be. Our presupposition here is that trying to let ontology speak about `\emph{all} that there is' (the accent is on the adverb, on this famous quote of Quine \cite{quine1948there}) can lead to annoying paradoxes and ambiguities. 

Instead research shall concentrate on clarifying what the verb means: in what sense, `what there is' \emph{is}? \emph{What is is-ness?} As category theorists, our -perhaps simplistic- answer is that, again paraphrasing Quine,
\begin{quote}
	being is \emph{being the object of a category}.
\end{quote}
Explaining why this is exactly Quine's motto, just shifted one universe higher, is the content of our §\ref{metaon}.
\subsubsection{Structure of the paper}
The remaining part of the first section draws a picture as accurate as possible, of the wheres and whys of structural Mathematics; its implications are the subject of several essays on the philosophy of Mathematics, like \cite{kromer2007tool,Marquis1997,marquis2010category,marquis2008geometrical}. This section has several different purposes: it provides an explicit statement of purposes for the entire polyptych \cite{black,homot}; it declares our stance on the foundation we choose, clarifying assumptions that we feel are usually neglected on essays on the topic (`where are the objects that Mathematics aims to describe? Where is the language by means of which this description is possible?') -of course without claiming to have solved the matter once and for all; we provide pointers as specific as possible in order to help the reader navigate the relevant literature.

The second section delves into the first major point of our presentation: large categories are universes where `Existence', intended as the sum of information acquired from the perceptual bundle we experience, that language organises and conceptualises. `Language' here is a shortcut to denote the power of a fixed large category $\clC$ to express well-formed formulas of (a certain fragment of) logic. Objects and morphisms of $\clC$ shall be considered respectively as types and terms of a language, \emph{the} internal language of $\clC$; now, the richer $\clC$ is, the more it is able to faithfully represent the cosmos we're thrown into. Among many possible choices for $\clC$, we take \emph{toposes} as the class of categories harbouring `set theories': the internal language of a topos is powerful enough to re-enact set theory, and subsequently propositional logic.

The third section deals with a specific example of a topos, useful for later examination: the categories of objects parameterised by a fixed `space of parameters' $I$. The `slice' category $\Set/I$ is a topos, and its internal language, namely its internal logic, is tightly linked to set-theoretic properties of the slicing set $I$. The logic we obtain in $\Set/I$ by casting the general definitions of subobject classifier (and internal language) is genuinely non-classical.

The fourth section contains a careful analysis of the internal language of $\Set/I$.

The fifth section contains an application of the tools we exposed until now: the seemingly paradoxical `nine copper coins' problem exposed in Jorge Luis Borges' \cite{Borges1963}, far from being paradoxical, admits a natural interpretation as a statement in $\Set/I$, for a suitable choice of $I$, and thus of the induced internal logic. We propose other examples of seemingly paradoxical statements in Borges' literary work that instead are admissible statements in the internal logic of \emph{some} topos: on Tl\"on entities may disappear if neglected: this means that $I$ is linearly ordered; Babylon's chaotic lottery resembles, with their obscure, impenetrable purposes, the chaotic behaviour of a dynamical system: this means that $I$ carries a semigroup action; Tl\"on's instantaneism, mimicking/mocking Berkeley, can be obtained assuming $I$ is a discrete, uncountable set.

We close the paper with a section on future development, vistas for future applications, and with a wrap-up of the discussion as we have unraveled so far. Ideally, §\ref{sec_prelim} and §\ref{int_lang} shall be skipped by readers already having some acquaintance with category theory; the second half of the paper makes however heavy use of the notation established before.
\subsection{On the choice of a meta-theory and a foundation} \label{meta_theory}
Along the 20$^\text{th}$ century, the discipline of Mathematics divided into different sub-classes, each with their specific problems and its specific language, just to find, soon after, unification under a single notion of \emph{structure}, through the notion of abstract category \cite{gtone}. This process led to an epistemological revision of Mathematics and has inspired, parallel to the development of operative tools, a revision of both the foundations of Mathematics and the purposes of its research.

According to many, it is undeniable that
\begin{quote}
	[the] mathematical uses of the tool `category theory' and epistemological considerations having category theory as their object cannot be separated, neither historically nor philosophically. \cite{kromer2007tool}
\end{quote}
Structural-mathematical practice, i.e. the practice of everyday Mathematics directed by structural meta-principles, produced a `natural' choice for the underlying meta-ontology of Mathematics
%\footnote{Perhaps improperly, the locution \emph{meta-ontology of Mathematics} is used here to refer to the totality of operative beliefs inspiring the ergonomic of mathematical objects. Some of these principles are: objects not enjoying a universal property shall be discarded; definitions that are isomorphism-invariant shall be preferred over those who are not; both these commandments are based on the idea that classes of mathematical objects arrange in coherent conglomerates exhibiting more structure than the mere aggregation of their elements: the requests of universality and isomorphism-invariance are meant not to destroy such additional structure. Examples of these meta-principles can be found in various other areas of Mathematics.}
 which, later, felt the need to be characterized more precisely. Similarly to what Carnap\footnote{Some words that philosophers should keep in mind, on the lawfulness of the use of abstract entities (specifically mathematical) in semantic reflection, also valid in ontology:
	\begin{quote}
		we take the position that the introduction of the new ways of speaking does not need any theoretical justification because it does not imply any assertion of reality [...].  it is a practical, not a theoretical question; it is the question of whether or not to accept the new linguistic forms. The acceptance cannot be judged as being either true or false because it is not an assertion. It can only be judged as being more or less expedient, fruitful, conducive to the aim for which the language is intended. Judgments of this kind supply the motivation for the decision of accepting or rejecting the kind of entities. \hfill \cite{carnap1956meaning}
	\end{quote}} suggested regarding semantics,
\begin{quote}
	mathematicians creating their discipline were not seeking to justify the constitution of the objects studied by making assumptions as to their ontology.\hfill  \cite{kromer2007tool}
\end{quote}
Beyond the attempts (above all, those of Bourbaki group: but see \cite{McL}, historical notes on Ch. 4, for a hint that Bourbaki didn't really get the point of structural Mathematics), what matters is that the habit of reasoning in terms of structures has suggested implicit epistemological and ontological attitudes. This matter would deserve an exhausting independent inquiry.

For our objectives it's enough to declare a differentiation that Kr\"omer
elaborated, inspired by \cite{Cor96}: the difference between \emph{structuralism} and \emph{structural Mathematics}:
\begin{enumtag}{s}
	\item \label{s:uno} Structuralism: the philosophical position regarding structures as the subject matter of Mathematics;
	\item \label{s:due} Structural Mathematics: the methodological approach to look in a given problem for the structure itself.
\end{enumtag}
Of course, \ref{s:uno} implies \ref{s:due} but the opposite is not always true:
\begin{remark} \label{weak_structuralism}
	That is, one can do structural Mathematics without being a structuralist and taking different, or even opposite, positions concerning structuralism itself.

	Nevertheless, the use of \CT as meta-language, despite the historical link with structuralism, doesn't make automatic the transition from \ref{s:due} to \ref{s:uno}; it just suggests that the ontology is not only dependent on the `ideology' (in a Quinean sense) of the theory, but it is instead influenced by the epistemological model inspired by formal language.
\end{remark}
Kr\"omer's distinction, however, has another virtue: instead of stumbling in a possibly not ambiguous definition of \textit{structure} (with the unwanted consequences that could arise in the operational practice), \ref{s:uno} can be reduced to (or can redefine) \ref{s:due}, saying that:
\begin{quote}
	\emph{structuralism is the claim that Mathematics
		is essentially structural Mathematics} \cite{kromer2007tool}
\end{quote}
This is the same thing as saying: the structural practice already is its philosophy.

Attempts to explain the term `structure' by Bourbaki in the years following the publication of the \textit{Elements des Mathématiques}, led to the first systematic elaboration of a philosophy that we could appropriately call \textit{structural Mathematics}. Its target is to `\textit{assembling all possible ways in which given set can be endowed with certain structure}' \cite{kromer2007tool}, and elaborate, in the programmatic paper \textit{The Architecture of Mathematics} (written by Dieudonné alone and published in 1950), a formal strategy. While specifying that `\textit{this definition is not sufficiently general for the needs of Mathematics}' \cite{Bourb50}, the author encoded a series of operational steps through which a structure on a collection is assembled set-theoretically. Adopting therefore a reductionist perspective in which
\begin{quote}
	the structure-less sets are the raw material of structure building which in Bourbaki’s analysis is `unearthed' in a quasi\hyp{}archaeological, reverse manner; they are the most general objects which can, in a rewriting from scratch of Mathematics, successively be endowed with ever more special and richer structures.\hfill  \cite{kromer2007tool}
\end{quote}
On balance, in Bourbaki's structuralism, the notion of set doesn't disappear definitively in front of the notion of structure. Times were not ripe to abandon set theory; the path towards an `integral' structuralism was still long, and culminated years after, with Lawvere's attempt at a foundation \theory{ETCS} of set theory first \cite{lawvere1964elementary} and \theory{ETCC} of category theory (and as a consequence, `of all Mathematics') after \cite{lajolla}, through structuralism.

To appreciate the depth and breadth of such an impressive piece of work, however, the word `foundation' must be taken in the particular sense intended by mathematicians:
\begin{quote}
	[\dots\unkern] a single system of first-order axioms in which all usual mathematical objects can be defined and all their usual properties proved.
\end{quote}
Such a position sounds at the same time a bit cryptic to unravel, and unsatisfactory; Lawvere's (and others') stance on the matter is that a foundation of Mathematics is \emph{de facto} just a set $\clL$ of first-order axioms organised in a Gentzen-like deductive system. The deductive system so generated reproduces Mathematics as we know and practice it, providing a formalisation for something that already exists and needs no further explanation, and that we call `Mathematics'.

It is not a vacuous truth that $\clL$ exists somewhere: point is, the fact that the theory so determined has a nontrivial model, i.e the fact that it can be interpreted inside a given familiar structure, is at the same time the key assumption we make, and the less relevant aspect of the construction itself.

Showing that $\clL$ `has a model' is --although slightly improperly-- meant to ensure that, \emph{assuming the existence of a naive set theory} (i.e., assuming the prior existence of structures called `sets'), axioms of $\clL$ can be satisfied by a naive set. Alternatively, and more crudely: assuming the existence of a model of \theory{ZFC}, $\clL$ has a model \emph{inside that model of \theory{ZFC}}.\footnote{It shall be made clear, ensuring that a given theory has a model isn't driven by psychological purposes only: on the one hand, purely syntactic Mathematics would be very difficult to parse, as opposed to the more colloquial practice of mathematical development; on the other hand (and this is more important), the only things syntax can see are equality and truth. To prove that a given statement is false, one either has to check all possible syntactic derivations leading to $\varphi$, finding none --this is unpractical, to say the least-- or to \emph{find a model} where $\lnot\varphi$ holds.}

\subsection{Our foundation, at last.} A series of works attempting to unhinge some aspects of ontology through category theory should at least try to tackle such a seemingly simple question as `where' are the symbols forming the first-order theory \theory{ETCC}. And yet, everyone just believes in -some flavour of- sets and solves the issue of `where' they are with a leap of faith from which all else must follow.



The usual choice for mathematicians imbued by syntacticism is to assume that, wherever and whatever they are, symbols `are', and our r\^ole in unveiling Mathematics is \emph{descriptive} rather than generative.\footnote{Inside (say) a constructivist foundation it is not legitimate to posit that axioms `create' mathematical objects; from this, the legitimacy of the question of where they are, and the equally legitimate answer `nowhere'. The only thing we can say is that they `make precise, albeit implicitly, the \emph{meaning} of mathematical objects' \cite{Agzz} (it seems to us that in Mathematics as well as in philosophy of language, meaning and denotation are safely kept separate). We take this principle -that the world/meta\hyp{}model exists and we can just attempt at describing it by means of the language/model- and we leverage it without further question.}

This state of affairs has, to the best of our moderate knowledge on the subject, various possible explanations:
\begin{itemize}
	\item On one hand, it constitutes the heritage of Bourbaki's authoritarian stance on formalism in pure Mathematics;
	\item on the other hand, a different position would result in barely any difference for the `working-class'; mathematicians are irreducible pragmatists, somewhat blind to the consequences of their philosophical stances.
\end{itemize}
So, symbols and letters do not exist outside of the Gentzen-like deductive system we specified together with $\clL$.

As arid as it may seem, this perspective proved itself to be quite useful in working Mathematics; consider for example the type declaration rules of a typed functional programming language: such a concise declaration as
\begin{minted}{haskell}
  data Nat = Z | S Nat
\end{minted}
makes no assumption on `what' \mintinline{haskell}{Z} and \mintinline{haskell}{S :: Nat -> Nat} are; instead, it treats these constructors as meaningful formally (in terms of the admissible derivations a well-formed expression is subject to) and intuitively (in terms of the fact that they model natural numbers: every data structure that has those two constructors \emph{must} be the type $\bbN$ of natural numbers -provided data constructors like \verb|S| are all injective).

Taken as an operative rule, this reveals exactly what is our stance towards foundations: we are `structuralist in the meta-theory', meaning that we treat the symbols of a first-order theory or the constructors of a type system regardless of their origin, provided the same relation occur between criptomorphic collections of labelled atoms.

In this precise sense, we are thus structuralists in the meta-theory, and yet we do so with a grain of salt, maintaining a transparent approach to the consequences and limits of this partialisation. On the one hand, pragmatism works; it generates rules of evaluation for the truth of sentences. On the other hand, this sounds like a Munchhausen-like explanation of its the value, in terms of itself. Yet there seems to be no way to do better: answering the initial question `where are the letters of \theory{ETCC}?' would result in no less than a foundation of language.

And this for no other reason than `our' meta-theory is something near to a structuralist theory of language; thus, a foundation for such a meta-theory shall inhabit a meta-meta-theory\dots{} and so on.

Thus, rather than trying to revert this state of affairs we silently comply to it as everyone else does; but we feel contempt after a brief and honest declaration of intents towards where our meta-theory lives. Such a meta-theory hinges again on work of Lawvere, and especially on his series of works on functorial semantics.
