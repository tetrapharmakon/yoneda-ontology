% La ``traduzione'' dei problemi dell'ontologia nel linguaggio di \CT permette di manipolare meglio nozioni (non solo, come si sa, matematiche) ma metamatematiche e metafisiche, e ci dota di un approccio più compatto e di una visione più ``leggera'' e occamista delle questioni vertenti su oggetti e esistenza. Non giustifichiamo questo approccio a priori ma ne testimoniamo la fecondità già provata in letteratura\footnote{Cf. Mt7,16, giusto per ingraziarsi i severiniani.}, soprattutto paragonata a quella degli approcci set-theoretic (di cui già è informata la totalità delle ontologie formali).
% Using the most advanced tools of mathematical investigations, we aim at 
% In \CT possiamo ``tradurre'' i problemi classici dell'ontologia, fornire
% un \emph{ambiente} ben definito nel quale questioni ritenute oggetto di dibattito filosofico possano illuminarsi in modi nuovi o scomparire. E questo non per qualche perverso istinto riduzionistico, ma per poterne parlare in termini efficaci e nel linguaggio adatto a inquadrarli: tentare, con gli strumenti più avanzati e raffinati dell'astrazione matematica, di rispondere a delle domande, produrre conoscenza, e non solo dibattito; inscrivere antiche o recenti questioni in un nuovo paradigma, volto a superare e al contempo far avanzare la ricerca.

% Come ogni paradigma lo dotiamo di una sintassi con la quale ``nominare'' concetti e dare definizioni, e di una semantica che produca modelli, e quindi contesti, entro i quali ``guardare'' le teorie; questa sintassi e questa semantica non ce le inventiamo: sono già nella matematica e da lì le preleviamo.
% \subsection{Towards an Operative Ontology}


% Riassumiamo i vantaggi del fare ontologia usando la teoria delle categorie:
% \begin{itemize}
% 	\item Prima di tutto in questo modo ontologia, ci si permetta la battuta, la si \emph{fa} effettivamente. Vale a dire, come mostreremo nel resto del lavoro e in altri successivi, si affrontano di petto le questioni e le si risolvono. Stiamo perciò suggerendo un approccio \emph{problem solving}
% 	\item Lo strutturalismo ``debole'' implicito in questa visione è da noi mantenuto solo a livello metateorico, e consente di guardare alle relazioni tra oggetti all'interno delle teorie. Anche qui, l'approccio relazionale è assunto in quanto proficuo in senso pratico, operativo, e - come vedremo al termine del paragrafo - non implica l'adesione incondizionata a uno strutturalismo ``forte''.
% 	\item Come conseguenza fondamentale del punto precedente nessun concetto viene studiato in senso ``assoluto'' (qualunque cosa ciò significhi) ma relativamente al contesto in cui opera, e alla teoria che stiamo adottando per definirlo.
% \end{itemize}

% Come si è detto in 1.2 l'uso di questi strumenti concettuali ha aiutato la pratica matematica e ha involontariamente ispirato una visione epistemologica, e poi ontologica, della disciplina, vale a dire dei suoi oggetti di studio. A coloro che obiettano che bisogna prima sapere \textit{cos'è} una struttura prima di lavorare con essa noi rispondiamo, con Kr\"omer, che
% \begin{quote}
% 	this reproach is empty and one tries to explain the clearer by the more obscure when giving priority to ontology in such situations [...]. Structure occurs in the dealing with something and does
% 	not exist independently of this dealing. \cite{kromer2007tool}
% \end{quote}
% Memori delle osservazioni di Carnap [nota \dots], non riteniamo che questo approccio ``operativo'' all'ontologia (che non è puro \textit{problem solving} ma anche chiarificazione concettuale) implichi necessariamente l'adesione incondizionata ad uno strutturalismo filosofico integrale - o a sue varianti specifiche come la teoria \textbf{ROS} -, esattamente come abbiamo visto non avvenire nel passaggio dalla \textit{structural mathematics} allo strutturalismo vero e proprio (o al bourbakismo). La sua importanza è principalmente metodologica. (\textit{Au contraire} risulta necessario per chi appoggia posizioni strutturaliste al di fuori della matematica cominciare a fare ontologia in termini categoriali, nelle modalità qui indicate).
% \end{italian}
%  \subsection{Existence: Persistence of Identity?} 
%  Ontology rests upon the principle of identity: it is this very principle that our category-theoretic approach aims to unhinge. 
%  \begin{italian}
%  E tuttavia formalizzare il concetto intuitivo di identità si rivela una questione estremamente spinosa: cosa significa che \emph{due cose sono, invece, una} è un problema che ci arrovella fin da quando otteniamo la ragione e la parola; ciò perché il problema è tanto elementare quanto sfuggente: l'unica maniera in cui possiamo esibire ragionamento certo è il calcolo; del resto, se la sintassi non vede che l'uguaglianza in senso più stretto possibile, la prassi deve diventare in fretta capace di una maggiore elasticità: per un istante ho postulato che ci fossero ``due'' cose, non una. E non è forse questo a renderle due? E questa terza cosa che le distingue, è davvero diversa da entrambe?

%  Usciti dalle nebbie delle speculazioni tradizionali, i filosofi a cavallo tra '800 e '900 si sono posti questo complicato problema: due vie sono possibili: la risposta fregeana \cite{} per cui ''$x$ esiste'' se e solo se ''$x$ è identico a qualcosa [banalmente, a sè stesso]'' alla soluzione logica quineana per cui ''essere è essere il valore di una variabile [vincolata]''. 
 
%  Il primo approccio si rivela comodo solo se si è realisti concettuali, ma è scarsamente informativo (cosa è l'uguaglianza tra due oggetti apparentemente diversi? Quel qualcosa che ci ha fatto sospettare lo fossero non è abbastanza a renderli tali?); coinvolge la nozione di identità, ed anzi scarica su questa l'onere di definire esistenza; questa non è la strada giusta: l'identità di fatto non esiste, perché ogni identità è un'identificazione, e ogni uguaglianza una relazione di equivalenza; la questione è tuttavia complessa abbastanza da dedicarvi un lavoro a parte (in effetti, due \cite{,}) di questo polittico.
 
%  Il secondo tipo di approccio ha ispirato l'interesse per la nozione di \emph{ontological committment} (l'insieme di assunti che ``si danno per scontati'' quando si parla di ontologia, o se ne partecipa) e per la conseguente definizione di ontologia (di una teoria) come ''dominio di oggetti su cui variano i quantificatori'' (cf. \cite{}): namely una teoria qualsiasi è impegnata sulle entità su cui variano i quantificatori dei suoi enunciati. 
 	
%  	Vedremo in \autoref{metaon} che la concezione Quineana fitta nella nostra visione di ontologia categoriale come conseguenza di una internalizzazione. 
 	
%  	Esiste una terza via, meno diffusa in letteratura ma decisamente meno opinabile: definire l'esistenza tramite la persistenza nel tempo. Diciamo che ''$x$ esiste'' se e solo se ''$x$ è identico a sè stesso in ogni frame temporale $\la T,<\ra$'', dove $T$ è un insieme non vuoto di istanti e $<$ una relazione binaria in $T$ (e la relazione di esistenza in $T$ è allora una relazione $(x,t)\mapsto x\mathrel{\tilde\in} t$; ``$x$ esiste in $T$'' se per ogni $t : T$ si ha $x\mathrel{\tilde\in} t$).
 	
%  	Come si vede questa definizione cattura una nozione intuitiva di esistenza, impiegando sia l'identità (con tutti i problemi che essa comporta) che il tempo, o meglio una opportuna logica temporale nella quale far "persistere" le entità. (E' facile scrivere cosa significa la relazione ``$x$ esiste in $T$'' in termini di (L)TL)
 	
% 	 Uno dei risultati di questo paper è che possiamo definire l'esistenza in maniera altrettanto intuitiva, senza riferirsi all'identità, né a un frame temporale; in effetti, fornendo un concetto più generale, dentro il quale si troveranno anche gli altri.
	 
% 	 Porremo la questione nel seguente modo: ciò che è variabile relativo ad $x$ è il grado, o \emph{forza} della sua esistenza; l'esistenza ``classica'' è esistenza in massimo grado nel linguaggio interno del topos che fa da Universo (per noi, un universo borgesiano); lì saremo in grado di indicare il ''grado'' di esistenza degli oggetti che lo abitano, senza presupporre di muoverci attraverso istanti di tempo (come potrebbe suggerire l'esempio delle monete) o punti dello spazio (come la freccia).%, oppure in altri inimmaginabili modi, magari attraverso diverse dimensioni. 
% 	 A seconda della struttura del dominio possiamo scegliere la logica da utilizzare e così il ''contesto'' più adatto all'intuizione che abbiamo dell'universo nel linguaggio naturale. 
 	
%  	La persistenza nel tempo non è perciò rimossa dalla descrizione, o negata; piuttosto, inglobata. E' un sottocaso del modello generale, precisamente quello in cui la proposizione sull'esistenza dell'oggetto è vera con forza 1 i tutti gli istanti (cfr. \autoref{}, nota 16).
 	
%  	Va da sè che, molto informalmente, l'esistenza in this conception non è altro che la modalità di "presenza" degli oggetti all'interno di un modello. E' quindi letteralmente ciò che possiamo \emph{farci} con gli oggetti, come possiamo porci rispetto a essi. 
 	
%  	Non è solo una nozione operativa di esistenza, vicina peraltro al nostro senso comune: per noi le cose esistono se possiamo toccarle, vederle, postularle (quando invisibili) in base a ipotesi su rapporti di causazione che hanno con entità osservabili, descriverle, contarle, utilizzarle; e ciò è indipendente dal \emph{come} esse esistano. E di conseguenza è anche una visione epistemica. 
 	
%  	Bypassata la domanda ''se le cose esistono'', in base alle nostre scelte metateoriche e fondazionali, l'esistenza riguarda i modi tramite i quali le cose entrano in relazione l'una con l'altra. Questo ci permette, ecco i vantaggi dell'ontologia categoriale, di sfruttare la visione strutturalista e poter descrivere e render cogenti non solo il nostro mondo ma realtà distanti, come Tl\"on, fornite di un'ontologia diversa. 
 	
%  	In linguaggio matematico questo "modo di comportarsi delle cose" non è altro che lo studio delle relazioni tra gli oggetti di una categoria.
 	
%  	Infine sarà, l'esistenza, anche context-dependent; varierà a seconda del linguaggio interno della teoria (cioè della categoria) nella quale operiamo. E questo permette di formalizzare la banale intuizione, spesso sfuggente agli occhi degli ontologi, per cui l'esistenza in un mondo come Tl\"on sarà presumibilmente diversa dalla nostra. L'ovvia constatazione che cambiando ontologia cambia il concetto di ''esistere'' diventa qui una cosneguenza automatica dell'uso di un linguaggio matematico. 
%  \end{italian}


% Ontology rests upon the principle of identity. It is this very principle that here we aim to unhinge.

% Cosa significa che \emph{due cose sono, invece, una} è un problema che ci arrovella fin da quando otteniamo la ragione e la parola; ciò perché il problema è tanto elementare quanto sfuggente: l'unica maniera in cui possiamo esibire ragionamento certo è il calcolo; del resto, se la sintassi non vede che l'uguaglianza in senso più stretto possibile, la prassi deve diventare in fretta capace di una maggiore elasticità: per un istante ho postulato che ci fossero ``due'' cose, non una. E non è forse questo a renderle due? E questa terza cosa che le distingue, è davvero diversa da entrambe?

% Ciò che risulterà evidente è che la nozione di identità è, appunto, \emph{context-dependent}, e questo risolve il dibattito che, almeno da [Geach,\dots], impegna i filosofi, in merito alla sua eventuale relatività ontologica.

% Sostituire la nozione classica vuol dire rivedere i fondamenti dell'ontologia: la stessa nozione centrale di \emph{esistenza}, nella tradizione quineana, si definisce tramite la nozione, più (illusoriamente) semplice e primitiva, di identità: ``\textit{A esiste}'' sse ``\textit{qualcosa è identico ad A}'' (questo ``qualcosa'' è una variabile vincolata ad un quantificatore esistenziale).

% Il motivo per cui riteniamo di dover agire in questa direzione è dovuto agli innumerevoli problemi che la nozione di identità classica (criterio di Leibniz, sue varianti, ma anche definizioni successive in sua vece) si porta dietro, rilevati da molti filosofi nel corso del secolo passato, e non superabili rifiutando solo l'identità leibniziana o abbracciando la prospettiva mereologica (ma ci riserviamo di parlarne in lavori successivi). Il motivo per cui molti filosofi, pur sottolineando l'inadeguatezza della nozione, non hanno mai seriamente proposto di sostituirla, crediamo sia per mancanza sia di un linguaggio adatto sia di alternative teoriche rigorose in esso espimibili.
% L'\textit{Homotopy Type Theory}, e più in generale la \CT, rispondono a queste esigenze, e attuano quella sostituzione finora mai realizzata. \fo{Esempio di come HoTT sia strutturalista nella metateoria è che in HoTT si può definire cos'è una categoria, e dopo averlo fatto si scopre che la sintassi interpreta ``essere uguali'' per due oggetti/termini di tipo categoria $A,B: \mathcal C$ come ``essere isomorfi'', o come ``essere omotopi'' quando $\mathcal C$ viene interpretato come un tipo \emph{di omotopia}, $A,B: \mathcal C$ come punti di questo spazio, e $A=_{\mathcal C}B: \sf Prop$ come un'omotopia tra $A$ e $B$. Vale anche la pena notare che in teoria dei tipi la relatività ontologica della nozione di identità \emph{è un assunto}: ogni tipo $X$ è equipaggiato con una ``sua'' nozione di identità $=_X$ che è locale, è ``la sua'' e nulla a a che vedere, a priori, con $=_Y$ per un altro tipo $Y$. Ogni uguaglianza istanziata per termini di tipo diverso è quindi inammissibile \emph{nel linguaggio} ancor prima che nella semantica.}\endfo



