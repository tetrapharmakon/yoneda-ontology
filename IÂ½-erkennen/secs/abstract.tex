\begin{abstract}
	We outline a `formal theory of scientific theories' rooted in the theory of profunctors; this poses the accent on the fact that the scope of scientific knowledge is to build a `meaningful connection' (i.e. a well-behaved adjunction) between a  linguistic asset (a `theoretical category' $\clT$) and the world said language ought to describe. Such world is often unfathomable, and thus we can only resort to a smaller fragment $\clO$ of it in our analysis: from this we build the category $[\clO^\op,\Set]$ of all possible displacements of observational terms $\clO$. The self-duality of the bicategory of profunctors accounts for the fact that theoretical and observational terms can exchange their r\^ole, providing evidence for the conventional nature of their being separated entities; moreover, to every profunctor $\fkR$ linking $\clT$ and $\clO$ one can associate an object $\clO\uplus_\fkR\clT$ obtained gluing together the two categories, and accounting for the mutual relations subsumed by $\fkR$. Representability of profunctors provides a categorical interpretation for the `Ramseyfication' operation.
\end{abstract}