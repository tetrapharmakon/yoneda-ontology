\section{Introduction}\label{sec_intro}
\epigraph{El mundo, desgraciadamente, es real.}{\cite{confutacion}}
This is the first chapter of a series of works aiming to touch a pretty wide range of topics.

Its purpose is to adopt a wide-ranging approach to a fragment of elementary problems in a certain branch of contemporary philosophy of Mathematics. More in detail, we attempt at laying a foundation for solving a number of problems in ontology employing pure mathematics; in particular, using the branch of mathematics known as \emph{category theory}.

As authors, we are aware that such an ambitious statement of purpose must be adequately motivated, bounded to a realistic goal, and properly framed in the current state of the art on the matter. This is the scope of the initial section of the present first manuscript.
\subsection{What is this series}
Since forever, mathematics studies three fundamental indefinite terms: \emph{form}, \emph{measure}, and \emph{inference}. Apperception makes us recognise that there are extended entities in space, persisting in time. From this, the necessity to measure how much these entities are extended, and to build a web of conceptual relations between them, explaining how they arrange `logically'.
Contamination between these three archetypal processes is certainly possible and common: in fact, mathematics happens exactly at the crossroad where algebra, geometry and logic intersect.

We can even say more: mathematics is a language engineered to systematically infer properties of the three mentioned indefinites; so, meta-mathematics done through mathematics (if such a thing even exists) exhibits the features of a \emph{ur-language}, a generative scheme for `all' possible languages. It is a language whose elements are the rules to give oneself a language, convey information to other selves, and allow deduction. It is a meta-object, a scheme of rules to generate objects/languages.

Taken this tentative definition, mathematics (not its history, not its philosophy, but its \emph{practice}) serves as a powerful tool to tackle the essential questions of ontology: what things are cogently, what makes them what they are and not different.

Quantitative thinking is a consistent (some would say `honest') way of approaching the deep problem of the cogency of entities; yet, it is undeniable that a certain philosophical debate has become indifferent, or even overtly hostile, to mathematical language. A tear in the veil that occurred a long time ago, due to different purposes and different specific vocabulary, can not be repaired by two people only. If, however, the reader of these notes asks for an extended motivation for our work, a wide-ranging project in which it fits, a long-term goal, in short, a \emph{program}, they will find it now: there is a piece of mathematics whose purpose is to solve philosophical problems, in the same sense certain mathematics `solves' the motion of celestial bodies.

It does not annihilate the question: it proposes models within which one can reformulate it; it highlights what is a trivial consequence of the axioms of that model and what instead is not, and requires the language to be expanded, modified, sharpened. We aim to approach this never-mentioned discipline as mathematicians. But we do it without elaborating `new' theorems; we draw connections between the modern mathematical practice to use them in the context of philosophical research.

Sure, solving once and for all the problems posited by ontology would be megalomaniac; we do not claim to have reaced such an ambitious objective. More modestly, we propose a starting point unhinging some well-established beliefs;\footnote{Above all else, the belief that ontology is too general to be approached quantitatively, and that it contains mathematical language as a proper subclass: it is instead the exact opposite, as the central idea of our work is that ontologies -there are many- are mathematical objects.} we humbly point the finger at some problems that prose is unable to notice, because it lacks a specific and technical language; we suggest that only in such a language, when words mean precise things and are tools of epistemic research instead of mere magic spells, a few essential questions of recent ontology dissolve in a thin thread of smoke, and others simply become `the wrong question': not false, just meaningless; meaningless as the endeavour to bestow a concrete attribute as a temperature to abstract concept like consciousness or justice.

We shall say at the outset that the ur-language we are tackling is not mathematics. Yet, mathematical language has an enormous potential to hint at what the elementary particles of the \emph{characteristica universalis} we're looking for should be made of.

It may also seem suspicious to employ mathematics to tackle questions that traditionally pertain to philosophy: some philosophers believe they are debating about problems more general than Mathematics, from a higher point of obsevation on the fundamental questions of Being. By proposing the `ontology as categories' point of view, our work aims to dismantle such a false belief.%: each and every debate must appear in a language.

In doing so, we believe we can provide a more adequate language, taken from mathematics, within which to frame some deep ontological questions, whose potential is often lost in the threads of ill-posed questions and clumsy answers. The reader will allow a tongue-in-cheek here, subsuming our position: in ontology, it is not a matter of making a \emph{correct use of language}, but rather a matter of \emph{using the correct language}.

This correct language is inherently mathematical, as only mathematics proved to be able to substantiate a qualitative analysis in quantitative terms. This language `must be' \emph{category theory}, as only category theory has the power to speak about a totality of structures of a given kind in a compelling way, treating mathematical \emph{theories} as mathematical \emph{objects}.

As of now, our work unravels in three different chapters, and it will attempt to cover a variety of topics: 
\begin{itemize}
    \item the present manuscript, \emph{Existence}, provides the tools to build a sufficiently expressive `theory of existence' inside a category. This first chapter has a distinctly foundational r\^ole; its scope to build the fundamentals of our toolset (category theory and categorical logic, as developed in \cite{mac1992sheaves,JohnstonePT,lambek1988introduction}). 
    
    As both a test-bench for our theory and a literary \emph{divertissement}, we propose a category-theoretic solution of Borges' paradoxes present in \cite{Borges1963}. In our final section, we relate our framework to more classical ancient and modern philosophers; we link topos theory to Berkeley's instantaneism and internal category theory to Quine's definition of the [domain of] existence of an entity as a domain of validity of quantifiers (intended as propositional functions, i.e. functions whose codomain is a space of truth values).
    \item A second chapter \cite{black}, currently in preparation, addresses the problem of \emph{identity}, and in particular its context-dependent nature. Our proof of concept here consists of a rephrasing of Black's classical `two spheres' paradox \cite{papear_di_black} in the elementary terms of invariance under a group of admissible transformations; this time the solution is provided by Klein's famous \emph{Erlangen program} group-theoretic foundation for geometry: the two interlocutors of Black's imaginary dialogue respectively live in an Euclidean and an affine world: this difference, not perceived by means of language, affects their understanding of the `two' spheres, and irredeemably prevents them from mutual intelligente.
    \item A third chapter \cite{homot}, currently in preparation, addresses again the problem of identity, but this time through the lens of algebraic topology, a branch of mathematics that in recent years defied well\hyp{}established ontological assumptions ; the many commonalities between category theory and homotopy theory suggest that `identity' is not a primitive concept, but instead depends on our concrete representation of mathematical entities. This can be formalised in various ways, among many the \emph{Homotopy Type Theory} foundation of \cite{hottbook,cwp}. \cite{homot} aims to be an introduction to the fundamental principles of HoTT \emph{ad usum delphini}: we investigate how when $X,Y : \clC$ are objects in a category, there often is a class of equivalences $W \subseteq \hom(\clC)$ prescribing that $X,Y$ shouldn't be distinguished in the associated ontology; equality is then defined \emph{ex post} in terms of $W$, varying as the ambient category $\clC$ does; this yields a $W$-parametric notion of identity $\equiv_W$, allowing categories to be categorified versions of \emph{Bishop sets}, i.e. pairs $(S,\rho)$ where $\rho$ is an equivalence relation on $S$ prescribing a $\rho$-equality.
\end{itemize}
Our main tenet in the present chapter is that ontolog\emph{ies} are mathematical objects: each ontology is a certain category $\clO$, inside which `Being' unravels as the sum of all statements that the internal language (see \ref{}) of $\clO$ can concoct. 

Of course, the more expressive is this language, the more expressive the resulting theory of existence will turn out to be. Our presupposition here is that trying to let ontology speak about `\emph{all} that there is' (the accent is on the adverb, on the famous quote of Quine \cite{}) can lead to annoying paradoxes and foxholes.

Instead research shall concentrate on clarifyin what the verb means: in what sense, `what there is' \emph{is}? What is is-ness? As category theorists, our -perhaps simplistic- answer is that, again paraphrasing Quine, 
\begin{quote}
being is \emph{being the object of a category}.
\end{quote}
Explaining why this is exactly Quine's motto, just shifted one universe higher, is the content of our §\ref{metaon}.
\subsubsection{Structure of the paper}
\todo[inline]{manca}
\subsection{On the choice of a meta-theory and a foundation}
Along the 20$^\text{th}$ century, the discipline of Mathematics divided into different subclasses, each with their specific problems and its specific language, just to find, soon after, unification under a single notion of \emph{structure}, through the notion of abstract category \cite{gtone}. This process led to an epistemological revision of mathematics and has inspired, parallel to the development of operative tools, a revision of both the foundations of mathematics and the purposes of its research. 

According to many, it is undeniable that
\begin{quote}
    [the] mathematical uses of the tool `category theory' and epistemological considerations having category theory as their object cannot be separated, neither historically nor philosophically. \cite{kromer2007tool}
\end{quote}
Structural-mathematical practice, i.e. the practice of everyday mathematics directed by structural meta-principles, produced a `natural' choice for the underlying metaontology of mathematics\footnote{Perhaps improperly, the locution \emph{metaontology of mathematics} is used here to refer to the totality of operative beliefs inspiring the ergonomy of mathematical objects. Some of these principles are: objects not enjoying a universal property shall be discarded; definitions that are isomorphism-invariant shall be preferred over those who are not; both these commandments are based on the idea that classes of mathematical objects arrange in coherent conglomerates exhibiting more structure than the mere aggregation of their elements: the requests of universality and isomorphism-invariance are meant not to destroy such additional structure. Examples of these meta-principles can be found in various other areas of Mathematics.} which, later, felt the need to be characterized more precisely. Similarly to what Carnap\footnote{Some words that philosophers should keep in mind, on the lawfulness of the use of abstract entities (specifically mathematical) in semantic reflection, also valid in ontology:
    \begin{quote}
        we take the position that the introduction of the new ways of speaking does not need any theoretical justification because it does not imply any assertion of reality [...].  it is a practical, not a theoretical question; it is the question of whether or not to accept the new linguistic forms. The acceptance cannot be judged as being either true or false because it is not an assertion. It can only be judged as being more or less expedient, fruitful, conducive to the aim for which the language is intended. Judgments of this kind supply the motivation for the decision of accepting or rejecting the kind of entities. \hfill \cite{carnap1956meaning}
    \end{quote}} suggested regarding semantics,
\begin{quote}
    mathematicians creating their discipline were not seeking to justify the constitution of the objects studied by making assumptions as to their ontology.\hfill  \cite{kromer2007tool}
\end{quote}
Beyond the attempts (above all, those of Bourbaki group: but see \cite{McL}, historical notes on Ch. 4, for a hint that Bourbaki didn't really get the point of structural mathematics), what matters is that the habit of reasoning in terms of structures has suggested implicit epistemological and ontological attitudes. This matter would deserve an exhausting independent inquiry.

For our objectives it's enough to declare a differentiation that Kr\"omer
elaborated, inspired by \cite{Cor96}: the difference between \emph{structuralism} and \emph{structural mathematics}:
\begin{enumtag}{s}
    \item \label{s:uno} Structuralism: the philosophical position regarding structures as the subject matter of mathematics;
    \item \label{s:due} Structural Mathematics: the methodological approach to look in a given problem for the structure itself.
\end{enumtag}
Of course, \ref{s:uno} implies \ref{s:due} but the opposite is not always true:
\begin{remark} \label{weak_structuralism}
    That is, one can do structural mathematics without being a structuralist and taking different, or even opposite, positions concerning structuralism itself. 

    Nevertheless, the use of \CT as meta-language, despite the historical link with structuralism, doesn't make automatic the transition from \ref{s:due} to \ref{s:uno}; it just suggests that the ontology is not only dependent on the `ideology' (in a Quinean sense) of the theory, but it is instead influenced by the epistemological model inspired by formal language.
\end{remark}
Kr\"omer's distinction, however, has another virtue: instead of stumbling in a possibly not ambiguous definition of \textit{structure} (with the unwanted consequences that could arise in the operational practice), \ref{s:uno} can be reduced to (or can redefine) \ref{s:due}, saying that:
\begin{quote}
    \emph{structuralism is the claim that mathematics
        is essentially structural mathematics} \cite{kromer2007tool}
\end{quote}
This is the same thing as saying: the structural practice already is its philosophy.

Attempts to explain the term `structure' by Bourbaki in the years following the publication of the \textit{Elements des Mathématiques}, led to the first systematic elaboration of a philosophy that we could appropriately call \textit{structural mathematics}. Its target is to `\textit{assembling all possible ways in which given set can be endowed with certain structure}' \cite{kromer2007tool}, and elaborate, in the programmatic paper \textit{The Architecture of Mathematics} (written by Dieudonné alone and published in 1950), a formal strategy. While specifying that `\textit{this definition is not sufficiently general for the needs of mathematics}' \cite{Bourb50}, the author encoded a series of operational steps through which a structure on a collection is assembled set-theoretically. Adopting therefore a reductionist perspective in which
\begin{quote}
    the structureless sets are the raw material of structure building which in Bourbaki’s analysis is `unearthed' in a quasi-archaeological, reverse manner; they are the most general objects which can, in a rewriting from scratch of mathematics, successively be endowed with ever more special and richer structures.\hfill  \cite{kromer2007tool}
\end{quote}
On balance, in Bourbaki's structuralism, the notion of set doesn't disappear definitively in front of the notion of structure. Times were not ripe to abandon set theory; the path towards an `integral' structuralism was still long, and culminated years after, with Lawvere's attempt at a foundation \theory{ETCS} of set theory first \cite{lawvere1964elementary} and \theory{ETCC} of category theory (and as a consequence, `of all Mathematics') after \cite{lajolla}, through structuralism.

To appreciate the depth and breadth of such an impressive piece of work, however, the word `foundation' must be taken in the particular sense intended by mathematicians:
\begin{quote}
    [\dots\unkern] a single system of first-order axioms in which all usual mathematical objects can be defined and all their usual properties proved.
\end{quote}
Such a position sounds at the same time a bit cryptic to unravel, and unsatisfactory; Lawvere's (and others') stance on the matter is that a foundation of mathematics is \emph{de facto} just a set $\clL$ of first-order axioms organised in a Gentzen-like deductive system. The deductive system so generated reproduces mathematics as we know and practice it, providing a formalisation for something that already exists and needs no further explanation, and that we call `mathematics'.

It is not a vacuous truth that $\clL$ exists somewhere: point is, the fact that the theory so determined has a nontrivial model, i.e the fact that it can be interpreted inside a given familiar structure, is at the same time the key assumption we make, and the less relevant aspect of the construction itself.

Showing that $\clL$ `has a model' is --although slightly improperly-- meant to ensure that, \emph{assuming the existence of a naive set theory} (i.e., assuming the prior existence of structures called `sets'), axioms of $\clL$ can be satisfied by a naive set. Alternatively, and more crudely: assuming the existence of a model of \theory{ZFC}, $\clL$ has a model \emph{inside that model of \theory{ZFC}}.\footnote{It shall be made clear, ensuring that a given theory has a model isn't driven by psychological purposes only: on the one hand, purely syntactic mathematics would be very difficult to parse, as opposed to the more colloquial practice of mathematical development; on the other hand (and this is more important), the only things syntax can see are equality and truth. To prove that a given statement is false, one either has to check all possible syntactic derivations leading to $\varphi$, finding none --this is unpractical, to say the least-- or to \emph{find a model} where $\lnot\varphi$ holds.}

\subsection{Our foundation, at last.} A series of works attempting to unhinge some aspects of ontology through category theory should at least try to tackle such a simple and yet diabolic question as `where' are the symbols forming the first-order theory \theory{ETCC}. And yet, everyone just believes in -some flavour of- sets and solves the issue of `where' they are with a leap of faith from which all else must follow.

This might appear somewhat circular: aren't sets in themselves already a mathematical object? How can they be a piece of the theory they aim to be a foundation of? In his \cite{lolli1977categorie} the author addresses the problem as follows:\footnote{Authors' translation: \emph{When mathematicians talk about models they do not have the impression to have exited a set-theoretic foundation. This impression is correct, and justified by the possibility to represent formal languages through set theory, to study the relations among structures and symbolic representations. When mathematicians turn their attention to models of a set theory, some philosophical questions can however no longer be avoided. The natural question of what is the relation between the sets that are models of a theory, and the sets of which the theory talks about is nothing but a question about the relation between the semantical, set-theoretic metatheory and the `object theory' we want to talk about. The two theories can coincide, and in fact the metatheory can even be a proper subtheory of the object theory.\\
\indent We choose a set whose elements, usually finite sets, represent the symbols of the language we want to study, and then with a concatenation operation that can be the `ordered pair' construction we define the set of words $L$ of the well-formed formulas, of terms, and the operation that associates a free variable with a formula, and so on for all syntactic notions.}}
\begin{quote}
    Quando un matematico parla di modelli non ha [\dots\unkern] l'impressione di uscire dall'ambito insiemistico. Questa impressione, che è corretta, è giustificata dalla possibilità di rappresentare i linguaggi formali con gli oggetti della teoria degli insiemi, di studiare in essa le relazioni tra le strutture e le rappresentazioni dei simboli. Quando l'attenzione è rivolta ai modelli di una teoria degli insiemi, certe questioni sofistiche non possono però essere più evitate. La domanda spontanea sulla relazione che intercorre tra gli insiemi che so­no modelli di una teoria e gli insiemi di cui parla la teoria non è altro che una domanda sulla relazione tra la metateoria semantica insiemi­stica e la teoria in esame, o teoria oggetto. Le due teorie possono coincidere, anzi la metateoria può essere anche una sottoteoria pro­pria della teoria oggetto. [\dots\unkern]

    Si sceglie un insieme i cui elementi, di solito insiemi finiti, rappresentano i simboli del linguaggio che si vuole studiare, quindi con una operazione di concatenazione che può essere la coppia ordinata si definisce l'insieme delle parole $L$ delle for­mule ben formate, dei termini, l'operazione che a una formula asso­cia le sue variabili libere, e così via per tutte le nozioni sintattiche.
\end{quote}
The idea that a subtheory $L'\subset L$ of the object theory can play the r\^ole of metatheory might appear baffling; in practice, the choice is to rely on one among two possible solutions. Pure Platonism assumes the existence of a hierarchy of universes harbouring the object theory; pure syntacticism exploits G\"odel's completeness theorem: every proof is a finite object, and every theorem proved in the metatheory is just a finite string of symbols. No need for a model.

Platonism has limits: in a fixed a class theory $\sfC$ ($\sf MK$, Morse-Kelley(-Mostowski); or $\sf NBG$, Von Neumann-Bernays-G\"odel), there's an object $V$ that plays the r\^ole of the universe of sets; in $V$, all mathematics can be enacted. Of course, consistency of $\sfC$ is only granted by an act of faith.

Syntacticism has limits: following it, one abjures any universality mathematics might claim. But syntacticism also has merits: undeniably (disgracefully, luckily) the World is real. And reality is complex enough to contain languages as purely syntactical objects; the percussion of a log with oxen bones, rather than prophecies over the entrails of a lamb, or intuitionistic type theory, all have the same purpose: intersubjective convection of meaning, deduced by a bundle of perceptions, so to gain an advantage, \emph{id est} some predictive power, over said perceptions. Of course: intuitionistic type theory is just \emph{slightly} more effective than hepatomancy.

Knowledge is obtained by collision and retro-propagation between Reality and the perceptual bundle it generates.

Accepting this, the urge to define seemingly abstract concepts like learning, conscience, and knowledge, together with precious continuous feedback coming from real objects, evidently determine an undeniable primacy of quantitative thinking, this time intended as machine learning and artificial intelligence, that sets (or should set if only more philosophers knew linear algebra) a new bar for research in philosophy of mind.

However, we refrain from entering such a deep rabbit-hole, as it would have catastrophic consequences on the quality, length, and depth of our exposition.

The usual choice for mathematicians is to assume that, wherever and whatever they are, these symbols `are', and our r\^ole in unveiling mathematics is \emph{descriptive} rather than generative.\footnote{Inside (say) a constructivist foundation it is not legitimate to posit that axioms `create' mathematical objects; from this, the legitimacy of the question of where they are, and the equally legitimate answer `nowhere'. The only thing we can say is that they `make precise, albeit implicitly, the \emph{meaning} of mathematical objects' \cite{Agzz} (it seems to us that in mathematics as well as in philosophy of language, meaning and denotation are safely kept separate). We take this principle -that the world/metamodel exists and we can just attempt at describing it by means of the language/model- as evident to anyone in a healthy state of mind, and we leverage it without further question.}

This state of affairs has, to the best of our moderate knowledge on the subject, various possible explanations:
\begin{itemize}
    \item On one hand, it constitutes the heritage of Bourbaki's authoritarian stance on formalism in pure mathematics;
    \item on the other hand, a different position would result in barely any difference for the `working-class'; mathematicians are irreducible pragmatists, somewhat blind to the consequences of their philosophical stances.
\end{itemize}
So, symbols and letters do not exist outside of the Gentzen-like deductive system we specified together with $\clL$.

As arid as it may seem, this perspective proved itself to be quite useful in working mathematics; consider for example the type declaration rules of a typed functional programming language: such a concise declaration as
\begin{lstlisting}[ language=Haskell
                  , basicstyle=\ttfamily\small
                  , keywordstyle=\color{blue!75}
                  , morekeywords={S,Z}
                  ]
  data Nat = Z | S Nat
\end{lstlisting}
makes no assumption on `what' \verb|Z| and \verb|S:: Nat -> Nat| are; instead, it treats these constructors as meaningful formally (in terms of the admissible derivations a well-formed expression is subject to) and intuitively (in terms of the fact that they model natural numbers: every data structure that has those two constructors must be the type $\bbN$ of natural numbers -provided data constructors like \verb|S| are all injectve).

Taken as an operative rule, this reveals exactly what is our stance towards foundations: we are `structuralist in the metatheory', meaning that we treat the symbols of a first-order theory or the constructors of a type system regardless of their origin, provided the same relation occur between criptomorphic collections of labelled atoms.

In this precise sense, we are thus structuralists in the metatheory, and yet we do so with a grain of salt, maintaining a transparent approach to the consequences and limits of this partialisation. On the one hand, pragmatism works; it generates rules of evaluation for the truth of sentences. On the other hand, this sounds like a Munchhausen-like explanation of its the value, in terms of itself. Yet there seems to be no way to do better: answering the initial question `where are the letters of \theory{ETCC}?' would result in no less than a foundation of language.

And this for no other reason than `our' metatheory is something near to a structuralist theory of language; thus, a foundation for such a metatheory shall inhabit a meta-metatheory\dots{} and so on.

Thus, rather than trying to revert this state of affairs we silently comply to it as everyone else does; but we feel contempt after a brief and honest declaration of intents towards where our metatheory lives. Such a metatheory hinges again on work of Lawvere, and especially on his series of works on functorial semantics.
