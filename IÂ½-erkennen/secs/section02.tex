\epigraph{Life is the life of the world to come, which a man earns by means of the letters.}{Sefer Ozar Eden Ganuz}
\section{Semantical conception of theories}
Along the XXth century there have been many attempts towards a formal definition of a scientific theory. The \emph{Wiener Kreis} verificationist paradigm/account, and Neurath's theory of `protocollar statements', has given the initial input to elaborate a completely semantic framework for scientific theories, and spurred the search for a pan-linguistic vision of philosophy of science.

The formal account in which e.g. Carnap \cite{?} provides his notion of `theory' is known in the literature as \emph{syntactical conception of theories} \cite{?}, while the term `semantic' is due to later developments. Yet, the field of epistemology that the logical neopositivism started can legitimately be called a `semantics of theories', because some of its features, if not the underlying ideology, are the same throughout the works of Carnap \cite{?},  Beth \cite{?}, and Suppes \cite{?}, all the way up to the recent canonical uses of physical handbooks.

{\color{red} Non è chiarissimo cosa ci sia scritto\dots}

	[scrivere quali sono queste caratteristiche]

	[sintesi delle varie concezioni; le teorie come classi di modelli $\clK$;
		le teorie come oggetti formali]

semantica non standard per teorie empiriche in cui le teorie sono sistemi formali e tutte le nozioni diventano oggetti matematici; 

More properly, a \emph{theory} becomes a structure $(F_\clL, \clK)$ where $F_\clL$ is a formal system, and $\clK$ the totality of all its interpretations. Our strategy is to further separate $F_\clL$ into two `vocabularies' (intended as syntactic categories of first order theories), one of them representing the \emph{pure} or \emph{theoretical} terms (see Plantinga \cite{}), and the other representing the \emph{applied} or \emph{observational} terms. 

Thus, for us a scientific theory is a triple $\bk{(\clT,\clO), \clK}$ whose first two elements form the logic $F_\clL=(\clT,\clO)$.% cui la prima coppia configura una logica (uno `spazio degli stati' che configura una logica).

The pair $(\clT,\clO)$ is subject to an additional admissibility condition, that is the existence of a meaningful relation between the theoretical world $\clT$ and the observational world $\clO$; the notion of `meaningful relation' between structured high-level systems is captured by a well-known mathematical object, a \emph{profunctor} between the two syntactic categories $\clT,\clO$. 

Having outlined a general theory of scientific theories in terms of profunctors is the main contribution of the present work.

Concludiamo questa sezione introduttiva con una discussione a proposito della `natura' di $\clT,\clO$.

Come facciamo osservare in \cite{canont1}, il problema di dove siano situati gli oggetti sintattici che incarnano una teoria linguistica può essere risolto agilmente in una prospettiva epserienzialista: il mondo esiste in maniera innegabile, ed è una struttura sufficientemente complessa da contenere al suo interno sufficienti oggetti costituenti `concreti' di un sistema formale. E' quindi semplicemente da una porzione del mondo che ricaviamo i simboli primitivi del linguaggio.

Questo problema, e la sua soluzione, si riflettono inevitabilmente nella maniera in cui le categorie $\clT$ e $\clO$ sono specificate. Nel nostro modello `il mondo' è una categoria $\clW$, unfathomable and given since the beginning of time, a cui non abbiamo altro modo di accedere se non attraverso funtori $\phi : \clL \to \clW$ (cf. \autoref{}) che rappresentano sottocategorie piccole costruite a partire dalle porzioni di $\clW$ a cui abbiamo accesso. La richiesta che $\clW$ sia sufficientemente espressiva si traduce nella richiesta (cf. \autoref{}) che $\clW$ ammetta tutti i colimiti (nel senso di \cite[??]{Bor1}). Una tale categoria si dice uno \emph{Yuggoth}, dal nome del finto pianeta dove i Mi-Go sono localizzati.

When things are put in this perspective, a few remarks are in order: 
\begin{itemize}
	\item La specificazione del dominio di $\clO$ determina il tipo di teoria che stiamo considerando (scientifica, strettamente empirica, logico-matematica, metafisica).
	\item Dire che $\clO$ determina le \emph{tipizzazioni} della teoria significa dire che svolge lo stesso ruolo della legge $\beta$ {\color{red} chi è beta?} nella semantica dello spazio degli stati, mentre la classe $\clK$ è isomorfa all'insieme $\mathcal{M}$ dello spazio degli stati. Il tipo di $\beta$ determina il tipo di $\mathcal{M}$ che determina il tipo di $T = (\mathcal{M}, \beta)$. Idem nel nostro approccio: $\clO 	= \{\alpha_1,\dots,\alpha_n\}$ determina il tipo, che implementa una logica che determina la classe $\clK$.
\end{itemize}
Questo chiude il cerchio sul problema della rappresentazione di un mondo $\clW$ in termini di una sua porzione $\clT$ a cui abbiamo accesso ermeneutico e da cui abbiamo ricavato un linguaggio.

Questa rappresentazione avviene mediante funtori $\clL \to \clW$ che, grazie alla proprietà di cocompletezza di $\clW$, si estendono a funtori di rappresentazione $[\clL^\op,\Set] \leftrightarrows \clW$; il mondo del resto è inconoscibile strictly speaking: invece che a $\clW$ noi abbiamo accesso ad un suo frammento osservazionale $\clL$, da cui ricaviamo, sempre grazie alla cocompletezza di $\clW$, una ulteriore rappresentazione $[\clL^\op,\Set] \leftrightarrows \clW$. In generale, questo è tutto quello che si può dire. 

Ad ipotesi aggiuntive (cf. \autoref{canvas_scienza}) sull'inclusione $\clL\subseteq \clW$, tuttavia, possiamo inferire che la totalità dei modelli $[\clL^\op,\Set]$ \emph{contiene una copia} di $\clW$. In questo senso, alle ipotesi di \autoref{canvas_scienza}, the unfathomable world is a full subcategory of the class of all modes in which language can create interpretation. Questo rovescia la prospettiva classica, e per quanto bold a statement it might seem, non è senza conseguenze: si vedano \autoref{remark_yuggoth_1}, \autoref{remark_yuggoth_2}.
%si può generalizzare tutto ciò dicendo appunto che a seconda di come è strutturata la classe $A_{\mathcal{F_L}}$ noi possiamo distinguere le teorie anche al di fuori dell'ambito strettamente scientifico. Questa considerazione va a scomparire proprio con ciò che emerge in sezione 6: eliminando questa distinzione ottieni un account più generale di trattamento delle teorie%

\subsection{The Two Dictionaries}
Nella concezione neopositivistica la distinzione tra legge teorica e legge empirica non è dovuta alla natura ipotetica della prima (anche una legge empirica può esserlo) quanto dal fatto che i due tipi di legge contengono tipi differenti di termini \cite{?}. La distinzione è quindi formale, e indica una approccio prettamente linguistico a questioni epistemologiche. La formulazione profuntoriale delle teorie scientifiche annulla con ancora maggior forza la distinzione tra l'impianto osservativo e quello teoretico/linguistico di una teoria; the two categories $\clO,\clT$ are carved out of the same substance of their Father $\clW$: one is just a portion of the world we can better control and describe.

Il presente lavoro intende argomentare nella seguente direzione: primo, la distinzione tra osservazionale e teoretico è veramente figmentale. Secondo, sposando questa tesi l'impianto matematico atto a descrivere una teoria generale delle teorie scientifiche diventa piu agile e fedele.

In termini profuntoriali infatti la distinzione tra osservazionale e teoretico svanisce in virtù di \autoref{da_collage} e di argomenti direttamente correlati: per prima cosa, la bicategoria dei profuntori, come definita in \autoref{}, mutuando da \cite{benabou2000distributors}, è auto-duale; ciò significa che un dato profuntore $p : \clT \pto \clO$ ammette una immagine speculare $p^\op : \clO \pto \clT$;\footnote{Questo fatto è reminiscent del fatto che bla bla una relazione non ha un dominio di definizione privilegiato; chiaramente, la categoria $\mathsf{Cat}$ ha una involuzione non banale data dall'opping, e questo rende leggermente più visibile, ma non meno tautologica, l'auto-involuzione di $\Prof$.} in secondo luogo, e certamente in maniera piu decisiva per la nostra tesi, come spiegato in \autoref{resoudre_la_tension} dato un generico profuntore $p : \clT \pto\clO$ possiamo costruire il ``collage'' di $\clT,\clO$ lungo $p$, ossia una nuova categoria $\clT\uplus_p\clO$, fitting in a span 
\[ \vcenter{\xymatrix{
	& \clT\uplus_p \clO \ar[dr]\ar[dl]& \\ 
	\clT  && \clO 
}} \] (cf. \autoref{def:dfib} and in particular \autoref{collage_explaned}) che permette di ricostruire i termini teoretici e osservazionali per proiezione.
\begin{remark}\label{hint_at_collage}
	Anche in questa visione `sintattica' \cite{giunti2016} una teoria è sempre una struttura che contiene un sistema formale $\mathcal{F_L}$ e la classe $\clK$ dei suoi modelli. La strategia carnapiana per rendere conto della presenza di entità `osservazionali' e quindi, a rigore, non formalizzabili, all'interno di teorie scientifiche è quella di considerare due diversi dizionari: $\clT$ che contiene \emph{termini teorici} e $\clO$ che contiene \emph{termini osservativi}. 
	%Intuitivamente $\mathcal{F_L} = \clT \cup \clO$, ma piu precisamente $\mathcal{F_L} = \clT \uplus_\varphi \clO$.
\end{remark}


\begin{definition}[Wiener Theory]
	A theory $(\clF_\clL, \clK_\bullet)$ consists of a formal system $\clF_\clL$ and a class of models $\{\mathcal{K}_n \mid n \in \N \}$ indexed by the set of natural numbers. 
	{\color{red} Qual è l'intuizione dietro questa definizione? Perché vuoi un insieme numerabile di modelli?\dots}
	The formal system $\clF_\clL$ consists in its own right of:
	\begin{itemize}
		\item A formal language $\mathcal{L}$ \fo{of what kind?}
		\item Two vocabularies: $\clT$ for theoretical terms and $\clO$ for observational terms 
		\item Given $\mathsf{L}$ a first-order logic $\mathcal{V}_{\mathcal{T}}^L = \clT \cup \mathsf{L}$ is a logical vocabulary of $T$
	\end{itemize}    
\end{definition}
The vocabularies exhaust all the terms of formal system in the sense that 
\[\mathcal{F_L} = \clT \cup \clO \cup \mathcal{V}_{\mathcal{T}}^L.\]

In carnapian, and in general neopositivistic, account a theory is expressible as a sentence formed by terms $\tau_1, \dots, \tau_k$ taken from one of the two dictionaries \cite{?}. 

In the Wiener Kreis paradigm, the formal specification of $\clO$ is left unclear; Carnap \cite{} posits the existence of \emph{correspondance rules} between $\clO$ and $\clT$, associating to each term $o$ of $\clO$, or O-term, its companion in $\clT$, or the T-term $\tau$ derived from $o$.\footnote{In general, Carnap holds that $\clO \subset \clT$, but at the same time he blurs the features of this identification of observational terms as `types of T-terms'.}

We can maintain a similar idea, just phrased in a slightly more precise way: we posit the existence of a function $\varphi$ that translates O-terms into T-terms.\footnote{In passing, it is worth to notice that Carnap's intuition fits nicely in our functorial framework: assume $\clO,\clT$ exhibit some kind of structure, and that $i : \clO\subseteq\clT$ \emph{as substructures} (e.g., assume that they are some sort of ordered sets, and that the order on $\clO$ is induced by the inclusion); then, a \emph{left} (resp., \emph{right}) \emph{translation function} $\varphi_L$ (resp., $\varphi_R$), is a left (resp., right) adjoint for the inclusion $i : \clO\hookrightarrow \clT$. We will not expand further on this idea.}

So, a Wiener definition of a theory is a suitable set of pairs $\{\bk{\tau,\varphi(\omega)} \mid \tau\in\clT, \omega \in\clO\}\subseteq \clT\times\clT$, where $\varphi: \clO \to \clT$ is called a \emph{translation function}. In this way it is trivially true that all the terms of a theory are in the first dictionary. This is akin to the \emph{graph} of the translation function $\varphi$, and not by chance: see \autoref{da_collage}.

The neopositivistic approach is to build an observational version of a theory $T$ following a procedure first outlined by Ramsey \cite{?}.

Più semplicemente, usando \autoref{}, e considerando che $\clO \subset \clT$, ci pare che la ramseyfication, non potendo nessun termine $\tau_k \in T$ uscire dal vocabolario teorico, sia un ri-tradurre nel linguaggio di $\clO$ il contenuto della teoria, sostituendo i termini teorici "puri" con delle variabili \cite{?}.
\begin{definition}
	Given an \emph{application function} $\psi: \clT \to \clO$, the Ramseyfication $T^\text{Ra}$ of a theory $T$ is obtained as  
	\[ 
		T^\text{Ra} = \{\bk{[\tau/x], \psi(\varphi (\omega))} \mid \omega\in\clO\} 
	\] {\color{red} chi è $x$?}
\end{definition}
In questa semantica si dice allora che $T^\text{Ra}$ rappresenta il \emph{contenuto osservativo} della teoria e in generale vale che: $T \cong T^\text{Ra}$. 



Non ci pare, al di là degli intenti riduzionistici del Wiener Kreis \cite{Weinb}, che ci siano motivi epistemologicamente validi per eseguire un tale procedimento, che del resto non ha avuto molto seguito nella storia successiva, ma la distinzione carnapiana ripresa nel nostro framework, per trarre alcune conclusioni on relations between theoretical and observational core.
%spiegare meglio perché la Ramsey-sentence da sola è un fallimento del riduzionismo viennese%

Già agli albori dell'applicazione di metodi formali in filosofia della scienza è evidente la difficoltà di ignorare la tesi duhemeana secondo la quale "tutta l'osservazione [in fisica] è carica di teoria". 

Le ambiguità carnapiane sul dominio di oggetti sui quali verterebbe la definizione viennese noi, coerenti col testo, le risolviamo indicando come dominio $\clT$. Cosa sia il mondo puro delle osservazioni al quale farebbe riferimento la Ramsey-version della teoria non è chiaro, se non un altro sotto-dizionario teorico, per l'appunto. Non a caso il neopositivismo da un iniziale fisicalismo approda ad un'ottica convenzionalista \cite{?}, in seguito ai falliti tentativi di formalizzazione di un contesto osservazionale extra-teorico. 

Seguendo \cite{psillos} il realismo strutturalista che ispira questi attempts si riduce ad affermare che una teoria $T$ is logically equivalent to the conjunction
\[\overline{T} \land (\overline{T} \rightarrow T)
\] where the second member is the \emph{meaning postulate}:
\begin{quotation}
	Carnap notes that this conditional has no factual content and takes it to be a meaning postulates \cite{psillos}
\end{quotation}
[spiegare con un esempio tipo paper Psillos].

This kind of realism is an if-then form: non è il mondo l'oggetto della scientific knowledge ma le condizioni che, date certe premesse, si verificano nel mondo strutturale di cui parlano le teorie. [esempio del tipo: non è più "la terra gira intorno al sole" ma "dati la terra e il sole, se si verificano (nella realtà) le condizioni $x_1,x_2,x_3$ allora sarà vero il fenomeno della rotazione terrestre". Da cui emerge anche la natura predittiva]. 

Il meaning postulate non era altro che the Wiener Kreis' version of the famous demarcation problem between science and metaphysics: the secular attempt to elaborate a precise criterion to distinguish a proposition belonging to empirical sciences from a metaphysical (or, at large, not scientifical) proposition. Un criterio che, a seconda dei contesti, è sempre risultato troppo stretto o troppo largo. 

Tolti i due estremi, gli enunciati protocollari da un lato \cite{?}, i discorsi di Heidegger dall'altro \cite{?}, esistono tutti i casi intermedi per i quali a meaning criterion (o una operazione come la drastica traduzione dei costrutti teorici in "referti osservativi puri" tramite ramseyfication) impedisce concretamente di individuare la demarcazione. Classical objections range from Popper \cite{?} to more recent sociology of science.
\subsubsection*{Structure of the paper}
Section 2 and 3 outline the mathematical background we need throughout the work; section 4 introduces our mai notions: a canvas, a world, a theory, a science. Sections 5 and 6 will provide evidence that the two dictionaries, theoretical and observable, live in a very tight relation; a broadly intended meaning for the notion of \emph{theory} asks for a precise understanding of the category of adjunctions between the theoretical and observational side.