\begin{abstract}
	The present paper approaches ontology and meta\hyp{}ontology through Mathematics, and more precisely through the theory of \emph{elementary toposes}; for us, an \emph{ontology} is a mathematical object: it is a category $\clE$, the universe of discourse in which our Mathematics (intended at large, as a theory of knowledge) can be deployed. The well-studied \emph{internal language} of such categories is expressive enough to talk about existence, sometimes in a nuanced way that is akin to the way in which different philosophers talked; technically speaking, the presence of an object $\Omega_\clE$ parametrizing the truth values of the internal propositional calculus prescribes the `modes of existence' for the objects of a fixed ontology/category.

	This approach resembles, but is more general than, the one leading to \emph{fuzzy logics}, as most choices of $\clE$ and thus of $\Omega_\clE$ yield nonclassical, many-valued logics.

	As both a test-bench for our theory, and a literary \emph{divertissement}, we propose a possible category-theoretic solution of the famous Tl\"on's ``nine copper coins'' paradox, and of other seemingly paradoxical construction in Jorge Luis Borges' literary work.

	We conclude with some vistas on the most promising applications of our future work.
\end{abstract}