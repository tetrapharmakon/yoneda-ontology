\section{Theories and models}
\label{sec:orge02f333}
In this section we exploit the terminology established before.
\begin{definition}[Theory]\label{teoria}
	A \emph{theory} $\clL$ is the syntactic category $\clT_L$ (cf. \cite{lambek1988introduction}) of a type theory $L$.
\end{definition}
The reader interested in how the construction of $\clT_L$ goes can take from \cite{lambek1988introduction} as standard reference, or \cite{abramskyno} for a shorter survey, in the simple case $L$ admits product and function types.

% From $\clT_L$ it is possible to carve a cartesian closed category, whose obects are denoted $\trn{B}$ as $B$ runs over the basic types of $L$. Then, we define recursively a \emph{semantic translation} operation from basic types $\trn{B} := B$ and
% \[\trn{T\times U}:=\trn{T}\times\trn{U} \,,\qd \trn{T\to U}:=\trn{T}\Rightarrow\trn{U}\,, \]
% and on typed terms as the set of rules in \cite[1.6.5]{abramskyno}.

% The resulting categorical semantics is \emph{sound}, i.e.
% \[ t=_\lambda u \implies \trn{t}=\trn{u} \]
% for every term $t,u$, and $=_\lambda$ is $\lambda$-reduction. It is however not \emph{complete}, in the sense that
% \[ \lsem t \rsem = \lsem u \rsem \text{\qd yet \qd} t\neq_\lambda u\,. \]
% It is possible to build a new category $\clC_\lambda$ providing a sound and complete term model of (a fragment of simply-typed) $\lambda$-calculus; yet, the specifics of this construction are of little interest for the present discussion.
\begin{definition}[World, Yuggoth]\label{mondo_yalda}
	A \emph{world} is a large category $\clW$; a \emph{Yuggoth}\footnote{\emph{Yuggoth} (also \emph{Iukkoth}, or {\yugg} in Chtuvian language) is an enormous trans-Neptunian planet whose orbit is perpendicular to the ecliptic plane of the solar system. A Yuggoth is a world so big to inspire a sense of unfathomable awe.} is a world that, as a category, admits all small colimits.
\end{definition}
\begin{definition}[Canvas, science]\label{canvas_scienza}
	Given a theory $\clL$ and a world $\clW$, a $\clL$-\emph{canvas} of $\clW$ is a functor
	\[\xymatrix{\clL \ar[r]^\phi & \clW.}\]

	A canvas $\phi : \clL \to \clW$ is a \emph{\science} if $\phi$ is a dense functor.
\end{definition}
\begin{remark}\label{remark_yuggoth_1}
	The NR paradigm exposed in \autoref{nr_para} now entails that given a canvas $\phi : \clL \to \clW$
	\begin{itemize}
		\item If $\clW$ is a world, we obtain a \emph{representation} functor
		      \[ \xymatrix{\clW \ar[r] & [\clL^\op, \Set];} \label{mvndvs}\]
		      this means: given a canvas $\phi$ of the world, the latter leaves an image on the canvas.
		\item If in addition $\clW$ is a Yuggoth, we obtain a NR-adjunction
		      \[\xymatrix{\clW \ar@<3pt>[r] & \ar@<3pt>[l] [\clL^\op, \Set];}\]
		      this has to be interpreted as: if $\clW$ is sufficiently expressive, then models of the theory that explains $\clW$ through $\phi$ can be used to acquire a two-way knowledge. Phenomena have a theoretical counterpart in $[\clL^\op,\Set]$ via the nerve; theoretical objects strive to describe phenomena via their realisation.
		\item If an $\clL$-canvas $\phi : \clL \to \clW$ is a \science, `the world' is a full subcategory of the class of all modes in which `language' can create interpretation.
	\end{itemize}
\end{remark}
\begin{remark}\label{remark_yuggoth_2}
	The terminology is chosen to inspire the following idea in the reader: science strives to define \emph{theories} that allow for the creation of world representations; said representations are descriptive when there is dialectic opposition between world and models; when such representation is faithful, we have reduced `the world' to a piece of the models created to represent it.

	The tongue-in-cheek here is, a science (in the usual sense of the world) can never attain the status of a \science, if not potentially; attempts to generate scientific knowledge are the attempts of
	\begin{itemize}
		\item recognizing the world $\clW$ as a sufficiently expressive object for it to contain phenomena and information;
		\item carve a language $L$, if necessary from a small subset of $\clC$, that is sufficiently `compact', but also sufficiently expressive for its syntactic category to admit a representation into the world;
		\item obtaining an \emph{adjunction} between $\clW$ and models of the worlds obtained as models of the syntactic theory $\clL$; this is meant to generate models starting from observed phenomena, and to predict new phenomena starting from models;
		\item obtaining that `language is a dense subset of the world', by this meaning that the adjunction outlined above is sufficiently well-behaved to describe the world as a fragment of the semantic interpretations obtained from~$\clL$.
	\end{itemize}
	It is evident that there is a tension between two opposite feature that $\clL$ must exhibit; it has to be not too large to remain tractable, but on the other hand it must be large enough in order to be able to speak about `everything' it aims to describe.
\end{remark}
Regarding our definition of \science, we can't help but admit we had this definition in mind \cite[2.1]{biologia}:
\begin{definition*}[\protect{\cite[2.1]{biologia}}]
	A \emph{scientific theory} $\clT$ consists of a formal structure $F$ and a class of interpretations $M_i$, shortly denoted as $\clT=\langle F,M_i\mid i\in I\rangle$. The structure $F$ consists on its won right of
	\begin{itemize}
		\item a language $\clL$, in which it is possible to formulate propositions. If $\clL$ is fully formalised, it will consist of a finite set of symbols, and a finite set of rules to determine which expressions are well-formed. This is commonly called \emph{technical language};
		\item A set $A$ of `axioms' or `postulates' in $\clL^\star$;
		\item A \emph{logical apparatus} $R$, whose elements are rules of inference and logical axioms, allowing to prove propositions.
	\end{itemize}
\end{definition*}
The language of category theory allows for a refined rephrasing of the previous definition: we say that a \emph{$\clS$-scientific theory} is the following arrangement of data:
\begin{enumtag}{st}
	\item a formal language $\clL$;
	\item the syntactic category $T_\clL$, obtained as in \cite{lambek1988introduction};
	\item the category of functors $[T_\clC, \clS]$, whose codomain is a Yuggoth.
\end{enumtag}
More than often, our theories will be $\Set$-scientific: in such case we just omit the specification of the semantic Yuggoth, and call them \emph{scientific theories}.

Since the category $[T_\clC, \Set]$ determines $\clL$ and $T_\clL$ completely, up to Cauchy-completion \cite{borceuso-cauchy}, we can see that the triple $(\clL, T_\clL, [T_\clL,\Set])$ can uniquely be recovered from its model category $[T_\clC, \Set]$. We thus comply to the additional abuse of notation to call `scientific theory' the category $[T_\clL,\Set]$ for some $T_\clL$.

So, a `coherent correspondence linking expressions of $\clF$ with semantic expressions' boils down to a functor; this is compatible with \cite[2.1]{biologia}, and in fact an improvement (the mass of results in category theory become readily available to speak about --scientific-- theories; not to mention that the concept of `formal structure' is never rigorously defined throughout \cite{biologia}).

%\subsection{The first rough attempt of some infant god}
Let us consider two categories $\clO,\clT$, respectively the \emph{observational} and the \emph{theoretical}. Even though their origin is never examined further, it is fruitful to think that $\clO,\clT\subseteq \clW$, i.e. that they are `carved' from the world, building respectively on the tangible experience (for $\clO$) and on a linguistic structure (for $\clL$).

If $\clW$ is a Yuggoth each pair of canvases
\[ \xymatrix{
		\clO \ar[r]^-\psi & \clW & \ar[l]_-\phi \clT
	} \label{2_canvases}\] gives rise, according to \eqref{mvndvs}, to representations
\[ \xymatrix{
	[\clO^\op,\Set] \ar@<.5em>[r] & \ar@<.5em>[l]^-{N_\psi} \clW \ar@<-.5em>[r]_-{N_\phi} \ar@{}[r]|-\perp \ar@{}[l]|-\perp & \ar@<-.5em>[l] [\clT^\op,\Set]
	} \label{2_reps}\]
The leftmost category is the category we have experimental access, starting from the fragment $\clO \subseteq \clW$ we can observe. The rightmost category is the category of symbols we can speak of, trying to reproduce the observed behaviour.
\begin{definition}
	We refine the terminology introduced above to speak of a \emph{theoretical} (resp., a \emph{observational}) \emph{science}, assuming that $\phi$ (resp., $\psi$) is a \science.
\end{definition}
Assuming that $\phi : \clT \to \clW$ is a theoretical \science, now, the representation functor $\clW \to [\clT^\op,\Set]$ above acquires a left adjoint.