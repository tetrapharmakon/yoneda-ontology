\documentclass{amsart}

\makeatletter
\def\@settitle{\begin{center}%
  \baselineskip14\p@\relax
  \bfseries
  \uppercasenonmath\@title
  \@title
  \ifx\@subtitle\@empty\else
     \\[1ex]\uppercasenonmath\@subtitle
     \footnotesize\mdseries\@subtitle
  \fi
  \end{center}%
}
\def\subtitle#1{\gdef\@subtitle{#1}}
\def\@subtitle{}
\makeatother

\author{Fouche and Denta}
\title{Categorical ontology I}
\subtitle{identity and Yoneda lemma}
\usepackage{fouche}
\setlength{\epigraphwidth}{.6\textwidth}
\begin{document}
\maketitle
\section{Introduction}
\epigraph{\textsc{Eadem svnt qvorvm vnvm potest svbstitvi alteri per transmvtationem}}{Leibniz, if he knew category theory}
Serve un paragrafo che spieghi diffusamente dove prendiamo le categorie, e di conseguenza da dove deriviamo il metalinguaggio in cui ha senso la nozione di uguaglianza (metalinguistica).
\section{Preliminaries on variable set theory}
A first result is the equivalence of $I$-variable sets and the slice $\Set/I$: 
\begin{proposition}
  
\end{proposition}
\begin{proof}
  
\end{proof}
\begin{remark}
  A more abstract look at this construction regards it as a particular instance of the \emph{Grothendieck construction}: the category of functors $\clC\to\Set$ is equivalent to the category of \emph{discrete fibrations} on $\clC$. In this case, $I$ is regarded as a discrete category, so that all functors $\clE \to I$ are discrete fibrations.
\end{remark}
We observe that the category of variable sets is a topos: we break the result into the verification of the various axioms:
\begin{proposition}
$\Set/I$ is cartesian closed.
\end{proposition}
\begin{proof}
  
\end{proof}
\begin{proposition}
$\Set/I$ has a subobject classifier.
\end{proposition}
\begin{proof}
  
\end{proof}
\begin{proposition}
$\Set/I$ is cocomplete and accessible.
\end{proposition}
\begin{proof}
  
\end{proof}
Accessibility is a corollary of Yoneda in the following form: every $F : I \to \Set$ is a colimit of representables
\[
F \cong \colim\Big(\clE(F) \xto{\Sigma} I \xto{y} \Set^I\Big)  
\]
($\clE(F)$ is small because in this case $\clE(F)\cong\coprod_{i\in I}Fi$)
\begin{itemize}
  \item classifo dei sottoggetti
  \item internal language
\end{itemize}
\section{}
\section{Nine copper coins}
\epigraph{Explicaron que una cosa es igualdad y otra identidad y formularon una especie de reductio
ad absurdum, o sea el caso hipotético de nueve hombres que en nueve sucesivas noches
padecen un vivo dolor. ¿No sería ridículo -interrogaron- pretender que ese dolor es el
mismo?}{JLB ---Tl\"on, Uqbar, Orbis Tertius}
\end{document}