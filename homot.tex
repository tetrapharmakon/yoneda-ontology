\documentclass{amsart}
\author{Fouche and Denta}
\title{No identity without homotopy}
\usepackage{fouche}
\begin{document}
\section{Introduction}
\section{Homotopy theory}
l'80\% del paper mi sa che si scrive scopiazzando dalla mia vecchia nota per i filosofi. Questa parte deve introdurre
\begin{itemize}
  \item il programma di Klein 
  \item la visione group-teoretica e come essa ha pavezzato la strada alle categorie
  \item le categorie e la nozione di omotopia/struttura modello
  \item \dots tutta roba che già c'è nel documento per i viennesi.
\end{itemize}
\section{No identity without homotopy}
\section{Two exactly similar spheres}
\epigraph{Le soleil est un globe froid, solide et homogène. Sa surface est divisée en carrés d’un mètre, qui sont les bases de longues pyramides renversées, filetées, longues de 696999 kilomètres, les pointes à un kilomètre du centre. }{A. Jarry ---Deuxième lettre à lord Kelvin.}
\end{document}