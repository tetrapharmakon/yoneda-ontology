\documentclass{amsart}

\usepackage{fouche}
% \usepackage{braket}
% \makeatletter
% \def\@settitle{\begin{center}%
%   % \baselineskip14\p@\relax
%   % \bfseries
%   % \uppercasenonmath\@title
%   \@title
%   \ifx\@subtitle\@empty\else
%      \\[1ex]\uppercasenonmath\@subtitle
%      \footnotesize\mdseries\@subtitle
%   \fi
%   \end{center}%
% }
% \def\subtitle#1{\gdef\@subtitle{#1}}
% \def\@subtitle{}
% \makeatother

\newcommand{\var}[3][]{%
\left[%
\begin{smallmatrix}
  #2            \\%
  #1 \downarrow \\%
  #3%
\end{smallmatrix}%
\right]%
}

\newenvironment{xsmallmatrix}[1]
  {\renewcommand\thickspace{\kern#1}\smallmatrix}
  {\endsmallmatrix}
%
\newcommand{\cvar}[3]{%
\left[
\begin{xsmallmatrix}{0em}%
  & #1         \\ 
  #2 & \downarrow \\ 
  & #3%
\end{xsmallmatrix}%
\right]
}

\newcommand{\po}[1][dr]{\save*!/#1+1.5pc/#1:(1,-1)@^{|-}\restore}
\newcommand{\pb}[1][dr]{\save*!/#1-1.5pc/#1:(-1,1)@^{|-}\restore}
\def\theory#1{\textsf{#1}}
\def\CT{\theory{CT}\@\xspace}
\setlength{\epigraphwidth}{.6\textwidth}
\setcounter{tocdepth}{1}
%
\def\ra{\rangle}
\def\la{\langle}
\def\lr#1#2{\la #1,#2\ra}
\def\tr{\textsf{t}}
\newcommand{\true}{\texttt{t}}
\def\id{\text{id}}

% %== authors' info
\author{Blinded authors}
%
%
%
% \author{Dario Dentamaro}
% \address{ Dario \textsc{Dentamaro}: Università degli Studi di Firenze, Dipartimento di Matematica e Informatica
%         }
% \email{dario.dentamaro@stud.unifi.it}
% \email{dariodentamaro26@gmail.com}
% %
% \author{Fosco Loregian (corresponding author)}
% \address{ Fosco \textsc{Loregian} (corresponding author): %
%           Tallinn University of Technology, %
%           Institute of Cybernetics, Akadeemia tee 15/2, %
%           12618 Tallinn, Estonia%
%         }
% \email{fosco.loregian@taltech.ee}
% \email{fosco.loregian@gmail.com}



%== metadata
\title{Categorical Ontology I: Existence}

\subjclass{18-03, 18B25, 18C10}

\usepackage{ xspace
           , proof
           , fontawesome
           , listings
           , anyfontsize
           , tikzsymbols
          %  , fontspec
           }
\usepackage{devanagari}

\def\eye{\text{\faEye}}
\def\noeye{\text{\faEyeSlash}}

% \renewcommand*{\ttdefault}{cmvtt}

\newenvironment{italian}{\color{green!60!black}}{\color{black}}


\def\tlon{Tl\"on\@\xspace}
\def\tlonian{Tl\"onian\@\xspace}
\def\tlonians{Tl\"onians\@\xspace}

\newcommand{\dia}[1]{
\begin{tikzpicture}
\node[draw,rounded corners=1, thin, inner sep=1pt] {\tiny #1};
\end{tikzpicture}
}

\def\S{\dia{S}}
\def\M{\dia{M}}
\def\Tu{\dia{Tu}}
\def\W{\dia{W}}
\def\Th{\dia{Th}}
\def\F{\dia{F}}
\def\Sa{\dia{Sa}}
\usepackage{CJKutf8}

\newcommand{\entails}[1]{\;\mathrel{\sdtstile{\scriptscriptstyle #1}{}}\;}

\newcommand{\bsmat}{\left[\begin{smallmatrix}}
  \newcommand{\esmat}{\end{smallmatrix}\right]}
%{\mathrel{\ooalign{$\vDash$\cr\hfil$\raisebox{-2pt}{$\scriptscriptstyle #1$}$\hfil}}}
% \usepackage{fontspec}
% \newfontfamily{\dev}[Script=Devanagari]{Halant Regular}

\makeatletter
\@addtoreset{footnote}{section}
\makeatother

\newenvironment{fo}[1]{\color{green!60!black} \textbf{Fouche said}: #1}{}
\newenvironment{de}[1]{\color{green!60!black} \textbf{Denta said}: #1}{}

\newcommand{\topicRule}{\todo[cyan, inline]{}}
\usepackage{booktabs}
\begin{document}
\begin{abstract}
  The present paper approaches ontology and metaontology through mathematics, and more precisely through category theory. We exploit the theory of \emph{elementary toposes} to claim that a satisfying ``theory of existence'', and more at large ontology itself, can both be obtained through category theory. In this perspective, an \emph{ontology} is a mathematical object: it is a category, the universe of discourse in which our mathematics (intended at large, as a theory of knowledge) can be deployed. The \emph{internal language} that all categories possess prescribes the modes of existence for the objects of a fixed ontology/category.

  This approach resembles, but is more general than, fuzzy logics, as most choices of $\clE$ and thus of $\Omega_\clE$ yield nonclassical, many-valued logics.
  
  Framed this way, ontology suddenly becomes more mathematical: a solid corpus of techniques can be used to backup philosophical intuition with a useful, modular language, suitable for a practical foundation. As both a test-bench for our theory, and a literary \emph{divertissement}, we propose a possible category-theoretic solution of Borges' famous paradoxes of Tl\"on's ``nine copper coins'', and of other seemingly paradoxical construction in his literary work. We then delve into the topic with some vistas on our future works.
\end{abstract}
\maketitle

\vspace*{\fill}
  \tableofcontents
  \section{Introduction}\label{sec_intro}
\epigraph{El mundo, desgraciadamente, es real.}{\cite{confutacion}}
This is the first chapter of a series of works aiming to touch a pretty wide range of topics.

Its purpose is to adopt a wide-ranging approach to a fragment of elementary problems in a certain branch of contemporary philosophy of Mathematics. More in detail, we attempt at laying a foundation for solving a number of problems in ontology employing pure Mathematics; in particular, using the branch of Mathematics known as \emph{category theory}.

As authors, we are aware that such an ambitious statement of purpose must be adequately motivated, bounded to a realistic goal, and properly framed in the current state of the art on the matter. This is the scope of the initial section of the present first manuscript.
\subsection{What is this series}
Since forever, Mathematics studies three fundamental indefinite terms: \emph{form}, \emph{measure}, and \emph{inference}. Apperception makes us recognise that there are extended entities in space, persisting in time. From this, the necessity to measure how much these entities are extended, and to build a web of conceptual relations between them, explaining how they arrange `logically'.
Contamination between these three archetypal processes is certainly possible and common: in fact, Mathematics happens exactly at the crossroad where algebra, geometry and logic intersect.

We can even say more: Mathematics is a language engineered to systematically infer properties of the three mentioned indefinite; so, meta-Mathematics done through Mathematics (if such a thing even exists) exhibits the features of a \emph{ur-language}, a generative scheme for `all' possible languages. It is a language whose elements are the rules to give oneself a language, convey information to other selves, and allow deduction. It is a meta-object, a scheme of rules to generate objects/languages.

Taken this tentative definition, Mathematics (not its history, not its philosophy, but its \emph{practice}) serves as a powerful tool to tackle the essential questions of ontology: what things are cogently, what makes them what they are and not different.

Quantitative thinking is a consistent (some would say `honest') way of approaching the deep problem of the cogency of entities; yet, it is undeniable that a certain philosophical debate has become indifferent, or even overtly hostile, to mathematical language. A tear in the veil that occurred a long time ago, due to different purposes and different specific vocabulary, can not be repaired by two people only. If, however, the reader of these notes asks for an extended motivation for our work, a wide-ranging project in which it fits, a long-term goal, in short, a \emph{program}, they will find it now: there is a piece of Mathematics whose purpose is to solve philosophical problems, in the same sense certain Mathematics `solves' the motion of celestial bodies.

It does not annihilate the question: it proposes models within which one can reformulate it; it highlights what is a trivial consequence of the axioms of that model and what instead is not, and requires the language to be expanded, modified, sharpened. We aim to approach this never-mentioned discipline as mathematicians. But we do it without elaborating `new' theorems; we draw connections between the modern mathematical practice to use them in the context of philosophical research.

Sure, solving once and for all the problems posited by ontology would be megalomaniac; we do not claim to have reached such an ambitious objective. More modestly, we propose a starting point unhinging some well-established beliefs;\footnote{Above all else, the belief that ontology is too general to be approached quantitatively, and that it contains mathematical language as a proper subclass: it is instead the exact opposite, as the central idea of our work is that ontologies -there are many- are mathematical objects.} we humbly point the finger at some problems that prose is unable to notice, because it lacks a specific and technical language; we suggest that only in such a language, when words mean precise things and are tools of epistemic research instead of mere magic spells, a few essential questions of recent ontology dissolve in a thin thread of smoke, and others simply become `the wrong question': not false, just meaningless; meaningless as the endeavour to bestow a concrete attribute as a temperature to abstract concept like consciousness or justice.

We shall say at the outset that the ur-language we are tackling is not Mathematics. Yet, mathematical language has an enormous potential to hint at what the elementary particles of the \emph{characteristica universalis} we're looking for should be made of.

It may also seem suspicious to employ Mathematics to tackle questions that traditionally pertain to philosophy: some philosophers believe they are debating about problems more general than Mathematics, from a higher point of observation on the fundamental questions of Being. By proposing the `ontology as categories' point of view, our work aims to dismantle such a false belief.%: each and every debate must appear in a language.

In doing so, we believe we can provide a more adequate language, taken from Mathematics, within which to frame some deep ontological questions, whose potential is often lost in the threads of ill-posed questions and clumsy answers. The reader will allow a tongue-in-cheek here, subsuming our position: in ontology, it is not a matter of making a \emph{correct use of language}, but rather a matter of \emph{using the correct language}.

This correct language is inherently mathematical, as only Mathematics proved to be able to substantiate a qualitative analysis in quantitative terms. This language `must be' \emph{category theory}, as only category theory has the power to speak about a totality of structures of a given kind in a compelling way, treating mathematical \emph{theories} as mathematical \emph{objects}.

As of now, our work unravels in three different chapters, and it will attempt to cover a variety of topics: 
\begin{itemize}
    \item the present manuscript, \emph{Existence}, provides the tools to build a sufficiently expressive `theory of existence' inside a category. This first chapter has a distinctly foundational r\^ole; its scope to build the fundamentals of our tool-set (category theory and categorical logic, as developed in \cite{mac1992sheaves,JohnstonePT,lambek1988introduction}). 
    
    As both a test-bench for our theory and a literary \emph{divertissement}, we propose a category-theoretic solution of Borges' paradoxes present in \cite{Borges1963}. In our final section, we relate our framework to more classical ancient and modern philosophers; we link topos theory to Berkeley's instantaneism and internal category theory to Quine's definition of the [domain of] existence of an entity as a domain of validity of quantifiers (intended as propositional functions, i.e. functions whose codomain is a space of truth values).
    \item A second chapter \cite{black}, currently in preparation, addresses the problem of \emph{identity}, and in particular its context-dependent nature. Our proof of concept here consists of a rephrasing of Black's classical `two spheres' paradox \cite{papear_di_black} in the elementary terms of invariance under a group of admissible transformations; this time the solution is provided by Klein's famous \emph{Erlangen program} group-theoretic foundation for geometry: the two interlocutors of Black's imaginary dialogue respectively live in an Euclidean and an affine world: this difference, not perceived by means of language, affects their understanding of the `two' spheres, and irredeemably prevents them from mutual intelligence.
    \item A third chapter \cite{homot}, currently in preparation, addresses again the problem of identity, but this time through the lens of algebraic topology, a branch of Mathematics that in recent years defied well\hyp{}established ontological assumptions ; the many commonalities between category theory and homotopy theory suggest that `identity' is not a primitive concept, but instead depends on our concrete representation of mathematical entities. This can be formalised in various ways, among many the \emph{Homotopy Type Theory} foundation of \cite{hottbook,cwp}. \cite{homot} aims to be an introduction to the fundamental principles of HoTT \emph{ad usum delphini}: we investigate how when $X,Y : \clC$ are objects in a category, there often is a class of equivalences $W \subseteq \hom(\clC)$ prescribing that $X,Y$ shouldn't be distinguished in the associated ontology; equality is then defined \emph{ex post} in terms of $W$, varying as the ambient category $\clC$ does; this yields a $W$-parametric notion of identity $\equiv_W$, allowing categories to be categorified versions of \emph{Bishop sets}, i.e. pairs $(S,\rho)$ where $\rho$ is an equivalence relation on $S$ prescribing a $\rho$-equality.
\end{itemize}
Our main tenet in the present chapter is that ontolog\emph{ies} are mathematical objects: each ontology is a certain category $\clO$, inside which `Being' unravels as the sum of all statements that the internal language (see \ref{da_lang}) of $\clO$ can concoct. 

Of course, the more expressive is this language, the more expressive the resulting theory of existence will turn out to be. Our presupposition here is that trying to let ontology speak about `\emph{all} that there is' (the accent is on the adverb, on the famous quote of Quine \cite{quine1948there}) can lead to annoying paradoxes and foxholes.

Instead research shall concentrate on clarifying what the verb means: in what sense, `what there is' \emph{is}? What is is-ness? As category theorists, our -perhaps simplistic- answer is that, again paraphrasing Quine, 
\begin{quote}
being is \emph{being the object of a category}.
\end{quote}
Explaining why this is exactly Quine's motto, just shifted one universe higher, is the content of our §\ref{metaon}.
\subsubsection{Structure of the paper}
The remaining part of the first section draws a picture as accurate as possible, of the wheres and whys of structural Mathematics; its implications are the subject of several essays on the philosophy of Mathematics, like \cite{kromer2007tool,Marquis1997,marquis2010category,marquis2008geometrical}. This section has several different purposes: it provides an explicit statement of purposes for the entire polyptych \cite{black,homot}; it declares our stance on the foundation we choose, clarifying assumptions that we feel are usually neglected on essays on the topic (`where are the objects that Mathematics aims to describe? Where is the language by means of which this description is possible?') -of course without claiming to have solved the matter once and for all; we provide pointers as specific as possible in order to help the reader navigate the relevant literature.

The second section delves into the first major point of our presentation: large categories are universes where `Being', intended as the sum of information acquired from the perceptual bundle we experience, that language organises and conceptualises. `Language' here is a shortcut to denote the power of a fixed large category $\clC$ to express well-formed formulas of (a certain fragment of) logic. Objects and morphisms of $\clC$ shall be considered respectively as types and terms of a language, \emph{the} internal language of $\clC$ (see \autoref{da_lang}); now, the richer $\clC$ is, the more it is able to faithfully represent the cosmos we're thrown into. Among many possible choices for $\clC$, we take \emph{toposes} as the class of categories harbouring `set theories': the internal language of a topos is powerful enough to re-enact set theory, and subsequently propositional logic.

The third section deals with a specific example of a topos, useful for later examination: the categories of objects parameterised by a fixed `space of parameters' $I$. The `slice' category $\Set/I$ is a topos, and its internal language, namely its internal logic, is tightly linked to set-theoretic properties of the slicing set $I$. The logic we obtain in $\Set/I$ by casting the general definitions of subobject classifier \ref{eletop} and internal language is genuinely non-classical.

The fourth section contains a careful analysis of the internal language of $\Set/I$.

The fifth section contains an application of the tools we exposed until now: the seemingly paradoxical `nine copper coins' problem exposed in Jorge Luis Borges' \cite{Borges1963}, far from being paradoxical, admits a natural interpretation as a statement in $\Set/I$, for a suitable choice of $I$, and thus of the induced internal logic. We propose other examples of seemingly paradoxical statements in Borges' literary work that instead are admissible statements in the internal logic of \emph{some} topos: on Tl\"on entities may disappear if neglected: this means that $I$ is linearly ordered; Babylon's chaotic lottery resembles, with their obscure, impenetrable purposes, the chaotic behaviour of a dynamical system: this means that $I$ carries a semigroup action; Tl\"on's instantaneism, mimicking/mocking Berkeley, can be obtained assuming $I$ is a discrete, uncountable set.

We close the paper with a section on future development, vistas for future applications, and with a wrap-up of the discussion as we have unraveled so far. Ideally, §\ref{sec_prelim} and §\ref{int_lang} shall be skipped by readers already having some acquaintance with category theory; the second half of the paper makes however heavy use of the notation established before.
\subsection{On the choice of a meta-theory and a foundation}
Along the 20$^\text{th}$ century, the discipline of Mathematics divided into different sub-classes, each with their specific problems and its specific language, just to find, soon after, unification under a single notion of \emph{structure}, through the notion of abstract category \cite{gtone}. This process led to an epistemological revision of Mathematics and has inspired, parallel to the development of operative tools, a revision of both the foundations of Mathematics and the purposes of its research. 

According to many, it is undeniable that
\begin{quote}
    [the] mathematical uses of the tool `category theory' and epistemological considerations having category theory as their object cannot be separated, neither historically nor philosophically. \cite{kromer2007tool}
\end{quote}
Structural-mathematical practice, i.e. the practice of everyday Mathematics directed by structural meta-principles, produced a `natural' choice for the underlying meta-ontology of Mathematics\footnote{Perhaps improperly, the locution \emph{meta-ontology of Mathematics} is used here to refer to the totality of operative beliefs inspiring the ergonomic of mathematical objects. Some of these principles are: objects not enjoying a universal property shall be discarded; definitions that are isomorphism-invariant shall be preferred over those who are not; both these commandments are based on the idea that classes of mathematical objects arrange in coherent conglomerates exhibiting more structure than the mere aggregation of their elements: the requests of universality and isomorphism-invariance are meant not to destroy such additional structure. Examples of these meta-principles can be found in various other areas of Mathematics.} which, later, felt the need to be characterized more precisely. Similarly to what Carnap\footnote{Some words that philosophers should keep in mind, on the lawfulness of the use of abstract entities (specifically mathematical) in semantic reflection, also valid in ontology:
    \begin{quote}
        we take the position that the introduction of the new ways of speaking does not need any theoretical justification because it does not imply any assertion of reality [...].  it is a practical, not a theoretical question; it is the question of whether or not to accept the new linguistic forms. The acceptance cannot be judged as being either true or false because it is not an assertion. It can only be judged as being more or less expedient, fruitful, conducive to the aim for which the language is intended. Judgments of this kind supply the motivation for the decision of accepting or rejecting the kind of entities. \hfill \cite{carnap1956meaning}
    \end{quote}} suggested regarding semantics,
\begin{quote}
    mathematicians creating their discipline were not seeking to justify the constitution of the objects studied by making assumptions as to their ontology.\hfill  \cite{kromer2007tool}
\end{quote}
Beyond the attempts (above all, those of Bourbaki group: but see \cite{McL}, historical notes on Ch. 4, for a hint that Bourbaki didn't really get the point of structural Mathematics), what matters is that the habit of reasoning in terms of structures has suggested implicit epistemological and ontological attitudes. This matter would deserve an exhausting independent inquiry.

For our objectives it's enough to declare a differentiation that Kr\"omer
elaborated, inspired by \cite{Cor96}: the difference between \emph{structuralism} and \emph{structural Mathematics}:
\begin{enumtag}{s}
    \item \label{s:uno} Structuralism: the philosophical position regarding structures as the subject matter of Mathematics;
    \item \label{s:due} Structural Mathematics: the methodological approach to look in a given problem for the structure itself.
\end{enumtag}
Of course, \ref{s:uno} implies \ref{s:due} but the opposite is not always true:
\begin{remark} \label{weak_structuralism}
    That is, one can do structural Mathematics without being a structuralist and taking different, or even opposite, positions concerning structuralism itself. 

    Nevertheless, the use of \CT as meta-language, despite the historical link with structuralism, doesn't make automatic the transition from \ref{s:due} to \ref{s:uno}; it just suggests that the ontology is not only dependent on the `ideology' (in a Quinean sense) of the theory, but it is instead influenced by the epistemological model inspired by formal language.
\end{remark}
Kr\"omer's distinction, however, has another virtue: instead of stumbling in a possibly not ambiguous definition of \textit{structure} (with the unwanted consequences that could arise in the operational practice), \ref{s:uno} can be reduced to (or can redefine) \ref{s:due}, saying that:
\begin{quote}
    \emph{structuralism is the claim that Mathematics
        is essentially structural Mathematics} \cite{kromer2007tool}
\end{quote}
This is the same thing as saying: the structural practice already is its philosophy.

Attempts to explain the term `structure' by Bourbaki in the years following the publication of the \textit{Elements des Mathématiques}, led to the first systematic elaboration of a philosophy that we could appropriately call \textit{structural Mathematics}. Its target is to `\textit{assembling all possible ways in which given set can be endowed with certain structure}' \cite{kromer2007tool}, and elaborate, in the programmatic paper \textit{The Architecture of Mathematics} (written by Dieudonné alone and published in 1950), a formal strategy. While specifying that `\textit{this definition is not sufficiently general for the needs of Mathematics}' \cite{Bourb50}, the author encoded a series of operational steps through which a structure on a collection is assembled set-theoretically. Adopting therefore a reductionist perspective in which
\begin{quote}
    the structure-less sets are the raw material of structure building which in Bourbaki’s analysis is `unearthed' in a quasi-archaeological, reverse manner; they are the most general objects which can, in a rewriting from scratch of Mathematics, successively be endowed with ever more special and richer structures.\hfill  \cite{kromer2007tool}
\end{quote}
On balance, in Bourbaki's structuralism, the notion of set doesn't disappear definitively in front of the notion of structure. Times were not ripe to abandon set theory; the path towards an `integral' structuralism was still long, and culminated years after, with Lawvere's attempt at a foundation \theory{ETCS} of set theory first \cite{lawvere1964elementary} and \theory{ETCC} of category theory (and as a consequence, `of all Mathematics') after \cite{lajolla}, through structuralism.

To appreciate the depth and breadth of such an impressive piece of work, however, the word `foundation' must be taken in the particular sense intended by mathematicians:
\begin{quote}
    [\dots\unkern] a single system of first-order axioms in which all usual mathematical objects can be defined and all their usual properties proved.
\end{quote}
Such a position sounds at the same time a bit cryptic to unravel, and unsatisfactory; Lawvere's (and others') stance on the matter is that a foundation of Mathematics is \emph{de facto} just a set $\clL$ of first-order axioms organised in a Gentzen-like deductive system. The deductive system so generated reproduces Mathematics as we know and practice it, providing a formalisation for something that already exists and needs no further explanation, and that we call `Mathematics'.

It is not a vacuous truth that $\clL$ exists somewhere: point is, the fact that the theory so determined has a nontrivial model, i.e the fact that it can be interpreted inside a given familiar structure, is at the same time the key assumption we make, and the less relevant aspect of the construction itself.

Showing that $\clL$ `has a model' is --although slightly improperly-- meant to ensure that, \emph{assuming the existence of a naive set theory} (i.e., assuming the prior existence of structures called `sets'), axioms of $\clL$ can be satisfied by a naive set. Alternatively, and more crudely: assuming the existence of a model of \theory{ZFC}, $\clL$ has a model \emph{inside that model of \theory{ZFC}}.\footnote{It shall be made clear, ensuring that a given theory has a model isn't driven by psychological purposes only: on the one hand, purely syntactic Mathematics would be very difficult to parse, as opposed to the more colloquial practice of mathematical development; on the other hand (and this is more important), the only things syntax can see are equality and truth. To prove that a given statement is false, one either has to check all possible syntactic derivations leading to $\varphi$, finding none --this is unpractical, to say the least-- or to \emph{find a model} where $\lnot\varphi$ holds.}

\subsection{Our foundation, at last.} A series of works attempting to unhinge some aspects of ontology through category theory should at least try to tackle such a simple and yet diabolic question as `where' are the symbols forming the first-order theory \theory{ETCC}. And yet, everyone just believes in -some flavour of- sets and solves the issue of `where' they are with a leap of faith from which all else must follow.

This might appear somewhat circular: aren't sets in themselves already a mathematical object? How can they be a piece of the theory they aim to be a foundation of? In his \cite{lolli1977categorie} the author addresses the problem as follows:\footnote{Authors' translation: \emph{When mathematicians talk about models they do not have the impression to have exited a set-theoretic foundation. This impression is correct, and justified by the possibility to represent formal languages through set theory, to study the relations among structures and symbolic representations. When mathematicians turn their attention to models of a set theory, some philosophical questions can however no longer be avoided. The natural question of what is the relation between the sets that are models of a theory, and the sets of which the theory talks about is nothing but a question about the relation between the semantic, set-theoretic meta-theory and the `object theory' we want to talk about. The two theories can coincide, and in fact the meta-theory can even be a proper sub-theory of the object theory.\\
\indent We choose a set whose elements, usually finite sets, represent the symbols of the language we want to study, and then with a concatenation operation that can be the `ordered pair' construction we define the set of words $L$ of the well-formed formulas, of terms, and the operation that associates a free variable with a formula, and so on for all syntactic notions.}}
\begin{quote}
    Quando un matematico parla di modelli non ha [\dots\unkern] l'impressione di uscire dall'ambito insiemistico. Questa impressione, che è corretta, è giustificata dalla possibilità di rappresentare i linguaggi formali con gli oggetti della teoria degli insiemi, di studiare in essa le relazioni tra le strutture e le rappresentazioni dei simboli. Quando l'attenzione è rivolta ai modelli di una teoria degli insiemi, certe questioni sofistiche non possono però essere più evitate. La domanda spontanea sulla relazione che intercorre tra gli insiemi che so­no modelli di una teoria e gli insiemi di cui parla la teoria non è altro che una domanda sulla relazione tra la metateoria semantica insiemi­stica e la teoria in esame, o teoria oggetto. Le due teorie possono coincidere, anzi la metateoria può essere anche una sottoteoria pro­pria della teoria oggetto. [\dots\unkern]

    Si sceglie un insieme i cui elementi, di solito insiemi finiti, rappresentano i simboli del linguaggio che si vuole studiare, quindi con una operazione di concatenazione che può essere la coppia ordinata si definisce l'insieme delle parole $L$ delle for­mule ben formate, dei termini, l'operazione che a una formula asso­cia le sue variabili libere, e così via per tutte le nozioni sintattiche.
\end{quote}
The idea that a subtheory $L'\subset L$ of the object theory can play the r\^ole of meta-theory might appear baffling; in practice, the choice is to rely on one among two possible solutions. Pure Platonism assumes the existence of a hierarchy of universes harbouring the object theory; pure syntacticism exploits G\"odel's completeness theorem: every proof is a finite object, and every theorem proved in the meta-theory is just a finite string of symbols. No need for a model.

Platonism has limits: in a fixed a class theory $\sfC$ ($\sf MK$, Morse-Kelley(-Mostowski); or $\sf NBG$, Von Neumann-Bernays-G\"odel), there's an object $V$ that plays the r\^ole of the universe of sets; in $V$, all Mathematics can be enacted. Of course, consistency of $\sfC$ is only granted by an act of faith.

Syntacticism has limits: following it, one abjures any universality Mathematics might claim. But syntacticism also has merits: undeniably (disgracefully, luckily) the World is real. And reality is complex enough to contain languages as purely syntactical objects; the percussion of a log with oxen bones, rather than prophecies over the entrails of a lamb, or intuitionistic type theory, all have the same purpose: intersubjective convection of meaning, deduced by a bundle of perceptions, so to gain an advantage, \emph{id est} some predictive power, over said perceptions. Of course: intuitionistic type theory is just \emph{slightly} more effective than hepatomancy.

Knowledge is obtained by collision and retro-propagation between Reality and the perceptual bundle it generates.

Accepting this, the urge to define seemingly abstract concepts like learning, conscience, and knowledge, together with precious continuous feedback coming from real objects, evidently determine an undeniable primacy of quantitative thinking, this time intended as machine learning and artificial intelligence, that sets (or should set if only more philosophers knew linear algebra) a new bar for research in philosophy of mind.

However, we refrain from entering such a deep rabbit-hole, as it would have catastrophic consequences on the quality, length, and depth of our exposition.

The usual choice for mathematicians is to assume that, wherever and whatever they are, these symbols `are', and our r\^ole in unveiling Mathematics is \emph{descriptive} rather than generative.\footnote{Inside (say) a constructivist foundation it is not legitimate to posit that axioms `create' mathematical objects; from this, the legitimacy of the question of where they are, and the equally legitimate answer `nowhere'. The only thing we can say is that they `make precise, albeit implicitly, the \emph{meaning} of mathematical objects' \cite{Agzz} (it seems to us that in Mathematics as well as in philosophy of language, meaning and denotation are safely kept separate). We take this principle -that the world/metamodel exists and we can just attempt at describing it by means of the language/model- as evident to anyone in a healthy state of mind, and we leverage it without further question.}

This state of affairs has, to the best of our moderate knowledge on the subject, various possible explanations:
\begin{itemize}
    \item On one hand, it constitutes the heritage of Bourbaki's authoritarian stance on formalism in pure Mathematics;
    \item on the other hand, a different position would result in barely any difference for the `working-class'; mathematicians are irreducible pragmatists, somewhat blind to the consequences of their philosophical stances.
\end{itemize}
So, symbols and letters do not exist outside of the Gentzen-like deductive system we specified together with $\clL$.

As arid as it may seem, this perspective proved itself to be quite useful in working Mathematics; consider for example the type declaration rules of a typed functional programming language: such a concise declaration as
\begin{lstlisting}[ language=Haskell
                  , basicstyle=\ttfamily\small
                  , keywordstyle=\color{blue!75}
                  , morekeywords={S,Z}
                  ]
  data Nat = Z | S Nat
\end{lstlisting}
makes no assumption on `what' \verb|Z| and \verb|S:: Nat -> Nat| are; instead, it treats these constructors as meaningful formally (in terms of the admissible derivations a well-formed expression is subject to) and intuitively (in terms of the fact that they model natural numbers: every data structure that has those two constructors must be the type $\bbN$ of natural numbers -provided data constructors like \verb|S| are all injective).

Taken as an operative rule, this reveals exactly what is our stance towards foundations: we are `structuralist in the meta-theory', meaning that we treat the symbols of a first-order theory or the constructors of a type system regardless of their origin, provided the same relation occur between criptomorphic collections of labelled atoms.

In this precise sense, we are thus structuralists in the meta-theory, and yet we do so with a grain of salt, maintaining a transparent approach to the consequences and limits of this partialisation. On the one hand, pragmatism works; it generates rules of evaluation for the truth of sentences. On the other hand, this sounds like a Munchhausen-like explanation of its the value, in terms of itself. Yet there seems to be no way to do better: answering the initial question `where are the letters of \theory{ETCC}?' would result in no less than a foundation of language.

And this for no other reason than `our' meta-theory is something near to a structuralist theory of language; thus, a foundation for such a meta-theory shall inhabit a meta-meta-theory\dots{} and so on.

Thus, rather than trying to revert this state of affairs we silently comply to it as everyone else does; but we feel contempt after a brief and honest declaration of intents towards where our meta-theory lives. Such a meta-theory hinges again on work of Lawvere, and especially on his series of works on functorial semantics.

  \section{Categories as places}\label{as_places}
The present section has double, complementary purposes: we would like to narrow the discussion down to the particular flavour in which we interpret the word ``category'', but also to expand its meaning so as to encompass its r\^ole as a foundation for mathematics. More or less, the idea is that a category is both an algebraic structure (a microcosm) and a metastructure in which \emph{all} other algebraic structures can be interpreted (a macrocosm).\footnote{As an aside, we shall at least mention the dangers of too much a naive approach towards the micro/macrocosm dichotomy: if all algebraic structures can be interpreted in a category, and categories are algebraic structures, there surely is such a thing as the theory of categories \emph{internal} to a given one. And large categories shall be thought as categories internal to the ``meta''category (unfortunate but unavoidable name) of categories. There surely is a well-developed and expressive theory of internal categories (see \cite[8]{Bor1}); but our reader surely has understood that the two ``categories'', albeit bearing the same name, shall be considered on totally different grounds: one merely is a structure; the other is a foundation for that, and others, structure.}

More in detail, a category provides with a sound graphical representation of the defining operation of a certain type of structure $\fkT$ (see \autoref{unialg} below; we take the word \emph{universal} in the sense of \cite[XV.1]{grillet2007abstract}).
Such a perspective allows to concretely build an object representing a given (fragment of ) a language $L$, and a topos (see \autoref{eletop}) obtained as sort of a universal semantic interpretation of $L$ as internal language. This construction is a classical piece of categorical logic, and will not be recalled in detail (yet, see our \autoref{da_lang}): the reader is invited to consult \cite[II.12, 13, 14]{lambek1988introduction}, and in particular
\begin{quote}
	J. Lambek proposed to use the \emph{free topos} [on a type theory/language] as ambient world to do mathematics in; [\dots\unkern] Being syntactically constructed, but universally determined, with higher-order intuitionistic type theory as internal language, [Lambek] saw [this structure] as a reconciliation of the three classical schools of philosophy of mathematics, namely formalism, Platonism, and intuitionism. \hfill\cite{free_topos}
\end{quote}
Interpretation, as defined in logical semantics \cite{gamut1991logic}, can be seen as a function $t: L^\star \to \Omega$ that associates elements of a set $\Omega$ to the free variables of a formula $\alpha$ in the language $L^\star$ generated by an alphabet $L$; along the history of category theory, subsequent refinements of this fundamental idea led to revolutionary notions as that of functorial semantics and internal logic of a topos.
As an aside, it shall be noted that the impulse towards this research was somewhat motivated by the refusal of set-theoretic foundations, opposed to type-theoretic ones.

In the following subsection, we give a more fine-grained presentation of the philosophical consequences that a ``metatheoretic structural'' perspective has on mathematical ontology.
\subsection{Theories and their models}
In \cite{lajolla} the author W. Lawvere builds a formal language \theory{ETAC} encompassing ``elementary'' category theory, and a theory \theory{ETCC} for the category of all categories, yielding a model for \theory{ETAC}. In this perspective category theory has a syntax \CT; in \CT, categories are terms. In addition, we are provided with a \emph{meta}theory, in which we can consider categories of categories, etc.:
\begin{quote}
	If $\Phi$ is any theorem of elementary theory of abstract categories, then $\forall \clA (\clA \models \Phi)$ is a theorem of basic theory of category of all categories. \hfill \cite{lajolla}
\end{quote}
After this, the author makes the rather ambiguous statement that ``\textit{every object in a world described by basic theory is, at least, a category}''. This is a key observation: what is the world described by \theory{ETAC}, what are its elements?

We posit that the statement shall be interpreted as follows: categories in mathematics carry a double nature. They surely are the structures in which the entities we are interested to describe organise themselves; but on the other hand, they inhabit a single, big (meta)category of all categories. Such a big structure is fixed once and for all, at the outset of our discussion, and it is the \emph{place} in which we can provide concrete models for ``small'' categories.

In other words (and this is true also in light of our \autoref{funsemanzi}), categories live on different, almost opposite, grounds: as syntactic objects, that can be used to model \emph{language}, and as semantic objects, that can be used to model \emph{meaning}.

To fix ideas with a particular example: we posit that there surely is such a thing as ``the category of groups''. But on the other hand, groups are just very specific kinds of sets, so groups are but a substructure of the only category that exists.

Sure, such an approach is quite unsatisfactory in a structural perspective. It bestows the category of sets with a privileged role that it does not have: sets are just \emph{one} of the possible choices for a foundation of mathematics. Instead, we would like to disengage the (purely syntactic) notion of structure from the (semantic) notion of interpretation.

The ``categories as places'' philosophy now provides such a disengagement, in order to approach the foundation of mathematics agnostically.%: it is not important what the foundational model contains, its universal property is important.

In this perspective one can easily fit various research tracks in categorical algebra, \cite{Janelidze2004}, functorial semantics \cite{lawvere1963functorial,hyland2007category}, categorical logic \cite{lambek1988introduction}, and topos theory \cite{JohnstonePT} that characterised the last sixty years of research in category theory.

Lawvere's \emph{functorial semantics} was introduced in the author's PhD thesis \cite{lawvere1963functorial} in order to provide a categorical axiomatisation of universal algebra, the part of mathematical logic whose subject is the abstract notion of mathematical structure: a semi-classical reference for universal algebra, mingled with a structuralist perspective, is \cite{manes2012algebraic}; see also \cite{sankappanavar}. We shall say at the outset that a more detailed and technical presentation of the basic ideas of functorial semantics is given in our \autoref{funsemanzi} below; here we aim neither at completeness nor at self-containment.

Everything starts with the following definition:
\begin{definition}\label{unialg}
	A \emph{type $\fkT$ of universal algebra} is a pair $(T,\underline{\alpha})$ where $T$ is a set called the (\emph{algebraic}) \emph{signature} of the theory, and $\underline\alpha$ a function $T \to \bbN$ that assigns to every element $t: T$ a natural number $n_t: \bbN$ called the \emph{arity} of the function symbol $t$.
\end{definition}
\begin{definition}
	A (\emph{universal}) \emph{algebra} of type $\fkT$ is a pair $(A,f^A)$ where $A$ is a set and $f^A: T \to \prod_{t: T} \Set(A^{n_t},A )$ is a function that sends every function symbol $t: T$ to a function $f^A_t: A^{n_t} \to A$; $f^A_t$ is called the $n_t$-ary operation on $A$ associated to the function symbol $t: T$.
\end{definition}
We could evidently have replaces $\Set$ with another category $\clC$ of our choice, provided the object $A^n: \clC$ still has a meaning for every $n: \bbN$ (to this end, it suffices that $\clC$ has finite products; we call such a $\clC$ a \emph{Cartesian} category). A universal algebra of type $\fkT$ in $\clC$ is now a pair $(A,f^A)$ where $A: \clC$ and $f^A: T \to \prod_{t: T} \clC(A^{n_t},A )$; it is however possible to go even further, enlarging the notion of ``type of algebra'' even more.

The abstract structure we are trying to classify is a \emph{sketch} (the terminology is neither new nor inexplicable: see \cite{ehresmann1968esquisses,coppey1984leccons, Bor2}) representing the most general arrangement of operations $f^A: A^n \to A$ and properties thereof\footnote{Examples of such properties are (left) alternativity: for all $x,y,z$, one has $f^A(x,f^A(x,y)) = f^A(f^A(x,x),y)$; associativity: $f^A(x,f^A(y,z)) = f^A(f^A(x,y),z)$; commutativity: $f^A(x,y)=f^A(y,x)$; and so on.} that coexist in an object $A$; such a sketch is pictorially represented as a (rooted directed) graph, modeling arities of the various function symbols determining a given type of algebra $\fkT$ (see also \cite[XV.3]{grillet2007abstract} for the definition of \emph{variety of algebras}).%: un'algebra universale soggetta a equazioni, ossia una coppia $(A,R)$ ove $A$ è un'algebra di tipo $\fkT$ e $R\subseteq A^\star \times A^\star$ è un sottoinsieme di coppie di parole in un opportuno monoide di Kleene modificato).

Given the ``theory'' $\fkT$ and the graph $G_{\fkT}$ that it represents, the category $\mathcal{L}_{\fkT}$ generated by $G_\fkT$ ``is'' the theory we aimed to study, and every functor $A: \clL_{\fkT} \to \Set$ with the property that $A([n+m]) \cong A[n] \times A[m]$ concretely realises via its image a \emph{representation} of $\clL_\fkT$ (and thus of $\fkT$) in $\Set$.
\begin{remark}\label{rmk_explicit_theoer}
	More concretely, there is a ``theory of groups''. Such a theory determines a graph $G_{\cate{Grp}}$ built in such a way to generate a category $\clL=\clL_{\cate{Grp}}$ with finite products. \emph{Models} of the theory of groups are functors $\clL \to \Set$ uniquely determined by the image of the ``generating object'' $[1]$ (the set $G=G[1]$ is the underlying set, or the \emph{carrier} of the algebraic structure in study; in our \autoref{unialg} the carrier is just the first member of the pair $(A,f^A)$); the request that $G$ is a product preserving functor entails that if $\clL$ is a theory and $G: \clL \to \Set$ one of its models, we must have $G[n]=G^n = G \times G \times\dots\times G$, and thus each function symbol $f: [n]\to [1]$ describing an abstract operation on $G$ receives an interpretation as a concrete function $f: G^n \to G$.
\end{remark}
Until now, we interpreted our theory $\clL$ in sets; but we could have chosen a different category $\clC$ at no additional cost, provided $\clC$ was endowed with finite products, in order to speak of the object $A^n = A\times \cdots\times A$ for all $n: \bbN$. In this fashion, we obtain the $\clC$-models of $\clL$, instead of its $\Set$-models: formally, and conceptually, the difference is all there.

Yet, the freedom to disengage language and meaning visibly has deep consequences: suddenly, and quite miraculously, we are allowed to speak of groups internal to the category of sets, i.e. functors $\clL_{\cate{Grp}} \to \Set$, topological groups, i.e. functors $\clL_{\cate{Grp}} \to \cate{Top}$ (so multiplication and inversion are continuous maps \emph{by this very choice}, without additional requests); we can treat monoids in the category of $R$-modules, i.e. $R$-algebras \cite[IV]{book337527}, and monoids in the category of posets (i.e. quantales \cite{Paseka2000}) all on the same conceptual ground.

It is evident that $\clL_{\fkT}$ is the same in all our choices; all that changes is the semantic universe in which the theory acquires meaning. This context-dependent definition of ``structure'' inspires our stance towards ontology, as the reader will notice throughout our \autoref{sec_coins} and \autoref{vistas}.

\subsection{The r\^ole of toposes}
Among many different Cartesian categories in which we can interpret a given theory $\clL$, toposes play a special r\^ole; this is mostly due to the fact that the \emph{internal language} every topos carries (in the sense of \autoref{da_lang}) is quite expressive.

To every theory $\clL$ one can associate a category, called the \emph{free topos} $\clE(\clL)$ on the theory (see \cite{lambek1988introduction}), such that there is a natural bijection between the $\clF$-models of $\clL$, in the sense of  \autoref{rmk_explicit_theoer}, and (a suitable choice of) morphisms of toposes\footnote{We refrain to enter the details of the definition of a morphism of toposes, but we glimpse at the definition: given two toposes $\clE,\clF$ a morphism $(f^*,f_*): \clE \to \clF$ consists of a pair of \emph{adjoint} functors (see \cite[3]{Bor1}) $f^*: \clE \leftrightarrows \clF: f_*$ with the property that $f^*$ commutes with finite limits (see \cite[2.8.2]{Bor1}).} $\clE(\clL) \to \clF$:
\[\cate{Mod}(\clL, \clF) \cong \hom(\clE(\clL), \clF).\]
In the present subsection we analyse how the construction of models of $\clL$ behaves when the semantics takes value in a category of presheaves (cf. \cite[??]{Bor1}).

Let's start stating a plain tautology, that still works as blatant motivation for our interest in toposes opposed as more general categories for our semantics. Sets can be canonically identified with the category $[1,\Set]$, so models of $\clL$ are tautologically identified to its $[1,\Set]$-models. It is then quite natural to wonder what $\clL$-models become when the semantics is taken in more general functor categories like $[\clC,\Set]$. This generalisation is compelling to our discussion: in case $\clC=I$ is a discrete category, we get back the well-known category of variable sets $\Set/I$ of \autoref{variabbo_set}.

Now, it turns out that $[\clC,\Set]$-model for an algebraic theory $\clL$, defined as functors $\clL \to [\clC,\Set]$ preserving finite products, correspond precisely to functors $\clC\to \Set$ such that each $FC$ is a $\clL$-model: this gives rise to the following ``commutative property'' for semantic interpretation:
\begin{quote}
	$\clL$-models in $[\clC,\Set]$ are precisely those models $\clC \to \Set$ that take value in the subcategory $\cate{Mod}_{\clL}(\Set)$ of models for $\clL$. In other words we can ``shift'' the $\cate{Mod}(-)$ construction in and out $[\clC,\Set]$ at our will:
	\[
		\cate{Mod}_{\clL}([C,\Set]) \cong [C, \cate{Mod}_{\clL}(\Set)]
	\]
\end{quote}
As the reader can see, the procedure of interpreting a given ``theory'' inside an abstract finitely complete category $\clK$ is something that is only possible when the theory is interpreted as a category, and when a model of the theory as a functor. This part of mathematics goes under many names: the one we will employ, i.e. \emph{categorical}, or \emph{functorial}, semantics \cite{lawvere1963functorial}, but also \emph{internalisation} of structures, \emph{categorical algebra}.

The internalisation paradigm sketched above suggests how ``small'' mathematicians often happily develop their mathematics without ever exiting a single (large) finitely complete category $\clK$, without even suspecting the presence of models for their theories outside $\clK$. To a category theorist, ``groups'' as abstract structures behave similarly to the disciples of the sect of the Phoenix \cite{fenix}: ``the name by which they are known to the world
is not the same as the one they themselves pronounce.'' They are a different, deeper structure than the one intended by their users.

By leaving the somewhat unsatisfying picture that ``all categories are small'' and by fixing a semantic universe like $\Set$), every large category works as a world in which one can speak mathematical language (i.e. ``study models for the theory of $\Omega$-structures'' as long as $\Omega$ is one of our abstract theories).

So, categories really exhibit a double nature: they are the theories we want to study, but they also are the \emph{places} where we want to realise those theories;looking from high enough, there is plenty of other places where one can move, other than the category $\Set$ of sets and functions (whose existence is, at the best of our knowledge, consequence of a postulate; we similarly posit that there exists a model for \theory{ETAC}). Small categories model theories, they are \emph{syntax}, in that they describe a relational structure using compositionality; but large categories offer a way to interpret the syntax, so they are a \emph{semantics}. Our stance is that a large relational structure is fixed once and for all, lying on the background, and allows for all other relational structures to be interpreted.

It is nearly impossible to underestimate the importance of this disengagement: syntax and semantics, once separated and given a limited ground of action, acquire their meaning.\footnote{Of course, this is dialectical opposition at its pinnacle, and by no means a sterile approach to category theory; see \cite{lawvere1996unity} for a visionary account of how ``dialectical philosophy can be modeled mathematically''.}

More technically, in our \autoref{da_lang} we recall how this perspective allows to interpret different kinds of logic in different kinds of categories: such an approach leads very far, to the purported equivalence between different flavours of logic\emph{s} and different classes of categories; the particular shape of semantics that you can interpret in $\clK$ is but a reflection of $\clK$'s nice categorical properties (e.g., having finite co/limits, nice choices of factorisation systems \cite[5.5]{Bor1}, \cite{FK}, a subobject classifier; or the property of the posets $\cate{Sub}(A)$ of subobjects of an object $A$ of being a complete, modular, distributive lattice\dots).

In light of \autoref{da_lang}, this last property has to do with the internal logic of the category: propositions are the set, or rather the \emph{type}, of ``elements'' for which they are ``true''; and in nice cases (like e.g. in toposes), they are also arrows with codomain a suitable \emph{type of truth values} $\Omega$).
\subsection{Categories are universes of discourse}
Somehow, the previous section posits that category theory as a whole is ``bigger than the mathematics itself'', and it works as one of its foundations; a category is a totality where all mathematics can be re-enacted; in this perspective, \theory{ETAC} works as a metalanguage in which we develop our approach. This is not very far from current mathematical practice, and in particular from Mac Lane's point of view:
\begin{quote}
	We can [\dots\unkern] ``ordinary'' mathematics as carried out exclusively within [a universe] $U$ (i.e. on elements of $U$) while $U$ itself and sets formed from $U$ are to be used for the construction of the desired large categories.\hfill \cite[I.6]{McL}
\end{quote}
since the unique large category we posit is ``the universe''. But this approach goes further, as it posits that \emph{mathematical theories are in themselves mathematical objects}, and as such, subject to the same analysis we perform on the object of which those theories speak about.

All in all, our main claim is that categories are of some use in clarifying a few philosophical problems, especially for what concerns the objects of discourse of ontology. The advantages of this approach are already visible in mathematics; we are trying to export them in the current philosophical debate.

Among many such advantages, we record those that are more influential to our work:
\begin{itemize}
	\item the possibility of reading theories in terms of relations: this allows to suspend our ontological commitment on the nature of objects. They are given, but embedded in a relational structure; this relational structure, i.e. a category modeling the ontology in study, is the object of our discourse. Even more: the two entities should not be considered as distinct: an ontology \emph{is} a category. This perspective is sketched in our \autoref{vistas}.
	\item a sharper, more precise, and less time-consuming conceptualisation process for ontology's subjects of study; if ontology is regarded as (the internal language of) a category, it becomes a \emph{context-dependent} entity, i.e. a function of the relational structure in which it is located.
	\item Ontology can be approached with a \emph{problem solving} attitude: answers to ontological problems are at least to some extent quantitatively determined; see our \autoref{sec_coins} for a roundup of examples in this direction.
	\item A weak form of structuralism, kept at an informal, metatheoretic level, is more than enough to appreciate how objects interact between each other inside a theory, and to deduce how they should behave by just applying the basic constructions of elementary mathematics coupled with category theory. To put it simple, the reader isn't forced to unconditionally adopt an hardcore-structuralist foundation such as e.g. the Structural Realism of \cite{bain2013category,eva2016category}; a similar attitude towards foundations is widespread in mathematical practice: many people do \textit{structural mathematics} without even knowing, or without worrying about, the subtleties of what a purely structuralist foundation implies. And this for the simple reason that structural mathematics \emph{works}: structural theories are easier to grok, they are modular, resilient to change, concise.
	\item In our system, ontology does not pose absolute questions: beings exist \emph{in a context}, relative to a relational structure (=a category) they are the objects of. Something is certainly lost: we claim that what is lost is a complicated and confused tangle of misunderstandings, caused by the habit of confusing natural and formal language. A fortunate analogy still involves foundations of Mathematics: bounding the existence of sets to a hierarchy of universes undoubtedly loses something. But what is this something? How can it be probed? And then, in what sense it ``exists''?
	      % Come conseguenza fondamentale del punto precedente nessun concetto viene studiato in senso ``assoluto'' (qualunque cosa ciò significhi) ma relativamente al contesto in cui opera, e alla teoria che stiamo adottando per definirlo.
\end{itemize}
Our main point is that the majority of, if not all, theories of existence seem burdened by the impossibility to disentangle existence from a ``context of existence''. Instead, the super-exponential progress of mathematics along the XX$^\text{th}$ century can only be explained by the general acceptance of the belief that things do not ``exist'' in themselves; instead they do as long as they are meaningful pieces of relational structures. Sure structuralism is a single philosophical current among many: it is ``the right one'', since it profoundly reshaped mathematics (=the prominent, most effective factory of epistemic knowledge humankind has ever produced).

As extremist as it may seem, our position fits into an already developed open debate; see for example the work of J.P. Marquis:
\begin{quote}
	[...] \emph{to be} is \emph{to be related}, and the ``essence'' of an ``entity'' is given by its relations to its ``environment''.
	\hfill \cite{Marquis1997}
\end{quote}
Sure, such a point of view is only acceptable when its fruitfulness has been determined by clarifying and formalising the relationship between an ontology regarded as ``all that there is'', and a (large) category as ``the place we inhabit (and in which we craft language)''. Our purpose is to add a grain of sand to this fertile area of discussion.

As mentioned in our \autoref{} these conceptual tools helped mathematical practice and has inadvertently inspired an epistemological and ontological quest about its objects of study. Someone might object that we should first know what is a structure before working with it; Kr\"omer's reply seems particularly compelling:
\begin{quote}
	this reproach is empty and one tries to explain the clearer by the more obscure when giving priority to ontology in such situations [...]. Structure occurs in the dealing with something and does
	not exist independently of this dealing. \cite{kromer2007tool}
\end{quote}
Category theory was invented without worrying about foundational issues; they were addressed just later. It effectively and faithfully involved all mathematics; we now claim structuralism provides a ``practical foundation'' for metaphysical problems. For pragmatism and what it means in category theory, see
\begin{quote}
	that structural mathematics is characterized as an activity by a treatment of things as if one were dealing with structures. From the pragmatist viewpoint, we do not know much more about structures than how to deal with them, after all. \hfill \cite{kromer2007tool}
\end{quote}
  \section{Preliminaries on variable set theory}
\epigraph{{\dn  a\326wy\3C4wAdFEn \8{B}tAEn \326wy\3C4wm@yAEn BArt . \\
			a\326wy\3C4wEnDnA\306wy\?v t/ kA pErd\?vnA .. 28..}}
% \epigraph{O scion of Bharat, all created beings are unmanifest before birth, manifest in life, and again unmanifest on death. So why grieve?}
{Śrīmadbhagavadgītā II, 28}
The present section introduces the main mathematical tool of our discussion: the theory of \emph{variable sets}. A variable set is just a family of sets indexed by another set $I$, i.e. (more formally) a class function $I \to \Set$. The collection of all such functions forms the object class of a category. Such categories are denoted as $\Set/I$ (read ``sets \emph{over} $I$'') and they are a particularly rich framework to re-enact mathematics in its entirety; here we explore the structure of $\Set/I$ in full detail.

We begin by assessing the equivalence between two different but equivalent descriptions of the category of variable sets: as class functions $I \to \Set$, or as functions $h : A \to I$ with fixed codomain $I$ (cf. \cite[1.6.1]{Bor1}). In some of our proofs it will be crucial to blur such a distinction between the category of functors $I \to \Set$ and the \emph{slice} category $\Set/I$; once the following result is proved, we will freely refer to any of these two categories as the category of \emph{variable sets} (indexed by $I$).
\begin{proposition}\label{variabbo_set}
	Let $I$ be a set, regarded as a discrete category, and let $\Set^I$ be the category of functors $F : I \to \Set$; moreover, let $\Set/I$ the slice category. Then, there is an equivalence (actually, an isomorphism, see \cite[1.5.1]{Bor1}) between $\Set^I$ and $\Set/I$.
\end{proposition}
\begin{proof}
	Our proof is based on the fact that we can represent the category $\Set^I$ as the category of $I$-indexed families of objects, i.e. with the category whose objects are $(\underline X)_I := \{X_i\mid i\in I\}$, and morphisms $(\underline X)_I\to (\underline Y)_I$ the families $\{f_i : X_i \to Y_i\mid i \in I\}$. Given this, the two categories obviously identify, as a functor $F : I \to \Set$ amounts to a choice of sets $A_i := F(i)$, and functoriality reduces to the property that identity arrows in $I$ go to identity functions $A_i \to A_i$.

	Let us consider an object $h : X\to I$ of $\Set/I$, and define a function $i\mapsto h^\leftarrow(i)$; of course, $(X(h))_I := \{h^\leftarrow(i) \mid i \in I\}$ is a $I$-indexed family, and since $I$ can be regarded as a discrete category, this is sufficient to define a functor $F_h : I \to \Set$.

	Let us define a functor in the opposite direction: let $F : I \to \Set$ be a functor. This defines a function $h_F : \coprod_{i\in I}Fi \to I$, where $\coprod_{i\in I} Fi$ is the disjoint union of all the sets $Fi$.

	The claim now follows from the fact that the correspondences $h\mapsto F_h$ and $F\mapsto h_F$ are mutually inverse.

	This is easy to verify: the function $F_{h_F}$ sends $i\in I$ to the set $h_F^\leftarrow(i)=Fi$, and the function $h_{F_h} \in \Set/I$ has domain $\coprod_{i\in I}F_h(i) = \coprod_{i\in I}h^\leftarrow(i)=X$ (as $i$ runs over the set $I$, the disjoint union of all preimages $h^\leftarrow(i)$ equals the domain of $h$, i.e. the set $X$).
\end{proof}
\begin{notation}
	The present remark is meant to establish a bit of terminology: by virtue of \autoref{variabbo_set} above, an object of the category of variable sets is equally denoted pair $(A,f : A \to I)$, as a function $h : I \to \Set$, or as the family of sets $\{h(i) \mid i \in I\} = \{A_i\mid i\in I\}$. We call the function $f$ the \emph{structure map} of the variable set $A$, and we call the function $F_h$ the functor \emph{associated}, or corresponding, to the variable set in study. Common parlance almost always blurs the distinction between these objects.
\end{notation}
\begin{remark}
	A more abstract look at this result establishes the equivalence $\Set/I\cong \Set^I$ as a particular instance of the \emph{Grothendieck construction} (see \cite[1.1]{Leinster2004}): for every small category $\clC$, the category of functors $\clC\to\Set$ is equivalent to the category of \emph{discrete fibrations} on $\clC$ (see \cite[1.1]{Leinster2004}). In this case, the domain $\clC=I$ is a discrete category, hence all functors $\clE \to I$ are, trivially, discrete fibrations.
\end{remark}
\begin{remark}
	The next crucial step of our analysis is the observation that the category of variable sets is a \emph{topos}: we break the result into the verification of the various axioms, as exposed in \autoref{eletop} and \autoref{grotop}. Our proof relies on the fact that the category of sets is itself a topos: in particular, it is cartesian closed, and admits the set $\{\perp,\top\}$ as subobject classifier.\footnote{We choose to employ a classical model of set theory, as opposed to an intuitionistic model where the classifier $\Omega$ consists of a more general Heyting algebra $H$; a general procedure to obtain a $\Omega$-many valued logic of set theory is to take the topos $\clE = \text{Sh}(H)$ of \emph{sheaves} on a Heyting algebra $H$: then, there is an isomorphism $\Omega_\clE\cong H$. The core of all our argument is very rarely affected by the choice to cut the complexity of our $\Omega$ to be the bare minimum; this is mainly due to expository reasons. The reader shall feel free to replace $\{\perp,\top\}$ with a more generic choice of Heyting algebra, and they are invited to adapt the arguments of section \autoref{sec_coins} accordingly.}
\end{remark}
\begin{proposition}\label{carclo}
	The category of variable sets is Cartesian closed in the sense of \cite[p.335]{Bor1}.
\end{proposition}
\begin{proof}
	We shall first show that the category of variable sets admits products: this is well-known as in $\Set/I$, products are precisely pullbacks (\cite[2.5.1]{Bor1}); note that \autoref{variabbo_set} gives an identification between the pullback $X\times_I Y$ as a set over $I$, and the $I$-indexed family of the preimages of $i$ under $h$:
	\[\vcenter{\scriptsize\xymatrix@!=1mm{
		& X\times_I Y \ar[dd]^h \ar@[lightgray][dr]\ar@[lightgray][dl]&  \\
		{\color{lightgray} X} \ar@[lightgray][dr]_{\color{lightgray} f}&& {\color{lightgray} Y} \ar@[lightgray][dl]^{\color{lightgray} g}\\
		& I &
		}}\iff i\mapsto h^\leftarrow(i) = \Big\{(x,y) \in X\times_I Y \mid h(x,y)=i\Big\}\]
	Now, this yields a canonical bijection $h^\leftarrow(i)\cong f^\leftarrow(i)\times g^\leftarrow(i)$. This is exactly the definition of the product of the two associated functors $F_f, F_g : I\to \Set$.

	To complete the proof, we shall show that each functor $\firstblank \times_I Y$ has a right adjoint $Y \pitchfork_I\firstblank$. The functor $\Set^I \to \Set^I : Z\mapsto Y\pitchfork_I Z$ where $Y\pitchfork_I Z : i \mapsto \Set(Y_i, Z_i)$ does the job. This, together with a straightforward verification, sets up the bijection
	\[\begin{array}{c}
			\xymatrix{X\times_I Y \ar[r] & Z}               \\ \hline
			\xymatrix{X \ar[r]           & Y\pitchfork_I Z}
		\end{array}\]
	and by a completely analogous argument (the construction  $(A,B)\mapsto A \times_I B$ is of course symmetric in its two arguments), we get a bijection
	\[\begin{array}{c}
			\xymatrix{X\times_I Y \ar[r] & Z}                \\ \hline
			\xymatrix{Y \ar[r]           & X\pitchfork_I Z;}
		\end{array}\]
	showing that also $X\times_I \firstblank$ has a right adjoint $X\pitchfork_I \firstblank$. This concludes the proof that the category of variable sets is Cartesian closed.
\end{proof}
\begin{proposition}\label{variable_sets_have_omega}
	The category of variable sets has a subobject classifier.
\end{proposition}
\begin{proof}
	From \autoref{eletop} we know that we shall find a variable set $\Omega$ such that there is a bijection
	\[\begin{array}{c}
			\xymatrix{ \chi : A \ar[r] & \Omega} \\ \hline
			\textsf{Sub}_I(A)
		\end{array}\]
	where $\textsf{Sub}_I(A)$ denotes the set of isomorphism classes of monomorphisms into $A$, in the category of variable sets.\footnote{A monomorphism into $A$ as an object of $\Set^I$ is nothing but a family of injections $s_i : S_i \to A_i$; a monomorphism in $\Set/I$ is a set $S$ in a commutative triangle
	\[\scriptsize
		\xymatrix@!=1mm{S\ar[rr]\ar[dr]_s && A\ar[dl]^a \\ &I.&}\]}
	For the sake of simplicity, for the rest of the proof we fix as category of variable sets the slice $\Set/I$.

	From this we make the following guess: as an object of $\Set/I$, $\Omega$ is the canonical projection $\pi_I : I\times \{\perp,\top\} \to I$. We are thus left with the verification that $\pi_I$ has the correct structure and universal property.

	First, we shall find a universal monomorphism $\true : * \to \Omega$ in $\Set/I$. Unwinding the definitions (in particular, since the identity function $\id_I : I \to I$ is evidently the terminal object in $\Set/I$), such a map amounts to an injective function $I\to \Omega$ having the projection $\pi_I : \Omega \times I \to I$ as left inverse.% This generalised element selects the $\top$ (read ``top'') truth value in $\Omega$.

	It turns out that the function $I \to I\times \{\perp,\top\}$ choosing the top-level copy of $I\cong I\times \{\top\}$ plays the r\^ole of $\true$: in the following, we shortly denote $\Omega_I$ such a product set.

	Now, a monomorphism $\cvar{S}{s}{I} \hookrightarrow \cvar{A}{a}{I}$ in $\Set/I$ is given by an injective function $S \hookrightarrow A$ that commutes with the structure maps of $S,A$; so, the commutative square
	\[
		\vcenter{\xymatrix{
		S\ar[d]\ar[r] & I \ar[d]^{\true}\\
		A \ar[r]_-{\chi_S} & \Omega_I
		}}
	\]
	is easily seen to be a pullback; in fact, every morphism of variable sets $\chi_S : A \to \Omega_I$ must send the element $a : A$ to a pair $(i,\epsilon) : I\times \{\perp,\top\}$. The pullback of $\chi$ and $\true$, as defined above, consists of the subset of the product $A\times I$ such that $\chi(a)=\true(i)=(i,\top)$; this defines a variable set $S=(\chi(i,\top))_I$, and such a correspondence is clearly invertible: every variable set arises in this way, and defines a ``characteristic'' function $\chi_S : (A,\cvar{A}{f}{I}) \to (\Omega_I,\pi_I)$:
	\[\chi_S(a) =
		\begin{cases}
			(f(a), \top)  & \text{ if } a\in S    \\
			(f(a), \perp) & \text{ if } a\notin S
		\end{cases}\]
	This concludes the proof of the fact that $\Set/I$ admits a subobject classifier.
\end{proof}
\begin{remark}
	A straightforward but important remark is now in order. The structure of subobject classifier of $\Omega_I$, and in particular the shape of a characteristic function $\chi_S : A \to \Omega_I$ for a subobject $S\subseteq A$ in $\Set/I$, is explicitly obtained using the structure map $f$ of the variable set $f : A\to I$.

	This will turn out to be very useful along our main section, where we shall note that a proposition in the internal language of $\Set/I$ amounts to a function $p : U \to \Omega_I$, having as domain a variable set $u : U \to I$, whose structure map uniquely determines the ``strength'' (see \autoref{very_importanta_force}) of the proposition $p$. In a nutshell, $p(x)$ is a truth value in $\Omega_I$; the fact that $p$ is a morphism \emph{of variable sets} however forces this truth value to be $(u(x),\epsilon) \in I\times \{\perp,\top\}$. We invite the baffled reader not to worry now; we will duly justify each of these conceptual steps along \autoref{sec_coins}.
\end{remark}
\begin{proposition}
	The category of variable sets is cocomplete and accessible.
\end{proposition}
\begin{proof}
	Cocompleteness can be shown appealing to \cite[??]{Bor1}: if $\clC$ is a small category, and $\clD$ a cocomplete category, the category $\clD^\clC$ of all functors $F : \clC \to \clD$ is cocomplete, and colimits are computed pointwise, meaning that given a diagram $\clJ \to \clD^\clC$ of functors $F_j$, $\colim F_j : \clD^\clC$ is the functor $C \mapsto \colim_\clJ F_j(C)$ (that exists in $\clD$ by assumption).
	Given that $\Set/I\cong \Set^I$, and that the category $\Set$ is cocomplete, we obtain the result.

	Accessibility is a corollary of Yoneda in the following form: every $F : I \to \Set$ is a colimit of representables
	\[
		F \cong \colim\Big(\clE(F) \xto{\Sigma} I \xto{y} \Set^I\Big)
	\]
	(the category of elements \cite{Bor1} $\clE(F)$ of $F:\Set^I$ is small because in this case $\clE(F)\cong\coprod_{i\in I}Fi$).
\end{proof}
\begin{corollary}
	The category of variable sets is a Grothendieck topos.
\end{corollary}

  \section{The internal language of variable sets}\label{int_lang}
\epigraph{I am hard but I am fair; there is no racial bigotry here. [\dots\unkern] Here you are all equally worthless.}{GySgt Hartman}
\begin{definition}\label{da_lang}
	The internal language of a topos $\clE$ is a formal language defined by \emph{types} and \emph{terms}; suitable terms form the class of variables. Other terms form the class of \emph{formul\ae}.
	\begin{itemize}
		\item \emph{Types} are the objects of $\clE$
		\item \emph{Terms} of type $X$ are morphisms of codomain $X$, usually denoted $\alpha,\beta,\sigma,\tau : U \to X$.
		      \begin{itemize}
			      \item Suitable terms are variables: the identity arrow of $X\in\clE$ is the variable  $x : X \to X$. For technical reasons we shall keep a countable number of variables of the same type distinguished:\footnote{These technical reasons lie on the evident necessity to be free to refer to the same free variable an unbounded number of times. This can be formalised in various ways: we refer the reader to \cite{lambek1988introduction} and \cite{JohnstonePT}.} $x,x',x'',\dots : X \to X$ are all interpreted as $1_X$.
		      \end{itemize}
		\item Generic terms may depend on multiple variables; the domain of a term of type $X$ is the \emph{domain of definition} of a term.
	\end{itemize}
	A number of inductive clauses define the other terms of the language:
	\begin{itemize}
		\item the identity arrow of an object $X\in\clE$ is a term of type $X$;
		\item given terms $\sigma : U \to X$ and $\tau :  V\to Y$ there exists a term $\lr{\sigma}{\tau}$ of type $X\times Y$ obtained from the pullback
		      \[\xymatrix{
			      W \ar[d]\ar[r]\ar[dr]|{\lr{\sigma}{\tau}} & X \times V \ar[d]\\
			      U\times Y \ar[r]& X\times Y
			      }\]
		\item Given terms $\sigma : U \to X, \tau : V \to X$ of the same type $X$, there is a term $[\sigma = \tau] : W \xto{\lr{\sigma}{\tau}} X\times X \xto{\delta_X} \Omega$, where $\delta_X : X\times X \to \Omega$ is defined as the classifying map of the mono $X \hookrightarrow X\times X$.
		\item Given a term $\sigma : U \to X$ and a term $f : X \to Y$, tere is a term $f[\sigma] := f\circ\sigma : U \to Y$.
		\item Given terms $\theta :  V \to Y^X$ and $\sigma : U\to X$, there is a term
		      \[
			      W\lr{\theta}{\sigma} \xto{}Y^X\times X \xto{\text{ev}} Y
		      \]
		\item In the particular case $Y=\Omega$, the term above is denoted
		      \[[\sigma\in\theta] : W\lr{\theta}{\sigma} \to \Omega\]
		\item If $x$ is a variable of type $X$, and $\sigma : X\times U \to Z$, there is a term
		      \[\lambda x.\sigma : U \xto{\eta} (X\times U)^X \xto{\sigma^X} Z^X\]
		      obtained as the mate of $\sigma$.
	\end{itemize}
	These rules can of course be also presented as the formation rules for a Gentzen-like deductive system: let us rewrite them in this formalism.
	\[ \begin{array}{cc}
			\infer{1_X : X \to X}{}                                                              &
			\infer{\lr{\sigma}{\tau} : W\lr{\theta}{\sigma} \to X \times Y }{\sigma : U \to X    &   & \tau : V \to Y}             \\[1em]
			\infer{[\sigma=\tau] : W \to \Omega}{\sigma : U \to X                                &   & \tau : V \to X}           &
			\infer{f[\sigma] : U \to Y}{\sigma : U \to X                                         &   & f : X \to Y}                \\[1em]
			\infer{W\lr{\theta}{\sigma} \xto{}Y^X\times X \xto{\text{ev}} Y}{\theta :  V \to Y^X &   & \sigma : U\to X}          &
			\infer{\lambda x.\sigma = \sigma^X\circ\eta : U \to (X\times U)^X \to Z^X}{ x: X     &   & \sigma : X\times U \to Z}
		\end{array}\]
	To formulas of the language of $\clE$ we apply the usual operations and rules of first-order logic: logical connectives are induced by the structure of internal Heyting algebra of $\Omega$: given formulas $\varphi,\psi$ we define
	\begin{itemize}
		\item $\varphi\lor \psi$ is the formula $W\lr{\varphi}{\psi} \to \Omega\times \Omega \xto{\lor} \Omega$;
		\item $\varphi\land\psi$ is the formula $W\lr{\varphi}{\psi} \to \Omega\times \Omega \xto{\land} \Omega$;
		\item $\varphi\Rightarrow\psi$ is the formula $W\lr{\varphi}{\psi} \to \Omega\times \Omega \xto{\Rightarrow} \Omega$;
		\item $\lnot\varphi$ is the formula $U \to \Omega \xto{\lnot} \Omega$.
	\end{itemize}
\end{definition}
Universal quantifiers admit an interpretation in the Mitchell-Bénabou language of $\clE$: the following definition comes from \cite[VI]{mac1992sheaves}.
\begin{definition}\label{quantifezzi}
	Let $\clE$ be a topos, and let $\varphi : U\times V \to \Omega$ be a formula defined on a product type $U\times V$. The variable $u:U$ can be now quantified over yielding a new formula
	\[\forall u.\varphi(u,v) : V \to \Omega\]
	which no longer contains $u$ as a free variable.

	The term $\forall u.\varphi(u,v)$ is obtained by composition of $\lambda u.\varphi(u,v) : V \to V\to\Omega^U$ with the unique right adjoint $\forall_t : \Omega^U \to \Omega$ to the precomposition $t^*:\Omega \to \Omega^U$\footnote{The arrow $t:U\to 1$ is the \emph{t}erminal map, and $t^* : \Omega^1\to \Omega^U$ is induced by precomposition; it sends a term $x : \Omega$ to the constant function $t^*x=\lambda u.x$.} in the diagram
	\[\xymatrix{
			V \ar[r]^{\lambda u.\varphi} & \Omega^U \ar[r]^{\forall_t} & \Omega
		}\]
	(see \cite[IV.9]{mac1992sheaves}). Similarly, we obtain a term $\exists u.\varphi(u,v)$ from the composition
	\[\xymatrix{
			V \ar[r]^{\lambda u.\varphi} & \Omega^U \ar[r]^{\exists_t} & \Omega
		}\]
	with the left adjoint $\exists_t \dashv t$.
\end{definition}
Each formula $\varphi : U \to \Omega$ defines a subobject $\{x\mid \varphi\} \subseteq U$ of its domain of definition; this is the subobject classified by $\varphi$, and must be thought as the subobject where ``$\varphi$ is true''.

If $\varphi : U\to\Omega$ is a formula, we say that $\varphi$ is \emph{universally valid} if $\{x\mid\varphi\}\cong U$. If $\varphi$ is universally valid in $\clE$, we write ``$\clE\Vdash \varphi$'' (read: ``$\clE$ believes in $\varphi$'').

Examples of universally valid formulas:
\begin{itemize}
	\item $\clE\Vdash [x=x]$
	\item $\clE\Vdash [(x \in_X \{x\mid\varphi\}) \iff \varphi]$
	\item $\clE\Vdash \varphi$ if and only if $\clE \Vdash \forall x.\varphi$
	\item $\clE\Vdash [\varphi \Rightarrow \lnot\lnot\varphi]$
\end{itemize}
Let us now glance at the internal language of variable sets. This will turn out to be the cornerstone for the analysis in \autoref{sec_coins}.

Here we just unwind \autoref{da_lang} at its very surface; we invite the reader to endeavour in the instructive exercise to fill all details properly.
\begin{definition}
	Types and terms of $\clL(\Set/I)$ are respectively arrows $\var{U}{I}$ with codomain $I$, and commutative triangles
	\[\xymatrix{
			U \ar[rr]^\sigma \ar[dr]_u && X\ar[dl]^x \\
			&I&
		}\]
	Given this, we define
	\begin{itemize}
		\item \emph{product terms} as functions $\sigma : U_1\times_I U_2 \times_I\dots \times_I U_n \to X$;
		\item terms $\la \sigma,\tau\ra$ obtained as the (diagonal map in the) pullback
		      \[\xymatrix{
			      W\la \sigma,\tau\ra \pb \ar[d]\ar[r]& X \times_I V \ar[d]\ar@[lightgray]@/^1pc/[ddr]\\
			      U \times_I Y \ar[r]\ar@[lightgray]@/_1pc/[drr] & X \times_I Y \ar@[lightgray][dr]\\
			      &&**[r] \color{lightgray} I \times_I I\cong I
			      }\]
		\item terms $[\sigma=\tau]$, obtained as compositions
		      \[\xymatrix{ W\la \sigma,\tau\ra \ar[r]^{\la \sigma,\tau\ra} & X \times_I Y \ar[r]^{\delta_X} & \Omega_I }\]
		      where $\delta_X : X \times_I X \to \Omega$ classifies the mono $m : X \to X\times_I X$ obtained by the universal property of the pullback $X\times_I X$ as the kernel pair of $\cvar{X}{x}{I}$.
	\end{itemize}
	The cartesian closed structure of $\Set/I$ (cf. \autoref{carclo}) yields terms (we denote $B^A:= A\pitchfork B$ in the notation of \autoref{carclo})
	\begin{itemize}
		\item $[\sigma\in\theta] : W\la \theta,\sigma\ra \to Y^X \times_I X \to \Omega_I$ from suitable $\sigma,\theta$;
		\item $U \to (X\times_I U)^X \xto{\sigma^X} Z^X$ ($\lambda$-abstraction on $x : X$) from a suitable $\sigma : X\times_I U \to Z$.
	\end{itemize}
\end{definition}
Quantifiers can be deduced in a similar way, starting from their general definition in \autoref{quantifezzi}.
\begin{remark}
	Like every other Grothendieck topos, the category $\Set/I$ has a \emph{natural number object} (see \cite[VI.1]{mac1992sheaves}, \cite[p.46]{lambek1988introduction}); here we shall outline its construction. It is a general fact that such a natural number object in the category of variable sets, consists of the constant functor on $\bbN : \Set$, when we realise variable sets as functors $I \to \Set$: thus, in fibered terms, the natural number object is just $\pi_I : \bbN \times I \to I$.

	A natural number object provides the category $\clE$ it lives in with a notion of \emph{recursion} and with a notion of $\clE$-induction principle: namely, we can interpret the sentence
	\[\textstyle\big(Q0\land \bigwedge_{i\le n} Qi\Rightarrow Q(i+1)\big)\Rightarrow \bigwedge_{n : \bbN} Qn\]
	for every $Q : \bbN \to \Omega_I$.

	In the category of variable sets, the universal property of $\pi_I : \bbN\times I \to I$ amounts to the following fact: given any diagram of solid arrows
	\begin{equation}\label{natnumeq}
		\vcenter{\xymatrix{
				I \ar[r]^0 \ar@{=}[d] & \bbN\times I\ar@{.>}[d]^u \ar[r]^{s\times I} & \bbN \times I\ar@{.>}[d]^u \\
				I \ar[r]_x & X \ar[r]_f & X
			}}
	\end{equation}
	where every arrow carry a structure of morphism over $I$ (and $0 : i \mapsto (0,i)$, $s\times I : (n,i) \mapsto (n+1,i)$), there is a unique way to complete it with the dotted arrow, i.e. with a function $u : \bbN \times I \to X$ such that
	\[u \circ (s\times I) = f \circ u.\]
	Clearly, $u$ must be defined by induction: if it exists, the commutativity of the left square amounts to the request that $u(0,i)=x(i)$ for every $i : I$. Given this, the inductive step is
	\[
		u(s(n,i)) = u(n+1,i) = f(u(n,i)).
	\]
	This recursively defines a function with the desired properties; it is clear that these requests uniquely determine $u$.
\end{remark}
Such a terse exposition obviously does not exhaust such a vast topic as recursion theory conducted with category-theoretic tools. The interested reader shall consult \cite{jacobs1997tutorial} for a crystal-clear introductory account, and \cite{cockett2008introduction,cockett2014total} for more recent and modern development of recursion theory.

The object of natural numbers of $\Set/I$ is easily seen to match the definition of the \emph{initial object} \cite[]{Bor1} of the category $\cate{Dyn}/I$ so defined:
\begin{itemize}
	\item the objects of $\cate{Dyn}/I$ are \emph{dynamical systems} in $\Set/I$, i.e. the triples $(x,X,f)$, where $X : \Set/I$ (say, with structure map $\xi : X \to I$), $x : (I,\id_I) \to (X,\xi)$ and $f : X \to X$ is an endo-morphism of variable sets;
	\item given two dynamical systems $(x,X,f)$ and $(y,Y,g)$ a \emph{morphism} between them is a function $u  :X \to Y$ such that the diagram
	      in \eqref{natnumeq} commutes in all its parts.
\end{itemize}
The reason why such a triple $(x,X,f)$ identifies as a (discrete) dynamical system is easily seen: the function $x : I \to X$ works as initial seed for a recursive application of $f$, in such a way that every $f : X \to X$ defines a sequence
\[u_{n+1} := f(u_n)\]
of its iterates. The system now lends itself to all sorts of questions: is there a fixed point for \(u_n(x)\)? Does the limit of \(u_n(x)\) belong to \(X\) (not obvious: consider \(X=[0,1[\) and \(u_n(x)\equiv 1-\frac{1}{n}\))? Is \(u_n(x)\) continuous in \(x\)? Etc.
  \section{Nine copper coins, and other toposes}\label{sec_coins}
\epigraph{
	Explicaron que una cosa es \emph{igualdad}, y otra \emph{identidad}, y formularon una especie de \emph{reductio ad absurdum}, o sea el caso hipotético de nueve hombres que en nueve sucesivas noches padecen un vivo dolor. ¿No sería ridículo -interrogaron- pretender que ese dolor es el mismo?
}{JLB ---Tl\"on, Uqbar, Orbis Tertius}

According to our description of the Mitchell-Bénabou language in the category of variable sets, \emph{propositions} are morphisms of the form
\[p : U \to \Omega_I\]

where $\Omega_I$ is the subobject classifier of $\Set/I$ described in \autoref{variable_sets_have_omega}; now, recall that
\begin{itemize}
	\item the object $\Omega_I = \{0,1\}\times I \to I$ becomes an object of $\Set/I$ when endowed with the projection $\pi_I : \Omega_I \to I$ on the second factor of its domain;
	\item the universal monic $\tr : I \to \Omega_I$ consists of a section of $\pi_I$, precisely the one that sends $i : I$ to the pair $(i,1) : \Omega_I$;
	\item every subobject $U \hookrightarrow A$ of an object $A$ results as a pullback (in $\Set/I$) along $\tr$:
	      \[\xymatrix@R=5mm@C=5mm{
		      U\ar[dr]^u\ar[rr]^u\ar[dd]_m&& I\ar[dd]^{\tr} \ar@{=}[dl]\\
		      &I& \\
		      A \ar[ur]\ar[rr]_{\chi_U}&& \ar[ul]\Omega_I
		      }\]
	      (see \autoref{variable_sets_have_omega} for a complete proof)
\end{itemize}
The set $I$ in this context acts as a \emph{multiplier} of truth values, in that every proposition can have a pair $(\epsilon, i)$ as truth value. We introduce the following notation: a proposition $p : U \to \Omega_I$ is \emph{true}, in context $x :U$, with \emph{strength} $t$, if $p(x) =(1,t)$ (resp., $p(x)=(0,t)$).
\begin{remark}\label{very_importanta_force}
	A proposition in the internal language of variable sets is a morphism of the following kind: a function $p : U \to \Omega_I$, defined on a certain domain, and such that
	\[
		\xymatrix{
			U\ar[d]_u\ar[r]^-p  & \{0,1\}\times I \ar[d]^{\pi_I}\\
			I \ar@{=}[r]& I
		}
	\]
	(it must be a morphism of variable sets!) This means that $\pi p(x : U) = u(x : U)$, so that $p(x) = (\epsilon_x, u(x))$ for $\epsilon_x =0,1$ and $u$ is uniquely determined by the "variable domain" $U$. This is an important observation: the strength with which $p$ is true/false is completely determined by the structure of its domain, in the form of the function $u : U \to I$ that renders the pair $(U,u)$ an object of $\Set/I$.
\end{remark}
To get a grip of the different roles of various classes of propositions, and given that our interest will be limited to a certain class of particular propositions that we will construct \emph{ex nihilo}, it is now convenient to discuss what constraints we have to put on the structure of $I$: of course, the richest this structure is, the better will the category $\Set/I$ behave: it is for example possible to equip $I$ with an order structure, or a natural topology. Among different choices of truth multiplier, yielding different categories of variable sets, and different kinds of internal logic therein, we will privilege those that make $I$ behave like a space of strengths: a dense, linear order with LUP, thus not really far from being a closed, bounded subset of the real line.

The main result of the present section is a roundup of examples showing that it is possible to concoct categories of variable sets where some seemingly paradoxical constructions coming from J.L. Borges' literary world have, instead, a perfectly ``classical'' behaviour when looked with the lenses of the logic of variable set theory.

Each of the examples in our roundup \autoref{bla}, \ref{bli}, \ref{blu} is organised as follows: we recall the shape of a paradoxical statement in Borges' literary world. Then, we show in which topos this reduces to an intuitive statement expressed in the syntax of a variable set category.

More than often, we use $I=[0,1]$ as base of variable sets; as already said, there are different reasons for this choice: the most intuitive is that if a truth value is given with a \emph{strength} $t: I$ it is a natural request to be able to \emph{compare} elements in this set; in particular, it should always be possible to assess what truth value is stronger.

For this reason, even if this assumption is never strictly necessary (the only constraint is that $I$ is totally, or partially, ordered set by a relation $\le$), a natural choice for $I$ is a \emph{continuum} (=a dense total order with LUP --see \cite{moschovakis2009descriptive}). An alternative choice drops the density assumption: in that case the (unique) finite total order $\Delta[n] = \{0 < 1 <\dots < n\}$, or the countable total order $I=\omega = \bigcup_n \Delta[n]$ are all pretty natural choices for $I$ (although it is way more natural for $I$ to have a minimum \emph{and} a maximum element).\footnote{We're only interested in the notion of an abstract interval here: a continuum $X$ endowed with an operation $X \to X \vee X$ of ``zooming'', uniquely defined by this property. In a famous paper Freyd characterises ``the interval'' as the terminal interval coalgebra: see \cite[§1]{freyd2008algebraic}; for our purposes, note that $[0,1]$ is a natural choice: it is a frame, thus a Heyting algebra $\fkH =([0,1],\land,\lor,\Rightarrow)$ with respect to the pseudo-complement operation given by $(x \Rightarrow z) := \bigvee_{x\land y \le z} y$ (it is immediate that $x \land a \le b$ if and only if $a \le x \Rightarrow b$ for every $a,b: [0,1]$).} In each of these cases ``classical'' logic is recovered as a projection: propositions $p$ can be true or false with strength $1$,\footnote{Here $I$ is represented as an interval whose minimal and maximal element are respectively $0$ and $1$; of course these are just placeholders, but it is harmless for the reader to visualise $I$ as the interval $[0,1]$.} the maximum element of $I$:
\begin{center}
	\begin{tikzpicture}
		\draw[gray!70,dashed] (2,1.25) -- (2,-.25) node[below] {$\{\perp,\top\}$};
		\draw[fill] (0,0) circle (1pt) node[left] {$0$};
		\draw[fill] (2,0) circle (1pt) node[right] (dis) {$1$};
		\draw (0,0) -- (2,0);
		\begin{scope}[yshift=1cm]
			\draw[fill] (0,0) circle (1pt) node[left] {$0$};
			\draw[fill] (2,0) circle (1pt) node[right] (dat) {$1$};
			\draw (0,0) -- (2,0);
			\node[right of=dis] {$\{\top\}\times I$};
			\node[right of=dat] {$\{\perp\}\times I$};
		\end{scope}
	\end{tikzpicture}
\end{center}
In order to aid the reader understand the explicit way in which $I$ ``multiplies'' truth values, we spell out explicitly the structure of the subobject classifier in $\Set/\Delta[2]$. In order to keep calling the minimum and maximum of $I$ respectively $0$ and $1$ we call $\frac{1}{2}$ the intermediate point of $\Delta[2]$.
\begin{remark}
	The subobject classifier of $\Set/\Delta[2]$ consists of the partially ordered set $\Delta[1]\times\Delta[2]$ that we can represent pictorially as a rectangle
	\[\begin{tikzpicture}[xscale=2]
			\foreach \i/\name in {0/0,.5/{\frac{1}{2}},1/0}{
			\foreach \j/\pos in {0/below,1/above}
			\fill (\i,\j) ellipse (1pt and 2pt) node[\pos] {\tiny $(\name,\j)$};
			}
			\draw (0,0) rectangle (1,1);
			\draw (.5,1) -- (.5,0);
		\end{tikzpicture}\]
	endowed with the product order. The resulting poset is partially ordered, and in fact a Heyting algebra, because it results as the product of two Heyting algebras: the Boole algebra $\{0<1\}$ and the frame of open subsets of the Sierpinski space $\{a,b\}$ (the topology is $\tau_S = \{\varnothing, \{a\}, \{a,b\}\}$).
\end{remark}
\begin{remark}\label{alcuni_set}
	Given that $I=[0,1]$ ndowed with its usual Euclidean topology is one of our most natural choices, we explicitly define some interesting sets obtained out of $\Omega_I$ or out of a proposition $p : U \to \Omega_I$ in the Mitchell-Bénabou language of $\Set/[0,1]$:%definiamo alcuni insiemi di interesse per una data proposizione $p : U \to \Omega_I$ per questa scelta di $I$:
	\begin{itemize}
		\item $A^\top = \{x : U \mid p(x) = (1,t_x), t_x > 0\} = p^\leftarrow(\{1\}\times (0,1])$ e $A^\perp = \{x : U \mid p(x) = (0,t_x), t_x > 0\}$; this is the subset of functions $u : U \to I$ such that $u^\leftarrow 0 = \varnothing$.
		\item $B^\top = \{x : U \mid p(x) = (1,1)\} = p^\leftarrow((1,1))$ and $B^\perp = \{x : U \mid p(x) = (0,1)\}$ ``true''s and ``false''s; these are obtaiend as fibers over the maximal element of $I$.
		\item $E_t^\top = \{ x : U \mid p(x)=(1,t)\}$ e $E_t^\perp = \{ x : U \mid p(x)=(0,t)\}$.
	\end{itemize}
\end{remark}
A crucial decision at some point will be about the regularity with which the strength of $p$ depends with respect to the variables on its domain of definition. Without a continuous dependence, small changes in context $x : U$ might drastically change the truth value $p(x)$.\footnote{There is no a priori reason to maintain that $p$ is a continuous proposition; one might argue that discontinuous changes in truth value of $p$ happen all the time in ``real life''; see the family of paradoxes based on so-called \emph{separating instants}: how well-defined the notion of ``time of death'' is? How well-defined the notion of ``instant in time''?}
\subsection{The unimaginable topos theory hidden in Borges' library}
Jorge Luis Borges' literary work is well-known for being made by paradoxical worlds; oftentimes, seemingly absurd consequences follow by stretching to their limit ideas from logic and mathematics: time, infinite, self-referentiality, duplication, recursion, the relativity of time, the illusory nature of our perceptions, the limits of language, its capacity to generate worlds.

In the present section we choose \emph{Fictions}, Borges' famous collection of novels, as source of inspiration for possible and impossible worlds and their ontology.

Usualmente la costruzione di un ``mondo impossibile'' (variamente impiegato nelle semantiche paraconsistenti) va circa come segue: since the classic work of Rescher and Brandom [\cite{10.2307/20127724}] it is obtained with recursive operations on (kripkean) standard worlds \dots ; noi rovesciamo tale prospettiva, e invece di depennare dal computo degli universi i mondi le cui caratteristiche do not comply with sensorial experience, or imply paradoxical entities/constructs, we accept their existence for bizarre that it may seem, and we try to deduce what kind of logic can consistently generate such statements.

The interest in such a literary calembour is manifold, and the results are surprising:
\begin{itemize}
	\item we unravel how a mathematically deep universe Borges has inadvertently created: of the many compromises we had to take in order to reconcile literature and the uniderlying mathematics,\footnote{See \autoref{our_beloved_interval} below: these ``compromises'' mainly amount to assumptions on the behaviour of space-time on Tl\"on and Babylon.} we believe no one is particularly far-fetched one;
	\item we unravel how \emph{relative} ontological assumptions are; they are not given: using category theory, ontology, far from being the presupposition on which it is based, is a byproduct of language itself. The more expressive language is, the more ontology; the fuzzier its capacity to assert truth, the fuzzier existence becomes;
	\item ``Fuzziness'' of existence, i.e. the fact \emph{entia} exist less than completely, is hard-coded in the language (in the sense of \autoref{da_lang}) of the category we decide to work in from time to time;
	\item ...
\end{itemize}
To sum up, readers willing to find an original result in this paper, might find it precisely here: we underline how Borges' alternative worlds (Babylon, Tl\"on \ dots) are mathematically consistent places, worthy of existence as much as our world, just based on a different internal logic. And they are so just thanks to a base-sensitive theory of existence --ontology breaks in a spectrum of ontolog\emph{ies}, one for each category/world.

The first paradox we aim to frame in the right topos is the famous nine copper coins argument, used by the philosophers of Tlön to construct a paradoxical object whose existence persists over time, in absence of a consciousness continually perceiving it and maintaining in a state of being.
\begin{example}[Nine copper coins]\label{bla}
	First, we recall the exact statement of the paradox:\footnote{The paradox appears in a primitive, mostly unchanged version in \cite{borges1997otras}, where instead of nine copper coins, a single arrow, shot by an anonymous archer, disappears among the woods. For what concerns Tl\"on's  nine copper coins, our translation comes from \cite{tlonEN}:
		\begin{quote}
			\hspace{.5em} Tuesday, $X$ crosses a deserted road and loses nine copper coins. On Thursday, $Y$ finds in the road four coins, somewhat rusted by Wednesday's rain. On Friday, $Z$ discovers three coins in the road. On Friday morning, $X$ finds two coins in the corridor of his house. The heresiarch would deduce from this story the reality - i.e., the continuity - of the nine coins which were recovered.

			\hspace{.5em} It is absurd (he affirmed) to imagine that four of the coins have not existed between Tuesday and Thursday, three between Tuesday and Friday afternoon, two between Tuesday and Friday morning. It is logical to think that they have existed - at least in some secret way, hidden from the comprehension of men - at every moment of those three periods.
		\end{quote}}
	\begin{quote}
		El martes, $X$ atraviesa un camino desierto y pierde nueve monedas de cobre.
		El jueves, $Y$ encuentra en el camino cuatro monedas, algo herrumbradas por la lluvia del miércoles. El viernes, $Z$ descubre tres monedas en el camino. El viernes de mañana, $X$ encuentra dos monedas en el corredor de su casa. El  quería deducir de esa historia la realidad -id est la continuidad- de las nueve monedas recuperadas.

		Es absurdo (afirmaba) imaginar que cuatro de las monedas no han existido entre el martes y el jueves, tres entre el martes y la tarde del viernes, dos entre el martes y la madrugada del viernes. Es lógico pensar que han existido -siquiera de algún modo secreto, de comprensión vedada a los hombres- en todos los momentos de esos tres plazos.
	\end{quote}
	Before going on with our analysis, it is important to remark that there is one and only one reason why the paradox of the nine copper coins is invalid: copper does not rust. Incidentally, we will be able to propose a rectification of this ``rust counterargument'' without appealing to the assumption that copper can rust on Tl\"on due to a purported difference between Earth's and Tl\"onian chemistry.

	It is obvious on what this construction leverages to build an efficient aporia: the unimaginable persistence of things through time, even without a perceivent consciousness.

	Expressed in natural language, our solution to the paradox goes more or less as follows: $X$ loses their coins on Tuesday, and the strength $\varphi$ with which they ``exist'' lowers; it grows back in the following days, going back to a maximum value when $X$ retrieves two of their coins on the front door. $Y$ findings of other coins raises their existence strength to a maximum. The coins that $Y$ has found rusted (or rather, the surface copper oxidized: this is possible, but water is rarely sufficient to ignite the process alone --certainly not in the space of a few hours.
	% \footnote{in una sorta di principio di wormhole, eventi indipendenti sulla terra sono dipendenti su Tlön, perché l'evento A influenza, in uno spazio pluridimensionale di scelte di z di verità, l'evento B in modi che gli sarebbero vietati se fosse sulla Terra.}
	\begin{remark}\label{our_beloved_interval}
		In this perspective, Tl\"on classifier of truth values can be taken as $\Omega_I = \{0<1\}\times I$, where $I$ is any set with more than one element; a minimal example can be $I=\{N,S\}$ (justifying this choice from inside Tl\"on is easy: the planet is subdivided into two emispheres; each of which now has its own logic ``line'' independent from the other), but as explained in \autoref{our_beloved_interval} a more natural choice for our purposes is the closed real interval $I=[0,1]$.

		This allows for a continuum of possible forces with which a truth value can be true or false;  it is tobe noted that $[0,1]$ is also the most natural place on which to interpret fuzzy logic, albeit the interest for $[0,1]$ therein can be easily and better motivated starting from probability theory. (There are interesting perspective on how to develop basic measure theory out of $[0,1]$. For example, measures valued in things like Banach space and more general topological	vector spaces have been considered.)
	\end{remark}
	We now start to formalise properly what we said until now. To set our basic assumptions straight, we proceed as follows:
	\begin{itemize}
		\item Without loss of generality, we can assume the set $C = \{c_1,\dots,c_9\}$ of the nine coins to be totally orered and partitioned in such a way that the first two coins are those retrieved by $X$ on Tuesday, the subsequent four are those found by $Y$ on their way, and the other three are those seen by$Z$. So,
		      \[C = C_X \sqcup C_Y \sqcup C_Z.\]
		      and $C_X = \{c_{X1}, c_{X2}\}$, $C_Y = \{c_{Y1},c_{Y2},c_{Y3},c_{Y4}\}$, $C_Z= \{c_{Z1}, c_{Z2}, c_{Z3}\}$ As already said, the truth multiplier $I$ is the closed interval $[0,1]$ with its canonical order --so with its canonical structure of Heyting algebra, and if needed, endowed with the usual topology inherited by the real line.
		\item Propositions of interest for us are of the following form:
		      \[\lambda gcd.p(g, c, d) : \{X,Y,Z\}\times C\times W \to \Omega_I\]
		      where $W \subseteq \{\S,\M,\Tu,\W, \Th,\F,\Sa\}$ is a set of days (strictly speaking, the paradox involves just the interval $[\Tu,\F]$). The value $p(g,c,d)$ models how in $g$'s frame of existence the coin $c$ exists at day $d$ with strength $p(g,c,d)$.
	\end{itemize}
	\begin{definition}[Admissible coins]
		We now define an \emph{admissible} configuration of coins any arrangement of $C$ such that the following two conditions are satisfied:
		\begin{enumtag}{ad}
			\item \label{ad:uno} for all day $d$ and coin $c$, we have
			\[
				\sum_{u: \{X,Y,Z\}} p(u,c,d) = (\top, 1)
			\]
			where we denote as ``sum'' the logical conjunction in $\Omega_I$: this means that day by day, the \emph{global} existence of the group of coins constantly attains the maximum; it is the \emph{local} existence that lowers when the initial conglomerate of coins is partitioned.
			\item \label{ad:due} Moreover,
			\[
				\begin{cases}
					\sum_{c_X: C_X} p(X,c_X,V) = (\top,1) \\
					\sum_{c_Y: C_Y} p(Y,c_Y,G) = (\top,1) \\
					\sum_{c_Z: C_Z} p(Z,c_Z,V) = (\top,1)
				\end{cases}
			\]
		\end{enumtag}
	\end{definition}
	In an admissible configuration the subsets $ C_X, C_Y, C_Z $ can only attain an existence $p(g,c,d) \lneq (\top,1)$; that is to say, \emph{no coin completely exists locally}. But for an hypothetical external observer, capable of observing the system, adding up the forces with which the various parts of $C$ exist, the coins \emph{globally} exist `` in some secret way, of understanding forbidden to men'' (or rather, to $ X, Y, Z $).
	\begin{center}
		\begin{figure}[h]
			\begin{tikzpicture}[xscale=6, yscale=4]
				\coordinate (1) at (0,0);
				\foreach \i/\j in {1/\W,2/\Th,3/\F}{
						\draw[gray!90] (\i/4,0) node[below] {\tiny \j} -- (\i/4,1);
					}
				\node[below,gray!90] at (0,0) {\Tu};
				\node[below,gray!90] at (1,0) {\Sa};
				\coordinate (1) at (0,0);
				\coordinate (2) at (1/2, 1/5);
				\coordinate (3) at (3/4,1);
				\draw[ultra thick,red] (1) -- (2) node[right] {\scriptsize $1/5$} -- (3);
				\draw[thin,yellow] (1) -- (2) -- (3);
				\draw[blue] (1) -- (1/2,1) node[above] {\scriptsize $p(Y, c_Y)=(\top,1)$} -- (3/4,1/4) node[right] {\scriptsize $1/4$};
				\node[fill=gray!60] (cloud) at (1/4,1/2) {\large\color{black} \faCloud};
				\draw[thick] (0,0) rectangle (1,1);
			\end{tikzpicture}
			\caption{A pictorial representation of the truth forces of coins in different days; we choose a minimally complex, piecewise linear model where strength of existence goes up and down linearly to join the points where \cite{tlonEN} gives complete information about the coins' configuration. $X$ is marked in red, $Z$ in yellow, $Y$ in blue. Time is considered as a continuum line, marked at weekdays for readability.}
			\label{fig_coins}
		\end{figure}
	\end{center}
\end{example}
\begin{remark}
	An important claim is now in order, as our construction might appear just a sleight of hand leaving open the main problem posed by the coin riddle: \emph{where} do the coins exist? Not to mention that to some extent, \emph{every} theory of existence posits that things are there in some secret way. Tl\"on's idealist might deny time (our model makes no strong assumption on what time is made of: discrete, continuous, homogeneous, slowed by traveling at high speed\dots); $B^\bot$ ontologists live comfortably in $B^\bot$, as defined in \autoref{alcuni_set} without being able to tell anything about how objects ``completely do not exist''.
	At the opposite side of the spectrum the pure empiricist lives in $B^\top$, and they're unable to tell how they ``completely exist'').

	This is where our analysis comes in handy; in particular, this is where our particular choice of $\Omega_I$ plays its r\^ole. In our model $B^\perp, B^\top$ are just particular slices of a single object where they coexist as particular fibers; the approach that looks at them as such is \emph{inherently better}, as it yields a quantitative answer framing both $B^\top$- and $B^\bot$-ontology as opposite sides of a spectrum made out of diverse colours.
\end{remark}
\begin{example}
	An example of an admissible configuration of coins is the following (cf. \autoref{fig_coins}): we assume strength of existence goes up and down linearly to join the points where we have information about total existence of the subsets $C_X,C_Y,C_Z$: in the remaining instants of time, the coins out of sight for $X$ share an equal amount of existence in such a way that constraints \ref{ad:uno} and \ref{ad:due} above are satisfied: in this particular example, $p(Y,c_Y,\Th) = (\top,1)$, whereas $p(X,C_X,\Th)=p(Z,C_Z,\Th)=(\top,1/5)$, and $p(Y,c_Y,\F)=(\top,1/4)$, whereas $p(X,C_X,\F)=p(Z,C_Z,\F)=(\top,1)$.
\end{example}
% \begin{italian}
% 	\begin{remark}\label{di_navi_e_numeri}
% 		L'aritmetica di Tlon; proposizioni a forza additiva; parallelismi tra la nave di Teseo e le nove monete.
% 	\end{remark}
% \end{italian}
\begin{remark}[Continuity for a proposition]\label{continuiti}
	Let $p : U \to \Omega_I$ be a proposition; here we investigate what does it mean for $p$ to be (globally) continuous with respect to the Euclidean topology on $I=[0,1]$, in the assumption that its domain of definition $U$ is metrisable (this is true for example when $U$ is a subset of space-time). The condition is that
	\[ \forall \epsilon > 0,\,\exists \delta > 0 : |x-y|< \delta \To |px-py| < \epsilon \]
	Da questo segue immediatamente che quando $p$ è continua nel suo dominio, i valori di verità di $p$ in configurazioni ``vicine'' in un senso opportuno sono dati con forze allo stesso modo vicine (chiaramente questa è una descrizione spannometrica della nozione di continuità\dots).

	All elementary topology results apply to such a proposition: for example the set of forces with which $p$ is true or false is a connected subset of $\Omega_I$, compact if $U$ was compact.
\end{remark}
Continuity and discontinuity of a proposition $p : U \to \Omega_I$ now capture quite well other pieces of Borges' literary universe: here we provide two examples. We refrain from a deep, quantitative analysis, and we invite the reader to fill the details of our reasoning as a pleasant re-reading exercise of \cite{babil} and \cite{tlonEN}.\footnote{The plot of \cite{babil} in a nutshell is: in Babylon, a lottery game infiltrates reality to the point that it ends up governing the actions of all men; freeing them from free will while coorting them into pawns of an infinite, inescapable, unknowable game, governed by an iron, seemingly chaotic, probabilistic logic.}
\begin{example}[Discontinuity, sapphire from Taprobana]\label{bli}
	Propositions $p : U \to \Omega_{[0,1]}$ that are allowed to be discontinuous in its variables depend unpredictably from their context: such propositions model seemingly chaotic events triggered as the end terms of a chain of disconnected prior events; in fact, if we assume a real base for the domain of $p$, its continuity as stated in \autoref{continuiti} roughly means that events in the same neighbourhood --``near'' in space or temporally contiguous-- can't have too much different truth/strength values.

	A model for such a universe, where ``terrible consequences'' sometimes follow from the ``impersonal drawings, whose purpose is unclear'' characterising the Company's actions: seemingly disconnected (that ``a sapphire from Taprobana be thrown into the waters of the Euphrates''; that ``a bird be released from the top of a certain tower''; that ``every hundred years a grain of sand be added to (or taken from) the countless grains of sand on a certain beach''), but generating nontrivial dynamics when inserted in a suitable sequence as in \autoref{fig_dynamics}.

	Let us consider the dynamical system $(\nabla\circ p, [0,1])$ obtained from the iterates of the composition $\nabla \circ p : [0,1] \to [0,1]$ ($\nabla : I\amalg I \to I$ is the fold map of $I$, obtained from the identity of $I$ and the universal property of the coproduct). We start from a proposition $p$ depending on a free variable $t : I$; then, $p(t) = (u(t),\epsilon): \Omega_I$ consists of a strength and a truth value; but $p$ can be evaluated on $u(t)$ (although this is not its universal property, the fold map $\nabla$ just forgets the truth value, keeping the force): the process can thus be iterated as follows, exploiting the universal property of the natural number object in $\Set/I$:
	\begin{center}
		\begin{figure}[h]
			\def\line{\draw (0,0) -- (1,0); \draw (0,.5) -- (1,.5);}
			\begin{tikzpicture}
				\line
				\fill (.395,0) circle (1pt);
				\foreach \i/\j/\k in {2.25/.382/0,4.5/.537/.5,6.75/.874/.5,9/.128/0}{
						\begin{scope}[xshift=\i cm]
							\line
							\fill (\j,\k) circle (1pt);
							\node at (-.5,.375) {$\overset{\scriptscriptstyle\nabla\circ p}\mapsto$};
						\end{scope}
					}
			\end{tikzpicture}
			\caption{A dynamical system induced by the Company's infinite, impersonal drawings.}
			\label{fig_dynamics}
		\end{figure}
	\end{center}
	It is clear that the limit behaviour of such a sequence strongly depends on the analytic properties of $f$ (cf. \cite{strogatz1996nonlinear} for an intro to one-dimensional dynamic systems and for their applications see \cite{wiggins2003introduction}). In a well-known procedure, a repeated series of drawings can chaotically deform the configuration space on which $p$ is evaluated.\footnote{There is of course countless termination conditions on such process $p$; the series can be periodic, it can stop when $p$ reaches maximum or minimum force, or a prescribed value, or is inside/outside a certain range of forces\dots; we leave such speculations to the mystagogues of the Company, or to the lions of Qaphqa.}
\end{example}
Another compelling example is that of the monotonicity of a proposition depending, say, on a certain number of observers who acknowledge the ``existence'' of an object $R$ (be it physical or conceptual) in large numbers, from which very reason the existence of $R$ gains strength.
\begin{example}[Continuity: a few birds, a horse]\label{blu}
	Let us consider objects whose existence strength depends \emph{monotonously} and continuously from their parameters: for example a proposition $p$ may be ``truer'' the more people observe it, because
	\begin{quote}
		things became duplicated in Tlön; they also tend to become effaced and lose their details when they are forgotten. A classic example is the doorway which survived so long it was visited by a beggar and disappeared at his death. At times some birds, a horse, have saved the ruins of an amphitheater.  \hfill\cite{tlonEN}
	\end{quote}
	In such a situation, we can note that the strength of existence of some ruins --modeled as it is the more naive to do, like a rigid body $R$ in space-- depends on the number of its observers:
	\[\textstyle p(R, n) = \big(\top, 1-\frac{1}{n}\big)\]
	\begin{figure}[h]
		\begin{center}
			\begin{tikzpicture}[xscale=1.25]
				\draw[->, thick, >=stealth] (0,0) -- (6,0);
				\foreach \j/\i in { 1/.2
						, 2/.35
						, 3/.55
						, 4/.8
						, 5/1
					}
				\node[opacity=\i] at (\j,.65) {\fontsize{30}{30}\selectfont \Tribar};
			\end{tikzpicture}
		\end{center}
		\caption{On Tl\"on, there are things that exist stronger the more you believe in them. This is a consequence of the strength $p(\Tribar)$ monotonically depending on an increasing variable $n$ (``trust'' in that existence, ``belief'' that the impossible tribar \Tribar exists.)}
	\end{figure}
\end{example}
A last, somewhat dramatic example. What happens if we change topology on $I$? For example, we could brutally forget the Euclidean topology of the closed interval $[0,1]$, and regard $I$ as the disjoint union $\{ \{t\} \mid t: [0,1]\}$ of its points; so, the subobject classifier becomes the disjoint union of $[0,1]$ copies of $\{\perp,\top\}$. (See \autoref{fig:berkeley} below for a picture.)
\begin{example}[Burning fields at the horizon]\label{incendiata}
	Our tenet in this section is that Berkeley idealism of infinite and disconnected instants of time finds a natural home in our framework if the classifier is chosen to be the object $\Omega_I = \coprod_{t : [0,1]} \{ \perp,\top\}$. Such a peculiar logical framework allows for language to be reshaped in light of berkeleyan instantaneism: the various terms of the perceptual bundle are recorded and stockpiled by instantaneous accretion, by disjoint sum of their constituents. This is exactly what happens on Tl\"on for words like ``round airy-light on dark'' or ``pale-orange-of-the-sky''; objects are determined by their simultaneity, instead of their logical dependence: accretion superimposes fictitious meaning on a temporal sequence; it is just an illusion, a mistake of perception tricked into an illicit interpolation.
	\begin{quote}
		\hspace{.5em} Spinoza ascribes to his inexhaustible divinity the attributes of extension and thought; no one in Tlön would understand the juxtaposition of the first (which is typical only of certain states) and the second - which is a perfect synonym of the cosmos. In other words, they do not conceive that the spatial persists in time. The perception of a cloud of smoke on the horizon and then of the burning field and then of the half-extinguished cigarette that produced the blaze is considered an example of association of ideas.   \hfill\cite{tlonEN}
	\end{quote}
	Instantaneism is, of course, way larger than a paragraph: we explore the consequences of this principle in our \autoref{berkelei} below, where we sketch a \todo{yadda yadda}
	\begin{center}
		\begin{figure}[h]
			\begin{tikzpicture}[xscale=4, yscale=2]
				\fill[gray!30] (0,0) rectangle (1,1);
				\foreach \i in {.1,1,...,10}
				\draw[xshift=\i, yshift=-\i, ultra thin, fill=gray!30, opacity=.5] (0,0) rectangle (1,1);
				\draw[->, >=stealth, thin] (-.1,0) -- (-.1,1.5);
				\draw[->, >=stealth, thin] (-.1,0) -- (.5,-3/5);
				\draw[gray!60, xshift=.5cm, yshift=.25cm] (0,0) .. controls (.5,0) and (0,-.5) .. (.5,0);
			\end{tikzpicture}
			\caption{Times as an infinite, and infinitely subdivided, sequence of distinguished instants: the ``Berkeley paradox'' of non existence of causality lives in a topos, as nearby slices $E_t, E_s$ for small $|s-t|$ give no predictive power on how (and if) the transition from $E_t$ to $E_s$ might happen.}
			\label{fig:berkeley}
		\end{figure}
	\end{center}
\end{example}
  \section{Vistas on ontologies}\label{vistas}
\epigraph{¿No basta un solo término repetido para desbaratar y confundir la serie del tiempo? ¿Los fervorosos que se entregan a una línea de Shakespeare no son, literalmente, Shakespeare?}{\cite{confutacion}}
Looking at our \autoref{sec_coins} it's noticeable how mathematically consistent Borges' universe becomes when described using category theory. As we already said, this has to be regarded as an experiment, a literary \emph{divertissement}, and a test-bench for our main claim, that as abstract as it may seem, ontology can be made quantitative. But there's more, in the choice to analyse Borges' literary work: we can go further. We have this goal in mind here, while sketching some long term goals our work can aspire to become. We begin with a more academic discussion of classical idealism \emph{à la} Berkeley; in light of our \autoref{incendiata}, this can still be linked to \tlon's universe, as Borges has often been fascinated by, and mocked, classical idealism (see \cite{confutacion}). Let's say clearly that endeavouring on such a wide ground as classical idealism isn't the purpose of our work; yet, in his novel, Borges regards \tlon's language and philosophy as a concrete realisation of Berkeley's theory of knowledge. We thus find natural to explore such link with a classical piece of philosophy, as far as it can be taken.

In the following subsections, we sketch a more wide-ranging plan; surely a less substantiated one, and yet aimed at laying a foundation for future work, ours or others', in a research track that we feel is fertile and promising.
\subsection{Answers to the idealist}\label{berkelei}
Let us reconsider \autoref{bla}; we have already mentioned Berkeley's famous view of experience as a perceptual bundle of stimuli that are incapable to cohere.

In whatever sense it accepts that there are nine coins, Berkeleyan instantaneism right after asks to cut every existence $p(g,c,d) < (\top,1)$ at the ``purely false'' value, so that our notion of admissibility for a configuration of coins becomes untenable, because we can't have $\forall c.\forall d.\sum_u p(u,c,d) = (\top,1)$. Only what exists completely deserves to be called being. On thiouour side of the barricade, however, we find instantaneism untenable: what exists \emph{completely} (what \emph{exists} completely)? Idealism is equally untenable, when the word is intended as ``the belief that what lacks a percipient conscience must suddenly disappear in thin air'' (and thus, to ensure the world to exist, there must be God).

However, as strange and counterintuitive as it may seem, instantaneism and idealism have their point on \tlon:\footnote{Somehow, \cite{tlonEN} isn't far from an artificial world built to mock idealism; cf. also \cite{borges1997otras} and the famous essay \cite{confutacion} therein.} we should then be able to find a tenable justification for both of them, possibly from within topos theory; possibly, ``making no pact with the impostor Jesus Christ'' (cf. \cite{tlonEN}).

As we have seen throughout the entire work, in the internal language of a topos a proposition $p$ takes its truth values in a much wider range of possibilities than a mere yes-no, all-nothing dualism; thus, existence gives way to a more nuanced notion that can be (in)valid with a certain strength $t : I$. The language we introduced so far was precisely meant to quantify this stray from classical logic. Here, itis meant to quantify how much Berkeley's ``cut'' above is a blunt one, easily falsifiable (to say the least) even on \tlon.

To put it shortly, instantaneism arises forgetting the topology and order on $I$ (cf. \autoref{something_on_I}), and idealism arises from quotienting one of the copy $I\times \{\top\}$ of $I\times \{\top,\perp\}$ to a point (cf. \autoref{fig_Omega}). The problem is thus phrased in a way that makes its solution completely natural: just don't be a XVIII$^\text{th}$ century Irish empiricist; if a subobject classifier carries additional structure, don't forget it.

Let us consider a simple proposition $p$, like ``the World is there'': let us assume that $p : U \to \Omega_I$ depends on a certain number of variables $\vec x$; now, when saying that closing their eyes the world disappears, David and George claim that $p(\eye,\firstblank) \neq p(\noeye,\firstblank)$, and even more, that $p(\noeye,\firstblank)=(\bot,1)$. A certain number of implicit assumptions are already made here: first, that $p$'s domain of definition splits into a product $E\times U'$, where $E = \{\eye,\noeye\}$ is a space of parameters taking into account the state of George and David's eyes, so that $p$ is evaluated on pairs $(e,\vec y)$. Second, that every force $p(\noeye,\vec y) < (\top,1)$ collapses to the minimum value ``completelyy false'' so that at any $(\noeye,\vec y)$ configuration the World is \emph{not} there. 

A more mathematically-inspired analysis of the situation is of course possible: formally, George and David collapsed a certain subset of the classifier $\Omega_I$ to a point, by means of a quotient map $Q$
\[
  \begin{tikzpicture}
    \draw[gray!70,dashed] (2,1.25) -- (2,-.25) node[below] {$\{\perp,\top\}$};
    \draw[fill] (0,0) circle (1pt) node[left] {$0$};
    \draw[fill] (2,0) circle (1pt) node[right] (dis) {$1$};
    \draw (0,0) -- (2,0);
    \begin{scope}[yshift=1cm]
      \draw[fill] (0,0) circle (1pt) node[left] {$0$};
      \draw[fill] (2,0) circle (1pt) node[right] (dat) {$1$};
      \draw (0,0) -- (2,0);
    \end{scope}
    \node at (3,.5) {$\overset{Q}\longrightarrow$};
    \begin{scope}[xshift=4cm]
      \fill[gray!20] (0,0) -| (2,1);
      \draw[gray!70,dashed] (2,1.25) -- (2,-.25) node[below] {$\{\perp,\top\}$};
      \draw[fill] (0,1) circle (1pt) node[left] {$0$};
      \draw[fill] (2,1) circle (1pt) node[right] (dis) {$1$};
      \draw (0,1) -- (2,1);
    \end{scope}
  \end{tikzpicture}
\]
(An even more blunt choice would have been to take just the fiber $\{\perp,\top\}\cong \{\perp,\top\}\times \{1\}$, thus falling into classical logic; but we have no information about how George and David handle falsehood.)

Now, what happens at $p(\eye,\vec y)$, as opposed to what happens at $p(\noeye,\vec y)$? Strictly speaking, we can't rule out the possibility that perception is affected by observers (it certainly is on \tlon, in Babylon, or in a quantum mechanics laboratory); so, George and David closing their eyes somehow affects the degree of existence of the World. In what way? In classical logic, this degree of existence is trivially binary; but in the right topos $\clE$ there can be even a continuous spectrum of different choices. So, the apparent paradox of idealism can be turned into a safe statement: observation --or lack thereof-- affects beings so that something changes between $p(\eye,\firstblank)$ and $p(\noeye,\firstblank)$. It does so in a complicated, perhaps even a mysterious way; but in a coherent fashion.

Yet, it is improper to say that the world vanished, so we can safely assume that this dent is so infinitesimal that it goes undetected by our instruments. (Yet, what kind of instrument detects ``existence'' as a pure concept?)

Of course, the practice of rewording identity and persistence-in-time as mathematical notions affects natural language as well. When David and George close their eyes, they are tricked into believing that the World had disappeared. But this loss of information is merely induced by the blunt quotient they performed on elements of $U$ in $Z^\top$ (cf. the notation in \autoref{alcuni_set}); on the contrary, whenever David and George admit that there is a continuous parameter (a ``strength'') modelling $p$'s truth in the sense defined here they implicitly accept to move towards a nonclassical universe of discourse (a topos). The rest is mere calculation.

From inside \tlon, this gives way to the epistemic vision of its inhabitants, where each of $X,Y,Z$ does not know what happens to the coins of the other observers, and in what secret way $C$ exists as a global entity: without an awakening about the shape of their formal logic, the topos-theoretic model is unusable by \tlonians.

Even this superficial analysis is already sufficient to make our point, that George and David peculiar theory of existence isn't barred from our model; on the contrary, it arises as a very specific example of internal logic. How this is done, follows from a standard procedure in problem-solving:
\begin{itemize}
  \item First, identify the constraints forcing your choice of $I$ (``what logic shapes your $I$?'' and vice versa ``what $I$ best approximates the logic you want to describe?'');
  \item Given an explicit description of $\Omega_I$, compute proposition strength (``where are you seeking $p$ to be true/false'', and at which strength?)
\end{itemize}
Points of evaluation can be provided by a temporal, spatial or any other kind of reference whatsoever: we are completely agnostic towards the shape of space-time, towards the structure or the properties of the set it forms; moreover, every configuration respecting the prescribed constraints is ``correct'' from within the topos. Finally, our approach is computational: given the initial piece of data (for the nine coins: the observer, a subset of coins, a day of the week) all else necessarily follows from a calculation.\footnote{Another source of agnosticism towards the structure of $I$ was anticipated in \autoref{existence}, and consists in the refusal to adopt a temporal logic framework; we can sometimes interpret the elements of $I$ as time instants, but this is not an obligation at all. A presentist, or a Borges' coherent idealist (capable to deduce from idealism the nonexistence of time, as in \cite{confutacion}) can establish the truth of $p$ without renouncing their ontological stance; they are just forced to accept the result of a calculation.}

We believe this intuitive approach to fuzzy logic through category theory has great potential for philosophical debate, as it fits George and David's counterintuitive perspective into a coherent frame, and it avoids the barren dichotomies in which the debate had stalled for a few centuries.
\subsection{Ye shall know them by their fruit}\label{frutti}
The main point of our paper can be summarised very concisely: an ``ontology'' is a category $\clO$, inside which ``Being unravels''. Every existence theory shall be reported, and is relative, to a fixed ontology $\clO$, the ``world we live in''; such existence theory coincides with the internal language $\clL(\clO)$ of the ontology/category (from now on we employ the two terms as synonyms), in the syntax\hyp{}semantics adjunction of \cite{syntax-semantics_duality}.

So determined, the internal language of an ontology $\clO$ is the collection of ``things that can be said'' about the elements of the ontology.

If, now, ontology is the study of Being, and if we are structuralist in the metatheory (cf. \autoref{sec_intro}), we cannot know beings but through their attributes. Secretly, this is \emph{Yoneda lemma} (cf. \cite[1.3.3]{Bor1}), the statement that the totality of modes of understanding a ``thing'' $X$ coincides with the totality of modes your language allows you to probe $X$. Things do not exist out of an ontology; objects do not exist out of a category; types do not exist out of a type theory. In relational structures, objects are known via their modes of interaction with other objects, and these are modelled as morphisms $U \to X$; Yoneda lemma posits that we shall ``know objects by their morphisms'': the object $X:\clC$ coincides with the totality of all morphisms $U\to X$, organised in a coherent bundle (a functor $\clC(\firstblank,X) : \clC \to \Set$).

All in all, an ontology is a mode of understanding the attributes of Being: a category, be it in an Aristotelic or in a structuralist sense. As a consequence \emph{meta}ontology, i.e. the totality of such ways of understanding, must coincide with the general theory of such individual modes: with \emph{category theory}.

We should say no more on the matter: everything else pertains to \emph{meta}ontology.

Indeed, questions as ``where is language'' and what general principles inspire it might have an anthropological or even neurophysiological answer; not an ontological one. Or at least, not without paying a price: operative ontology as sketched here has limits. It can't speak of Being out of the one ignited by itself. (read as \emph{category theory has limits}: it cannot speak efficiently of objects out of a fixed universe of discourse. Implicit: category theory also has merits.)
\subsection{Metaontology} \label{metaon}
The gist of our \autoref{bla} is that $X,Y,Z$ can't assess the existence of the coins classically; they just have access to partial information allowing neither a global statement of existence on the set $C$ of coins lost by $X$, nor an unbiased claim about the meaning thereof. Coins are untouched as a global lost conglomerate, yet their strength of existence is very likely to change \emph{locally}.

This begs the important questions of to which extent language determines ontology. And to which extent it constrains its expressive power? To which extent the inhabitants of \tlon fail to see what is exactly ``a topos further'' (persistence of existence through time). To what extent \emph{we} fail to see that\dots

Far from claiming we resolved such metaontological issues, here we make a hopefully clarifying statement: according to Quine and his school, ontology can be defined as the ``domain over which [logical, or natural language's] quantifiers run''. This is not wrong; in fact, it is perfectly compatible with our views. But we work at a raised complexity level, in the following sense. In our framework, ontology \emph{à la} Quine still is a category, because (cf. \autoref{quantifezzi}) a quantifier can be described as a certain specific kind of functor
\[\forall,\exists : PX \to PY\]
between powersets regarded as \emph{internal} categories of our ambient ontology $\clO$; in light of this, it's easy to imagine this pattern to continue: if Quine calls metaontology what we call ontology, i.e. the metacategory grounding his propositional calculus is a single ambient category \emph{among many}, we live ``one universe higher'' in the cumulative hierarchy of foundations and meta-foundations: our ontology possesses a higher dimensionality, and harbours Quinean theory as an internal structure (see \autoref{internista} for the definition of an internal category). So, it shall exist, \emph{somewhere} -at least in some secret way, hidden from the comprehension of men- a language that calls ontology what we call metaontology (and thus a language that declassifies Quine's to a sub-ontology -whatever this means).

Finding the bottom of this tower of turtles is the aim of the track of research, of very ancient tradition, within which we want to insert the present work.

Of course our work does not make a single comment on how, if, and when, this ambitious foundational goal can be achieved. We just find remarkable that framing Quine's definition in this bigger picture is not far from the so-called practice of ``negative thinking'' in category theory: negative thinking is the belief that a high-dimensional/complex entity can be understood through the analogy with its low-dimensional/complexity counterparts; see \cite{nlab:category-order,nlab:neg-think} for a minimal introduction to the principle, \cite{baez2010lectures} for a practical introduction, and see \cite{gowers2007} for what Gowers calls ``backwards generalisation''.
\subsection{Conclusion}
The circle closes on, and motivates better, our initial foundational choice: categories and their theory correspond 1:1 to ontology and its theory. However, countless important issues remain open: in what sense this is satisfying? In what sense the scope of our analysis is not limited by this choice? What's his foundation? Is this a faithful way to describe such an elusive concept as ``Being''?

None of these questions is naive; in fact, each legitimately pertains to metaontology and has no definitive answer. More or less our stance is as follows: approaching problems in ontology with a reasonable amount of mathematical knowledge is fruitful. Yet, the problem of what is a foundation for that mathematics remains (fortunately!) wide open; it pertains to metaontology, whose ambitious effort is to clarify ``what there is''. We believe the philosophers' job to work in synergy with quantitative knowledge, approaching the issue with complementary tools.
  \appendix
  \section{Category theory}
\epigraph{El atanor está apagado -repitió- y están llenos de polvo los alambiques. En este tramo
	de mi larga jornada uso de otros instrumentos.}{\cite{arena}}
\subsection{Fundamentals of CT}
Throughout the paper we employ standard basic category-theoretic terminology, and thus we refrain from giving a self contained exposition of elementary definitions. Instead, we rely on famous and wide-spread sources like \cite{Bor1,Bor2,McL,riehlcontext,leinster2014basic,simmons2011introduction}.

Precise references for the basic definitions can be found
\begin{itemize}
	\item for the definition of category, functor, and natural transformation, in \cite[1.2.1]{Bor1}, \cite[I.2]{McL}, \cite[1.2.2]{Bor1}, \cite[I.3]{McL}, \cite[1.3.1]{Bor1}.
	\item The Yoneda lemma is stated as \cite[1.3.3]{Bor1}, \cite[III.2]{McL}.
	\item For the definition of co/limit and adjunction, in \cite[2.6.2]{Bor1}, \cite[III.3]{McL}, \cite[2.6.6]{Bor1}, \cite[III.4]{McL} (consider in particular the definitions of \emph{pullback}, \emph{product}, \emph{terminal object}).
	\item For the definition of accessible and locally presentable category in \cite[5.3.1]{Bor2}, \cite[5.2.1]{Bor2}, \cite{Adamek1994}.%, \cite[]{}.
	\item Basic facts about ordinal and cardinal numbers can be found in \cite{kunen}; another comprehensive reference on basic and non-basic set theory is \cite{jech2013set}.
	\item The standard source for Lawvere functorial semantics is Lawvere's PhD thesis \cite{lawvere1963functorial}; more modern accounts are \cite{hyland2007category}.
	\item Standard references for topos theory are \cite{mac1992sheaves,JohnstonePT}. See in particular \cite[VI.5]{mac1992sheaves} and \cite[5.4]{JohnstonePT} for what concerns the Mitchell-Bénabou language of a topos.
\end{itemize}
\subsection{Toposes}\leavevmode
For us, an \emph{ordinal number} will be an isomorphism class of well\hyp{}ordered sets, and a \emph{cardinal number} is any ordinal which is not in bijection with a smaller ordinal. Every set $X$ admits a unique \emph{cardinality}, i.e. a least ordinal $\kappa$ with a bijection $\kappa \cong X$ such that there are no bijections from a smaller ordinal. We freely employ results that depend on the axiom of choice when needed. A cardinal $\kappa$ is \emph{regular} if no set of cardinality $\kappa$ is the union of fewer than $\kappa$ sets of cardinality less than $\kappa$; all cardinals in the following subsection are assumed regular without further mention.

Let $\kappa$ be a cardinal; we say that a category $\clA$ is $\kappa$\hyp{}filtered if for every category $\clJ\in\Qat_{<\kappa}$ with less than $\kappa$ objects, $\clA$ is injective with respect to the cone completion $\clJ\to \clJ^\rhd$; this means that every diagram
\[
	\vcenter{\xymatrix{
			\clJ\ar[d]\ar[r]^D & \clA \\
			\clJ^\rhd\ar@{.>}[ur]_{\bar D}
		}}
\]
has a dotted filler $\bar D : \clJ^\rhd \to \clA$.

We say that a category $\clC$ admits filtered colimits if for every filtered category $\clA$ and every diagram $D : \clA \to \clC$, the colimit $\colim D$ exists as an object of $\clC$. Of course, whenever an ordinal $\alpha$ is regarded as a category, it is a filtered category, so a category that admits all $\kappa$\hyp{}filtered colimits admits all colimits of chains
\[
	C_0 \to C_1 \to \cdots \to C_\alpha \to\cdots
\]
with less than $\kappa$ terms. A useful, completely elementary result is that the existence of colimits over all ordinals less than $\kappa$ implies the existence of $\kappa$\hyp{}filtered colimits; this relies on the fact that every filtered category $\clA$ admits a cofinal functor (see \cite{Bor1}) from an ordinal $\alpha_\clA$.

We say that a functor $F : \clA \to \clB$ \emph{commutes with} or \emph{preserves} filtered colimits if whenever $\clJ$ is a filtered category, $D : \clJ \to \clA$ is a diagram with colimit $L=\colim_\clJ D_j$, then $F(L)$ is the colimit of the composition $F\circ D$. Anoter common name for such an $F$ is a \emph{finitary} functor, or a functor \emph{with rank $\omega$}.
\begin{definition}\label{accepre}
	Let $\clC$ be a category;
	\begin{itemize}
		\item We say that $\clC$ is \emph{$\kappa$\hyp{}accessible} if it admits $\kappa$\hyp{}filtered colimits, and if it has a \emph{small} subcategory $\clS\subset \clA$ of $\kappa$\hyp{}presentable objects such that every $A\in\clA$ is a $\kappa$\hyp{}filtered colimit of objects in $\clS$.
		\item We say that $\clC$ is \emph{(locally) $\kappa$\hyp{}presentable} if it is accessible and cocomplete.
	\end{itemize}
	The theory of presentable and accessible categories is a cornerstone of \emph{categorical logic}, i.e. of the translation of model theory into the language of category theory.

	Accessible and presentable categories admit \emph{representation theorems}:
	\begin{itemize}
		\item A category $\clC$ is accessible if and only if it is equivalent to the ind\hyp{}completion $\text{Ind}_\kappa(\clS)$ of a small category, i.e. to the completion of a small category $\clS$ under  $\kappa$\hyp{}filtered colimits;
		\item A category $\clC$ is presentable if and only if it is a full reflective subcategory of a category of presheaves $i : \clC \to \Qat(\clS^\op,\Set)$, such that the embedding functor $i$ commutes with $\kappa$\hyp{}filtered colimts.
	\end{itemize}
\end{definition}
All categories of usual algebraic structures are accessible, and they are locally presentable as soon as they are cocomplete; an example of a category which is $\aleph_1$\hyp{}presentable but not $\aleph_0$\hyp{}presentable: the category of metric spaces and short maps.

We now glance at \emph{topos theory}:
\begin{definition}\label{eletop}
	An \emph{elementary topos} is a category $\clE$
	\begin{itemize}
		\item which is \emph{Cartesian closed}, i.e. each functor $\firstblank\times A$ has a right adjoint $[A, \firstblank]$;
		\item having a \emph{subobject classifier}, i.e. an object $\Omega\in\clE$ such that the functor $\text{Sub} : \clE^\op\to \Set$ sending $A$ into the set of isomorphism classes of monomorphisms $\var{U}{A}$ is representable by the object $\Omega$.
	\end{itemize}
	The natural bijection $\clE(A,\Omega)\cong\text{Sub}(A)$ is obtained pulling back the monomorphism $U\subseteq A$ along a \emph{universal arrow} $t : 1\to \Omega$, as in the diagram
	\[
		\vcenter{\xymatrix{
				U \pb\ar[r]\ar[d]& 1\ar[d]^t \\
				A \ar[r]_{\chi_U}& \Omega
			}}
	\]
	so, the bijection is induced by the map $\var{U}{A}\mapsto \chi_U$.
\end{definition}
\begin{definition}\label{grotop}
	A \emph{Grothendieck topos} is an elementary topos that, in addition, is locally presentable.
\end{definition}
The well-known \emph{Giraud theorem} gives a proof for the difficult implication of the following \emph{recognition principle} for Grothendieck toposes:
\begin{theorem}
	Let $\clE$ be a category; then $\clE$ is a Grothendieck topos if and only if it is a left exact reflection of a category $\Qat(\clA^\op,\Set)$ of presheaves on a small category $\clA$.
\end{theorem}
(recall that a \emph{left exact reflection} of $\clC$ is a reflective subcategory $\clR\hookrightarrow \clC$ such that the reflector $r : \clC \to \clR$ preserves finite limits. It is a reasonably easy exercise to prove that a left exact reflection of a Grothendieck topos is again a Grothendieck topos; Giraud proved that all Grothendieck toposes arise this way.)
\subsection{A little primer on algebraic theories}\label{funsemanzi}
The scope of this short subsection is to collect a reasonably self-contained account of functorial semantics. It is unrealistic to aim at such a big target as providing a complete account of it in a single appendix; the reader is warmly invited to parallel their study with more classical references as \cite{lawvere1963functorial}.
\begin{definition}[Lawvere theory]\label{lo_tiori}
	A \emph{Lawvere theory} is a category having objects the natural numbers, and where the sum on natural numbers has the universal property of a categorical product, as defined e.g. in \cite[2.1.4]{Bor1}.
\end{definition}
Let us denote $[n]$ the typical object of $\clL$. Unwinding the definition, we deduce that in a Lawvere theory $\clL$ the sum of natural numbers $[n+m]$ is equipped with two morphisms $[n] \leftarrow [n+m] \to [m]$ exhibiting the universal property of the product.

Every Lawvere theory comes equipped with a functor $p : \cate{Fin}^\op \to \clC$ that is the identity on objects and preserves finite products. A convenient shorthand to refer to the Lawvere theory $\clL$ is thus as the functor $p$, or as the pair $(p,\clL)$.
\begin{definition}
	The category $\cate{Law}$ of Lawvere theories has objects the Lawvere theories, understood as functors $p : \cate{Fin}^\op \to \clL$, and morphisms the functors $h :  p\to q$ such that the triangle
	\[\xymatrix{
			& \cate{Fin}^\op \ar[dr]^q \ar[dl]_p & \\
			\clL \ar[rr]_h && \clM
		}\]
	is commutative. It is evident that $\cate{Law}$ is the subcategory of the undercategory $\cate{Fin^\op}/\Qat$ (see e.g. \cite[I.6]{McL}for a precise definition) made by those functors that preserve finite products.
\end{definition}
\begin{remark}
	The category $\textsf{Law}$ has no nonidentity 2-cells; this is a consequence of the fact that a natural transformation $\alpha : h \To k$ that makes the triangle ``commute'', i.e. $\alpha * p = \id_q$ must be the identity on all objects.
\end{remark}
\begin{example}[The trivial theories]
	The category $\cate{Fin}^\op$, opposite to the category of finite sets and functions, is the initial object in the category  $\cate{Law}$; the terminal object is constructed as follows: the category $\clT$ has objects the natural numbers, and $\clT([n],[m])=\{*\}$ for every $n,m: \bbN$. It is evident that given this definition, there is a unique identity-on-objects functor $\clL \to \clT$ for every other Lawvere theory $(p,\clL)$.
\end{example}
\begin{definition}[Model of a Lawvere theory]
	A \emph{model} for a Lawvere theory $(p ,\clL)$ consists of a product-preserving functor $L : \clL \to \Set$. The subcategory $[\clL,\Set]_\times \subset [\clL, \Set]$ of models of the theory $\clL$ is \emph{full}, i.e. a morphism of models $L \to L'$ consists of a natural transformation $\alpha : L \Rightarrow L'$ between the two functors.
\end{definition}
Observe that the mere request that $\alpha : L \to L'$ is a natural transformation between product preserving functors means that $\alpha_{[n]} : L[n] \to L'[n]$ coincides with the product $(\alpha_{[1]})^n : L[1]^n \to L'[1]^n$.
\begin{proposition}
	Let $p : \cate{Fin}^\op\to \clL$ be a Lawvere theory. Then, the following conditions are equivalent for a functor $L : \clL \to \Set$:
	\begin{itemize}
		\item $L$ is a model for the Lawvere theory $(p,\clL)$;
		\item the composition $L\circ p : \cate{Fin}^\op \to \Set$ preserves finite products;
		\item there exists a set $A$ such that $L\circ p = \Set(j[n],A)$.
	\end{itemize}
\end{proposition}
\begin{corollary}\label{da_pull}
	The square
	\[
		\xymatrix{
			\cate{Mod}(p,\clL) \ar[d]_u \ar[r]^r & [\clL,\Set]\ar[d]^{p^*} \\
			\Set \ar[r]_{N_j} & [\cate{Fin}^\op,\Set]
		}
	\]
	is a pullback of categories. The functor $u$ is completely determined by the fact that $u(L) = L[1]$, $r$ is an inclusion, and $N_j(A) = \lambda F.\Set(F,A)$ is the functor induced by the inclusion $j : \cate{Fin} \subset \Set$.
\end{corollary}
\begin{corollary}
	The category of models $\cate{Mod}(p,\clL)$ of a Lawvere theory is a locally presentable, accessibly embedded, complete and cocomplete subcategory of $[\clL,\Set]$. Moreover, the forgetful functor $u : \cate{Mod}(p,\clL) \to \Set$ of \autoref{da_pull} is \emph{monadic} in the sense of \cite[4.4.1]{Bor2}. A complete proof of all these facts is in \cite[3.4.5]{Bor2}, \cite[3.9.1]{Bor2}, \cite[5.2.2.a]{Bor2}. A terse argument goes as follows: the functors $p^*, N_j$ are accessible right adjoints between locally presentable categories; therefore, so is the pullback diagram: $r$ is a fully faithful, accessible right adjoint, and $u$ is an accessible right adjoint, that moreover reflects isomorphisms. It can be directly proved that it preserves the colimits of split coequalizers, and thus the adjunction $f \dashv u$ is monadic by \cite[4.4.4]{Bor2}.
\end{corollary}
The last technical remark that we collect sheds a light on the discorso prolisso in \autoref{as_places}: the models of a thelory $\clL$ interpreted in the category of models of a theory $M$ correspond to the models of a theory $\clL \otimes \clM$, defined by a suitable universal property:
\[
	\cate{Mod}(\clL\otimes \clM,\Set)  \cong
	\cate{Mod}(\clL, \cate{Mod}(\clM,\Set))  \cong
	\cate{Mod}(\clM, \cate{Mod}(\clL,\Set)).
\]
\begin{definition}
	Given two theories $\clL$ and $\clM$ it is possible to construct a new theory called the \emph{tensor product} $\clL \otimes \clM$; this new theory can be characterized by the following universal property: the models of $\clL \otimes \clM$ consist of the category of $\clL$-models interpreted in the category of $\clM$-models or, equivalently (and this is remarkable) of $\clM$-models interpreted in the category of $\clL$-models.
\end{definition}
\begin{theorem}
	(\cite[4.6.2]{Bor2}) There is an equivalence between the following two categories:
	\begin{itemize}
		\item $\cate{Law}$, regarded as a non-full subcategory of the category $\textsf{Fin}^\op/\Qat$, i.e. where a morphism of Lawvere theories consists of a functor $h : \clL \to \clM$ that preserves finite products;
		\item \emph{finitary} monads, i.e. those monads that preserve filtered colimits, and morphisms of monads in the sense of \cite[4.5.8]{Bor2}.
	\end{itemize}
\end{theorem}
\begin{proof}
	The proof goes as follows: given a Lawvere theory $p : \textsf{Fin}^\op \to \clL$, we have shown that the functor $u : \cate{Mod}(p) \to \Set$ in the pullback square \autoref{da_pull} has a left adjoint $f : \Set \to \cate{Mod}(p)$; the composition $uf$ is thus a monad on $\Set$. This is functorial, when a morphism of monads is defined
\end{proof}
\begin{definition}
	\label{internista}
	Let  $\clC$ be a category with finite limits. An \emph{internal category} $\underline{A}$ consists of a pair $(A_0, A_1)$ of objects of $\clC$, endowed with morphisms $s,t,c,i$ as in the following diagram
	\[
		\xymatrix{
		A_1 \times_{A_0} A_1 \ar[r]|-c \ar@<6pt>[r]^-{p_0}  \ar@<-6pt>[r]_-{p_1} & A_1 \ar@<6pt>[r]^s  \ar@<-6pt>[r]_t & A_0 \ar[l]|i
		}
	\]
	where $A_1 \times_{A_0} A_1$ is obtained pulling back $s,t$. These data must satisfy the following axioms, resembling the category axioms:
	\begin{enumtag}{ic}
		\item $i$ equalises the pair $(s,t)$, i.e. $s\circ i= t\circ i$, and this composition makes the identity $1_{A_0}$;
		\item $t\circ p_1 = t\circ c$ and $s\circ p_0 = s \circ c$;
		\item $c\circ\la 1_{A_1},i\circ s\ra = c\circ\la i\circ t, 1_{A_1}\ra$, and this composition makes the identity $1_{A_1}$;
		\item $c$ is \emph{associative}, i.e. $c \circ (1_{A_1}\times_{A_0} c)= c \circ (c \times_{A_0} 1_{A_1})$.
	\end{enumtag}
\end{definition}
These axioms are meant to re-enact the definition of category, as an abstract ``object of objects'' and ``object of morphisms'', endowed with maps $s,t: A_1 \to A_0$ sending every morphism to its \emph{source} and \emph{target}, with an \emph{identity selector} map $i : A_0 \to A_1$ and with a composition partial binary operation $c : A_1 \times A_1 \to A_1$ whose domain is the object of \emph{composable} arrows. Cf. \cite[8]{Bor1} for a torough discussion.
It has to be noted however that the theory of categories is \emph{not} an algebraic theory in the sense of our \autoref{lo_tiori}, and this just because the composition operation is only \emph{partially} defined over the domain of composable arrows, i.e. on the pullback $A_1\times_{A_0} A_1$. Theories whose syntax allows for a number of partially defined relation and function symbol are termed \emph{essentially algebraic}. See \cite{Adamek1994} for a torough discussion of the topic.

  \bibliography{allofthem}{}
  \bibliographystyle{amsalpha}
\end{document}
