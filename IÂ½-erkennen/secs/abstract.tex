\begin{abstract}
	We outline a `formal theory of scientific theories' rooted in the theory of profunctors; the category theoretic asset stresses the fact that the scope of scientific knowledge is to build `meaningful connections' (i.e. well-behaved adjunctions) between a linguistic object (a `theoretical category' $\clT$) and the world $\clW$ said language ought to describe. Such world is often unfathomable, and thus we can only resort to a smaller fragment of it in our analysis: this is the `observational category' $\clO \subseteq \clW$. 
	From this we build the category $[\clO^\op,\Set]$ of all possible displacements of observational terms $\clO$. The self-duality of the bicategory of profunctors accounts for the fact that theoretical and observational terms can exchange their r\^ole without subtantial changes in the resulting predictive\hyp{}descriptive theory; this provides evidence for the idea that their separation is a mere linguistic convention; to every profunctor $\fkR$ linking $\clT$ and $\clO$ one can associate an object $\clO\uplus_\fkR\clT$ obtained \emph{gluing} together the two categories, and accounting for the mutual relations subsumed by $\fkR$. 
	Under mild assumptions, such an arrangement of functors, profunctors, and gluings provides a categorical interpretation for the `Ramseyfication' operation, in a very explicit sense: in a scientific theory, if a computation entails a certain behaviour for the system the theory describes, then saturating its theoretical variables with actual observed terms, we obtain the entailment \emph{in the world}.
\end{abstract}