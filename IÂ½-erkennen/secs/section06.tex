\section{Towards a universal notion of theory}\label{sec_6_universal}
This concluding section wraps up the various topics touched along the paper.

\medskip
As we have seen in \autoref{resoudre_la_tension}, profunctor theory is just a mathematization of the well-known tenet that epistemology is a relational theory: scientific theories are but well-behaved adjunctions between the part of the world that we want to model (this part doesn't have to be physical), the part of the world to which we have experimental access, and the linguistic paraphernalia that we use to represent the latter in terms of the former.

Theories can't be told apart in terms of their objects of study; instead, they can be classified in terms of the web of relations that they entertain together with others theories/categories.

This entails that
\begin{itemize}
	\item There is no substantial difference between the syntactic categories $\clT,\clO\subset \clW$, i.e. between observative and theoretical terms. Far from being a step back towards an efficient representation of reality, this elegantly gets rid of the early gawky attempts towards a `naturalisation of epistemology', originally thought to even happen \emph{inside syntax}.

	      No theory can exit language; this does not mean that a theory isn't telling something about the world: instead, theories --and metatheories about the world-- are linguistic objects above all else. The way in which this linguistic practice unravels, on the other hand, is too loose to be functional; it is, instead, relational.
	\item Being able to exchange the r\^oles of $\clT,\clO$ is reflected in the model in the property of every profunctor (i.e. Ramsey map) $\fkR : \clT \pto \clO$ to be `swapped' into a Ramsey map $\fkR^\op : \clO \pto \clT$. The observational and theoretical categories bear this name as a result of nothing but an arbitrary labeling. This gives way to all sorts of fruitful interpretations: it becomes possible to label as `sciences' sufficiently expressive descriptions of possible worlds (few would object that the complicated hierarchy of sub-worlds in which Eä is divided, or the narration by which it became the world as we know it, form a `science'), or the --strictly speaking-- unobservable phenomena that occur in Physics as well as in theology.

	      Being `scientific' is thus not a property of the object we want to describe; instead, `scientificity' is a measure of the faithfulness of described phenomena in a `world' $\clW$, and of the ability of descriptions to cast predictions on the behaviour of the system. This is akin to scientific practice: if quantum mechanics gave more correct predictions about the world accepting that uncertainty is induced by an \emph{ogbanje}, physicists would study Igbo mythology instead of functional analysis.

	      % \item Una 'teoria delle teorie' non sarà object-dependent ma context-dependent, dove context è l'insieme delle aggiunzioni che si stabiliscono tra le categorie dentro le quali scegliamo di muoverci. 
\end{itemize}